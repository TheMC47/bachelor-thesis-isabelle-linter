\chapter{Evaluation}\label{chapter:evaluation}
In this chapter we evaluate the linter in ``real-world'' conditions, in
order to evaluate its performance. Moreover, the experiments give insight to
what extent the best practices that are checked by the linter are applied
in various Isabelle theories.

The tests are performed on a machine with a 10 Core 2.4 GHz processor and 46 
GB of RAM, running CentOS 8 and Isabelle with the Isabelle2021 February version.

\section{Approach}
The linter is evaluated against two classes of Isabelle theories:
the HOL theories, intended to represent theories from official Isabelle
libraries, and a random set of 20 sessions from the Archive of Formal 
Proofs \footnote{Lp, LTL\_Master\_Theorem, Constructive\_Cryptography\_CM, Recursion-Addition, Randomised\_Social\_Choice, Parity\_Game, LTL, Possibilistic\_Noninterference, IMP2\_Binary\_Heap, DataRefinementIBP, Transitive-Closure, Stewart\_Apollonius, Minkowskis\_Theorem, Stellar\_Quorums, Smooth\_Manifolds, Category2, VerifyThis2019, No\_FTL\_observers, BDD, Pairing\_Heap}. The bundles of lints employed
are respectively the \textit{foundational} and the \textit{afp} bundles.

The command line tool is utilized
to perform these experiments. For the AFP sessions, a base logic image
is supplied (option \texttt{-b}) to each session in order to speed up 
retrieving the snapshots. The values related to timing
are averaged across 10 runs. Crucially, however, these values only 
consider
the time taken to lint the theories, without including the time
needed to fetch the snapshots (which is dominant when using the
command line tool). 

\section{Results}
\subsection{Report summary}
The report summary for linting Isabelle/HOL and the selected
AFP sessions can be seen on \autoref{tab:hol-summary} and
\autoref{tab:afp-summary} respectively. The linter detected a total
of 252 lints in Isabelle/HOL and 575 lints in the 
AFP sessions. This is a significant difference, considering
the sizes of both sets: Isabelle/HOL consists
of 111 theories with around 113 thousand lines of theories, whereas
the AFP selection contains 135 theories with around 73
thousand lines. In relative terms, one lint got triggered every
448 lines in Isabelle/HOL versus every 126 lines in
the AFP sessions. The distribution of these lints
is however comparable: those with high severity were \SI{28.97}{\percent} of the
lints detected in Isabelle/HOL, which is lower than the
\SI{45.39}{\percent} in the AFP sessions. Respectively for medium severity 
lints it is \SI{58.35}{\percent} versus \SI{44.17}{\percent} and \SI{14.68}{\percent} versus \SI{10.43}{\percent} for the low
severity lints.

\begin{table}
    \centering
    
\begin{tabular}{llr}
\toprule
Severity & Name & Number of occurences     \\
\midrule
High   & \hyperref[lint:unrestrictedauto]{Unrestricted auto} &  71 \\
       & \hyperref[lint:globalattr]{Global attribute on unnamed lemma} &   2 \\
Medium & \hyperref[lint:complexisar]{Complex Isar initial method} &  62 \\
       & \hyperref[lint:implicitrule]{Implicit rule} &   55 \\
       & \hyperref[lint:complexmethod]{Complex method} &   20 \\
       & \hyperref[lint:lemmatrans]{Lemma-transforming attribute} &   3 \\
       & \hyperref[lint:applyisarswitch]{Apply-Isar switch} &   2 \\
Low & \hyperref[lint:useby]{Use by} &  37 \\
\bottomrule
& & Total: 252

\end{tabular}
    \caption{Lint summary of Isabelle/HOL}
    \label{tab:hol-summary}
    
\end{table}
\begin{table}
    \centering
\begin{tabular}{llr}
\toprule
Severity & Name & Number of occurences      \\
\midrule
High  & \hyperref[lint:unrestrictedauto]{Unrestricted auto} &  233 \\
      & \hyperref[lint:globalattr]{Global attribute on unnamed lemma} &   27 \\
      & \hyperref[lint:counterexample]{Counter-example finder} &    1 \\
Medium & \hyperref[lint:complexmethod]{Complex method} &   81 \\
       & \hyperref[lint:complexisar]{Complex Isar initial method} &   70 \\
       & \hyperref[lint:applyisarswitch]{Apply-Isar switch} &   68 \\
       & \hyperref[lint:implicitrule]{Implicit rule} &   33 \\
       & \hyperref[lint:lemmatrans]{Lemma-transforming attribute} &    2 \\
Low & \hyperref[lint:useby]{Use by} &   60 \\
\bottomrule
& & Total: 575
\end{tabular}
    \caption{Lint summary of the AFP selection}
    \label{tab:afp-summary}
\end{table}

\subsection{Performance}
The median time taken to lint a theory is 20.7 milliseconds with a mean
of 53.55 milliseconds. This means that the results should be viewed
skeptically: the theory sample used has a bias towards smaller 
theories. In fact, the median length of the theories is 452 lines with
a mean of 771.5 lines, but the longest theory, \textit{HOL.List}, has 8199
lines. Another
caveat to mention is that the number of lines of a theory is not the
only metric that could be applied to quantify the size of a theory.
However, it is a simpler and more common property than the number of commands.

\autoref{fig:timing} and \autoref{fig:timing_small}
show the time taken to process a theory, relative
to the number of lines it has, and the number of results reported.
We could observe two main aspects: First, larger 
theories (in terms of number of lines) tending to take longer to process,
and second, the processing time tending to increase with the number of 
lints, for theories of similar length. 

\begin{figure}
    \centering
    %% Creator: Matplotlib, PGF backend
%%
%% To include the figure in your LaTeX document, write
%%   \input{<filename>.pgf}
%%
%% Make sure the required packages are loaded in your preamble
%%   \usepackage{pgf}
%%
%% Figures using additional raster images can only be included by \input if
%% they are in the same directory as the main LaTeX file. For loading figures
%% from other directories you can use the `import` package
%%   \usepackage{import}
%%
%% and then include the figures with
%%   \import{<path to file>}{<filename>.pgf}
%%
%% Matplotlib used the following preamble
%%
\begingroup%
\makeatletter%
\begin{pgfpicture}%
\pgfpathrectangle{\pgfpointorigin}{\pgfqpoint{6.000000in}{4.000000in}}%
\pgfusepath{use as bounding box, clip}%
\begin{pgfscope}%
\pgfsetbuttcap%
\pgfsetmiterjoin%
\pgfsetlinewidth{0.000000pt}%
\definecolor{currentstroke}{rgb}{1.000000,1.000000,1.000000}%
\pgfsetstrokecolor{currentstroke}%
\pgfsetstrokeopacity{0.000000}%
\pgfsetdash{}{0pt}%
\pgfpathmoveto{\pgfqpoint{0.000000in}{0.000000in}}%
\pgfpathlineto{\pgfqpoint{6.000000in}{0.000000in}}%
\pgfpathlineto{\pgfqpoint{6.000000in}{4.000000in}}%
\pgfpathlineto{\pgfqpoint{0.000000in}{4.000000in}}%
\pgfpathclose%
\pgfusepath{}%
\end{pgfscope}%
\begin{pgfscope}%
\pgfsetbuttcap%
\pgfsetmiterjoin%
\definecolor{currentfill}{rgb}{1.000000,1.000000,1.000000}%
\pgfsetfillcolor{currentfill}%
\pgfsetlinewidth{0.000000pt}%
\definecolor{currentstroke}{rgb}{0.000000,0.000000,0.000000}%
\pgfsetstrokecolor{currentstroke}%
\pgfsetstrokeopacity{0.000000}%
\pgfsetdash{}{0pt}%
\pgfpathmoveto{\pgfqpoint{0.750000in}{0.500000in}}%
\pgfpathlineto{\pgfqpoint{5.400000in}{0.500000in}}%
\pgfpathlineto{\pgfqpoint{5.400000in}{3.520000in}}%
\pgfpathlineto{\pgfqpoint{0.750000in}{3.520000in}}%
\pgfpathclose%
\pgfusepath{fill}%
\end{pgfscope}%
\begin{pgfscope}%
\pgfpathrectangle{\pgfqpoint{0.750000in}{0.500000in}}{\pgfqpoint{4.650000in}{3.020000in}}%
\pgfusepath{clip}%
\pgfsetbuttcap%
\pgfsetroundjoin%
\definecolor{currentfill}{rgb}{0.121569,0.466667,0.705882}%
\pgfsetfillcolor{currentfill}%
\pgfsetlinewidth{1.003750pt}%
\definecolor{currentstroke}{rgb}{0.121569,0.466667,0.705882}%
\pgfsetstrokecolor{currentstroke}%
\pgfsetdash{}{0pt}%
\pgfpathmoveto{\pgfqpoint{1.181136in}{1.831930in}}%
\pgfpathcurveto{\pgfqpoint{1.185648in}{1.831930in}}{\pgfqpoint{1.189975in}{1.833722in}}{\pgfqpoint{1.193165in}{1.836912in}}%
\pgfpathcurveto{\pgfqpoint{1.196354in}{1.840102in}}{\pgfqpoint{1.198147in}{1.844429in}}{\pgfqpoint{1.198147in}{1.848940in}}%
\pgfpathcurveto{\pgfqpoint{1.198147in}{1.853452in}}{\pgfqpoint{1.196354in}{1.857779in}}{\pgfqpoint{1.193165in}{1.860969in}}%
\pgfpathcurveto{\pgfqpoint{1.189975in}{1.864158in}}{\pgfqpoint{1.185648in}{1.865951in}}{\pgfqpoint{1.181136in}{1.865951in}}%
\pgfpathcurveto{\pgfqpoint{1.176625in}{1.865951in}}{\pgfqpoint{1.172298in}{1.864158in}}{\pgfqpoint{1.169108in}{1.860969in}}%
\pgfpathcurveto{\pgfqpoint{1.165918in}{1.857779in}}{\pgfqpoint{1.164126in}{1.853452in}}{\pgfqpoint{1.164126in}{1.848940in}}%
\pgfpathcurveto{\pgfqpoint{1.164126in}{1.844429in}}{\pgfqpoint{1.165918in}{1.840102in}}{\pgfqpoint{1.169108in}{1.836912in}}%
\pgfpathcurveto{\pgfqpoint{1.172298in}{1.833722in}}{\pgfqpoint{1.176625in}{1.831930in}}{\pgfqpoint{1.181136in}{1.831930in}}%
\pgfpathclose%
\pgfusepath{stroke,fill}%
\end{pgfscope}%
\begin{pgfscope}%
\pgfpathrectangle{\pgfqpoint{0.750000in}{0.500000in}}{\pgfqpoint{4.650000in}{3.020000in}}%
\pgfusepath{clip}%
\pgfsetbuttcap%
\pgfsetroundjoin%
\definecolor{currentfill}{rgb}{0.121569,0.466667,0.705882}%
\pgfsetfillcolor{currentfill}%
\pgfsetlinewidth{1.003750pt}%
\definecolor{currentstroke}{rgb}{0.121569,0.466667,0.705882}%
\pgfsetstrokecolor{currentstroke}%
\pgfsetdash{}{0pt}%
\pgfpathmoveto{\pgfqpoint{0.995413in}{0.635806in}}%
\pgfpathcurveto{\pgfqpoint{0.998603in}{0.635806in}}{\pgfqpoint{1.001663in}{0.637073in}}{\pgfqpoint{1.003918in}{0.639329in}}%
\pgfpathcurveto{\pgfqpoint{1.006174in}{0.641584in}}{\pgfqpoint{1.007441in}{0.644644in}}{\pgfqpoint{1.007441in}{0.647834in}}%
\pgfpathcurveto{\pgfqpoint{1.007441in}{0.651024in}}{\pgfqpoint{1.006174in}{0.654084in}}{\pgfqpoint{1.003918in}{0.656339in}}%
\pgfpathcurveto{\pgfqpoint{1.001663in}{0.658595in}}{\pgfqpoint{0.998603in}{0.659862in}}{\pgfqpoint{0.995413in}{0.659862in}}%
\pgfpathcurveto{\pgfqpoint{0.992223in}{0.659862in}}{\pgfqpoint{0.989163in}{0.658595in}}{\pgfqpoint{0.986908in}{0.656339in}}%
\pgfpathcurveto{\pgfqpoint{0.984652in}{0.654084in}}{\pgfqpoint{0.983385in}{0.651024in}}{\pgfqpoint{0.983385in}{0.647834in}}%
\pgfpathcurveto{\pgfqpoint{0.983385in}{0.644644in}}{\pgfqpoint{0.984652in}{0.641584in}}{\pgfqpoint{0.986908in}{0.639329in}}%
\pgfpathcurveto{\pgfqpoint{0.989163in}{0.637073in}}{\pgfqpoint{0.992223in}{0.635806in}}{\pgfqpoint{0.995413in}{0.635806in}}%
\pgfpathclose%
\pgfusepath{stroke,fill}%
\end{pgfscope}%
\begin{pgfscope}%
\pgfpathrectangle{\pgfqpoint{0.750000in}{0.500000in}}{\pgfqpoint{4.650000in}{3.020000in}}%
\pgfusepath{clip}%
\pgfsetbuttcap%
\pgfsetroundjoin%
\definecolor{currentfill}{rgb}{0.121569,0.466667,0.705882}%
\pgfsetfillcolor{currentfill}%
\pgfsetlinewidth{1.003750pt}%
\definecolor{currentstroke}{rgb}{0.121569,0.466667,0.705882}%
\pgfsetstrokecolor{currentstroke}%
\pgfsetdash{}{0pt}%
\pgfpathmoveto{\pgfqpoint{1.945182in}{1.907175in}}%
\pgfpathcurveto{\pgfqpoint{1.953418in}{1.907175in}}{\pgfqpoint{1.961319in}{1.910447in}}{\pgfqpoint{1.967142in}{1.916271in}}%
\pgfpathcurveto{\pgfqpoint{1.972966in}{1.922095in}}{\pgfqpoint{1.976239in}{1.929995in}}{\pgfqpoint{1.976239in}{1.938231in}}%
\pgfpathcurveto{\pgfqpoint{1.976239in}{1.946468in}}{\pgfqpoint{1.972966in}{1.954368in}}{\pgfqpoint{1.967142in}{1.960192in}}%
\pgfpathcurveto{\pgfqpoint{1.961319in}{1.966015in}}{\pgfqpoint{1.953418in}{1.969288in}}{\pgfqpoint{1.945182in}{1.969288in}}%
\pgfpathcurveto{\pgfqpoint{1.936946in}{1.969288in}}{\pgfqpoint{1.929046in}{1.966015in}}{\pgfqpoint{1.923222in}{1.960192in}}%
\pgfpathcurveto{\pgfqpoint{1.917398in}{1.954368in}}{\pgfqpoint{1.914126in}{1.946468in}}{\pgfqpoint{1.914126in}{1.938231in}}%
\pgfpathcurveto{\pgfqpoint{1.914126in}{1.929995in}}{\pgfqpoint{1.917398in}{1.922095in}}{\pgfqpoint{1.923222in}{1.916271in}}%
\pgfpathcurveto{\pgfqpoint{1.929046in}{1.910447in}}{\pgfqpoint{1.936946in}{1.907175in}}{\pgfqpoint{1.945182in}{1.907175in}}%
\pgfpathclose%
\pgfusepath{stroke,fill}%
\end{pgfscope}%
\begin{pgfscope}%
\pgfpathrectangle{\pgfqpoint{0.750000in}{0.500000in}}{\pgfqpoint{4.650000in}{3.020000in}}%
\pgfusepath{clip}%
\pgfsetbuttcap%
\pgfsetroundjoin%
\definecolor{currentfill}{rgb}{0.121569,0.466667,0.705882}%
\pgfsetfillcolor{currentfill}%
\pgfsetlinewidth{1.003750pt}%
\definecolor{currentstroke}{rgb}{0.121569,0.466667,0.705882}%
\pgfsetstrokecolor{currentstroke}%
\pgfsetdash{}{0pt}%
\pgfpathmoveto{\pgfqpoint{1.667113in}{1.099036in}}%
\pgfpathcurveto{\pgfqpoint{1.673493in}{1.099036in}}{\pgfqpoint{1.679612in}{1.101570in}}{\pgfqpoint{1.684123in}{1.106082in}}%
\pgfpathcurveto{\pgfqpoint{1.688634in}{1.110593in}}{\pgfqpoint{1.691169in}{1.116712in}}{\pgfqpoint{1.691169in}{1.123092in}}%
\pgfpathcurveto{\pgfqpoint{1.691169in}{1.129472in}}{\pgfqpoint{1.688634in}{1.135591in}}{\pgfqpoint{1.684123in}{1.140102in}}%
\pgfpathcurveto{\pgfqpoint{1.679612in}{1.144613in}}{\pgfqpoint{1.673493in}{1.147148in}}{\pgfqpoint{1.667113in}{1.147148in}}%
\pgfpathcurveto{\pgfqpoint{1.660733in}{1.147148in}}{\pgfqpoint{1.654614in}{1.144613in}}{\pgfqpoint{1.650103in}{1.140102in}}%
\pgfpathcurveto{\pgfqpoint{1.645591in}{1.135591in}}{\pgfqpoint{1.643057in}{1.129472in}}{\pgfqpoint{1.643057in}{1.123092in}}%
\pgfpathcurveto{\pgfqpoint{1.643057in}{1.116712in}}{\pgfqpoint{1.645591in}{1.110593in}}{\pgfqpoint{1.650103in}{1.106082in}}%
\pgfpathcurveto{\pgfqpoint{1.654614in}{1.101570in}}{\pgfqpoint{1.660733in}{1.099036in}}{\pgfqpoint{1.667113in}{1.099036in}}%
\pgfpathclose%
\pgfusepath{stroke,fill}%
\end{pgfscope}%
\begin{pgfscope}%
\pgfpathrectangle{\pgfqpoint{0.750000in}{0.500000in}}{\pgfqpoint{4.650000in}{3.020000in}}%
\pgfusepath{clip}%
\pgfsetbuttcap%
\pgfsetroundjoin%
\definecolor{currentfill}{rgb}{0.121569,0.466667,0.705882}%
\pgfsetfillcolor{currentfill}%
\pgfsetlinewidth{1.003750pt}%
\definecolor{currentstroke}{rgb}{0.121569,0.466667,0.705882}%
\pgfsetstrokecolor{currentstroke}%
\pgfsetdash{}{0pt}%
\pgfpathmoveto{\pgfqpoint{2.928485in}{1.885645in}}%
\pgfpathcurveto{\pgfqpoint{2.938739in}{1.885645in}}{\pgfqpoint{2.948574in}{1.889719in}}{\pgfqpoint{2.955825in}{1.896969in}}%
\pgfpathcurveto{\pgfqpoint{2.963076in}{1.904220in}}{\pgfqpoint{2.967150in}{1.914055in}}{\pgfqpoint{2.967150in}{1.924310in}}%
\pgfpathcurveto{\pgfqpoint{2.967150in}{1.934564in}}{\pgfqpoint{2.963076in}{1.944399in}}{\pgfqpoint{2.955825in}{1.951650in}}%
\pgfpathcurveto{\pgfqpoint{2.948574in}{1.958901in}}{\pgfqpoint{2.938739in}{1.962975in}}{\pgfqpoint{2.928485in}{1.962975in}}%
\pgfpathcurveto{\pgfqpoint{2.918231in}{1.962975in}}{\pgfqpoint{2.908395in}{1.958901in}}{\pgfqpoint{2.901145in}{1.951650in}}%
\pgfpathcurveto{\pgfqpoint{2.893894in}{1.944399in}}{\pgfqpoint{2.889820in}{1.934564in}}{\pgfqpoint{2.889820in}{1.924310in}}%
\pgfpathcurveto{\pgfqpoint{2.889820in}{1.914055in}}{\pgfqpoint{2.893894in}{1.904220in}}{\pgfqpoint{2.901145in}{1.896969in}}%
\pgfpathcurveto{\pgfqpoint{2.908395in}{1.889719in}}{\pgfqpoint{2.918231in}{1.885645in}}{\pgfqpoint{2.928485in}{1.885645in}}%
\pgfpathclose%
\pgfusepath{stroke,fill}%
\end{pgfscope}%
\begin{pgfscope}%
\pgfpathrectangle{\pgfqpoint{0.750000in}{0.500000in}}{\pgfqpoint{4.650000in}{3.020000in}}%
\pgfusepath{clip}%
\pgfsetbuttcap%
\pgfsetroundjoin%
\definecolor{currentfill}{rgb}{0.121569,0.466667,0.705882}%
\pgfsetfillcolor{currentfill}%
\pgfsetlinewidth{1.003750pt}%
\definecolor{currentstroke}{rgb}{0.121569,0.466667,0.705882}%
\pgfsetstrokecolor{currentstroke}%
\pgfsetdash{}{0pt}%
\pgfpathmoveto{\pgfqpoint{1.028946in}{0.637053in}}%
\pgfpathcurveto{\pgfqpoint{1.031551in}{0.637053in}}{\pgfqpoint{1.034049in}{0.638088in}}{\pgfqpoint{1.035891in}{0.639929in}}%
\pgfpathcurveto{\pgfqpoint{1.037732in}{0.641771in}}{\pgfqpoint{1.038767in}{0.644269in}}{\pgfqpoint{1.038767in}{0.646874in}}%
\pgfpathcurveto{\pgfqpoint{1.038767in}{0.649478in}}{\pgfqpoint{1.037732in}{0.651977in}}{\pgfqpoint{1.035891in}{0.653818in}}%
\pgfpathcurveto{\pgfqpoint{1.034049in}{0.655660in}}{\pgfqpoint{1.031551in}{0.656695in}}{\pgfqpoint{1.028946in}{0.656695in}}%
\pgfpathcurveto{\pgfqpoint{1.026342in}{0.656695in}}{\pgfqpoint{1.023844in}{0.655660in}}{\pgfqpoint{1.022002in}{0.653818in}}%
\pgfpathcurveto{\pgfqpoint{1.020160in}{0.651977in}}{\pgfqpoint{1.019125in}{0.649478in}}{\pgfqpoint{1.019125in}{0.646874in}}%
\pgfpathcurveto{\pgfqpoint{1.019125in}{0.644269in}}{\pgfqpoint{1.020160in}{0.641771in}}{\pgfqpoint{1.022002in}{0.639929in}}%
\pgfpathcurveto{\pgfqpoint{1.023844in}{0.638088in}}{\pgfqpoint{1.026342in}{0.637053in}}{\pgfqpoint{1.028946in}{0.637053in}}%
\pgfpathclose%
\pgfusepath{stroke,fill}%
\end{pgfscope}%
\begin{pgfscope}%
\pgfpathrectangle{\pgfqpoint{0.750000in}{0.500000in}}{\pgfqpoint{4.650000in}{3.020000in}}%
\pgfusepath{clip}%
\pgfsetbuttcap%
\pgfsetroundjoin%
\definecolor{currentfill}{rgb}{0.121569,0.466667,0.705882}%
\pgfsetfillcolor{currentfill}%
\pgfsetlinewidth{1.003750pt}%
\definecolor{currentstroke}{rgb}{0.121569,0.466667,0.705882}%
\pgfsetstrokecolor{currentstroke}%
\pgfsetdash{}{0pt}%
\pgfpathmoveto{\pgfqpoint{1.415870in}{0.856931in}}%
\pgfpathcurveto{\pgfqpoint{1.420381in}{0.856931in}}{\pgfqpoint{1.424708in}{0.858724in}}{\pgfqpoint{1.427898in}{0.861913in}}%
\pgfpathcurveto{\pgfqpoint{1.431088in}{0.865103in}}{\pgfqpoint{1.432881in}{0.869430in}}{\pgfqpoint{1.432881in}{0.873942in}}%
\pgfpathcurveto{\pgfqpoint{1.432881in}{0.878453in}}{\pgfqpoint{1.431088in}{0.882780in}}{\pgfqpoint{1.427898in}{0.885970in}}%
\pgfpathcurveto{\pgfqpoint{1.424708in}{0.889160in}}{\pgfqpoint{1.420381in}{0.890952in}}{\pgfqpoint{1.415870in}{0.890952in}}%
\pgfpathcurveto{\pgfqpoint{1.411359in}{0.890952in}}{\pgfqpoint{1.407032in}{0.889160in}}{\pgfqpoint{1.403842in}{0.885970in}}%
\pgfpathcurveto{\pgfqpoint{1.400652in}{0.882780in}}{\pgfqpoint{1.398860in}{0.878453in}}{\pgfqpoint{1.398860in}{0.873942in}}%
\pgfpathcurveto{\pgfqpoint{1.398860in}{0.869430in}}{\pgfqpoint{1.400652in}{0.865103in}}{\pgfqpoint{1.403842in}{0.861913in}}%
\pgfpathcurveto{\pgfqpoint{1.407032in}{0.858724in}}{\pgfqpoint{1.411359in}{0.856931in}}{\pgfqpoint{1.415870in}{0.856931in}}%
\pgfpathclose%
\pgfusepath{stroke,fill}%
\end{pgfscope}%
\begin{pgfscope}%
\pgfpathrectangle{\pgfqpoint{0.750000in}{0.500000in}}{\pgfqpoint{4.650000in}{3.020000in}}%
\pgfusepath{clip}%
\pgfsetbuttcap%
\pgfsetroundjoin%
\definecolor{currentfill}{rgb}{0.121569,0.466667,0.705882}%
\pgfsetfillcolor{currentfill}%
\pgfsetlinewidth{1.003750pt}%
\definecolor{currentstroke}{rgb}{0.121569,0.466667,0.705882}%
\pgfsetstrokecolor{currentstroke}%
\pgfsetdash{}{0pt}%
\pgfpathmoveto{\pgfqpoint{2.060743in}{1.003426in}}%
\pgfpathcurveto{\pgfqpoint{2.065616in}{1.003426in}}{\pgfqpoint{2.070290in}{1.005362in}}{\pgfqpoint{2.073735in}{1.008808in}}%
\pgfpathcurveto{\pgfqpoint{2.077181in}{1.012253in}}{\pgfqpoint{2.079117in}{1.016927in}}{\pgfqpoint{2.079117in}{1.021800in}}%
\pgfpathcurveto{\pgfqpoint{2.079117in}{1.026672in}}{\pgfqpoint{2.077181in}{1.031346in}}{\pgfqpoint{2.073735in}{1.034791in}}%
\pgfpathcurveto{\pgfqpoint{2.070290in}{1.038237in}}{\pgfqpoint{2.065616in}{1.040173in}}{\pgfqpoint{2.060743in}{1.040173in}}%
\pgfpathcurveto{\pgfqpoint{2.055871in}{1.040173in}}{\pgfqpoint{2.051197in}{1.038237in}}{\pgfqpoint{2.047752in}{1.034791in}}%
\pgfpathcurveto{\pgfqpoint{2.044306in}{1.031346in}}{\pgfqpoint{2.042370in}{1.026672in}}{\pgfqpoint{2.042370in}{1.021800in}}%
\pgfpathcurveto{\pgfqpoint{2.042370in}{1.016927in}}{\pgfqpoint{2.044306in}{1.012253in}}{\pgfqpoint{2.047752in}{1.008808in}}%
\pgfpathcurveto{\pgfqpoint{2.051197in}{1.005362in}}{\pgfqpoint{2.055871in}{1.003426in}}{\pgfqpoint{2.060743in}{1.003426in}}%
\pgfpathclose%
\pgfusepath{stroke,fill}%
\end{pgfscope}%
\begin{pgfscope}%
\pgfpathrectangle{\pgfqpoint{0.750000in}{0.500000in}}{\pgfqpoint{4.650000in}{3.020000in}}%
\pgfusepath{clip}%
\pgfsetbuttcap%
\pgfsetroundjoin%
\definecolor{currentfill}{rgb}{0.121569,0.466667,0.705882}%
\pgfsetfillcolor{currentfill}%
\pgfsetlinewidth{1.003750pt}%
\definecolor{currentstroke}{rgb}{0.121569,0.466667,0.705882}%
\pgfsetstrokecolor{currentstroke}%
\pgfsetdash{}{0pt}%
\pgfpathmoveto{\pgfqpoint{1.110974in}{0.676131in}}%
\pgfpathcurveto{\pgfqpoint{1.114164in}{0.676131in}}{\pgfqpoint{1.117224in}{0.677398in}}{\pgfqpoint{1.119479in}{0.679654in}}%
\pgfpathcurveto{\pgfqpoint{1.121735in}{0.681909in}}{\pgfqpoint{1.123002in}{0.684969in}}{\pgfqpoint{1.123002in}{0.688159in}}%
\pgfpathcurveto{\pgfqpoint{1.123002in}{0.691349in}}{\pgfqpoint{1.121735in}{0.694409in}}{\pgfqpoint{1.119479in}{0.696664in}}%
\pgfpathcurveto{\pgfqpoint{1.117224in}{0.698920in}}{\pgfqpoint{1.114164in}{0.700187in}}{\pgfqpoint{1.110974in}{0.700187in}}%
\pgfpathcurveto{\pgfqpoint{1.107784in}{0.700187in}}{\pgfqpoint{1.104725in}{0.698920in}}{\pgfqpoint{1.102469in}{0.696664in}}%
\pgfpathcurveto{\pgfqpoint{1.100213in}{0.694409in}}{\pgfqpoint{1.098946in}{0.691349in}}{\pgfqpoint{1.098946in}{0.688159in}}%
\pgfpathcurveto{\pgfqpoint{1.098946in}{0.684969in}}{\pgfqpoint{1.100213in}{0.681909in}}{\pgfqpoint{1.102469in}{0.679654in}}%
\pgfpathcurveto{\pgfqpoint{1.104725in}{0.677398in}}{\pgfqpoint{1.107784in}{0.676131in}}{\pgfqpoint{1.110974in}{0.676131in}}%
\pgfpathclose%
\pgfusepath{stroke,fill}%
\end{pgfscope}%
\begin{pgfscope}%
\pgfpathrectangle{\pgfqpoint{0.750000in}{0.500000in}}{\pgfqpoint{4.650000in}{3.020000in}}%
\pgfusepath{clip}%
\pgfsetbuttcap%
\pgfsetroundjoin%
\definecolor{currentfill}{rgb}{0.121569,0.466667,0.705882}%
\pgfsetfillcolor{currentfill}%
\pgfsetlinewidth{1.003750pt}%
\definecolor{currentstroke}{rgb}{0.121569,0.466667,0.705882}%
\pgfsetstrokecolor{currentstroke}%
\pgfsetdash{}{0pt}%
\pgfpathmoveto{\pgfqpoint{1.191454in}{1.968747in}}%
\pgfpathcurveto{\pgfqpoint{1.195966in}{1.968747in}}{\pgfqpoint{1.200293in}{1.970539in}}{\pgfqpoint{1.203483in}{1.973729in}}%
\pgfpathcurveto{\pgfqpoint{1.206672in}{1.976919in}}{\pgfqpoint{1.208465in}{1.981246in}}{\pgfqpoint{1.208465in}{1.985757in}}%
\pgfpathcurveto{\pgfqpoint{1.208465in}{1.990268in}}{\pgfqpoint{1.206672in}{1.994595in}}{\pgfqpoint{1.203483in}{1.997785in}}%
\pgfpathcurveto{\pgfqpoint{1.200293in}{2.000975in}}{\pgfqpoint{1.195966in}{2.002767in}}{\pgfqpoint{1.191454in}{2.002767in}}%
\pgfpathcurveto{\pgfqpoint{1.186943in}{2.002767in}}{\pgfqpoint{1.182616in}{2.000975in}}{\pgfqpoint{1.179426in}{1.997785in}}%
\pgfpathcurveto{\pgfqpoint{1.176236in}{1.994595in}}{\pgfqpoint{1.174444in}{1.990268in}}{\pgfqpoint{1.174444in}{1.985757in}}%
\pgfpathcurveto{\pgfqpoint{1.174444in}{1.981246in}}{\pgfqpoint{1.176236in}{1.976919in}}{\pgfqpoint{1.179426in}{1.973729in}}%
\pgfpathcurveto{\pgfqpoint{1.182616in}{1.970539in}}{\pgfqpoint{1.186943in}{1.968747in}}{\pgfqpoint{1.191454in}{1.968747in}}%
\pgfpathclose%
\pgfusepath{stroke,fill}%
\end{pgfscope}%
\begin{pgfscope}%
\pgfpathrectangle{\pgfqpoint{0.750000in}{0.500000in}}{\pgfqpoint{4.650000in}{3.020000in}}%
\pgfusepath{clip}%
\pgfsetbuttcap%
\pgfsetroundjoin%
\definecolor{currentfill}{rgb}{0.121569,0.466667,0.705882}%
\pgfsetfillcolor{currentfill}%
\pgfsetlinewidth{1.003750pt}%
\definecolor{currentstroke}{rgb}{0.121569,0.466667,0.705882}%
\pgfsetstrokecolor{currentstroke}%
\pgfsetdash{}{0pt}%
\pgfpathmoveto{\pgfqpoint{1.220345in}{1.045671in}}%
\pgfpathcurveto{\pgfqpoint{1.225217in}{1.045671in}}{\pgfqpoint{1.229891in}{1.047607in}}{\pgfqpoint{1.233337in}{1.051053in}}%
\pgfpathcurveto{\pgfqpoint{1.236782in}{1.054498in}}{\pgfqpoint{1.238718in}{1.059172in}}{\pgfqpoint{1.238718in}{1.064045in}}%
\pgfpathcurveto{\pgfqpoint{1.238718in}{1.068917in}}{\pgfqpoint{1.236782in}{1.073591in}}{\pgfqpoint{1.233337in}{1.077037in}}%
\pgfpathcurveto{\pgfqpoint{1.229891in}{1.080482in}}{\pgfqpoint{1.225217in}{1.082418in}}{\pgfqpoint{1.220345in}{1.082418in}}%
\pgfpathcurveto{\pgfqpoint{1.215472in}{1.082418in}}{\pgfqpoint{1.210798in}{1.080482in}}{\pgfqpoint{1.207353in}{1.077037in}}%
\pgfpathcurveto{\pgfqpoint{1.203907in}{1.073591in}}{\pgfqpoint{1.201971in}{1.068917in}}{\pgfqpoint{1.201971in}{1.064045in}}%
\pgfpathcurveto{\pgfqpoint{1.201971in}{1.059172in}}{\pgfqpoint{1.203907in}{1.054498in}}{\pgfqpoint{1.207353in}{1.051053in}}%
\pgfpathcurveto{\pgfqpoint{1.210798in}{1.047607in}}{\pgfqpoint{1.215472in}{1.045671in}}{\pgfqpoint{1.220345in}{1.045671in}}%
\pgfpathclose%
\pgfusepath{stroke,fill}%
\end{pgfscope}%
\begin{pgfscope}%
\pgfpathrectangle{\pgfqpoint{0.750000in}{0.500000in}}{\pgfqpoint{4.650000in}{3.020000in}}%
\pgfusepath{clip}%
\pgfsetbuttcap%
\pgfsetroundjoin%
\definecolor{currentfill}{rgb}{0.121569,0.466667,0.705882}%
\pgfsetfillcolor{currentfill}%
\pgfsetlinewidth{1.003750pt}%
\definecolor{currentstroke}{rgb}{0.121569,0.466667,0.705882}%
\pgfsetstrokecolor{currentstroke}%
\pgfsetdash{}{0pt}%
\pgfpathmoveto{\pgfqpoint{1.433411in}{1.088162in}}%
\pgfpathcurveto{\pgfqpoint{1.442802in}{1.088162in}}{\pgfqpoint{1.451809in}{1.091893in}}{\pgfqpoint{1.458449in}{1.098533in}}%
\pgfpathcurveto{\pgfqpoint{1.465090in}{1.105174in}}{\pgfqpoint{1.468821in}{1.114181in}}{\pgfqpoint{1.468821in}{1.123572in}}%
\pgfpathcurveto{\pgfqpoint{1.468821in}{1.132963in}}{\pgfqpoint{1.465090in}{1.141970in}}{\pgfqpoint{1.458449in}{1.148611in}}%
\pgfpathcurveto{\pgfqpoint{1.451809in}{1.155251in}}{\pgfqpoint{1.442802in}{1.158982in}}{\pgfqpoint{1.433411in}{1.158982in}}%
\pgfpathcurveto{\pgfqpoint{1.424020in}{1.158982in}}{\pgfqpoint{1.415013in}{1.155251in}}{\pgfqpoint{1.408372in}{1.148611in}}%
\pgfpathcurveto{\pgfqpoint{1.401732in}{1.141970in}}{\pgfqpoint{1.398001in}{1.132963in}}{\pgfqpoint{1.398001in}{1.123572in}}%
\pgfpathcurveto{\pgfqpoint{1.398001in}{1.114181in}}{\pgfqpoint{1.401732in}{1.105174in}}{\pgfqpoint{1.408372in}{1.098533in}}%
\pgfpathcurveto{\pgfqpoint{1.415013in}{1.091893in}}{\pgfqpoint{1.424020in}{1.088162in}}{\pgfqpoint{1.433411in}{1.088162in}}%
\pgfpathclose%
\pgfusepath{stroke,fill}%
\end{pgfscope}%
\begin{pgfscope}%
\pgfpathrectangle{\pgfqpoint{0.750000in}{0.500000in}}{\pgfqpoint{4.650000in}{3.020000in}}%
\pgfusepath{clip}%
\pgfsetbuttcap%
\pgfsetroundjoin%
\definecolor{currentfill}{rgb}{0.121569,0.466667,0.705882}%
\pgfsetfillcolor{currentfill}%
\pgfsetlinewidth{1.003750pt}%
\definecolor{currentstroke}{rgb}{0.121569,0.466667,0.705882}%
\pgfsetstrokecolor{currentstroke}%
\pgfsetdash{}{0pt}%
\pgfpathmoveto{\pgfqpoint{1.289475in}{1.167803in}}%
\pgfpathcurveto{\pgfqpoint{1.296842in}{1.167803in}}{\pgfqpoint{1.303908in}{1.170730in}}{\pgfqpoint{1.309117in}{1.175939in}}%
\pgfpathcurveto{\pgfqpoint{1.314326in}{1.181148in}}{\pgfqpoint{1.317253in}{1.188214in}}{\pgfqpoint{1.317253in}{1.195581in}}%
\pgfpathcurveto{\pgfqpoint{1.317253in}{1.202948in}}{\pgfqpoint{1.314326in}{1.210014in}}{\pgfqpoint{1.309117in}{1.215223in}}%
\pgfpathcurveto{\pgfqpoint{1.303908in}{1.220432in}}{\pgfqpoint{1.296842in}{1.223359in}}{\pgfqpoint{1.289475in}{1.223359in}}%
\pgfpathcurveto{\pgfqpoint{1.282108in}{1.223359in}}{\pgfqpoint{1.275042in}{1.220432in}}{\pgfqpoint{1.269833in}{1.215223in}}%
\pgfpathcurveto{\pgfqpoint{1.264624in}{1.210014in}}{\pgfqpoint{1.261697in}{1.202948in}}{\pgfqpoint{1.261697in}{1.195581in}}%
\pgfpathcurveto{\pgfqpoint{1.261697in}{1.188214in}}{\pgfqpoint{1.264624in}{1.181148in}}{\pgfqpoint{1.269833in}{1.175939in}}%
\pgfpathcurveto{\pgfqpoint{1.275042in}{1.170730in}}{\pgfqpoint{1.282108in}{1.167803in}}{\pgfqpoint{1.289475in}{1.167803in}}%
\pgfpathclose%
\pgfusepath{stroke,fill}%
\end{pgfscope}%
\begin{pgfscope}%
\pgfpathrectangle{\pgfqpoint{0.750000in}{0.500000in}}{\pgfqpoint{4.650000in}{3.020000in}}%
\pgfusepath{clip}%
\pgfsetbuttcap%
\pgfsetroundjoin%
\definecolor{currentfill}{rgb}{0.121569,0.466667,0.705882}%
\pgfsetfillcolor{currentfill}%
\pgfsetlinewidth{1.003750pt}%
\definecolor{currentstroke}{rgb}{0.121569,0.466667,0.705882}%
\pgfsetstrokecolor{currentstroke}%
\pgfsetdash{}{0pt}%
\pgfpathmoveto{\pgfqpoint{1.441665in}{1.043703in}}%
\pgfpathcurveto{\pgfqpoint{1.449479in}{1.043703in}}{\pgfqpoint{1.456973in}{1.046807in}}{\pgfqpoint{1.462499in}{1.052333in}}%
\pgfpathcurveto{\pgfqpoint{1.468024in}{1.057858in}}{\pgfqpoint{1.471128in}{1.065352in}}{\pgfqpoint{1.471128in}{1.073166in}}%
\pgfpathcurveto{\pgfqpoint{1.471128in}{1.080979in}}{\pgfqpoint{1.468024in}{1.088474in}}{\pgfqpoint{1.462499in}{1.093999in}}%
\pgfpathcurveto{\pgfqpoint{1.456973in}{1.099524in}}{\pgfqpoint{1.449479in}{1.102629in}}{\pgfqpoint{1.441665in}{1.102629in}}%
\pgfpathcurveto{\pgfqpoint{1.433852in}{1.102629in}}{\pgfqpoint{1.426357in}{1.099524in}}{\pgfqpoint{1.420832in}{1.093999in}}%
\pgfpathcurveto{\pgfqpoint{1.415307in}{1.088474in}}{\pgfqpoint{1.412202in}{1.080979in}}{\pgfqpoint{1.412202in}{1.073166in}}%
\pgfpathcurveto{\pgfqpoint{1.412202in}{1.065352in}}{\pgfqpoint{1.415307in}{1.057858in}}{\pgfqpoint{1.420832in}{1.052333in}}%
\pgfpathcurveto{\pgfqpoint{1.426357in}{1.046807in}}{\pgfqpoint{1.433852in}{1.043703in}}{\pgfqpoint{1.441665in}{1.043703in}}%
\pgfpathclose%
\pgfusepath{stroke,fill}%
\end{pgfscope}%
\begin{pgfscope}%
\pgfpathrectangle{\pgfqpoint{0.750000in}{0.500000in}}{\pgfqpoint{4.650000in}{3.020000in}}%
\pgfusepath{clip}%
\pgfsetbuttcap%
\pgfsetroundjoin%
\definecolor{currentfill}{rgb}{0.121569,0.466667,0.705882}%
\pgfsetfillcolor{currentfill}%
\pgfsetlinewidth{1.003750pt}%
\definecolor{currentstroke}{rgb}{0.121569,0.466667,0.705882}%
\pgfsetstrokecolor{currentstroke}%
\pgfsetdash{}{0pt}%
\pgfpathmoveto{\pgfqpoint{1.130578in}{0.871529in}}%
\pgfpathcurveto{\pgfqpoint{1.137711in}{0.871529in}}{\pgfqpoint{1.144553in}{0.874363in}}{\pgfqpoint{1.149597in}{0.879406in}}%
\pgfpathcurveto{\pgfqpoint{1.154640in}{0.884450in}}{\pgfqpoint{1.157474in}{0.891292in}}{\pgfqpoint{1.157474in}{0.898425in}}%
\pgfpathcurveto{\pgfqpoint{1.157474in}{0.905557in}}{\pgfqpoint{1.154640in}{0.912399in}}{\pgfqpoint{1.149597in}{0.917443in}}%
\pgfpathcurveto{\pgfqpoint{1.144553in}{0.922486in}}{\pgfqpoint{1.137711in}{0.925320in}}{\pgfqpoint{1.130578in}{0.925320in}}%
\pgfpathcurveto{\pgfqpoint{1.123446in}{0.925320in}}{\pgfqpoint{1.116604in}{0.922486in}}{\pgfqpoint{1.111560in}{0.917443in}}%
\pgfpathcurveto{\pgfqpoint{1.106517in}{0.912399in}}{\pgfqpoint{1.103683in}{0.905557in}}{\pgfqpoint{1.103683in}{0.898425in}}%
\pgfpathcurveto{\pgfqpoint{1.103683in}{0.891292in}}{\pgfqpoint{1.106517in}{0.884450in}}{\pgfqpoint{1.111560in}{0.879406in}}%
\pgfpathcurveto{\pgfqpoint{1.116604in}{0.874363in}}{\pgfqpoint{1.123446in}{0.871529in}}{\pgfqpoint{1.130578in}{0.871529in}}%
\pgfpathclose%
\pgfusepath{stroke,fill}%
\end{pgfscope}%
\begin{pgfscope}%
\pgfpathrectangle{\pgfqpoint{0.750000in}{0.500000in}}{\pgfqpoint{4.650000in}{3.020000in}}%
\pgfusepath{clip}%
\pgfsetbuttcap%
\pgfsetroundjoin%
\definecolor{currentfill}{rgb}{0.121569,0.466667,0.705882}%
\pgfsetfillcolor{currentfill}%
\pgfsetlinewidth{1.003750pt}%
\definecolor{currentstroke}{rgb}{0.121569,0.466667,0.705882}%
\pgfsetstrokecolor{currentstroke}%
\pgfsetdash{}{0pt}%
\pgfpathmoveto{\pgfqpoint{1.347256in}{0.923807in}}%
\pgfpathcurveto{\pgfqpoint{1.354147in}{0.923807in}}{\pgfqpoint{1.360756in}{0.926545in}}{\pgfqpoint{1.365629in}{0.931418in}}%
\pgfpathcurveto{\pgfqpoint{1.370502in}{0.936290in}}{\pgfqpoint{1.373239in}{0.942900in}}{\pgfqpoint{1.373239in}{0.949791in}}%
\pgfpathcurveto{\pgfqpoint{1.373239in}{0.956682in}}{\pgfqpoint{1.370502in}{0.963291in}}{\pgfqpoint{1.365629in}{0.968164in}}%
\pgfpathcurveto{\pgfqpoint{1.360756in}{0.973037in}}{\pgfqpoint{1.354147in}{0.975775in}}{\pgfqpoint{1.347256in}{0.975775in}}%
\pgfpathcurveto{\pgfqpoint{1.340365in}{0.975775in}}{\pgfqpoint{1.333755in}{0.973037in}}{\pgfqpoint{1.328882in}{0.968164in}}%
\pgfpathcurveto{\pgfqpoint{1.324010in}{0.963291in}}{\pgfqpoint{1.321272in}{0.956682in}}{\pgfqpoint{1.321272in}{0.949791in}}%
\pgfpathcurveto{\pgfqpoint{1.321272in}{0.942900in}}{\pgfqpoint{1.324010in}{0.936290in}}{\pgfqpoint{1.328882in}{0.931418in}}%
\pgfpathcurveto{\pgfqpoint{1.333755in}{0.926545in}}{\pgfqpoint{1.340365in}{0.923807in}}{\pgfqpoint{1.347256in}{0.923807in}}%
\pgfpathclose%
\pgfusepath{stroke,fill}%
\end{pgfscope}%
\begin{pgfscope}%
\pgfpathrectangle{\pgfqpoint{0.750000in}{0.500000in}}{\pgfqpoint{4.650000in}{3.020000in}}%
\pgfusepath{clip}%
\pgfsetbuttcap%
\pgfsetroundjoin%
\definecolor{currentfill}{rgb}{0.121569,0.466667,0.705882}%
\pgfsetfillcolor{currentfill}%
\pgfsetlinewidth{1.003750pt}%
\definecolor{currentstroke}{rgb}{0.121569,0.466667,0.705882}%
\pgfsetstrokecolor{currentstroke}%
\pgfsetdash{}{0pt}%
\pgfpathmoveto{\pgfqpoint{1.051646in}{0.641373in}}%
\pgfpathcurveto{\pgfqpoint{1.054250in}{0.641373in}}{\pgfqpoint{1.056749in}{0.642408in}}{\pgfqpoint{1.058590in}{0.644250in}}%
\pgfpathcurveto{\pgfqpoint{1.060432in}{0.646092in}}{\pgfqpoint{1.061467in}{0.648590in}}{\pgfqpoint{1.061467in}{0.651194in}}%
\pgfpathcurveto{\pgfqpoint{1.061467in}{0.653799in}}{\pgfqpoint{1.060432in}{0.656297in}}{\pgfqpoint{1.058590in}{0.658139in}}%
\pgfpathcurveto{\pgfqpoint{1.056749in}{0.659981in}}{\pgfqpoint{1.054250in}{0.661015in}}{\pgfqpoint{1.051646in}{0.661015in}}%
\pgfpathcurveto{\pgfqpoint{1.049041in}{0.661015in}}{\pgfqpoint{1.046543in}{0.659981in}}{\pgfqpoint{1.044701in}{0.658139in}}%
\pgfpathcurveto{\pgfqpoint{1.042860in}{0.656297in}}{\pgfqpoint{1.041825in}{0.653799in}}{\pgfqpoint{1.041825in}{0.651194in}}%
\pgfpathcurveto{\pgfqpoint{1.041825in}{0.648590in}}{\pgfqpoint{1.042860in}{0.646092in}}{\pgfqpoint{1.044701in}{0.644250in}}%
\pgfpathcurveto{\pgfqpoint{1.046543in}{0.642408in}}{\pgfqpoint{1.049041in}{0.641373in}}{\pgfqpoint{1.051646in}{0.641373in}}%
\pgfpathclose%
\pgfusepath{stroke,fill}%
\end{pgfscope}%
\begin{pgfscope}%
\pgfpathrectangle{\pgfqpoint{0.750000in}{0.500000in}}{\pgfqpoint{4.650000in}{3.020000in}}%
\pgfusepath{clip}%
\pgfsetbuttcap%
\pgfsetroundjoin%
\definecolor{currentfill}{rgb}{0.121569,0.466667,0.705882}%
\pgfsetfillcolor{currentfill}%
\pgfsetlinewidth{1.003750pt}%
\definecolor{currentstroke}{rgb}{0.121569,0.466667,0.705882}%
\pgfsetstrokecolor{currentstroke}%
\pgfsetdash{}{0pt}%
\pgfpathmoveto{\pgfqpoint{1.053194in}{0.642814in}}%
\pgfpathcurveto{\pgfqpoint{1.055798in}{0.642814in}}{\pgfqpoint{1.058296in}{0.643848in}}{\pgfqpoint{1.060138in}{0.645690in}}%
\pgfpathcurveto{\pgfqpoint{1.061980in}{0.647532in}}{\pgfqpoint{1.063015in}{0.650030in}}{\pgfqpoint{1.063015in}{0.652635in}}%
\pgfpathcurveto{\pgfqpoint{1.063015in}{0.655239in}}{\pgfqpoint{1.061980in}{0.657737in}}{\pgfqpoint{1.060138in}{0.659579in}}%
\pgfpathcurveto{\pgfqpoint{1.058296in}{0.661421in}}{\pgfqpoint{1.055798in}{0.662456in}}{\pgfqpoint{1.053194in}{0.662456in}}%
\pgfpathcurveto{\pgfqpoint{1.050589in}{0.662456in}}{\pgfqpoint{1.048091in}{0.661421in}}{\pgfqpoint{1.046249in}{0.659579in}}%
\pgfpathcurveto{\pgfqpoint{1.044407in}{0.657737in}}{\pgfqpoint{1.043373in}{0.655239in}}{\pgfqpoint{1.043373in}{0.652635in}}%
\pgfpathcurveto{\pgfqpoint{1.043373in}{0.650030in}}{\pgfqpoint{1.044407in}{0.647532in}}{\pgfqpoint{1.046249in}{0.645690in}}%
\pgfpathcurveto{\pgfqpoint{1.048091in}{0.643848in}}{\pgfqpoint{1.050589in}{0.642814in}}{\pgfqpoint{1.053194in}{0.642814in}}%
\pgfpathclose%
\pgfusepath{stroke,fill}%
\end{pgfscope}%
\begin{pgfscope}%
\pgfpathrectangle{\pgfqpoint{0.750000in}{0.500000in}}{\pgfqpoint{4.650000in}{3.020000in}}%
\pgfusepath{clip}%
\pgfsetbuttcap%
\pgfsetroundjoin%
\definecolor{currentfill}{rgb}{0.121569,0.466667,0.705882}%
\pgfsetfillcolor{currentfill}%
\pgfsetlinewidth{1.003750pt}%
\definecolor{currentstroke}{rgb}{0.121569,0.466667,0.705882}%
\pgfsetstrokecolor{currentstroke}%
\pgfsetdash{}{0pt}%
\pgfpathmoveto{\pgfqpoint{1.206931in}{0.676418in}}%
\pgfpathcurveto{\pgfqpoint{1.209536in}{0.676418in}}{\pgfqpoint{1.212034in}{0.677453in}}{\pgfqpoint{1.213876in}{0.679294in}}%
\pgfpathcurveto{\pgfqpoint{1.215717in}{0.681136in}}{\pgfqpoint{1.216752in}{0.683634in}}{\pgfqpoint{1.216752in}{0.686239in}}%
\pgfpathcurveto{\pgfqpoint{1.216752in}{0.688843in}}{\pgfqpoint{1.215717in}{0.691341in}}{\pgfqpoint{1.213876in}{0.693183in}}%
\pgfpathcurveto{\pgfqpoint{1.212034in}{0.695025in}}{\pgfqpoint{1.209536in}{0.696060in}}{\pgfqpoint{1.206931in}{0.696060in}}%
\pgfpathcurveto{\pgfqpoint{1.204327in}{0.696060in}}{\pgfqpoint{1.201829in}{0.695025in}}{\pgfqpoint{1.199987in}{0.693183in}}%
\pgfpathcurveto{\pgfqpoint{1.198145in}{0.691341in}}{\pgfqpoint{1.197110in}{0.688843in}}{\pgfqpoint{1.197110in}{0.686239in}}%
\pgfpathcurveto{\pgfqpoint{1.197110in}{0.683634in}}{\pgfqpoint{1.198145in}{0.681136in}}{\pgfqpoint{1.199987in}{0.679294in}}%
\pgfpathcurveto{\pgfqpoint{1.201829in}{0.677453in}}{\pgfqpoint{1.204327in}{0.676418in}}{\pgfqpoint{1.206931in}{0.676418in}}%
\pgfpathclose%
\pgfusepath{stroke,fill}%
\end{pgfscope}%
\begin{pgfscope}%
\pgfpathrectangle{\pgfqpoint{0.750000in}{0.500000in}}{\pgfqpoint{4.650000in}{3.020000in}}%
\pgfusepath{clip}%
\pgfsetbuttcap%
\pgfsetroundjoin%
\definecolor{currentfill}{rgb}{0.121569,0.466667,0.705882}%
\pgfsetfillcolor{currentfill}%
\pgfsetlinewidth{1.003750pt}%
\definecolor{currentstroke}{rgb}{0.121569,0.466667,0.705882}%
\pgfsetstrokecolor{currentstroke}%
\pgfsetdash{}{0pt}%
\pgfpathmoveto{\pgfqpoint{1.191970in}{0.689092in}}%
\pgfpathcurveto{\pgfqpoint{1.195160in}{0.689092in}}{\pgfqpoint{1.198220in}{0.690360in}}{\pgfqpoint{1.200475in}{0.692615in}}%
\pgfpathcurveto{\pgfqpoint{1.202731in}{0.694871in}}{\pgfqpoint{1.203998in}{0.697931in}}{\pgfqpoint{1.203998in}{0.701121in}}%
\pgfpathcurveto{\pgfqpoint{1.203998in}{0.704310in}}{\pgfqpoint{1.202731in}{0.707370in}}{\pgfqpoint{1.200475in}{0.709626in}}%
\pgfpathcurveto{\pgfqpoint{1.198220in}{0.711881in}}{\pgfqpoint{1.195160in}{0.713149in}}{\pgfqpoint{1.191970in}{0.713149in}}%
\pgfpathcurveto{\pgfqpoint{1.188780in}{0.713149in}}{\pgfqpoint{1.185721in}{0.711881in}}{\pgfqpoint{1.183465in}{0.709626in}}%
\pgfpathcurveto{\pgfqpoint{1.181210in}{0.707370in}}{\pgfqpoint{1.179942in}{0.704310in}}{\pgfqpoint{1.179942in}{0.701121in}}%
\pgfpathcurveto{\pgfqpoint{1.179942in}{0.697931in}}{\pgfqpoint{1.181210in}{0.694871in}}{\pgfqpoint{1.183465in}{0.692615in}}%
\pgfpathcurveto{\pgfqpoint{1.185721in}{0.690360in}}{\pgfqpoint{1.188780in}{0.689092in}}{\pgfqpoint{1.191970in}{0.689092in}}%
\pgfpathclose%
\pgfusepath{stroke,fill}%
\end{pgfscope}%
\begin{pgfscope}%
\pgfpathrectangle{\pgfqpoint{0.750000in}{0.500000in}}{\pgfqpoint{4.650000in}{3.020000in}}%
\pgfusepath{clip}%
\pgfsetbuttcap%
\pgfsetroundjoin%
\definecolor{currentfill}{rgb}{0.121569,0.466667,0.705882}%
\pgfsetfillcolor{currentfill}%
\pgfsetlinewidth{1.003750pt}%
\definecolor{currentstroke}{rgb}{0.121569,0.466667,0.705882}%
\pgfsetstrokecolor{currentstroke}%
\pgfsetdash{}{0pt}%
\pgfpathmoveto{\pgfqpoint{1.010890in}{0.660348in}}%
\pgfpathcurveto{\pgfqpoint{1.014573in}{0.660348in}}{\pgfqpoint{1.018106in}{0.661812in}}{\pgfqpoint{1.020711in}{0.664416in}}%
\pgfpathcurveto{\pgfqpoint{1.023315in}{0.667021in}}{\pgfqpoint{1.024779in}{0.670554in}}{\pgfqpoint{1.024779in}{0.674237in}}%
\pgfpathcurveto{\pgfqpoint{1.024779in}{0.677921in}}{\pgfqpoint{1.023315in}{0.681454in}}{\pgfqpoint{1.020711in}{0.684058in}}%
\pgfpathcurveto{\pgfqpoint{1.018106in}{0.686663in}}{\pgfqpoint{1.014573in}{0.688126in}}{\pgfqpoint{1.010890in}{0.688126in}}%
\pgfpathcurveto{\pgfqpoint{1.007207in}{0.688126in}}{\pgfqpoint{1.003674in}{0.686663in}}{\pgfqpoint{1.001069in}{0.684058in}}%
\pgfpathcurveto{\pgfqpoint{0.998464in}{0.681454in}}{\pgfqpoint{0.997001in}{0.677921in}}{\pgfqpoint{0.997001in}{0.674237in}}%
\pgfpathcurveto{\pgfqpoint{0.997001in}{0.670554in}}{\pgfqpoint{0.998464in}{0.667021in}}{\pgfqpoint{1.001069in}{0.664416in}}%
\pgfpathcurveto{\pgfqpoint{1.003674in}{0.661812in}}{\pgfqpoint{1.007207in}{0.660348in}}{\pgfqpoint{1.010890in}{0.660348in}}%
\pgfpathclose%
\pgfusepath{stroke,fill}%
\end{pgfscope}%
\begin{pgfscope}%
\pgfpathrectangle{\pgfqpoint{0.750000in}{0.500000in}}{\pgfqpoint{4.650000in}{3.020000in}}%
\pgfusepath{clip}%
\pgfsetbuttcap%
\pgfsetroundjoin%
\definecolor{currentfill}{rgb}{0.121569,0.466667,0.705882}%
\pgfsetfillcolor{currentfill}%
\pgfsetlinewidth{1.003750pt}%
\definecolor{currentstroke}{rgb}{0.121569,0.466667,0.705882}%
\pgfsetstrokecolor{currentstroke}%
\pgfsetdash{}{0pt}%
\pgfpathmoveto{\pgfqpoint{2.179916in}{1.540585in}}%
\pgfpathcurveto{\pgfqpoint{2.189124in}{1.540585in}}{\pgfqpoint{2.197957in}{1.544243in}}{\pgfqpoint{2.204468in}{1.550755in}}%
\pgfpathcurveto{\pgfqpoint{2.210980in}{1.557266in}}{\pgfqpoint{2.214638in}{1.566099in}}{\pgfqpoint{2.214638in}{1.575307in}}%
\pgfpathcurveto{\pgfqpoint{2.214638in}{1.584515in}}{\pgfqpoint{2.210980in}{1.593348in}}{\pgfqpoint{2.204468in}{1.599859in}}%
\pgfpathcurveto{\pgfqpoint{2.197957in}{1.606371in}}{\pgfqpoint{2.189124in}{1.610029in}}{\pgfqpoint{2.179916in}{1.610029in}}%
\pgfpathcurveto{\pgfqpoint{2.170708in}{1.610029in}}{\pgfqpoint{2.161875in}{1.606371in}}{\pgfqpoint{2.155364in}{1.599859in}}%
\pgfpathcurveto{\pgfqpoint{2.148852in}{1.593348in}}{\pgfqpoint{2.145194in}{1.584515in}}{\pgfqpoint{2.145194in}{1.575307in}}%
\pgfpathcurveto{\pgfqpoint{2.145194in}{1.566099in}}{\pgfqpoint{2.148852in}{1.557266in}}{\pgfqpoint{2.155364in}{1.550755in}}%
\pgfpathcurveto{\pgfqpoint{2.161875in}{1.544243in}}{\pgfqpoint{2.170708in}{1.540585in}}{\pgfqpoint{2.179916in}{1.540585in}}%
\pgfpathclose%
\pgfusepath{stroke,fill}%
\end{pgfscope}%
\begin{pgfscope}%
\pgfpathrectangle{\pgfqpoint{0.750000in}{0.500000in}}{\pgfqpoint{4.650000in}{3.020000in}}%
\pgfusepath{clip}%
\pgfsetbuttcap%
\pgfsetroundjoin%
\definecolor{currentfill}{rgb}{0.121569,0.466667,0.705882}%
\pgfsetfillcolor{currentfill}%
\pgfsetlinewidth{1.003750pt}%
\definecolor{currentstroke}{rgb}{0.121569,0.466667,0.705882}%
\pgfsetstrokecolor{currentstroke}%
\pgfsetdash{}{0pt}%
\pgfpathmoveto{\pgfqpoint{1.061448in}{0.743626in}}%
\pgfpathcurveto{\pgfqpoint{1.064052in}{0.743626in}}{\pgfqpoint{1.066551in}{0.744661in}}{\pgfqpoint{1.068392in}{0.746502in}}%
\pgfpathcurveto{\pgfqpoint{1.070234in}{0.748344in}}{\pgfqpoint{1.071269in}{0.750842in}}{\pgfqpoint{1.071269in}{0.753447in}}%
\pgfpathcurveto{\pgfqpoint{1.071269in}{0.756051in}}{\pgfqpoint{1.070234in}{0.758550in}}{\pgfqpoint{1.068392in}{0.760391in}}%
\pgfpathcurveto{\pgfqpoint{1.066551in}{0.762233in}}{\pgfqpoint{1.064052in}{0.763268in}}{\pgfqpoint{1.061448in}{0.763268in}}%
\pgfpathcurveto{\pgfqpoint{1.058843in}{0.763268in}}{\pgfqpoint{1.056345in}{0.762233in}}{\pgfqpoint{1.054504in}{0.760391in}}%
\pgfpathcurveto{\pgfqpoint{1.052662in}{0.758550in}}{\pgfqpoint{1.051627in}{0.756051in}}{\pgfqpoint{1.051627in}{0.753447in}}%
\pgfpathcurveto{\pgfqpoint{1.051627in}{0.750842in}}{\pgfqpoint{1.052662in}{0.748344in}}{\pgfqpoint{1.054504in}{0.746502in}}%
\pgfpathcurveto{\pgfqpoint{1.056345in}{0.744661in}}{\pgfqpoint{1.058843in}{0.743626in}}{\pgfqpoint{1.061448in}{0.743626in}}%
\pgfpathclose%
\pgfusepath{stroke,fill}%
\end{pgfscope}%
\begin{pgfscope}%
\pgfpathrectangle{\pgfqpoint{0.750000in}{0.500000in}}{\pgfqpoint{4.650000in}{3.020000in}}%
\pgfusepath{clip}%
\pgfsetbuttcap%
\pgfsetroundjoin%
\definecolor{currentfill}{rgb}{0.121569,0.466667,0.705882}%
\pgfsetfillcolor{currentfill}%
\pgfsetlinewidth{1.003750pt}%
\definecolor{currentstroke}{rgb}{0.121569,0.466667,0.705882}%
\pgfsetstrokecolor{currentstroke}%
\pgfsetdash{}{0pt}%
\pgfpathmoveto{\pgfqpoint{1.846130in}{0.843532in}}%
\pgfpathcurveto{\pgfqpoint{1.850248in}{0.843532in}}{\pgfqpoint{1.854198in}{0.845168in}}{\pgfqpoint{1.857110in}{0.848080in}}%
\pgfpathcurveto{\pgfqpoint{1.860022in}{0.850992in}}{\pgfqpoint{1.861658in}{0.854942in}}{\pgfqpoint{1.861658in}{0.859060in}}%
\pgfpathcurveto{\pgfqpoint{1.861658in}{0.863178in}}{\pgfqpoint{1.860022in}{0.867128in}}{\pgfqpoint{1.857110in}{0.870040in}}%
\pgfpathcurveto{\pgfqpoint{1.854198in}{0.872952in}}{\pgfqpoint{1.850248in}{0.874588in}}{\pgfqpoint{1.846130in}{0.874588in}}%
\pgfpathcurveto{\pgfqpoint{1.842012in}{0.874588in}}{\pgfqpoint{1.838061in}{0.872952in}}{\pgfqpoint{1.835150in}{0.870040in}}%
\pgfpathcurveto{\pgfqpoint{1.832238in}{0.867128in}}{\pgfqpoint{1.830601in}{0.863178in}}{\pgfqpoint{1.830601in}{0.859060in}}%
\pgfpathcurveto{\pgfqpoint{1.830601in}{0.854942in}}{\pgfqpoint{1.832238in}{0.850992in}}{\pgfqpoint{1.835150in}{0.848080in}}%
\pgfpathcurveto{\pgfqpoint{1.838061in}{0.845168in}}{\pgfqpoint{1.842012in}{0.843532in}}{\pgfqpoint{1.846130in}{0.843532in}}%
\pgfpathclose%
\pgfusepath{stroke,fill}%
\end{pgfscope}%
\begin{pgfscope}%
\pgfpathrectangle{\pgfqpoint{0.750000in}{0.500000in}}{\pgfqpoint{4.650000in}{3.020000in}}%
\pgfusepath{clip}%
\pgfsetbuttcap%
\pgfsetroundjoin%
\definecolor{currentfill}{rgb}{0.121569,0.466667,0.705882}%
\pgfsetfillcolor{currentfill}%
\pgfsetlinewidth{1.003750pt}%
\definecolor{currentstroke}{rgb}{0.121569,0.466667,0.705882}%
\pgfsetstrokecolor{currentstroke}%
\pgfsetdash{}{0pt}%
\pgfpathmoveto{\pgfqpoint{1.036685in}{0.638493in}}%
\pgfpathcurveto{\pgfqpoint{1.039289in}{0.638493in}}{\pgfqpoint{1.041788in}{0.639528in}}{\pgfqpoint{1.043629in}{0.641370in}}%
\pgfpathcurveto{\pgfqpoint{1.045471in}{0.643211in}}{\pgfqpoint{1.046506in}{0.645710in}}{\pgfqpoint{1.046506in}{0.648314in}}%
\pgfpathcurveto{\pgfqpoint{1.046506in}{0.650919in}}{\pgfqpoint{1.045471in}{0.653417in}}{\pgfqpoint{1.043629in}{0.655259in}}%
\pgfpathcurveto{\pgfqpoint{1.041788in}{0.657100in}}{\pgfqpoint{1.039289in}{0.658135in}}{\pgfqpoint{1.036685in}{0.658135in}}%
\pgfpathcurveto{\pgfqpoint{1.034080in}{0.658135in}}{\pgfqpoint{1.031582in}{0.657100in}}{\pgfqpoint{1.029740in}{0.655259in}}%
\pgfpathcurveto{\pgfqpoint{1.027899in}{0.653417in}}{\pgfqpoint{1.026864in}{0.650919in}}{\pgfqpoint{1.026864in}{0.648314in}}%
\pgfpathcurveto{\pgfqpoint{1.026864in}{0.645710in}}{\pgfqpoint{1.027899in}{0.643211in}}{\pgfqpoint{1.029740in}{0.641370in}}%
\pgfpathcurveto{\pgfqpoint{1.031582in}{0.639528in}}{\pgfqpoint{1.034080in}{0.638493in}}{\pgfqpoint{1.036685in}{0.638493in}}%
\pgfpathclose%
\pgfusepath{stroke,fill}%
\end{pgfscope}%
\begin{pgfscope}%
\pgfpathrectangle{\pgfqpoint{0.750000in}{0.500000in}}{\pgfqpoint{4.650000in}{3.020000in}}%
\pgfusepath{clip}%
\pgfsetbuttcap%
\pgfsetroundjoin%
\definecolor{currentfill}{rgb}{0.121569,0.466667,0.705882}%
\pgfsetfillcolor{currentfill}%
\pgfsetlinewidth{1.003750pt}%
\definecolor{currentstroke}{rgb}{0.121569,0.466667,0.705882}%
\pgfsetstrokecolor{currentstroke}%
\pgfsetdash{}{0pt}%
\pgfpathmoveto{\pgfqpoint{2.100468in}{3.273314in}}%
\pgfpathcurveto{\pgfqpoint{2.105677in}{3.273314in}}{\pgfqpoint{2.110673in}{3.275384in}}{\pgfqpoint{2.114357in}{3.279067in}}%
\pgfpathcurveto{\pgfqpoint{2.118040in}{3.282751in}}{\pgfqpoint{2.120109in}{3.287747in}}{\pgfqpoint{2.120109in}{3.292956in}}%
\pgfpathcurveto{\pgfqpoint{2.120109in}{3.298165in}}{\pgfqpoint{2.118040in}{3.303162in}}{\pgfqpoint{2.114357in}{3.306845in}}%
\pgfpathcurveto{\pgfqpoint{2.110673in}{3.310529in}}{\pgfqpoint{2.105677in}{3.312598in}}{\pgfqpoint{2.100468in}{3.312598in}}%
\pgfpathcurveto{\pgfqpoint{2.095259in}{3.312598in}}{\pgfqpoint{2.090262in}{3.310529in}}{\pgfqpoint{2.086579in}{3.306845in}}%
\pgfpathcurveto{\pgfqpoint{2.082895in}{3.303162in}}{\pgfqpoint{2.080826in}{3.298165in}}{\pgfqpoint{2.080826in}{3.292956in}}%
\pgfpathcurveto{\pgfqpoint{2.080826in}{3.287747in}}{\pgfqpoint{2.082895in}{3.282751in}}{\pgfqpoint{2.086579in}{3.279067in}}%
\pgfpathcurveto{\pgfqpoint{2.090262in}{3.275384in}}{\pgfqpoint{2.095259in}{3.273314in}}{\pgfqpoint{2.100468in}{3.273314in}}%
\pgfpathclose%
\pgfusepath{stroke,fill}%
\end{pgfscope}%
\begin{pgfscope}%
\pgfpathrectangle{\pgfqpoint{0.750000in}{0.500000in}}{\pgfqpoint{4.650000in}{3.020000in}}%
\pgfusepath{clip}%
\pgfsetbuttcap%
\pgfsetroundjoin%
\definecolor{currentfill}{rgb}{0.121569,0.466667,0.705882}%
\pgfsetfillcolor{currentfill}%
\pgfsetlinewidth{1.003750pt}%
\definecolor{currentstroke}{rgb}{0.121569,0.466667,0.705882}%
\pgfsetstrokecolor{currentstroke}%
\pgfsetdash{}{0pt}%
\pgfpathmoveto{\pgfqpoint{1.257489in}{0.741899in}}%
\pgfpathcurveto{\pgfqpoint{1.260679in}{0.741899in}}{\pgfqpoint{1.263739in}{0.743166in}}{\pgfqpoint{1.265995in}{0.745422in}}%
\pgfpathcurveto{\pgfqpoint{1.268250in}{0.747677in}}{\pgfqpoint{1.269518in}{0.750737in}}{\pgfqpoint{1.269518in}{0.753927in}}%
\pgfpathcurveto{\pgfqpoint{1.269518in}{0.757117in}}{\pgfqpoint{1.268250in}{0.760177in}}{\pgfqpoint{1.265995in}{0.762432in}}%
\pgfpathcurveto{\pgfqpoint{1.263739in}{0.764688in}}{\pgfqpoint{1.260679in}{0.765955in}}{\pgfqpoint{1.257489in}{0.765955in}}%
\pgfpathcurveto{\pgfqpoint{1.254300in}{0.765955in}}{\pgfqpoint{1.251240in}{0.764688in}}{\pgfqpoint{1.248984in}{0.762432in}}%
\pgfpathcurveto{\pgfqpoint{1.246729in}{0.760177in}}{\pgfqpoint{1.245461in}{0.757117in}}{\pgfqpoint{1.245461in}{0.753927in}}%
\pgfpathcurveto{\pgfqpoint{1.245461in}{0.750737in}}{\pgfqpoint{1.246729in}{0.747677in}}{\pgfqpoint{1.248984in}{0.745422in}}%
\pgfpathcurveto{\pgfqpoint{1.251240in}{0.743166in}}{\pgfqpoint{1.254300in}{0.741899in}}{\pgfqpoint{1.257489in}{0.741899in}}%
\pgfpathclose%
\pgfusepath{stroke,fill}%
\end{pgfscope}%
\begin{pgfscope}%
\pgfpathrectangle{\pgfqpoint{0.750000in}{0.500000in}}{\pgfqpoint{4.650000in}{3.020000in}}%
\pgfusepath{clip}%
\pgfsetbuttcap%
\pgfsetroundjoin%
\definecolor{currentfill}{rgb}{0.121569,0.466667,0.705882}%
\pgfsetfillcolor{currentfill}%
\pgfsetlinewidth{1.003750pt}%
\definecolor{currentstroke}{rgb}{0.121569,0.466667,0.705882}%
\pgfsetstrokecolor{currentstroke}%
\pgfsetdash{}{0pt}%
\pgfpathmoveto{\pgfqpoint{1.428768in}{0.734277in}}%
\pgfpathcurveto{\pgfqpoint{1.432451in}{0.734277in}}{\pgfqpoint{1.435984in}{0.735741in}}{\pgfqpoint{1.438589in}{0.738345in}}%
\pgfpathcurveto{\pgfqpoint{1.441193in}{0.740950in}}{\pgfqpoint{1.442657in}{0.744483in}}{\pgfqpoint{1.442657in}{0.748166in}}%
\pgfpathcurveto{\pgfqpoint{1.442657in}{0.751850in}}{\pgfqpoint{1.441193in}{0.755383in}}{\pgfqpoint{1.438589in}{0.757987in}}%
\pgfpathcurveto{\pgfqpoint{1.435984in}{0.760592in}}{\pgfqpoint{1.432451in}{0.762055in}}{\pgfqpoint{1.428768in}{0.762055in}}%
\pgfpathcurveto{\pgfqpoint{1.425084in}{0.762055in}}{\pgfqpoint{1.421551in}{0.760592in}}{\pgfqpoint{1.418947in}{0.757987in}}%
\pgfpathcurveto{\pgfqpoint{1.416342in}{0.755383in}}{\pgfqpoint{1.414879in}{0.751850in}}{\pgfqpoint{1.414879in}{0.748166in}}%
\pgfpathcurveto{\pgfqpoint{1.414879in}{0.744483in}}{\pgfqpoint{1.416342in}{0.740950in}}{\pgfqpoint{1.418947in}{0.738345in}}%
\pgfpathcurveto{\pgfqpoint{1.421551in}{0.735741in}}{\pgfqpoint{1.425084in}{0.734277in}}{\pgfqpoint{1.428768in}{0.734277in}}%
\pgfpathclose%
\pgfusepath{stroke,fill}%
\end{pgfscope}%
\begin{pgfscope}%
\pgfpathrectangle{\pgfqpoint{0.750000in}{0.500000in}}{\pgfqpoint{4.650000in}{3.020000in}}%
\pgfusepath{clip}%
\pgfsetbuttcap%
\pgfsetroundjoin%
\definecolor{currentfill}{rgb}{0.121569,0.466667,0.705882}%
\pgfsetfillcolor{currentfill}%
\pgfsetlinewidth{1.003750pt}%
\definecolor{currentstroke}{rgb}{0.121569,0.466667,0.705882}%
\pgfsetstrokecolor{currentstroke}%
\pgfsetdash{}{0pt}%
\pgfpathmoveto{\pgfqpoint{1.004183in}{0.635133in}}%
\pgfpathcurveto{\pgfqpoint{1.006788in}{0.635133in}}{\pgfqpoint{1.009286in}{0.636168in}}{\pgfqpoint{1.011128in}{0.638009in}}%
\pgfpathcurveto{\pgfqpoint{1.012969in}{0.639851in}}{\pgfqpoint{1.014004in}{0.642349in}}{\pgfqpoint{1.014004in}{0.644954in}}%
\pgfpathcurveto{\pgfqpoint{1.014004in}{0.647558in}}{\pgfqpoint{1.012969in}{0.650056in}}{\pgfqpoint{1.011128in}{0.651898in}}%
\pgfpathcurveto{\pgfqpoint{1.009286in}{0.653740in}}{\pgfqpoint{1.006788in}{0.654775in}}{\pgfqpoint{1.004183in}{0.654775in}}%
\pgfpathcurveto{\pgfqpoint{1.001579in}{0.654775in}}{\pgfqpoint{0.999080in}{0.653740in}}{\pgfqpoint{0.997239in}{0.651898in}}%
\pgfpathcurveto{\pgfqpoint{0.995397in}{0.650056in}}{\pgfqpoint{0.994362in}{0.647558in}}{\pgfqpoint{0.994362in}{0.644954in}}%
\pgfpathcurveto{\pgfqpoint{0.994362in}{0.642349in}}{\pgfqpoint{0.995397in}{0.639851in}}{\pgfqpoint{0.997239in}{0.638009in}}%
\pgfpathcurveto{\pgfqpoint{0.999080in}{0.636168in}}{\pgfqpoint{1.001579in}{0.635133in}}{\pgfqpoint{1.004183in}{0.635133in}}%
\pgfpathclose%
\pgfusepath{stroke,fill}%
\end{pgfscope}%
\begin{pgfscope}%
\pgfpathrectangle{\pgfqpoint{0.750000in}{0.500000in}}{\pgfqpoint{4.650000in}{3.020000in}}%
\pgfusepath{clip}%
\pgfsetbuttcap%
\pgfsetroundjoin%
\definecolor{currentfill}{rgb}{0.121569,0.466667,0.705882}%
\pgfsetfillcolor{currentfill}%
\pgfsetlinewidth{1.003750pt}%
\definecolor{currentstroke}{rgb}{0.121569,0.466667,0.705882}%
\pgfsetstrokecolor{currentstroke}%
\pgfsetdash{}{0pt}%
\pgfpathmoveto{\pgfqpoint{1.066607in}{0.770041in}}%
\pgfpathcurveto{\pgfqpoint{1.071118in}{0.770041in}}{\pgfqpoint{1.075445in}{0.771833in}}{\pgfqpoint{1.078635in}{0.775023in}}%
\pgfpathcurveto{\pgfqpoint{1.081825in}{0.778213in}}{\pgfqpoint{1.083617in}{0.782540in}}{\pgfqpoint{1.083617in}{0.787051in}}%
\pgfpathcurveto{\pgfqpoint{1.083617in}{0.791562in}}{\pgfqpoint{1.081825in}{0.795889in}}{\pgfqpoint{1.078635in}{0.799079in}}%
\pgfpathcurveto{\pgfqpoint{1.075445in}{0.802269in}}{\pgfqpoint{1.071118in}{0.804061in}}{\pgfqpoint{1.066607in}{0.804061in}}%
\pgfpathcurveto{\pgfqpoint{1.062096in}{0.804061in}}{\pgfqpoint{1.057769in}{0.802269in}}{\pgfqpoint{1.054579in}{0.799079in}}%
\pgfpathcurveto{\pgfqpoint{1.051389in}{0.795889in}}{\pgfqpoint{1.049597in}{0.791562in}}{\pgfqpoint{1.049597in}{0.787051in}}%
\pgfpathcurveto{\pgfqpoint{1.049597in}{0.782540in}}{\pgfqpoint{1.051389in}{0.778213in}}{\pgfqpoint{1.054579in}{0.775023in}}%
\pgfpathcurveto{\pgfqpoint{1.057769in}{0.771833in}}{\pgfqpoint{1.062096in}{0.770041in}}{\pgfqpoint{1.066607in}{0.770041in}}%
\pgfpathclose%
\pgfusepath{stroke,fill}%
\end{pgfscope}%
\begin{pgfscope}%
\pgfpathrectangle{\pgfqpoint{0.750000in}{0.500000in}}{\pgfqpoint{4.650000in}{3.020000in}}%
\pgfusepath{clip}%
\pgfsetbuttcap%
\pgfsetroundjoin%
\definecolor{currentfill}{rgb}{0.121569,0.466667,0.705882}%
\pgfsetfillcolor{currentfill}%
\pgfsetlinewidth{1.003750pt}%
\definecolor{currentstroke}{rgb}{0.121569,0.466667,0.705882}%
\pgfsetstrokecolor{currentstroke}%
\pgfsetdash{}{0pt}%
\pgfpathmoveto{\pgfqpoint{1.117165in}{0.996527in}}%
\pgfpathcurveto{\pgfqpoint{1.124758in}{0.996527in}}{\pgfqpoint{1.132042in}{0.999544in}}{\pgfqpoint{1.137411in}{1.004914in}}%
\pgfpathcurveto{\pgfqpoint{1.142781in}{1.010283in}}{\pgfqpoint{1.145798in}{1.017567in}}{\pgfqpoint{1.145798in}{1.025160in}}%
\pgfpathcurveto{\pgfqpoint{1.145798in}{1.032753in}}{\pgfqpoint{1.142781in}{1.040037in}}{\pgfqpoint{1.137411in}{1.045406in}}%
\pgfpathcurveto{\pgfqpoint{1.132042in}{1.050776in}}{\pgfqpoint{1.124758in}{1.053793in}}{\pgfqpoint{1.117165in}{1.053793in}}%
\pgfpathcurveto{\pgfqpoint{1.109572in}{1.053793in}}{\pgfqpoint{1.102288in}{1.050776in}}{\pgfqpoint{1.096919in}{1.045406in}}%
\pgfpathcurveto{\pgfqpoint{1.091549in}{1.040037in}}{\pgfqpoint{1.088532in}{1.032753in}}{\pgfqpoint{1.088532in}{1.025160in}}%
\pgfpathcurveto{\pgfqpoint{1.088532in}{1.017567in}}{\pgfqpoint{1.091549in}{1.010283in}}{\pgfqpoint{1.096919in}{1.004914in}}%
\pgfpathcurveto{\pgfqpoint{1.102288in}{0.999544in}}{\pgfqpoint{1.109572in}{0.996527in}}{\pgfqpoint{1.117165in}{0.996527in}}%
\pgfpathclose%
\pgfusepath{stroke,fill}%
\end{pgfscope}%
\begin{pgfscope}%
\pgfpathrectangle{\pgfqpoint{0.750000in}{0.500000in}}{\pgfqpoint{4.650000in}{3.020000in}}%
\pgfusepath{clip}%
\pgfsetbuttcap%
\pgfsetroundjoin%
\definecolor{currentfill}{rgb}{0.121569,0.466667,0.705882}%
\pgfsetfillcolor{currentfill}%
\pgfsetlinewidth{1.003750pt}%
\definecolor{currentstroke}{rgb}{0.121569,0.466667,0.705882}%
\pgfsetstrokecolor{currentstroke}%
\pgfsetdash{}{0pt}%
\pgfpathmoveto{\pgfqpoint{1.046487in}{0.842127in}}%
\pgfpathcurveto{\pgfqpoint{1.051360in}{0.842127in}}{\pgfqpoint{1.056033in}{0.844063in}}{\pgfqpoint{1.059479in}{0.847508in}}%
\pgfpathcurveto{\pgfqpoint{1.062924in}{0.850954in}}{\pgfqpoint{1.064860in}{0.855627in}}{\pgfqpoint{1.064860in}{0.860500in}}%
\pgfpathcurveto{\pgfqpoint{1.064860in}{0.865373in}}{\pgfqpoint{1.062924in}{0.870046in}}{\pgfqpoint{1.059479in}{0.873492in}}%
\pgfpathcurveto{\pgfqpoint{1.056033in}{0.876937in}}{\pgfqpoint{1.051360in}{0.878873in}}{\pgfqpoint{1.046487in}{0.878873in}}%
\pgfpathcurveto{\pgfqpoint{1.041614in}{0.878873in}}{\pgfqpoint{1.036941in}{0.876937in}}{\pgfqpoint{1.033495in}{0.873492in}}%
\pgfpathcurveto{\pgfqpoint{1.030050in}{0.870046in}}{\pgfqpoint{1.028114in}{0.865373in}}{\pgfqpoint{1.028114in}{0.860500in}}%
\pgfpathcurveto{\pgfqpoint{1.028114in}{0.855627in}}{\pgfqpoint{1.030050in}{0.850954in}}{\pgfqpoint{1.033495in}{0.847508in}}%
\pgfpathcurveto{\pgfqpoint{1.036941in}{0.844063in}}{\pgfqpoint{1.041614in}{0.842127in}}{\pgfqpoint{1.046487in}{0.842127in}}%
\pgfpathclose%
\pgfusepath{stroke,fill}%
\end{pgfscope}%
\begin{pgfscope}%
\pgfpathrectangle{\pgfqpoint{0.750000in}{0.500000in}}{\pgfqpoint{4.650000in}{3.020000in}}%
\pgfusepath{clip}%
\pgfsetbuttcap%
\pgfsetroundjoin%
\definecolor{currentfill}{rgb}{0.121569,0.466667,0.705882}%
\pgfsetfillcolor{currentfill}%
\pgfsetlinewidth{1.003750pt}%
\definecolor{currentstroke}{rgb}{0.121569,0.466667,0.705882}%
\pgfsetstrokecolor{currentstroke}%
\pgfsetdash{}{0pt}%
\pgfpathmoveto{\pgfqpoint{0.977356in}{1.139387in}}%
\pgfpathcurveto{\pgfqpoint{0.980546in}{1.139387in}}{\pgfqpoint{0.983606in}{1.140655in}}{\pgfqpoint{0.985862in}{1.142910in}}%
\pgfpathcurveto{\pgfqpoint{0.988117in}{1.145166in}}{\pgfqpoint{0.989385in}{1.148225in}}{\pgfqpoint{0.989385in}{1.151415in}}%
\pgfpathcurveto{\pgfqpoint{0.989385in}{1.154605in}}{\pgfqpoint{0.988117in}{1.157665in}}{\pgfqpoint{0.985862in}{1.159921in}}%
\pgfpathcurveto{\pgfqpoint{0.983606in}{1.162176in}}{\pgfqpoint{0.980546in}{1.163444in}}{\pgfqpoint{0.977356in}{1.163444in}}%
\pgfpathcurveto{\pgfqpoint{0.974167in}{1.163444in}}{\pgfqpoint{0.971107in}{1.162176in}}{\pgfqpoint{0.968851in}{1.159921in}}%
\pgfpathcurveto{\pgfqpoint{0.966596in}{1.157665in}}{\pgfqpoint{0.965328in}{1.154605in}}{\pgfqpoint{0.965328in}{1.151415in}}%
\pgfpathcurveto{\pgfqpoint{0.965328in}{1.148225in}}{\pgfqpoint{0.966596in}{1.145166in}}{\pgfqpoint{0.968851in}{1.142910in}}%
\pgfpathcurveto{\pgfqpoint{0.971107in}{1.140655in}}{\pgfqpoint{0.974167in}{1.139387in}}{\pgfqpoint{0.977356in}{1.139387in}}%
\pgfpathclose%
\pgfusepath{stroke,fill}%
\end{pgfscope}%
\begin{pgfscope}%
\pgfpathrectangle{\pgfqpoint{0.750000in}{0.500000in}}{\pgfqpoint{4.650000in}{3.020000in}}%
\pgfusepath{clip}%
\pgfsetbuttcap%
\pgfsetroundjoin%
\definecolor{currentfill}{rgb}{0.121569,0.466667,0.705882}%
\pgfsetfillcolor{currentfill}%
\pgfsetlinewidth{1.003750pt}%
\definecolor{currentstroke}{rgb}{0.121569,0.466667,0.705882}%
\pgfsetstrokecolor{currentstroke}%
\pgfsetdash{}{0pt}%
\pgfpathmoveto{\pgfqpoint{1.038748in}{0.863413in}}%
\pgfpathcurveto{\pgfqpoint{1.042432in}{0.863413in}}{\pgfqpoint{1.045965in}{0.864877in}}{\pgfqpoint{1.048569in}{0.867481in}}%
\pgfpathcurveto{\pgfqpoint{1.051174in}{0.870086in}}{\pgfqpoint{1.052637in}{0.873619in}}{\pgfqpoint{1.052637in}{0.877302in}}%
\pgfpathcurveto{\pgfqpoint{1.052637in}{0.880985in}}{\pgfqpoint{1.051174in}{0.884518in}}{\pgfqpoint{1.048569in}{0.887123in}}%
\pgfpathcurveto{\pgfqpoint{1.045965in}{0.889727in}}{\pgfqpoint{1.042432in}{0.891191in}}{\pgfqpoint{1.038748in}{0.891191in}}%
\pgfpathcurveto{\pgfqpoint{1.035065in}{0.891191in}}{\pgfqpoint{1.031532in}{0.889727in}}{\pgfqpoint{1.028927in}{0.887123in}}%
\pgfpathcurveto{\pgfqpoint{1.026323in}{0.884518in}}{\pgfqpoint{1.024860in}{0.880985in}}{\pgfqpoint{1.024860in}{0.877302in}}%
\pgfpathcurveto{\pgfqpoint{1.024860in}{0.873619in}}{\pgfqpoint{1.026323in}{0.870086in}}{\pgfqpoint{1.028927in}{0.867481in}}%
\pgfpathcurveto{\pgfqpoint{1.031532in}{0.864877in}}{\pgfqpoint{1.035065in}{0.863413in}}{\pgfqpoint{1.038748in}{0.863413in}}%
\pgfpathclose%
\pgfusepath{stroke,fill}%
\end{pgfscope}%
\begin{pgfscope}%
\pgfpathrectangle{\pgfqpoint{0.750000in}{0.500000in}}{\pgfqpoint{4.650000in}{3.020000in}}%
\pgfusepath{clip}%
\pgfsetbuttcap%
\pgfsetroundjoin%
\definecolor{currentfill}{rgb}{0.121569,0.466667,0.705882}%
\pgfsetfillcolor{currentfill}%
\pgfsetlinewidth{1.003750pt}%
\definecolor{currentstroke}{rgb}{0.121569,0.466667,0.705882}%
\pgfsetstrokecolor{currentstroke}%
\pgfsetdash{}{0pt}%
\pgfpathmoveto{\pgfqpoint{1.031526in}{0.631292in}}%
\pgfpathcurveto{\pgfqpoint{1.034130in}{0.631292in}}{\pgfqpoint{1.036629in}{0.632327in}}{\pgfqpoint{1.038470in}{0.634169in}}%
\pgfpathcurveto{\pgfqpoint{1.040312in}{0.636010in}}{\pgfqpoint{1.041347in}{0.638509in}}{\pgfqpoint{1.041347in}{0.641113in}}%
\pgfpathcurveto{\pgfqpoint{1.041347in}{0.643718in}}{\pgfqpoint{1.040312in}{0.646216in}}{\pgfqpoint{1.038470in}{0.648058in}}%
\pgfpathcurveto{\pgfqpoint{1.036629in}{0.649899in}}{\pgfqpoint{1.034130in}{0.650934in}}{\pgfqpoint{1.031526in}{0.650934in}}%
\pgfpathcurveto{\pgfqpoint{1.028921in}{0.650934in}}{\pgfqpoint{1.026423in}{0.649899in}}{\pgfqpoint{1.024581in}{0.648058in}}%
\pgfpathcurveto{\pgfqpoint{1.022740in}{0.646216in}}{\pgfqpoint{1.021705in}{0.643718in}}{\pgfqpoint{1.021705in}{0.641113in}}%
\pgfpathcurveto{\pgfqpoint{1.021705in}{0.638509in}}{\pgfqpoint{1.022740in}{0.636010in}}{\pgfqpoint{1.024581in}{0.634169in}}%
\pgfpathcurveto{\pgfqpoint{1.026423in}{0.632327in}}{\pgfqpoint{1.028921in}{0.631292in}}{\pgfqpoint{1.031526in}{0.631292in}}%
\pgfpathclose%
\pgfusepath{stroke,fill}%
\end{pgfscope}%
\begin{pgfscope}%
\pgfpathrectangle{\pgfqpoint{0.750000in}{0.500000in}}{\pgfqpoint{4.650000in}{3.020000in}}%
\pgfusepath{clip}%
\pgfsetbuttcap%
\pgfsetroundjoin%
\definecolor{currentfill}{rgb}{0.121569,0.466667,0.705882}%
\pgfsetfillcolor{currentfill}%
\pgfsetlinewidth{1.003750pt}%
\definecolor{currentstroke}{rgb}{0.121569,0.466667,0.705882}%
\pgfsetstrokecolor{currentstroke}%
\pgfsetdash{}{0pt}%
\pgfpathmoveto{\pgfqpoint{1.391623in}{0.680750in}}%
\pgfpathcurveto{\pgfqpoint{1.396134in}{0.680750in}}{\pgfqpoint{1.400461in}{0.682542in}}{\pgfqpoint{1.403651in}{0.685732in}}%
\pgfpathcurveto{\pgfqpoint{1.406841in}{0.688922in}}{\pgfqpoint{1.408633in}{0.693249in}}{\pgfqpoint{1.408633in}{0.697760in}}%
\pgfpathcurveto{\pgfqpoint{1.408633in}{0.702271in}}{\pgfqpoint{1.406841in}{0.706598in}}{\pgfqpoint{1.403651in}{0.709788in}}%
\pgfpathcurveto{\pgfqpoint{1.400461in}{0.712978in}}{\pgfqpoint{1.396134in}{0.714770in}}{\pgfqpoint{1.391623in}{0.714770in}}%
\pgfpathcurveto{\pgfqpoint{1.387112in}{0.714770in}}{\pgfqpoint{1.382785in}{0.712978in}}{\pgfqpoint{1.379595in}{0.709788in}}%
\pgfpathcurveto{\pgfqpoint{1.376405in}{0.706598in}}{\pgfqpoint{1.374613in}{0.702271in}}{\pgfqpoint{1.374613in}{0.697760in}}%
\pgfpathcurveto{\pgfqpoint{1.374613in}{0.693249in}}{\pgfqpoint{1.376405in}{0.688922in}}{\pgfqpoint{1.379595in}{0.685732in}}%
\pgfpathcurveto{\pgfqpoint{1.382785in}{0.682542in}}{\pgfqpoint{1.387112in}{0.680750in}}{\pgfqpoint{1.391623in}{0.680750in}}%
\pgfpathclose%
\pgfusepath{stroke,fill}%
\end{pgfscope}%
\begin{pgfscope}%
\pgfpathrectangle{\pgfqpoint{0.750000in}{0.500000in}}{\pgfqpoint{4.650000in}{3.020000in}}%
\pgfusepath{clip}%
\pgfsetbuttcap%
\pgfsetroundjoin%
\definecolor{currentfill}{rgb}{0.121569,0.466667,0.705882}%
\pgfsetfillcolor{currentfill}%
\pgfsetlinewidth{1.003750pt}%
\definecolor{currentstroke}{rgb}{0.121569,0.466667,0.705882}%
\pgfsetstrokecolor{currentstroke}%
\pgfsetdash{}{0pt}%
\pgfpathmoveto{\pgfqpoint{0.972713in}{0.631772in}}%
\pgfpathcurveto{\pgfqpoint{0.975318in}{0.631772in}}{\pgfqpoint{0.977816in}{0.632807in}}{\pgfqpoint{0.979658in}{0.634649in}}%
\pgfpathcurveto{\pgfqpoint{0.981500in}{0.636490in}}{\pgfqpoint{0.982534in}{0.638989in}}{\pgfqpoint{0.982534in}{0.641593in}}%
\pgfpathcurveto{\pgfqpoint{0.982534in}{0.644198in}}{\pgfqpoint{0.981500in}{0.646696in}}{\pgfqpoint{0.979658in}{0.648538in}}%
\pgfpathcurveto{\pgfqpoint{0.977816in}{0.650379in}}{\pgfqpoint{0.975318in}{0.651414in}}{\pgfqpoint{0.972713in}{0.651414in}}%
\pgfpathcurveto{\pgfqpoint{0.970109in}{0.651414in}}{\pgfqpoint{0.967611in}{0.650379in}}{\pgfqpoint{0.965769in}{0.648538in}}%
\pgfpathcurveto{\pgfqpoint{0.963927in}{0.646696in}}{\pgfqpoint{0.962892in}{0.644198in}}{\pgfqpoint{0.962892in}{0.641593in}}%
\pgfpathcurveto{\pgfqpoint{0.962892in}{0.638989in}}{\pgfqpoint{0.963927in}{0.636490in}}{\pgfqpoint{0.965769in}{0.634649in}}%
\pgfpathcurveto{\pgfqpoint{0.967611in}{0.632807in}}{\pgfqpoint{0.970109in}{0.631772in}}{\pgfqpoint{0.972713in}{0.631772in}}%
\pgfpathclose%
\pgfusepath{stroke,fill}%
\end{pgfscope}%
\begin{pgfscope}%
\pgfpathrectangle{\pgfqpoint{0.750000in}{0.500000in}}{\pgfqpoint{4.650000in}{3.020000in}}%
\pgfusepath{clip}%
\pgfsetbuttcap%
\pgfsetroundjoin%
\definecolor{currentfill}{rgb}{0.121569,0.466667,0.705882}%
\pgfsetfillcolor{currentfill}%
\pgfsetlinewidth{1.003750pt}%
\definecolor{currentstroke}{rgb}{0.121569,0.466667,0.705882}%
\pgfsetstrokecolor{currentstroke}%
\pgfsetdash{}{0pt}%
\pgfpathmoveto{\pgfqpoint{1.317334in}{0.743339in}}%
\pgfpathcurveto{\pgfqpoint{1.320524in}{0.743339in}}{\pgfqpoint{1.323583in}{0.744606in}}{\pgfqpoint{1.325839in}{0.746862in}}%
\pgfpathcurveto{\pgfqpoint{1.328094in}{0.749118in}}{\pgfqpoint{1.329362in}{0.752177in}}{\pgfqpoint{1.329362in}{0.755367in}}%
\pgfpathcurveto{\pgfqpoint{1.329362in}{0.758557in}}{\pgfqpoint{1.328094in}{0.761617in}}{\pgfqpoint{1.325839in}{0.763872in}}%
\pgfpathcurveto{\pgfqpoint{1.323583in}{0.766128in}}{\pgfqpoint{1.320524in}{0.767395in}}{\pgfqpoint{1.317334in}{0.767395in}}%
\pgfpathcurveto{\pgfqpoint{1.314144in}{0.767395in}}{\pgfqpoint{1.311084in}{0.766128in}}{\pgfqpoint{1.308828in}{0.763872in}}%
\pgfpathcurveto{\pgfqpoint{1.306573in}{0.761617in}}{\pgfqpoint{1.305306in}{0.758557in}}{\pgfqpoint{1.305306in}{0.755367in}}%
\pgfpathcurveto{\pgfqpoint{1.305306in}{0.752177in}}{\pgfqpoint{1.306573in}{0.749118in}}{\pgfqpoint{1.308828in}{0.746862in}}%
\pgfpathcurveto{\pgfqpoint{1.311084in}{0.744606in}}{\pgfqpoint{1.314144in}{0.743339in}}{\pgfqpoint{1.317334in}{0.743339in}}%
\pgfpathclose%
\pgfusepath{stroke,fill}%
\end{pgfscope}%
\begin{pgfscope}%
\pgfpathrectangle{\pgfqpoint{0.750000in}{0.500000in}}{\pgfqpoint{4.650000in}{3.020000in}}%
\pgfusepath{clip}%
\pgfsetbuttcap%
\pgfsetroundjoin%
\definecolor{currentfill}{rgb}{0.121569,0.466667,0.705882}%
\pgfsetfillcolor{currentfill}%
\pgfsetlinewidth{1.003750pt}%
\definecolor{currentstroke}{rgb}{0.121569,0.466667,0.705882}%
\pgfsetstrokecolor{currentstroke}%
\pgfsetdash{}{0pt}%
\pgfpathmoveto{\pgfqpoint{1.831684in}{0.862911in}}%
\pgfpathcurveto{\pgfqpoint{1.837793in}{0.862911in}}{\pgfqpoint{1.843652in}{0.865338in}}{\pgfqpoint{1.847971in}{0.869657in}}%
\pgfpathcurveto{\pgfqpoint{1.852290in}{0.873976in}}{\pgfqpoint{1.854717in}{0.879835in}}{\pgfqpoint{1.854717in}{0.885943in}}%
\pgfpathcurveto{\pgfqpoint{1.854717in}{0.892051in}}{\pgfqpoint{1.852290in}{0.897910in}}{\pgfqpoint{1.847971in}{0.902229in}}%
\pgfpathcurveto{\pgfqpoint{1.843652in}{0.906548in}}{\pgfqpoint{1.837793in}{0.908975in}}{\pgfqpoint{1.831684in}{0.908975in}}%
\pgfpathcurveto{\pgfqpoint{1.825576in}{0.908975in}}{\pgfqpoint{1.819717in}{0.906548in}}{\pgfqpoint{1.815398in}{0.902229in}}%
\pgfpathcurveto{\pgfqpoint{1.811079in}{0.897910in}}{\pgfqpoint{1.808652in}{0.892051in}}{\pgfqpoint{1.808652in}{0.885943in}}%
\pgfpathcurveto{\pgfqpoint{1.808652in}{0.879835in}}{\pgfqpoint{1.811079in}{0.873976in}}{\pgfqpoint{1.815398in}{0.869657in}}%
\pgfpathcurveto{\pgfqpoint{1.819717in}{0.865338in}}{\pgfqpoint{1.825576in}{0.862911in}}{\pgfqpoint{1.831684in}{0.862911in}}%
\pgfpathclose%
\pgfusepath{stroke,fill}%
\end{pgfscope}%
\begin{pgfscope}%
\pgfpathrectangle{\pgfqpoint{0.750000in}{0.500000in}}{\pgfqpoint{4.650000in}{3.020000in}}%
\pgfusepath{clip}%
\pgfsetbuttcap%
\pgfsetroundjoin%
\definecolor{currentfill}{rgb}{0.121569,0.466667,0.705882}%
\pgfsetfillcolor{currentfill}%
\pgfsetlinewidth{1.003750pt}%
\definecolor{currentstroke}{rgb}{0.121569,0.466667,0.705882}%
\pgfsetstrokecolor{currentstroke}%
\pgfsetdash{}{0pt}%
\pgfpathmoveto{\pgfqpoint{1.056805in}{0.645407in}}%
\pgfpathcurveto{\pgfqpoint{1.059995in}{0.645407in}}{\pgfqpoint{1.063054in}{0.646674in}}{\pgfqpoint{1.065310in}{0.648930in}}%
\pgfpathcurveto{\pgfqpoint{1.067566in}{0.651186in}}{\pgfqpoint{1.068833in}{0.654245in}}{\pgfqpoint{1.068833in}{0.657435in}}%
\pgfpathcurveto{\pgfqpoint{1.068833in}{0.660625in}}{\pgfqpoint{1.067566in}{0.663685in}}{\pgfqpoint{1.065310in}{0.665940in}}%
\pgfpathcurveto{\pgfqpoint{1.063054in}{0.668196in}}{\pgfqpoint{1.059995in}{0.669463in}}{\pgfqpoint{1.056805in}{0.669463in}}%
\pgfpathcurveto{\pgfqpoint{1.053615in}{0.669463in}}{\pgfqpoint{1.050555in}{0.668196in}}{\pgfqpoint{1.048300in}{0.665940in}}%
\pgfpathcurveto{\pgfqpoint{1.046044in}{0.663685in}}{\pgfqpoint{1.044777in}{0.660625in}}{\pgfqpoint{1.044777in}{0.657435in}}%
\pgfpathcurveto{\pgfqpoint{1.044777in}{0.654245in}}{\pgfqpoint{1.046044in}{0.651186in}}{\pgfqpoint{1.048300in}{0.648930in}}%
\pgfpathcurveto{\pgfqpoint{1.050555in}{0.646674in}}{\pgfqpoint{1.053615in}{0.645407in}}{\pgfqpoint{1.056805in}{0.645407in}}%
\pgfpathclose%
\pgfusepath{stroke,fill}%
\end{pgfscope}%
\begin{pgfscope}%
\pgfpathrectangle{\pgfqpoint{0.750000in}{0.500000in}}{\pgfqpoint{4.650000in}{3.020000in}}%
\pgfusepath{clip}%
\pgfsetbuttcap%
\pgfsetroundjoin%
\definecolor{currentfill}{rgb}{0.121569,0.466667,0.705882}%
\pgfsetfillcolor{currentfill}%
\pgfsetlinewidth{1.003750pt}%
\definecolor{currentstroke}{rgb}{0.121569,0.466667,0.705882}%
\pgfsetstrokecolor{currentstroke}%
\pgfsetdash{}{0pt}%
\pgfpathmoveto{\pgfqpoint{1.103236in}{0.644734in}}%
\pgfpathcurveto{\pgfqpoint{1.105840in}{0.644734in}}{\pgfqpoint{1.108338in}{0.645769in}}{\pgfqpoint{1.110180in}{0.647610in}}%
\pgfpathcurveto{\pgfqpoint{1.112022in}{0.649452in}}{\pgfqpoint{1.113057in}{0.651950in}}{\pgfqpoint{1.113057in}{0.654555in}}%
\pgfpathcurveto{\pgfqpoint{1.113057in}{0.657159in}}{\pgfqpoint{1.112022in}{0.659658in}}{\pgfqpoint{1.110180in}{0.661499in}}%
\pgfpathcurveto{\pgfqpoint{1.108338in}{0.663341in}}{\pgfqpoint{1.105840in}{0.664376in}}{\pgfqpoint{1.103236in}{0.664376in}}%
\pgfpathcurveto{\pgfqpoint{1.100631in}{0.664376in}}{\pgfqpoint{1.098133in}{0.663341in}}{\pgfqpoint{1.096291in}{0.661499in}}%
\pgfpathcurveto{\pgfqpoint{1.094450in}{0.659658in}}{\pgfqpoint{1.093415in}{0.657159in}}{\pgfqpoint{1.093415in}{0.654555in}}%
\pgfpathcurveto{\pgfqpoint{1.093415in}{0.651950in}}{\pgfqpoint{1.094450in}{0.649452in}}{\pgfqpoint{1.096291in}{0.647610in}}%
\pgfpathcurveto{\pgfqpoint{1.098133in}{0.645769in}}{\pgfqpoint{1.100631in}{0.644734in}}{\pgfqpoint{1.103236in}{0.644734in}}%
\pgfpathclose%
\pgfusepath{stroke,fill}%
\end{pgfscope}%
\begin{pgfscope}%
\pgfpathrectangle{\pgfqpoint{0.750000in}{0.500000in}}{\pgfqpoint{4.650000in}{3.020000in}}%
\pgfusepath{clip}%
\pgfsetbuttcap%
\pgfsetroundjoin%
\definecolor{currentfill}{rgb}{0.121569,0.466667,0.705882}%
\pgfsetfillcolor{currentfill}%
\pgfsetlinewidth{1.003750pt}%
\definecolor{currentstroke}{rgb}{0.121569,0.466667,0.705882}%
\pgfsetstrokecolor{currentstroke}%
\pgfsetdash{}{0pt}%
\pgfpathmoveto{\pgfqpoint{1.112522in}{0.643007in}}%
\pgfpathcurveto{\pgfqpoint{1.115712in}{0.643007in}}{\pgfqpoint{1.118771in}{0.644274in}}{\pgfqpoint{1.121027in}{0.646530in}}%
\pgfpathcurveto{\pgfqpoint{1.123283in}{0.648785in}}{\pgfqpoint{1.124550in}{0.651845in}}{\pgfqpoint{1.124550in}{0.655035in}}%
\pgfpathcurveto{\pgfqpoint{1.124550in}{0.658225in}}{\pgfqpoint{1.123283in}{0.661284in}}{\pgfqpoint{1.121027in}{0.663540in}}%
\pgfpathcurveto{\pgfqpoint{1.118771in}{0.665796in}}{\pgfqpoint{1.115712in}{0.667063in}}{\pgfqpoint{1.112522in}{0.667063in}}%
\pgfpathcurveto{\pgfqpoint{1.109332in}{0.667063in}}{\pgfqpoint{1.106272in}{0.665796in}}{\pgfqpoint{1.104017in}{0.663540in}}%
\pgfpathcurveto{\pgfqpoint{1.101761in}{0.661284in}}{\pgfqpoint{1.100494in}{0.658225in}}{\pgfqpoint{1.100494in}{0.655035in}}%
\pgfpathcurveto{\pgfqpoint{1.100494in}{0.651845in}}{\pgfqpoint{1.101761in}{0.648785in}}{\pgfqpoint{1.104017in}{0.646530in}}%
\pgfpathcurveto{\pgfqpoint{1.106272in}{0.644274in}}{\pgfqpoint{1.109332in}{0.643007in}}{\pgfqpoint{1.112522in}{0.643007in}}%
\pgfpathclose%
\pgfusepath{stroke,fill}%
\end{pgfscope}%
\begin{pgfscope}%
\pgfpathrectangle{\pgfqpoint{0.750000in}{0.500000in}}{\pgfqpoint{4.650000in}{3.020000in}}%
\pgfusepath{clip}%
\pgfsetbuttcap%
\pgfsetroundjoin%
\definecolor{currentfill}{rgb}{0.121569,0.466667,0.705882}%
\pgfsetfillcolor{currentfill}%
\pgfsetlinewidth{1.003750pt}%
\definecolor{currentstroke}{rgb}{0.121569,0.466667,0.705882}%
\pgfsetstrokecolor{currentstroke}%
\pgfsetdash{}{0pt}%
\pgfpathmoveto{\pgfqpoint{1.108395in}{0.641087in}}%
\pgfpathcurveto{\pgfqpoint{1.111585in}{0.641087in}}{\pgfqpoint{1.114644in}{0.642354in}}{\pgfqpoint{1.116900in}{0.644609in}}%
\pgfpathcurveto{\pgfqpoint{1.119155in}{0.646865in}}{\pgfqpoint{1.120423in}{0.649925in}}{\pgfqpoint{1.120423in}{0.653115in}}%
\pgfpathcurveto{\pgfqpoint{1.120423in}{0.656305in}}{\pgfqpoint{1.119155in}{0.659364in}}{\pgfqpoint{1.116900in}{0.661620in}}%
\pgfpathcurveto{\pgfqpoint{1.114644in}{0.663875in}}{\pgfqpoint{1.111585in}{0.665143in}}{\pgfqpoint{1.108395in}{0.665143in}}%
\pgfpathcurveto{\pgfqpoint{1.105205in}{0.665143in}}{\pgfqpoint{1.102145in}{0.663875in}}{\pgfqpoint{1.099890in}{0.661620in}}%
\pgfpathcurveto{\pgfqpoint{1.097634in}{0.659364in}}{\pgfqpoint{1.096367in}{0.656305in}}{\pgfqpoint{1.096367in}{0.653115in}}%
\pgfpathcurveto{\pgfqpoint{1.096367in}{0.649925in}}{\pgfqpoint{1.097634in}{0.646865in}}{\pgfqpoint{1.099890in}{0.644609in}}%
\pgfpathcurveto{\pgfqpoint{1.102145in}{0.642354in}}{\pgfqpoint{1.105205in}{0.641087in}}{\pgfqpoint{1.108395in}{0.641087in}}%
\pgfpathclose%
\pgfusepath{stroke,fill}%
\end{pgfscope}%
\begin{pgfscope}%
\pgfpathrectangle{\pgfqpoint{0.750000in}{0.500000in}}{\pgfqpoint{4.650000in}{3.020000in}}%
\pgfusepath{clip}%
\pgfsetbuttcap%
\pgfsetroundjoin%
\definecolor{currentfill}{rgb}{0.121569,0.466667,0.705882}%
\pgfsetfillcolor{currentfill}%
\pgfsetlinewidth{1.003750pt}%
\definecolor{currentstroke}{rgb}{0.121569,0.466667,0.705882}%
\pgfsetstrokecolor{currentstroke}%
\pgfsetdash{}{0pt}%
\pgfpathmoveto{\pgfqpoint{1.064027in}{0.640413in}}%
\pgfpathcurveto{\pgfqpoint{1.066632in}{0.640413in}}{\pgfqpoint{1.069130in}{0.641448in}}{\pgfqpoint{1.070972in}{0.643290in}}%
\pgfpathcurveto{\pgfqpoint{1.072814in}{0.645132in}}{\pgfqpoint{1.073848in}{0.647630in}}{\pgfqpoint{1.073848in}{0.650234in}}%
\pgfpathcurveto{\pgfqpoint{1.073848in}{0.652839in}}{\pgfqpoint{1.072814in}{0.655337in}}{\pgfqpoint{1.070972in}{0.657179in}}%
\pgfpathcurveto{\pgfqpoint{1.069130in}{0.659020in}}{\pgfqpoint{1.066632in}{0.660055in}}{\pgfqpoint{1.064027in}{0.660055in}}%
\pgfpathcurveto{\pgfqpoint{1.061423in}{0.660055in}}{\pgfqpoint{1.058925in}{0.659020in}}{\pgfqpoint{1.057083in}{0.657179in}}%
\pgfpathcurveto{\pgfqpoint{1.055241in}{0.655337in}}{\pgfqpoint{1.054207in}{0.652839in}}{\pgfqpoint{1.054207in}{0.650234in}}%
\pgfpathcurveto{\pgfqpoint{1.054207in}{0.647630in}}{\pgfqpoint{1.055241in}{0.645132in}}{\pgfqpoint{1.057083in}{0.643290in}}%
\pgfpathcurveto{\pgfqpoint{1.058925in}{0.641448in}}{\pgfqpoint{1.061423in}{0.640413in}}{\pgfqpoint{1.064027in}{0.640413in}}%
\pgfpathclose%
\pgfusepath{stroke,fill}%
\end{pgfscope}%
\begin{pgfscope}%
\pgfpathrectangle{\pgfqpoint{0.750000in}{0.500000in}}{\pgfqpoint{4.650000in}{3.020000in}}%
\pgfusepath{clip}%
\pgfsetbuttcap%
\pgfsetroundjoin%
\definecolor{currentfill}{rgb}{0.121569,0.466667,0.705882}%
\pgfsetfillcolor{currentfill}%
\pgfsetlinewidth{1.003750pt}%
\definecolor{currentstroke}{rgb}{0.121569,0.466667,0.705882}%
\pgfsetstrokecolor{currentstroke}%
\pgfsetdash{}{0pt}%
\pgfpathmoveto{\pgfqpoint{1.816723in}{0.962212in}}%
\pgfpathcurveto{\pgfqpoint{1.823614in}{0.962212in}}{\pgfqpoint{1.830224in}{0.964950in}}{\pgfqpoint{1.835097in}{0.969822in}}%
\pgfpathcurveto{\pgfqpoint{1.839969in}{0.974695in}}{\pgfqpoint{1.842707in}{0.981305in}}{\pgfqpoint{1.842707in}{0.988195in}}%
\pgfpathcurveto{\pgfqpoint{1.842707in}{0.995086in}}{\pgfqpoint{1.839969in}{1.001696in}}{\pgfqpoint{1.835097in}{1.006569in}}%
\pgfpathcurveto{\pgfqpoint{1.830224in}{1.011441in}}{\pgfqpoint{1.823614in}{1.014179in}}{\pgfqpoint{1.816723in}{1.014179in}}%
\pgfpathcurveto{\pgfqpoint{1.809832in}{1.014179in}}{\pgfqpoint{1.803223in}{1.011441in}}{\pgfqpoint{1.798350in}{1.006569in}}%
\pgfpathcurveto{\pgfqpoint{1.793478in}{1.001696in}}{\pgfqpoint{1.790740in}{0.995086in}}{\pgfqpoint{1.790740in}{0.988195in}}%
\pgfpathcurveto{\pgfqpoint{1.790740in}{0.981305in}}{\pgfqpoint{1.793478in}{0.974695in}}{\pgfqpoint{1.798350in}{0.969822in}}%
\pgfpathcurveto{\pgfqpoint{1.803223in}{0.964950in}}{\pgfqpoint{1.809832in}{0.962212in}}{\pgfqpoint{1.816723in}{0.962212in}}%
\pgfpathclose%
\pgfusepath{stroke,fill}%
\end{pgfscope}%
\begin{pgfscope}%
\pgfpathrectangle{\pgfqpoint{0.750000in}{0.500000in}}{\pgfqpoint{4.650000in}{3.020000in}}%
\pgfusepath{clip}%
\pgfsetbuttcap%
\pgfsetroundjoin%
\definecolor{currentfill}{rgb}{0.121569,0.466667,0.705882}%
\pgfsetfillcolor{currentfill}%
\pgfsetlinewidth{1.003750pt}%
\definecolor{currentstroke}{rgb}{0.121569,0.466667,0.705882}%
\pgfsetstrokecolor{currentstroke}%
\pgfsetdash{}{0pt}%
\pgfpathmoveto{\pgfqpoint{1.562385in}{0.788502in}}%
\pgfpathcurveto{\pgfqpoint{1.568494in}{0.788502in}}{\pgfqpoint{1.574352in}{0.790929in}}{\pgfqpoint{1.578672in}{0.795248in}}%
\pgfpathcurveto{\pgfqpoint{1.582991in}{0.799567in}}{\pgfqpoint{1.585418in}{0.805426in}}{\pgfqpoint{1.585418in}{0.811534in}}%
\pgfpathcurveto{\pgfqpoint{1.585418in}{0.817642in}}{\pgfqpoint{1.582991in}{0.823501in}}{\pgfqpoint{1.578672in}{0.827820in}}%
\pgfpathcurveto{\pgfqpoint{1.574352in}{0.832139in}}{\pgfqpoint{1.568494in}{0.834566in}}{\pgfqpoint{1.562385in}{0.834566in}}%
\pgfpathcurveto{\pgfqpoint{1.556277in}{0.834566in}}{\pgfqpoint{1.550418in}{0.832139in}}{\pgfqpoint{1.546099in}{0.827820in}}%
\pgfpathcurveto{\pgfqpoint{1.541780in}{0.823501in}}{\pgfqpoint{1.539353in}{0.817642in}}{\pgfqpoint{1.539353in}{0.811534in}}%
\pgfpathcurveto{\pgfqpoint{1.539353in}{0.805426in}}{\pgfqpoint{1.541780in}{0.799567in}}{\pgfqpoint{1.546099in}{0.795248in}}%
\pgfpathcurveto{\pgfqpoint{1.550418in}{0.790929in}}{\pgfqpoint{1.556277in}{0.788502in}}{\pgfqpoint{1.562385in}{0.788502in}}%
\pgfpathclose%
\pgfusepath{stroke,fill}%
\end{pgfscope}%
\begin{pgfscope}%
\pgfpathrectangle{\pgfqpoint{0.750000in}{0.500000in}}{\pgfqpoint{4.650000in}{3.020000in}}%
\pgfusepath{clip}%
\pgfsetbuttcap%
\pgfsetroundjoin%
\definecolor{currentfill}{rgb}{0.121569,0.466667,0.705882}%
\pgfsetfillcolor{currentfill}%
\pgfsetlinewidth{1.003750pt}%
\definecolor{currentstroke}{rgb}{0.121569,0.466667,0.705882}%
\pgfsetstrokecolor{currentstroke}%
\pgfsetdash{}{0pt}%
\pgfpathmoveto{\pgfqpoint{1.261101in}{0.679387in}}%
\pgfpathcurveto{\pgfqpoint{1.265973in}{0.679387in}}{\pgfqpoint{1.270647in}{0.681323in}}{\pgfqpoint{1.274093in}{0.684768in}}%
\pgfpathcurveto{\pgfqpoint{1.277538in}{0.688214in}}{\pgfqpoint{1.279474in}{0.692887in}}{\pgfqpoint{1.279474in}{0.697760in}}%
\pgfpathcurveto{\pgfqpoint{1.279474in}{0.702633in}}{\pgfqpoint{1.277538in}{0.707306in}}{\pgfqpoint{1.274093in}{0.710752in}}%
\pgfpathcurveto{\pgfqpoint{1.270647in}{0.714197in}}{\pgfqpoint{1.265973in}{0.716133in}}{\pgfqpoint{1.261101in}{0.716133in}}%
\pgfpathcurveto{\pgfqpoint{1.256228in}{0.716133in}}{\pgfqpoint{1.251554in}{0.714197in}}{\pgfqpoint{1.248109in}{0.710752in}}%
\pgfpathcurveto{\pgfqpoint{1.244663in}{0.707306in}}{\pgfqpoint{1.242727in}{0.702633in}}{\pgfqpoint{1.242727in}{0.697760in}}%
\pgfpathcurveto{\pgfqpoint{1.242727in}{0.692887in}}{\pgfqpoint{1.244663in}{0.688214in}}{\pgfqpoint{1.248109in}{0.684768in}}%
\pgfpathcurveto{\pgfqpoint{1.251554in}{0.681323in}}{\pgfqpoint{1.256228in}{0.679387in}}{\pgfqpoint{1.261101in}{0.679387in}}%
\pgfpathclose%
\pgfusepath{stroke,fill}%
\end{pgfscope}%
\begin{pgfscope}%
\pgfpathrectangle{\pgfqpoint{0.750000in}{0.500000in}}{\pgfqpoint{4.650000in}{3.020000in}}%
\pgfusepath{clip}%
\pgfsetbuttcap%
\pgfsetroundjoin%
\definecolor{currentfill}{rgb}{0.121569,0.466667,0.705882}%
\pgfsetfillcolor{currentfill}%
\pgfsetlinewidth{1.003750pt}%
\definecolor{currentstroke}{rgb}{0.121569,0.466667,0.705882}%
\pgfsetstrokecolor{currentstroke}%
\pgfsetdash{}{0pt}%
\pgfpathmoveto{\pgfqpoint{1.018112in}{0.632252in}}%
\pgfpathcurveto{\pgfqpoint{1.020717in}{0.632252in}}{\pgfqpoint{1.023215in}{0.633287in}}{\pgfqpoint{1.025057in}{0.635129in}}%
\pgfpathcurveto{\pgfqpoint{1.026899in}{0.636971in}}{\pgfqpoint{1.027933in}{0.639469in}}{\pgfqpoint{1.027933in}{0.642073in}}%
\pgfpathcurveto{\pgfqpoint{1.027933in}{0.644678in}}{\pgfqpoint{1.026899in}{0.647176in}}{\pgfqpoint{1.025057in}{0.649018in}}%
\pgfpathcurveto{\pgfqpoint{1.023215in}{0.650859in}}{\pgfqpoint{1.020717in}{0.651894in}}{\pgfqpoint{1.018112in}{0.651894in}}%
\pgfpathcurveto{\pgfqpoint{1.015508in}{0.651894in}}{\pgfqpoint{1.013010in}{0.650859in}}{\pgfqpoint{1.011168in}{0.649018in}}%
\pgfpathcurveto{\pgfqpoint{1.009326in}{0.647176in}}{\pgfqpoint{1.008292in}{0.644678in}}{\pgfqpoint{1.008292in}{0.642073in}}%
\pgfpathcurveto{\pgfqpoint{1.008292in}{0.639469in}}{\pgfqpoint{1.009326in}{0.636971in}}{\pgfqpoint{1.011168in}{0.635129in}}%
\pgfpathcurveto{\pgfqpoint{1.013010in}{0.633287in}}{\pgfqpoint{1.015508in}{0.632252in}}{\pgfqpoint{1.018112in}{0.632252in}}%
\pgfpathclose%
\pgfusepath{stroke,fill}%
\end{pgfscope}%
\begin{pgfscope}%
\pgfpathrectangle{\pgfqpoint{0.750000in}{0.500000in}}{\pgfqpoint{4.650000in}{3.020000in}}%
\pgfusepath{clip}%
\pgfsetbuttcap%
\pgfsetroundjoin%
\definecolor{currentfill}{rgb}{0.121569,0.466667,0.705882}%
\pgfsetfillcolor{currentfill}%
\pgfsetlinewidth{1.003750pt}%
\definecolor{currentstroke}{rgb}{0.121569,0.466667,0.705882}%
\pgfsetstrokecolor{currentstroke}%
\pgfsetdash{}{0pt}%
\pgfpathmoveto{\pgfqpoint{1.078989in}{0.649534in}}%
\pgfpathcurveto{\pgfqpoint{1.081593in}{0.649534in}}{\pgfqpoint{1.084091in}{0.650569in}}{\pgfqpoint{1.085933in}{0.652411in}}%
\pgfpathcurveto{\pgfqpoint{1.087775in}{0.654253in}}{\pgfqpoint{1.088809in}{0.656751in}}{\pgfqpoint{1.088809in}{0.659355in}}%
\pgfpathcurveto{\pgfqpoint{1.088809in}{0.661960in}}{\pgfqpoint{1.087775in}{0.664458in}}{\pgfqpoint{1.085933in}{0.666300in}}%
\pgfpathcurveto{\pgfqpoint{1.084091in}{0.668142in}}{\pgfqpoint{1.081593in}{0.669176in}}{\pgfqpoint{1.078989in}{0.669176in}}%
\pgfpathcurveto{\pgfqpoint{1.076384in}{0.669176in}}{\pgfqpoint{1.073886in}{0.668142in}}{\pgfqpoint{1.072044in}{0.666300in}}%
\pgfpathcurveto{\pgfqpoint{1.070202in}{0.664458in}}{\pgfqpoint{1.069168in}{0.661960in}}{\pgfqpoint{1.069168in}{0.659355in}}%
\pgfpathcurveto{\pgfqpoint{1.069168in}{0.656751in}}{\pgfqpoint{1.070202in}{0.654253in}}{\pgfqpoint{1.072044in}{0.652411in}}%
\pgfpathcurveto{\pgfqpoint{1.073886in}{0.650569in}}{\pgfqpoint{1.076384in}{0.649534in}}{\pgfqpoint{1.078989in}{0.649534in}}%
\pgfpathclose%
\pgfusepath{stroke,fill}%
\end{pgfscope}%
\begin{pgfscope}%
\pgfpathrectangle{\pgfqpoint{0.750000in}{0.500000in}}{\pgfqpoint{4.650000in}{3.020000in}}%
\pgfusepath{clip}%
\pgfsetbuttcap%
\pgfsetroundjoin%
\definecolor{currentfill}{rgb}{0.121569,0.466667,0.705882}%
\pgfsetfillcolor{currentfill}%
\pgfsetlinewidth{1.003750pt}%
\definecolor{currentstroke}{rgb}{0.121569,0.466667,0.705882}%
\pgfsetstrokecolor{currentstroke}%
\pgfsetdash{}{0pt}%
\pgfpathmoveto{\pgfqpoint{1.643382in}{0.727357in}}%
\pgfpathcurveto{\pgfqpoint{1.647500in}{0.727357in}}{\pgfqpoint{1.651450in}{0.728994in}}{\pgfqpoint{1.654362in}{0.731905in}}%
\pgfpathcurveto{\pgfqpoint{1.657274in}{0.734817in}}{\pgfqpoint{1.658910in}{0.738767in}}{\pgfqpoint{1.658910in}{0.742886in}}%
\pgfpathcurveto{\pgfqpoint{1.658910in}{0.747004in}}{\pgfqpoint{1.657274in}{0.750954in}}{\pgfqpoint{1.654362in}{0.753866in}}%
\pgfpathcurveto{\pgfqpoint{1.651450in}{0.756778in}}{\pgfqpoint{1.647500in}{0.758414in}}{\pgfqpoint{1.643382in}{0.758414in}}%
\pgfpathcurveto{\pgfqpoint{1.639263in}{0.758414in}}{\pgfqpoint{1.635313in}{0.756778in}}{\pgfqpoint{1.632401in}{0.753866in}}%
\pgfpathcurveto{\pgfqpoint{1.629489in}{0.750954in}}{\pgfqpoint{1.627853in}{0.747004in}}{\pgfqpoint{1.627853in}{0.742886in}}%
\pgfpathcurveto{\pgfqpoint{1.627853in}{0.738767in}}{\pgfqpoint{1.629489in}{0.734817in}}{\pgfqpoint{1.632401in}{0.731905in}}%
\pgfpathcurveto{\pgfqpoint{1.635313in}{0.728994in}}{\pgfqpoint{1.639263in}{0.727357in}}{\pgfqpoint{1.643382in}{0.727357in}}%
\pgfpathclose%
\pgfusepath{stroke,fill}%
\end{pgfscope}%
\begin{pgfscope}%
\pgfpathrectangle{\pgfqpoint{0.750000in}{0.500000in}}{\pgfqpoint{4.650000in}{3.020000in}}%
\pgfusepath{clip}%
\pgfsetbuttcap%
\pgfsetroundjoin%
\definecolor{currentfill}{rgb}{0.121569,0.466667,0.705882}%
\pgfsetfillcolor{currentfill}%
\pgfsetlinewidth{1.003750pt}%
\definecolor{currentstroke}{rgb}{0.121569,0.466667,0.705882}%
\pgfsetstrokecolor{currentstroke}%
\pgfsetdash{}{0pt}%
\pgfpathmoveto{\pgfqpoint{1.038233in}{0.632252in}}%
\pgfpathcurveto{\pgfqpoint{1.040837in}{0.632252in}}{\pgfqpoint{1.043335in}{0.633287in}}{\pgfqpoint{1.045177in}{0.635129in}}%
\pgfpathcurveto{\pgfqpoint{1.047019in}{0.636971in}}{\pgfqpoint{1.048053in}{0.639469in}}{\pgfqpoint{1.048053in}{0.642073in}}%
\pgfpathcurveto{\pgfqpoint{1.048053in}{0.644678in}}{\pgfqpoint{1.047019in}{0.647176in}}{\pgfqpoint{1.045177in}{0.649018in}}%
\pgfpathcurveto{\pgfqpoint{1.043335in}{0.650859in}}{\pgfqpoint{1.040837in}{0.651894in}}{\pgfqpoint{1.038233in}{0.651894in}}%
\pgfpathcurveto{\pgfqpoint{1.035628in}{0.651894in}}{\pgfqpoint{1.033130in}{0.650859in}}{\pgfqpoint{1.031288in}{0.649018in}}%
\pgfpathcurveto{\pgfqpoint{1.029446in}{0.647176in}}{\pgfqpoint{1.028412in}{0.644678in}}{\pgfqpoint{1.028412in}{0.642073in}}%
\pgfpathcurveto{\pgfqpoint{1.028412in}{0.639469in}}{\pgfqpoint{1.029446in}{0.636971in}}{\pgfqpoint{1.031288in}{0.635129in}}%
\pgfpathcurveto{\pgfqpoint{1.033130in}{0.633287in}}{\pgfqpoint{1.035628in}{0.632252in}}{\pgfqpoint{1.038233in}{0.632252in}}%
\pgfpathclose%
\pgfusepath{stroke,fill}%
\end{pgfscope}%
\begin{pgfscope}%
\pgfpathrectangle{\pgfqpoint{0.750000in}{0.500000in}}{\pgfqpoint{4.650000in}{3.020000in}}%
\pgfusepath{clip}%
\pgfsetbuttcap%
\pgfsetroundjoin%
\definecolor{currentfill}{rgb}{0.121569,0.466667,0.705882}%
\pgfsetfillcolor{currentfill}%
\pgfsetlinewidth{1.003750pt}%
\definecolor{currentstroke}{rgb}{0.121569,0.466667,0.705882}%
\pgfsetstrokecolor{currentstroke}%
\pgfsetdash{}{0pt}%
\pgfpathmoveto{\pgfqpoint{1.588180in}{0.772713in}}%
\pgfpathcurveto{\pgfqpoint{1.597203in}{0.772713in}}{\pgfqpoint{1.605857in}{0.776297in}}{\pgfqpoint{1.612237in}{0.782677in}}%
\pgfpathcurveto{\pgfqpoint{1.618616in}{0.789057in}}{\pgfqpoint{1.622201in}{0.797711in}}{\pgfqpoint{1.622201in}{0.806733in}}%
\pgfpathcurveto{\pgfqpoint{1.622201in}{0.815756in}}{\pgfqpoint{1.618616in}{0.824410in}}{\pgfqpoint{1.612237in}{0.830790in}}%
\pgfpathcurveto{\pgfqpoint{1.605857in}{0.837169in}}{\pgfqpoint{1.597203in}{0.840754in}}{\pgfqpoint{1.588180in}{0.840754in}}%
\pgfpathcurveto{\pgfqpoint{1.579158in}{0.840754in}}{\pgfqpoint{1.570504in}{0.837169in}}{\pgfqpoint{1.564124in}{0.830790in}}%
\pgfpathcurveto{\pgfqpoint{1.557744in}{0.824410in}}{\pgfqpoint{1.554160in}{0.815756in}}{\pgfqpoint{1.554160in}{0.806733in}}%
\pgfpathcurveto{\pgfqpoint{1.554160in}{0.797711in}}{\pgfqpoint{1.557744in}{0.789057in}}{\pgfqpoint{1.564124in}{0.782677in}}%
\pgfpathcurveto{\pgfqpoint{1.570504in}{0.776297in}}{\pgfqpoint{1.579158in}{0.772713in}}{\pgfqpoint{1.588180in}{0.772713in}}%
\pgfpathclose%
\pgfusepath{stroke,fill}%
\end{pgfscope}%
\begin{pgfscope}%
\pgfpathrectangle{\pgfqpoint{0.750000in}{0.500000in}}{\pgfqpoint{4.650000in}{3.020000in}}%
\pgfusepath{clip}%
\pgfsetbuttcap%
\pgfsetroundjoin%
\definecolor{currentfill}{rgb}{0.121569,0.466667,0.705882}%
\pgfsetfillcolor{currentfill}%
\pgfsetlinewidth{1.003750pt}%
\definecolor{currentstroke}{rgb}{0.121569,0.466667,0.705882}%
\pgfsetstrokecolor{currentstroke}%
\pgfsetdash{}{0pt}%
\pgfpathmoveto{\pgfqpoint{1.662470in}{0.803207in}}%
\pgfpathcurveto{\pgfqpoint{1.666588in}{0.803207in}}{\pgfqpoint{1.670538in}{0.804843in}}{\pgfqpoint{1.673450in}{0.807755in}}%
\pgfpathcurveto{\pgfqpoint{1.676362in}{0.810667in}}{\pgfqpoint{1.677998in}{0.814617in}}{\pgfqpoint{1.677998in}{0.818735in}}%
\pgfpathcurveto{\pgfqpoint{1.677998in}{0.822853in}}{\pgfqpoint{1.676362in}{0.826803in}}{\pgfqpoint{1.673450in}{0.829715in}}%
\pgfpathcurveto{\pgfqpoint{1.670538in}{0.832627in}}{\pgfqpoint{1.666588in}{0.834263in}}{\pgfqpoint{1.662470in}{0.834263in}}%
\pgfpathcurveto{\pgfqpoint{1.658352in}{0.834263in}}{\pgfqpoint{1.654402in}{0.832627in}}{\pgfqpoint{1.651490in}{0.829715in}}%
\pgfpathcurveto{\pgfqpoint{1.648578in}{0.826803in}}{\pgfqpoint{1.646942in}{0.822853in}}{\pgfqpoint{1.646942in}{0.818735in}}%
\pgfpathcurveto{\pgfqpoint{1.646942in}{0.814617in}}{\pgfqpoint{1.648578in}{0.810667in}}{\pgfqpoint{1.651490in}{0.807755in}}%
\pgfpathcurveto{\pgfqpoint{1.654402in}{0.804843in}}{\pgfqpoint{1.658352in}{0.803207in}}{\pgfqpoint{1.662470in}{0.803207in}}%
\pgfpathclose%
\pgfusepath{stroke,fill}%
\end{pgfscope}%
\begin{pgfscope}%
\pgfpathrectangle{\pgfqpoint{0.750000in}{0.500000in}}{\pgfqpoint{4.650000in}{3.020000in}}%
\pgfusepath{clip}%
\pgfsetbuttcap%
\pgfsetroundjoin%
\definecolor{currentfill}{rgb}{0.121569,0.466667,0.705882}%
\pgfsetfillcolor{currentfill}%
\pgfsetlinewidth{1.003750pt}%
\definecolor{currentstroke}{rgb}{0.121569,0.466667,0.705882}%
\pgfsetstrokecolor{currentstroke}%
\pgfsetdash{}{0pt}%
\pgfpathmoveto{\pgfqpoint{1.155341in}{0.675710in}}%
\pgfpathcurveto{\pgfqpoint{1.159025in}{0.675710in}}{\pgfqpoint{1.162558in}{0.677174in}}{\pgfqpoint{1.165162in}{0.679778in}}%
\pgfpathcurveto{\pgfqpoint{1.167767in}{0.682383in}}{\pgfqpoint{1.169230in}{0.685916in}}{\pgfqpoint{1.169230in}{0.689599in}}%
\pgfpathcurveto{\pgfqpoint{1.169230in}{0.693282in}}{\pgfqpoint{1.167767in}{0.696815in}}{\pgfqpoint{1.165162in}{0.699420in}}%
\pgfpathcurveto{\pgfqpoint{1.162558in}{0.702025in}}{\pgfqpoint{1.159025in}{0.703488in}}{\pgfqpoint{1.155341in}{0.703488in}}%
\pgfpathcurveto{\pgfqpoint{1.151658in}{0.703488in}}{\pgfqpoint{1.148125in}{0.702025in}}{\pgfqpoint{1.145521in}{0.699420in}}%
\pgfpathcurveto{\pgfqpoint{1.142916in}{0.696815in}}{\pgfqpoint{1.141453in}{0.693282in}}{\pgfqpoint{1.141453in}{0.689599in}}%
\pgfpathcurveto{\pgfqpoint{1.141453in}{0.685916in}}{\pgfqpoint{1.142916in}{0.682383in}}{\pgfqpoint{1.145521in}{0.679778in}}%
\pgfpathcurveto{\pgfqpoint{1.148125in}{0.677174in}}{\pgfqpoint{1.151658in}{0.675710in}}{\pgfqpoint{1.155341in}{0.675710in}}%
\pgfpathclose%
\pgfusepath{stroke,fill}%
\end{pgfscope}%
\begin{pgfscope}%
\pgfpathrectangle{\pgfqpoint{0.750000in}{0.500000in}}{\pgfqpoint{4.650000in}{3.020000in}}%
\pgfusepath{clip}%
\pgfsetbuttcap%
\pgfsetroundjoin%
\definecolor{currentfill}{rgb}{0.121569,0.466667,0.705882}%
\pgfsetfillcolor{currentfill}%
\pgfsetlinewidth{1.003750pt}%
\definecolor{currentstroke}{rgb}{0.121569,0.466667,0.705882}%
\pgfsetstrokecolor{currentstroke}%
\pgfsetdash{}{0pt}%
\pgfpathmoveto{\pgfqpoint{1.636159in}{0.789723in}}%
\pgfpathcurveto{\pgfqpoint{1.640670in}{0.789723in}}{\pgfqpoint{1.644997in}{0.791515in}}{\pgfqpoint{1.648187in}{0.794705in}}%
\pgfpathcurveto{\pgfqpoint{1.651377in}{0.797895in}}{\pgfqpoint{1.653169in}{0.802222in}}{\pgfqpoint{1.653169in}{0.806733in}}%
\pgfpathcurveto{\pgfqpoint{1.653169in}{0.811245in}}{\pgfqpoint{1.651377in}{0.815572in}}{\pgfqpoint{1.648187in}{0.818762in}}%
\pgfpathcurveto{\pgfqpoint{1.644997in}{0.821951in}}{\pgfqpoint{1.640670in}{0.823744in}}{\pgfqpoint{1.636159in}{0.823744in}}%
\pgfpathcurveto{\pgfqpoint{1.631648in}{0.823744in}}{\pgfqpoint{1.627321in}{0.821951in}}{\pgfqpoint{1.624131in}{0.818762in}}%
\pgfpathcurveto{\pgfqpoint{1.620941in}{0.815572in}}{\pgfqpoint{1.619149in}{0.811245in}}{\pgfqpoint{1.619149in}{0.806733in}}%
\pgfpathcurveto{\pgfqpoint{1.619149in}{0.802222in}}{\pgfqpoint{1.620941in}{0.797895in}}{\pgfqpoint{1.624131in}{0.794705in}}%
\pgfpathcurveto{\pgfqpoint{1.627321in}{0.791515in}}{\pgfqpoint{1.631648in}{0.789723in}}{\pgfqpoint{1.636159in}{0.789723in}}%
\pgfpathclose%
\pgfusepath{stroke,fill}%
\end{pgfscope}%
\begin{pgfscope}%
\pgfpathrectangle{\pgfqpoint{0.750000in}{0.500000in}}{\pgfqpoint{4.650000in}{3.020000in}}%
\pgfusepath{clip}%
\pgfsetbuttcap%
\pgfsetroundjoin%
\definecolor{currentfill}{rgb}{0.121569,0.466667,0.705882}%
\pgfsetfillcolor{currentfill}%
\pgfsetlinewidth{1.003750pt}%
\definecolor{currentstroke}{rgb}{0.121569,0.466667,0.705882}%
\pgfsetstrokecolor{currentstroke}%
\pgfsetdash{}{0pt}%
\pgfpathmoveto{\pgfqpoint{1.342097in}{0.688508in}}%
\pgfpathcurveto{\pgfqpoint{1.346969in}{0.688508in}}{\pgfqpoint{1.351643in}{0.690444in}}{\pgfqpoint{1.355089in}{0.693889in}}%
\pgfpathcurveto{\pgfqpoint{1.358534in}{0.697335in}}{\pgfqpoint{1.360470in}{0.702009in}}{\pgfqpoint{1.360470in}{0.706881in}}%
\pgfpathcurveto{\pgfqpoint{1.360470in}{0.711754in}}{\pgfqpoint{1.358534in}{0.716428in}}{\pgfqpoint{1.355089in}{0.719873in}}%
\pgfpathcurveto{\pgfqpoint{1.351643in}{0.723319in}}{\pgfqpoint{1.346969in}{0.725254in}}{\pgfqpoint{1.342097in}{0.725254in}}%
\pgfpathcurveto{\pgfqpoint{1.337224in}{0.725254in}}{\pgfqpoint{1.332550in}{0.723319in}}{\pgfqpoint{1.329105in}{0.719873in}}%
\pgfpathcurveto{\pgfqpoint{1.325659in}{0.716428in}}{\pgfqpoint{1.323723in}{0.711754in}}{\pgfqpoint{1.323723in}{0.706881in}}%
\pgfpathcurveto{\pgfqpoint{1.323723in}{0.702009in}}{\pgfqpoint{1.325659in}{0.697335in}}{\pgfqpoint{1.329105in}{0.693889in}}%
\pgfpathcurveto{\pgfqpoint{1.332550in}{0.690444in}}{\pgfqpoint{1.337224in}{0.688508in}}{\pgfqpoint{1.342097in}{0.688508in}}%
\pgfpathclose%
\pgfusepath{stroke,fill}%
\end{pgfscope}%
\begin{pgfscope}%
\pgfpathrectangle{\pgfqpoint{0.750000in}{0.500000in}}{\pgfqpoint{4.650000in}{3.020000in}}%
\pgfusepath{clip}%
\pgfsetbuttcap%
\pgfsetroundjoin%
\definecolor{currentfill}{rgb}{0.121569,0.466667,0.705882}%
\pgfsetfillcolor{currentfill}%
\pgfsetlinewidth{1.003750pt}%
\definecolor{currentstroke}{rgb}{0.121569,0.466667,0.705882}%
\pgfsetstrokecolor{currentstroke}%
\pgfsetdash{}{0pt}%
\pgfpathmoveto{\pgfqpoint{0.984063in}{0.645214in}}%
\pgfpathcurveto{\pgfqpoint{0.986668in}{0.645214in}}{\pgfqpoint{0.989166in}{0.646249in}}{\pgfqpoint{0.991008in}{0.648090in}}%
\pgfpathcurveto{\pgfqpoint{0.992849in}{0.649932in}}{\pgfqpoint{0.993884in}{0.652430in}}{\pgfqpoint{0.993884in}{0.655035in}}%
\pgfpathcurveto{\pgfqpoint{0.993884in}{0.657639in}}{\pgfqpoint{0.992849in}{0.660138in}}{\pgfqpoint{0.991008in}{0.661979in}}%
\pgfpathcurveto{\pgfqpoint{0.989166in}{0.663821in}}{\pgfqpoint{0.986668in}{0.664856in}}{\pgfqpoint{0.984063in}{0.664856in}}%
\pgfpathcurveto{\pgfqpoint{0.981459in}{0.664856in}}{\pgfqpoint{0.978960in}{0.663821in}}{\pgfqpoint{0.977119in}{0.661979in}}%
\pgfpathcurveto{\pgfqpoint{0.975277in}{0.660138in}}{\pgfqpoint{0.974242in}{0.657639in}}{\pgfqpoint{0.974242in}{0.655035in}}%
\pgfpathcurveto{\pgfqpoint{0.974242in}{0.652430in}}{\pgfqpoint{0.975277in}{0.649932in}}{\pgfqpoint{0.977119in}{0.648090in}}%
\pgfpathcurveto{\pgfqpoint{0.978960in}{0.646249in}}{\pgfqpoint{0.981459in}{0.645214in}}{\pgfqpoint{0.984063in}{0.645214in}}%
\pgfpathclose%
\pgfusepath{stroke,fill}%
\end{pgfscope}%
\begin{pgfscope}%
\pgfpathrectangle{\pgfqpoint{0.750000in}{0.500000in}}{\pgfqpoint{4.650000in}{3.020000in}}%
\pgfusepath{clip}%
\pgfsetbuttcap%
\pgfsetroundjoin%
\definecolor{currentfill}{rgb}{0.121569,0.466667,0.705882}%
\pgfsetfillcolor{currentfill}%
\pgfsetlinewidth{1.003750pt}%
\definecolor{currentstroke}{rgb}{0.121569,0.466667,0.705882}%
\pgfsetstrokecolor{currentstroke}%
\pgfsetdash{}{0pt}%
\pgfpathmoveto{\pgfqpoint{2.180432in}{0.849912in}}%
\pgfpathcurveto{\pgfqpoint{2.183622in}{0.849912in}}{\pgfqpoint{2.186681in}{0.851179in}}{\pgfqpoint{2.188937in}{0.853435in}}%
\pgfpathcurveto{\pgfqpoint{2.191193in}{0.855691in}}{\pgfqpoint{2.192460in}{0.858750in}}{\pgfqpoint{2.192460in}{0.861940in}}%
\pgfpathcurveto{\pgfqpoint{2.192460in}{0.865130in}}{\pgfqpoint{2.191193in}{0.868190in}}{\pgfqpoint{2.188937in}{0.870445in}}%
\pgfpathcurveto{\pgfqpoint{2.186681in}{0.872701in}}{\pgfqpoint{2.183622in}{0.873968in}}{\pgfqpoint{2.180432in}{0.873968in}}%
\pgfpathcurveto{\pgfqpoint{2.177242in}{0.873968in}}{\pgfqpoint{2.174182in}{0.872701in}}{\pgfqpoint{2.171927in}{0.870445in}}%
\pgfpathcurveto{\pgfqpoint{2.169671in}{0.868190in}}{\pgfqpoint{2.168404in}{0.865130in}}{\pgfqpoint{2.168404in}{0.861940in}}%
\pgfpathcurveto{\pgfqpoint{2.168404in}{0.858750in}}{\pgfqpoint{2.169671in}{0.855691in}}{\pgfqpoint{2.171927in}{0.853435in}}%
\pgfpathcurveto{\pgfqpoint{2.174182in}{0.851179in}}{\pgfqpoint{2.177242in}{0.849912in}}{\pgfqpoint{2.180432in}{0.849912in}}%
\pgfpathclose%
\pgfusepath{stroke,fill}%
\end{pgfscope}%
\begin{pgfscope}%
\pgfpathrectangle{\pgfqpoint{0.750000in}{0.500000in}}{\pgfqpoint{4.650000in}{3.020000in}}%
\pgfusepath{clip}%
\pgfsetbuttcap%
\pgfsetroundjoin%
\definecolor{currentfill}{rgb}{0.121569,0.466667,0.705882}%
\pgfsetfillcolor{currentfill}%
\pgfsetlinewidth{1.003750pt}%
\definecolor{currentstroke}{rgb}{0.121569,0.466667,0.705882}%
\pgfsetstrokecolor{currentstroke}%
\pgfsetdash{}{0pt}%
\pgfpathmoveto{\pgfqpoint{1.659890in}{0.702874in}}%
\pgfpathcurveto{\pgfqpoint{1.664008in}{0.702874in}}{\pgfqpoint{1.667958in}{0.704511in}}{\pgfqpoint{1.670870in}{0.707422in}}%
\pgfpathcurveto{\pgfqpoint{1.673782in}{0.710334in}}{\pgfqpoint{1.675419in}{0.714284in}}{\pgfqpoint{1.675419in}{0.718403in}}%
\pgfpathcurveto{\pgfqpoint{1.675419in}{0.722521in}}{\pgfqpoint{1.673782in}{0.726471in}}{\pgfqpoint{1.670870in}{0.729383in}}%
\pgfpathcurveto{\pgfqpoint{1.667958in}{0.732295in}}{\pgfqpoint{1.664008in}{0.733931in}}{\pgfqpoint{1.659890in}{0.733931in}}%
\pgfpathcurveto{\pgfqpoint{1.655772in}{0.733931in}}{\pgfqpoint{1.651822in}{0.732295in}}{\pgfqpoint{1.648910in}{0.729383in}}%
\pgfpathcurveto{\pgfqpoint{1.645998in}{0.726471in}}{\pgfqpoint{1.644362in}{0.722521in}}{\pgfqpoint{1.644362in}{0.718403in}}%
\pgfpathcurveto{\pgfqpoint{1.644362in}{0.714284in}}{\pgfqpoint{1.645998in}{0.710334in}}{\pgfqpoint{1.648910in}{0.707422in}}%
\pgfpathcurveto{\pgfqpoint{1.651822in}{0.704511in}}{\pgfqpoint{1.655772in}{0.702874in}}{\pgfqpoint{1.659890in}{0.702874in}}%
\pgfpathclose%
\pgfusepath{stroke,fill}%
\end{pgfscope}%
\begin{pgfscope}%
\pgfpathrectangle{\pgfqpoint{0.750000in}{0.500000in}}{\pgfqpoint{4.650000in}{3.020000in}}%
\pgfusepath{clip}%
\pgfsetbuttcap%
\pgfsetroundjoin%
\definecolor{currentfill}{rgb}{0.121569,0.466667,0.705882}%
\pgfsetfillcolor{currentfill}%
\pgfsetlinewidth{1.003750pt}%
\definecolor{currentstroke}{rgb}{0.121569,0.466667,0.705882}%
\pgfsetstrokecolor{currentstroke}%
\pgfsetdash{}{0pt}%
\pgfpathmoveto{\pgfqpoint{1.557226in}{0.676471in}}%
\pgfpathcurveto{\pgfqpoint{1.561345in}{0.676471in}}{\pgfqpoint{1.565295in}{0.678107in}}{\pgfqpoint{1.568207in}{0.681019in}}%
\pgfpathcurveto{\pgfqpoint{1.571119in}{0.683931in}}{\pgfqpoint{1.572755in}{0.687881in}}{\pgfqpoint{1.572755in}{0.691999in}}%
\pgfpathcurveto{\pgfqpoint{1.572755in}{0.696118in}}{\pgfqpoint{1.571119in}{0.700068in}}{\pgfqpoint{1.568207in}{0.702980in}}%
\pgfpathcurveto{\pgfqpoint{1.565295in}{0.705891in}}{\pgfqpoint{1.561345in}{0.707528in}}{\pgfqpoint{1.557226in}{0.707528in}}%
\pgfpathcurveto{\pgfqpoint{1.553108in}{0.707528in}}{\pgfqpoint{1.549158in}{0.705891in}}{\pgfqpoint{1.546246in}{0.702980in}}%
\pgfpathcurveto{\pgfqpoint{1.543334in}{0.700068in}}{\pgfqpoint{1.541698in}{0.696118in}}{\pgfqpoint{1.541698in}{0.691999in}}%
\pgfpathcurveto{\pgfqpoint{1.541698in}{0.687881in}}{\pgfqpoint{1.543334in}{0.683931in}}{\pgfqpoint{1.546246in}{0.681019in}}%
\pgfpathcurveto{\pgfqpoint{1.549158in}{0.678107in}}{\pgfqpoint{1.553108in}{0.676471in}}{\pgfqpoint{1.557226in}{0.676471in}}%
\pgfpathclose%
\pgfusepath{stroke,fill}%
\end{pgfscope}%
\begin{pgfscope}%
\pgfpathrectangle{\pgfqpoint{0.750000in}{0.500000in}}{\pgfqpoint{4.650000in}{3.020000in}}%
\pgfusepath{clip}%
\pgfsetbuttcap%
\pgfsetroundjoin%
\definecolor{currentfill}{rgb}{0.121569,0.466667,0.705882}%
\pgfsetfillcolor{currentfill}%
\pgfsetlinewidth{1.003750pt}%
\definecolor{currentstroke}{rgb}{0.121569,0.466667,0.705882}%
\pgfsetstrokecolor{currentstroke}%
\pgfsetdash{}{0pt}%
\pgfpathmoveto{\pgfqpoint{1.265744in}{0.665856in}}%
\pgfpathcurveto{\pgfqpoint{1.268348in}{0.665856in}}{\pgfqpoint{1.270847in}{0.666891in}}{\pgfqpoint{1.272688in}{0.668733in}}%
\pgfpathcurveto{\pgfqpoint{1.274530in}{0.670575in}}{\pgfqpoint{1.275565in}{0.673073in}}{\pgfqpoint{1.275565in}{0.675677in}}%
\pgfpathcurveto{\pgfqpoint{1.275565in}{0.678282in}}{\pgfqpoint{1.274530in}{0.680780in}}{\pgfqpoint{1.272688in}{0.682622in}}%
\pgfpathcurveto{\pgfqpoint{1.270847in}{0.684464in}}{\pgfqpoint{1.268348in}{0.685498in}}{\pgfqpoint{1.265744in}{0.685498in}}%
\pgfpathcurveto{\pgfqpoint{1.263139in}{0.685498in}}{\pgfqpoint{1.260641in}{0.684464in}}{\pgfqpoint{1.258799in}{0.682622in}}%
\pgfpathcurveto{\pgfqpoint{1.256958in}{0.680780in}}{\pgfqpoint{1.255923in}{0.678282in}}{\pgfqpoint{1.255923in}{0.675677in}}%
\pgfpathcurveto{\pgfqpoint{1.255923in}{0.673073in}}{\pgfqpoint{1.256958in}{0.670575in}}{\pgfqpoint{1.258799in}{0.668733in}}%
\pgfpathcurveto{\pgfqpoint{1.260641in}{0.666891in}}{\pgfqpoint{1.263139in}{0.665856in}}{\pgfqpoint{1.265744in}{0.665856in}}%
\pgfpathclose%
\pgfusepath{stroke,fill}%
\end{pgfscope}%
\begin{pgfscope}%
\pgfpathrectangle{\pgfqpoint{0.750000in}{0.500000in}}{\pgfqpoint{4.650000in}{3.020000in}}%
\pgfusepath{clip}%
\pgfsetbuttcap%
\pgfsetroundjoin%
\definecolor{currentfill}{rgb}{0.121569,0.466667,0.705882}%
\pgfsetfillcolor{currentfill}%
\pgfsetlinewidth{1.003750pt}%
\definecolor{currentstroke}{rgb}{0.121569,0.466667,0.705882}%
\pgfsetstrokecolor{currentstroke}%
\pgfsetdash{}{0pt}%
\pgfpathmoveto{\pgfqpoint{2.034949in}{0.763501in}}%
\pgfpathcurveto{\pgfqpoint{2.038138in}{0.763501in}}{\pgfqpoint{2.041198in}{0.764769in}}{\pgfqpoint{2.043454in}{0.767024in}}%
\pgfpathcurveto{\pgfqpoint{2.045709in}{0.769280in}}{\pgfqpoint{2.046977in}{0.772340in}}{\pgfqpoint{2.046977in}{0.775530in}}%
\pgfpathcurveto{\pgfqpoint{2.046977in}{0.778719in}}{\pgfqpoint{2.045709in}{0.781779in}}{\pgfqpoint{2.043454in}{0.784035in}}%
\pgfpathcurveto{\pgfqpoint{2.041198in}{0.786290in}}{\pgfqpoint{2.038138in}{0.787558in}}{\pgfqpoint{2.034949in}{0.787558in}}%
\pgfpathcurveto{\pgfqpoint{2.031759in}{0.787558in}}{\pgfqpoint{2.028699in}{0.786290in}}{\pgfqpoint{2.026443in}{0.784035in}}%
\pgfpathcurveto{\pgfqpoint{2.024188in}{0.781779in}}{\pgfqpoint{2.022920in}{0.778719in}}{\pgfqpoint{2.022920in}{0.775530in}}%
\pgfpathcurveto{\pgfqpoint{2.022920in}{0.772340in}}{\pgfqpoint{2.024188in}{0.769280in}}{\pgfqpoint{2.026443in}{0.767024in}}%
\pgfpathcurveto{\pgfqpoint{2.028699in}{0.764769in}}{\pgfqpoint{2.031759in}{0.763501in}}{\pgfqpoint{2.034949in}{0.763501in}}%
\pgfpathclose%
\pgfusepath{stroke,fill}%
\end{pgfscope}%
\begin{pgfscope}%
\pgfpathrectangle{\pgfqpoint{0.750000in}{0.500000in}}{\pgfqpoint{4.650000in}{3.020000in}}%
\pgfusepath{clip}%
\pgfsetbuttcap%
\pgfsetroundjoin%
\definecolor{currentfill}{rgb}{0.121569,0.466667,0.705882}%
\pgfsetfillcolor{currentfill}%
\pgfsetlinewidth{1.003750pt}%
\definecolor{currentstroke}{rgb}{0.121569,0.466667,0.705882}%
\pgfsetstrokecolor{currentstroke}%
\pgfsetdash{}{0pt}%
\pgfpathmoveto{\pgfqpoint{1.170818in}{0.633213in}}%
\pgfpathcurveto{\pgfqpoint{1.173423in}{0.633213in}}{\pgfqpoint{1.175921in}{0.634247in}}{\pgfqpoint{1.177763in}{0.636089in}}%
\pgfpathcurveto{\pgfqpoint{1.179605in}{0.637931in}}{\pgfqpoint{1.180639in}{0.640429in}}{\pgfqpoint{1.180639in}{0.643033in}}%
\pgfpathcurveto{\pgfqpoint{1.180639in}{0.645638in}}{\pgfqpoint{1.179605in}{0.648136in}}{\pgfqpoint{1.177763in}{0.649978in}}%
\pgfpathcurveto{\pgfqpoint{1.175921in}{0.651820in}}{\pgfqpoint{1.173423in}{0.652854in}}{\pgfqpoint{1.170818in}{0.652854in}}%
\pgfpathcurveto{\pgfqpoint{1.168214in}{0.652854in}}{\pgfqpoint{1.165716in}{0.651820in}}{\pgfqpoint{1.163874in}{0.649978in}}%
\pgfpathcurveto{\pgfqpoint{1.162032in}{0.648136in}}{\pgfqpoint{1.160998in}{0.645638in}}{\pgfqpoint{1.160998in}{0.643033in}}%
\pgfpathcurveto{\pgfqpoint{1.160998in}{0.640429in}}{\pgfqpoint{1.162032in}{0.637931in}}{\pgfqpoint{1.163874in}{0.636089in}}%
\pgfpathcurveto{\pgfqpoint{1.165716in}{0.634247in}}{\pgfqpoint{1.168214in}{0.633213in}}{\pgfqpoint{1.170818in}{0.633213in}}%
\pgfpathclose%
\pgfusepath{stroke,fill}%
\end{pgfscope}%
\begin{pgfscope}%
\pgfpathrectangle{\pgfqpoint{0.750000in}{0.500000in}}{\pgfqpoint{4.650000in}{3.020000in}}%
\pgfusepath{clip}%
\pgfsetbuttcap%
\pgfsetroundjoin%
\definecolor{currentfill}{rgb}{0.121569,0.466667,0.705882}%
\pgfsetfillcolor{currentfill}%
\pgfsetlinewidth{1.003750pt}%
\definecolor{currentstroke}{rgb}{0.121569,0.466667,0.705882}%
\pgfsetstrokecolor{currentstroke}%
\pgfsetdash{}{0pt}%
\pgfpathmoveto{\pgfqpoint{1.191454in}{0.651168in}}%
\pgfpathcurveto{\pgfqpoint{1.194644in}{0.651168in}}{\pgfqpoint{1.197704in}{0.652435in}}{\pgfqpoint{1.199960in}{0.654691in}}%
\pgfpathcurveto{\pgfqpoint{1.202215in}{0.656946in}}{\pgfqpoint{1.203483in}{0.660006in}}{\pgfqpoint{1.203483in}{0.663196in}}%
\pgfpathcurveto{\pgfqpoint{1.203483in}{0.666386in}}{\pgfqpoint{1.202215in}{0.669445in}}{\pgfqpoint{1.199960in}{0.671701in}}%
\pgfpathcurveto{\pgfqpoint{1.197704in}{0.673957in}}{\pgfqpoint{1.194644in}{0.675224in}}{\pgfqpoint{1.191454in}{0.675224in}}%
\pgfpathcurveto{\pgfqpoint{1.188264in}{0.675224in}}{\pgfqpoint{1.185205in}{0.673957in}}{\pgfqpoint{1.182949in}{0.671701in}}%
\pgfpathcurveto{\pgfqpoint{1.180694in}{0.669445in}}{\pgfqpoint{1.179426in}{0.666386in}}{\pgfqpoint{1.179426in}{0.663196in}}%
\pgfpathcurveto{\pgfqpoint{1.179426in}{0.660006in}}{\pgfqpoint{1.180694in}{0.656946in}}{\pgfqpoint{1.182949in}{0.654691in}}%
\pgfpathcurveto{\pgfqpoint{1.185205in}{0.652435in}}{\pgfqpoint{1.188264in}{0.651168in}}{\pgfqpoint{1.191454in}{0.651168in}}%
\pgfpathclose%
\pgfusepath{stroke,fill}%
\end{pgfscope}%
\begin{pgfscope}%
\pgfpathrectangle{\pgfqpoint{0.750000in}{0.500000in}}{\pgfqpoint{4.650000in}{3.020000in}}%
\pgfusepath{clip}%
\pgfsetbuttcap%
\pgfsetroundjoin%
\definecolor{currentfill}{rgb}{0.121569,0.466667,0.705882}%
\pgfsetfillcolor{currentfill}%
\pgfsetlinewidth{1.003750pt}%
\definecolor{currentstroke}{rgb}{0.121569,0.466667,0.705882}%
\pgfsetstrokecolor{currentstroke}%
\pgfsetdash{}{0pt}%
\pgfpathmoveto{\pgfqpoint{1.654215in}{0.742379in}}%
\pgfpathcurveto{\pgfqpoint{1.657405in}{0.742379in}}{\pgfqpoint{1.660465in}{0.743646in}}{\pgfqpoint{1.662721in}{0.745902in}}%
\pgfpathcurveto{\pgfqpoint{1.664976in}{0.748157in}}{\pgfqpoint{1.666244in}{0.751217in}}{\pgfqpoint{1.666244in}{0.754407in}}%
\pgfpathcurveto{\pgfqpoint{1.666244in}{0.757597in}}{\pgfqpoint{1.664976in}{0.760657in}}{\pgfqpoint{1.662721in}{0.762912in}}%
\pgfpathcurveto{\pgfqpoint{1.660465in}{0.765168in}}{\pgfqpoint{1.657405in}{0.766435in}}{\pgfqpoint{1.654215in}{0.766435in}}%
\pgfpathcurveto{\pgfqpoint{1.651025in}{0.766435in}}{\pgfqpoint{1.647966in}{0.765168in}}{\pgfqpoint{1.645710in}{0.762912in}}%
\pgfpathcurveto{\pgfqpoint{1.643455in}{0.760657in}}{\pgfqpoint{1.642187in}{0.757597in}}{\pgfqpoint{1.642187in}{0.754407in}}%
\pgfpathcurveto{\pgfqpoint{1.642187in}{0.751217in}}{\pgfqpoint{1.643455in}{0.748157in}}{\pgfqpoint{1.645710in}{0.745902in}}%
\pgfpathcurveto{\pgfqpoint{1.647966in}{0.743646in}}{\pgfqpoint{1.651025in}{0.742379in}}{\pgfqpoint{1.654215in}{0.742379in}}%
\pgfpathclose%
\pgfusepath{stroke,fill}%
\end{pgfscope}%
\begin{pgfscope}%
\pgfpathrectangle{\pgfqpoint{0.750000in}{0.500000in}}{\pgfqpoint{4.650000in}{3.020000in}}%
\pgfusepath{clip}%
\pgfsetbuttcap%
\pgfsetroundjoin%
\definecolor{currentfill}{rgb}{0.121569,0.466667,0.705882}%
\pgfsetfillcolor{currentfill}%
\pgfsetlinewidth{1.003750pt}%
\definecolor{currentstroke}{rgb}{0.121569,0.466667,0.705882}%
\pgfsetstrokecolor{currentstroke}%
\pgfsetdash{}{0pt}%
\pgfpathmoveto{\pgfqpoint{1.988518in}{0.874462in}}%
\pgfpathcurveto{\pgfqpoint{1.993727in}{0.874462in}}{\pgfqpoint{1.998723in}{0.876532in}}{\pgfqpoint{2.002407in}{0.880215in}}%
\pgfpathcurveto{\pgfqpoint{2.006090in}{0.883899in}}{\pgfqpoint{2.008160in}{0.888895in}}{\pgfqpoint{2.008160in}{0.894104in}}%
\pgfpathcurveto{\pgfqpoint{2.008160in}{0.899313in}}{\pgfqpoint{2.006090in}{0.904310in}}{\pgfqpoint{2.002407in}{0.907993in}}%
\pgfpathcurveto{\pgfqpoint{1.998723in}{0.911676in}}{\pgfqpoint{1.993727in}{0.913746in}}{\pgfqpoint{1.988518in}{0.913746in}}%
\pgfpathcurveto{\pgfqpoint{1.983309in}{0.913746in}}{\pgfqpoint{1.978312in}{0.911676in}}{\pgfqpoint{1.974629in}{0.907993in}}%
\pgfpathcurveto{\pgfqpoint{1.970945in}{0.904310in}}{\pgfqpoint{1.968876in}{0.899313in}}{\pgfqpoint{1.968876in}{0.894104in}}%
\pgfpathcurveto{\pgfqpoint{1.968876in}{0.888895in}}{\pgfqpoint{1.970945in}{0.883899in}}{\pgfqpoint{1.974629in}{0.880215in}}%
\pgfpathcurveto{\pgfqpoint{1.978312in}{0.876532in}}{\pgfqpoint{1.983309in}{0.874462in}}{\pgfqpoint{1.988518in}{0.874462in}}%
\pgfpathclose%
\pgfusepath{stroke,fill}%
\end{pgfscope}%
\begin{pgfscope}%
\pgfpathrectangle{\pgfqpoint{0.750000in}{0.500000in}}{\pgfqpoint{4.650000in}{3.020000in}}%
\pgfusepath{clip}%
\pgfsetbuttcap%
\pgfsetroundjoin%
\definecolor{currentfill}{rgb}{0.121569,0.466667,0.705882}%
\pgfsetfillcolor{currentfill}%
\pgfsetlinewidth{1.003750pt}%
\definecolor{currentstroke}{rgb}{0.121569,0.466667,0.705882}%
\pgfsetstrokecolor{currentstroke}%
\pgfsetdash{}{0pt}%
\pgfpathmoveto{\pgfqpoint{2.169082in}{0.916020in}}%
\pgfpathcurveto{\pgfqpoint{2.173200in}{0.916020in}}{\pgfqpoint{2.177150in}{0.917656in}}{\pgfqpoint{2.180062in}{0.920568in}}%
\pgfpathcurveto{\pgfqpoint{2.182974in}{0.923480in}}{\pgfqpoint{2.184610in}{0.927430in}}{\pgfqpoint{2.184610in}{0.931549in}}%
\pgfpathcurveto{\pgfqpoint{2.184610in}{0.935667in}}{\pgfqpoint{2.182974in}{0.939617in}}{\pgfqpoint{2.180062in}{0.942529in}}%
\pgfpathcurveto{\pgfqpoint{2.177150in}{0.945441in}}{\pgfqpoint{2.173200in}{0.947077in}}{\pgfqpoint{2.169082in}{0.947077in}}%
\pgfpathcurveto{\pgfqpoint{2.164964in}{0.947077in}}{\pgfqpoint{2.161014in}{0.945441in}}{\pgfqpoint{2.158102in}{0.942529in}}%
\pgfpathcurveto{\pgfqpoint{2.155190in}{0.939617in}}{\pgfqpoint{2.153554in}{0.935667in}}{\pgfqpoint{2.153554in}{0.931549in}}%
\pgfpathcurveto{\pgfqpoint{2.153554in}{0.927430in}}{\pgfqpoint{2.155190in}{0.923480in}}{\pgfqpoint{2.158102in}{0.920568in}}%
\pgfpathcurveto{\pgfqpoint{2.161014in}{0.917656in}}{\pgfqpoint{2.164964in}{0.916020in}}{\pgfqpoint{2.169082in}{0.916020in}}%
\pgfpathclose%
\pgfusepath{stroke,fill}%
\end{pgfscope}%
\begin{pgfscope}%
\pgfpathrectangle{\pgfqpoint{0.750000in}{0.500000in}}{\pgfqpoint{4.650000in}{3.020000in}}%
\pgfusepath{clip}%
\pgfsetbuttcap%
\pgfsetroundjoin%
\definecolor{currentfill}{rgb}{0.121569,0.466667,0.705882}%
\pgfsetfillcolor{currentfill}%
\pgfsetlinewidth{1.003750pt}%
\definecolor{currentstroke}{rgb}{0.121569,0.466667,0.705882}%
\pgfsetstrokecolor{currentstroke}%
\pgfsetdash{}{0pt}%
\pgfpathmoveto{\pgfqpoint{1.513891in}{0.734996in}}%
\pgfpathcurveto{\pgfqpoint{1.518402in}{0.734996in}}{\pgfqpoint{1.522729in}{0.736789in}}{\pgfqpoint{1.525919in}{0.739979in}}%
\pgfpathcurveto{\pgfqpoint{1.529109in}{0.743168in}}{\pgfqpoint{1.530901in}{0.747496in}}{\pgfqpoint{1.530901in}{0.752007in}}%
\pgfpathcurveto{\pgfqpoint{1.530901in}{0.756518in}}{\pgfqpoint{1.529109in}{0.760845in}}{\pgfqpoint{1.525919in}{0.764035in}}%
\pgfpathcurveto{\pgfqpoint{1.522729in}{0.767225in}}{\pgfqpoint{1.518402in}{0.769017in}}{\pgfqpoint{1.513891in}{0.769017in}}%
\pgfpathcurveto{\pgfqpoint{1.509380in}{0.769017in}}{\pgfqpoint{1.505053in}{0.767225in}}{\pgfqpoint{1.501863in}{0.764035in}}%
\pgfpathcurveto{\pgfqpoint{1.498673in}{0.760845in}}{\pgfqpoint{1.496881in}{0.756518in}}{\pgfqpoint{1.496881in}{0.752007in}}%
\pgfpathcurveto{\pgfqpoint{1.496881in}{0.747496in}}{\pgfqpoint{1.498673in}{0.743168in}}{\pgfqpoint{1.501863in}{0.739979in}}%
\pgfpathcurveto{\pgfqpoint{1.505053in}{0.736789in}}{\pgfqpoint{1.509380in}{0.734996in}}{\pgfqpoint{1.513891in}{0.734996in}}%
\pgfpathclose%
\pgfusepath{stroke,fill}%
\end{pgfscope}%
\begin{pgfscope}%
\pgfpathrectangle{\pgfqpoint{0.750000in}{0.500000in}}{\pgfqpoint{4.650000in}{3.020000in}}%
\pgfusepath{clip}%
\pgfsetbuttcap%
\pgfsetroundjoin%
\definecolor{currentfill}{rgb}{0.121569,0.466667,0.705882}%
\pgfsetfillcolor{currentfill}%
\pgfsetlinewidth{1.003750pt}%
\definecolor{currentstroke}{rgb}{0.121569,0.466667,0.705882}%
\pgfsetstrokecolor{currentstroke}%
\pgfsetdash{}{0pt}%
\pgfpathmoveto{\pgfqpoint{1.118713in}{0.638505in}}%
\pgfpathcurveto{\pgfqpoint{1.123224in}{0.638505in}}{\pgfqpoint{1.127551in}{0.640297in}}{\pgfqpoint{1.130741in}{0.643487in}}%
\pgfpathcurveto{\pgfqpoint{1.133931in}{0.646677in}}{\pgfqpoint{1.135723in}{0.651004in}}{\pgfqpoint{1.135723in}{0.655515in}}%
\pgfpathcurveto{\pgfqpoint{1.135723in}{0.660026in}}{\pgfqpoint{1.133931in}{0.664353in}}{\pgfqpoint{1.130741in}{0.667543in}}%
\pgfpathcurveto{\pgfqpoint{1.127551in}{0.670733in}}{\pgfqpoint{1.123224in}{0.672525in}}{\pgfqpoint{1.118713in}{0.672525in}}%
\pgfpathcurveto{\pgfqpoint{1.114201in}{0.672525in}}{\pgfqpoint{1.109874in}{0.670733in}}{\pgfqpoint{1.106685in}{0.667543in}}%
\pgfpathcurveto{\pgfqpoint{1.103495in}{0.664353in}}{\pgfqpoint{1.101702in}{0.660026in}}{\pgfqpoint{1.101702in}{0.655515in}}%
\pgfpathcurveto{\pgfqpoint{1.101702in}{0.651004in}}{\pgfqpoint{1.103495in}{0.646677in}}{\pgfqpoint{1.106685in}{0.643487in}}%
\pgfpathcurveto{\pgfqpoint{1.109874in}{0.640297in}}{\pgfqpoint{1.114201in}{0.638505in}}{\pgfqpoint{1.118713in}{0.638505in}}%
\pgfpathclose%
\pgfusepath{stroke,fill}%
\end{pgfscope}%
\begin{pgfscope}%
\pgfpathrectangle{\pgfqpoint{0.750000in}{0.500000in}}{\pgfqpoint{4.650000in}{3.020000in}}%
\pgfusepath{clip}%
\pgfsetbuttcap%
\pgfsetroundjoin%
\definecolor{currentfill}{rgb}{0.121569,0.466667,0.705882}%
\pgfsetfillcolor{currentfill}%
\pgfsetlinewidth{1.003750pt}%
\definecolor{currentstroke}{rgb}{0.121569,0.466667,0.705882}%
\pgfsetstrokecolor{currentstroke}%
\pgfsetdash{}{0pt}%
\pgfpathmoveto{\pgfqpoint{0.970134in}{0.627932in}}%
\pgfpathcurveto{\pgfqpoint{0.972738in}{0.627932in}}{\pgfqpoint{0.975237in}{0.628967in}}{\pgfqpoint{0.977078in}{0.630808in}}%
\pgfpathcurveto{\pgfqpoint{0.978920in}{0.632650in}}{\pgfqpoint{0.979955in}{0.635148in}}{\pgfqpoint{0.979955in}{0.637753in}}%
\pgfpathcurveto{\pgfqpoint{0.979955in}{0.640357in}}{\pgfqpoint{0.978920in}{0.642856in}}{\pgfqpoint{0.977078in}{0.644697in}}%
\pgfpathcurveto{\pgfqpoint{0.975237in}{0.646539in}}{\pgfqpoint{0.972738in}{0.647574in}}{\pgfqpoint{0.970134in}{0.647574in}}%
\pgfpathcurveto{\pgfqpoint{0.967529in}{0.647574in}}{\pgfqpoint{0.965031in}{0.646539in}}{\pgfqpoint{0.963189in}{0.644697in}}%
\pgfpathcurveto{\pgfqpoint{0.961348in}{0.642856in}}{\pgfqpoint{0.960313in}{0.640357in}}{\pgfqpoint{0.960313in}{0.637753in}}%
\pgfpathcurveto{\pgfqpoint{0.960313in}{0.635148in}}{\pgfqpoint{0.961348in}{0.632650in}}{\pgfqpoint{0.963189in}{0.630808in}}%
\pgfpathcurveto{\pgfqpoint{0.965031in}{0.628967in}}{\pgfqpoint{0.967529in}{0.627932in}}{\pgfqpoint{0.970134in}{0.627932in}}%
\pgfpathclose%
\pgfusepath{stroke,fill}%
\end{pgfscope}%
\begin{pgfscope}%
\pgfpathrectangle{\pgfqpoint{0.750000in}{0.500000in}}{\pgfqpoint{4.650000in}{3.020000in}}%
\pgfusepath{clip}%
\pgfsetbuttcap%
\pgfsetroundjoin%
\definecolor{currentfill}{rgb}{0.121569,0.466667,0.705882}%
\pgfsetfillcolor{currentfill}%
\pgfsetlinewidth{1.003750pt}%
\definecolor{currentstroke}{rgb}{0.121569,0.466667,0.705882}%
\pgfsetstrokecolor{currentstroke}%
\pgfsetdash{}{0pt}%
\pgfpathmoveto{\pgfqpoint{2.467787in}{0.890304in}}%
\pgfpathcurveto{\pgfqpoint{2.472996in}{0.890304in}}{\pgfqpoint{2.477993in}{0.892374in}}{\pgfqpoint{2.481676in}{0.896057in}}%
\pgfpathcurveto{\pgfqpoint{2.485360in}{0.899740in}}{\pgfqpoint{2.487429in}{0.904737in}}{\pgfqpoint{2.487429in}{0.909946in}}%
\pgfpathcurveto{\pgfqpoint{2.487429in}{0.915155in}}{\pgfqpoint{2.485360in}{0.920151in}}{\pgfqpoint{2.481676in}{0.923835in}}%
\pgfpathcurveto{\pgfqpoint{2.477993in}{0.927518in}}{\pgfqpoint{2.472996in}{0.929588in}}{\pgfqpoint{2.467787in}{0.929588in}}%
\pgfpathcurveto{\pgfqpoint{2.462578in}{0.929588in}}{\pgfqpoint{2.457582in}{0.927518in}}{\pgfqpoint{2.453899in}{0.923835in}}%
\pgfpathcurveto{\pgfqpoint{2.450215in}{0.920151in}}{\pgfqpoint{2.448146in}{0.915155in}}{\pgfqpoint{2.448146in}{0.909946in}}%
\pgfpathcurveto{\pgfqpoint{2.448146in}{0.904737in}}{\pgfqpoint{2.450215in}{0.899740in}}{\pgfqpoint{2.453899in}{0.896057in}}%
\pgfpathcurveto{\pgfqpoint{2.457582in}{0.892374in}}{\pgfqpoint{2.462578in}{0.890304in}}{\pgfqpoint{2.467787in}{0.890304in}}%
\pgfpathclose%
\pgfusepath{stroke,fill}%
\end{pgfscope}%
\begin{pgfscope}%
\pgfpathrectangle{\pgfqpoint{0.750000in}{0.500000in}}{\pgfqpoint{4.650000in}{3.020000in}}%
\pgfusepath{clip}%
\pgfsetbuttcap%
\pgfsetroundjoin%
\definecolor{currentfill}{rgb}{0.121569,0.466667,0.705882}%
\pgfsetfillcolor{currentfill}%
\pgfsetlinewidth{1.003750pt}%
\definecolor{currentstroke}{rgb}{0.121569,0.466667,0.705882}%
\pgfsetstrokecolor{currentstroke}%
\pgfsetdash{}{0pt}%
\pgfpathmoveto{\pgfqpoint{1.008310in}{0.634653in}}%
\pgfpathcurveto{\pgfqpoint{1.010915in}{0.634653in}}{\pgfqpoint{1.013413in}{0.635687in}}{\pgfqpoint{1.015255in}{0.637529in}}%
\pgfpathcurveto{\pgfqpoint{1.017097in}{0.639371in}}{\pgfqpoint{1.018131in}{0.641869in}}{\pgfqpoint{1.018131in}{0.644474in}}%
\pgfpathcurveto{\pgfqpoint{1.018131in}{0.647078in}}{\pgfqpoint{1.017097in}{0.649576in}}{\pgfqpoint{1.015255in}{0.651418in}}%
\pgfpathcurveto{\pgfqpoint{1.013413in}{0.653260in}}{\pgfqpoint{1.010915in}{0.654295in}}{\pgfqpoint{1.008310in}{0.654295in}}%
\pgfpathcurveto{\pgfqpoint{1.005706in}{0.654295in}}{\pgfqpoint{1.003208in}{0.653260in}}{\pgfqpoint{1.001366in}{0.651418in}}%
\pgfpathcurveto{\pgfqpoint{0.999524in}{0.649576in}}{\pgfqpoint{0.998489in}{0.647078in}}{\pgfqpoint{0.998489in}{0.644474in}}%
\pgfpathcurveto{\pgfqpoint{0.998489in}{0.641869in}}{\pgfqpoint{0.999524in}{0.639371in}}{\pgfqpoint{1.001366in}{0.637529in}}%
\pgfpathcurveto{\pgfqpoint{1.003208in}{0.635687in}}{\pgfqpoint{1.005706in}{0.634653in}}{\pgfqpoint{1.008310in}{0.634653in}}%
\pgfpathclose%
\pgfusepath{stroke,fill}%
\end{pgfscope}%
\begin{pgfscope}%
\pgfpathrectangle{\pgfqpoint{0.750000in}{0.500000in}}{\pgfqpoint{4.650000in}{3.020000in}}%
\pgfusepath{clip}%
\pgfsetbuttcap%
\pgfsetroundjoin%
\definecolor{currentfill}{rgb}{0.121569,0.466667,0.705882}%
\pgfsetfillcolor{currentfill}%
\pgfsetlinewidth{1.003750pt}%
\definecolor{currentstroke}{rgb}{0.121569,0.466667,0.705882}%
\pgfsetstrokecolor{currentstroke}%
\pgfsetdash{}{0pt}%
\pgfpathmoveto{\pgfqpoint{1.730052in}{0.953892in}}%
\pgfpathcurveto{\pgfqpoint{1.732657in}{0.953892in}}{\pgfqpoint{1.735155in}{0.954926in}}{\pgfqpoint{1.736997in}{0.956768in}}%
\pgfpathcurveto{\pgfqpoint{1.738839in}{0.958610in}}{\pgfqpoint{1.739873in}{0.961108in}}{\pgfqpoint{1.739873in}{0.963713in}}%
\pgfpathcurveto{\pgfqpoint{1.739873in}{0.966317in}}{\pgfqpoint{1.738839in}{0.968815in}}{\pgfqpoint{1.736997in}{0.970657in}}%
\pgfpathcurveto{\pgfqpoint{1.735155in}{0.972499in}}{\pgfqpoint{1.732657in}{0.973533in}}{\pgfqpoint{1.730052in}{0.973533in}}%
\pgfpathcurveto{\pgfqpoint{1.727448in}{0.973533in}}{\pgfqpoint{1.724950in}{0.972499in}}{\pgfqpoint{1.723108in}{0.970657in}}%
\pgfpathcurveto{\pgfqpoint{1.721266in}{0.968815in}}{\pgfqpoint{1.720232in}{0.966317in}}{\pgfqpoint{1.720232in}{0.963713in}}%
\pgfpathcurveto{\pgfqpoint{1.720232in}{0.961108in}}{\pgfqpoint{1.721266in}{0.958610in}}{\pgfqpoint{1.723108in}{0.956768in}}%
\pgfpathcurveto{\pgfqpoint{1.724950in}{0.954926in}}{\pgfqpoint{1.727448in}{0.953892in}}{\pgfqpoint{1.730052in}{0.953892in}}%
\pgfpathclose%
\pgfusepath{stroke,fill}%
\end{pgfscope}%
\begin{pgfscope}%
\pgfpathrectangle{\pgfqpoint{0.750000in}{0.500000in}}{\pgfqpoint{4.650000in}{3.020000in}}%
\pgfusepath{clip}%
\pgfsetbuttcap%
\pgfsetroundjoin%
\definecolor{currentfill}{rgb}{0.121569,0.466667,0.705882}%
\pgfsetfillcolor{currentfill}%
\pgfsetlinewidth{1.003750pt}%
\definecolor{currentstroke}{rgb}{0.121569,0.466667,0.705882}%
\pgfsetstrokecolor{currentstroke}%
\pgfsetdash{}{0pt}%
\pgfpathmoveto{\pgfqpoint{1.812080in}{0.785164in}}%
\pgfpathcurveto{\pgfqpoint{1.815764in}{0.785164in}}{\pgfqpoint{1.819297in}{0.786627in}}{\pgfqpoint{1.821901in}{0.789232in}}%
\pgfpathcurveto{\pgfqpoint{1.824506in}{0.791836in}}{\pgfqpoint{1.825969in}{0.795369in}}{\pgfqpoint{1.825969in}{0.799052in}}%
\pgfpathcurveto{\pgfqpoint{1.825969in}{0.802736in}}{\pgfqpoint{1.824506in}{0.806269in}}{\pgfqpoint{1.821901in}{0.808873in}}%
\pgfpathcurveto{\pgfqpoint{1.819297in}{0.811478in}}{\pgfqpoint{1.815764in}{0.812941in}}{\pgfqpoint{1.812080in}{0.812941in}}%
\pgfpathcurveto{\pgfqpoint{1.808397in}{0.812941in}}{\pgfqpoint{1.804864in}{0.811478in}}{\pgfqpoint{1.802259in}{0.808873in}}%
\pgfpathcurveto{\pgfqpoint{1.799655in}{0.806269in}}{\pgfqpoint{1.798191in}{0.802736in}}{\pgfqpoint{1.798191in}{0.799052in}}%
\pgfpathcurveto{\pgfqpoint{1.798191in}{0.795369in}}{\pgfqpoint{1.799655in}{0.791836in}}{\pgfqpoint{1.802259in}{0.789232in}}%
\pgfpathcurveto{\pgfqpoint{1.804864in}{0.786627in}}{\pgfqpoint{1.808397in}{0.785164in}}{\pgfqpoint{1.812080in}{0.785164in}}%
\pgfpathclose%
\pgfusepath{stroke,fill}%
\end{pgfscope}%
\begin{pgfscope}%
\pgfpathrectangle{\pgfqpoint{0.750000in}{0.500000in}}{\pgfqpoint{4.650000in}{3.020000in}}%
\pgfusepath{clip}%
\pgfsetbuttcap%
\pgfsetroundjoin%
\definecolor{currentfill}{rgb}{0.121569,0.466667,0.705882}%
\pgfsetfillcolor{currentfill}%
\pgfsetlinewidth{1.003750pt}%
\definecolor{currentstroke}{rgb}{0.121569,0.466667,0.705882}%
\pgfsetstrokecolor{currentstroke}%
\pgfsetdash{}{0pt}%
\pgfpathmoveto{\pgfqpoint{1.259037in}{0.656928in}}%
\pgfpathcurveto{\pgfqpoint{1.262227in}{0.656928in}}{\pgfqpoint{1.265287in}{0.658196in}}{\pgfqpoint{1.267542in}{0.660451in}}%
\pgfpathcurveto{\pgfqpoint{1.269798in}{0.662707in}}{\pgfqpoint{1.271065in}{0.665767in}}{\pgfqpoint{1.271065in}{0.668957in}}%
\pgfpathcurveto{\pgfqpoint{1.271065in}{0.672146in}}{\pgfqpoint{1.269798in}{0.675206in}}{\pgfqpoint{1.267542in}{0.677462in}}%
\pgfpathcurveto{\pgfqpoint{1.265287in}{0.679717in}}{\pgfqpoint{1.262227in}{0.680985in}}{\pgfqpoint{1.259037in}{0.680985in}}%
\pgfpathcurveto{\pgfqpoint{1.255847in}{0.680985in}}{\pgfqpoint{1.252788in}{0.679717in}}{\pgfqpoint{1.250532in}{0.677462in}}%
\pgfpathcurveto{\pgfqpoint{1.248276in}{0.675206in}}{\pgfqpoint{1.247009in}{0.672146in}}{\pgfqpoint{1.247009in}{0.668957in}}%
\pgfpathcurveto{\pgfqpoint{1.247009in}{0.665767in}}{\pgfqpoint{1.248276in}{0.662707in}}{\pgfqpoint{1.250532in}{0.660451in}}%
\pgfpathcurveto{\pgfqpoint{1.252788in}{0.658196in}}{\pgfqpoint{1.255847in}{0.656928in}}{\pgfqpoint{1.259037in}{0.656928in}}%
\pgfpathclose%
\pgfusepath{stroke,fill}%
\end{pgfscope}%
\begin{pgfscope}%
\pgfpathrectangle{\pgfqpoint{0.750000in}{0.500000in}}{\pgfqpoint{4.650000in}{3.020000in}}%
\pgfusepath{clip}%
\pgfsetbuttcap%
\pgfsetroundjoin%
\definecolor{currentfill}{rgb}{0.121569,0.466667,0.705882}%
\pgfsetfillcolor{currentfill}%
\pgfsetlinewidth{1.003750pt}%
\definecolor{currentstroke}{rgb}{0.121569,0.466667,0.705882}%
\pgfsetstrokecolor{currentstroke}%
\pgfsetdash{}{0pt}%
\pgfpathmoveto{\pgfqpoint{2.080863in}{3.117124in}}%
\pgfpathcurveto{\pgfqpoint{2.085736in}{3.117124in}}{\pgfqpoint{2.090410in}{3.119060in}}{\pgfqpoint{2.093855in}{3.122505in}}%
\pgfpathcurveto{\pgfqpoint{2.097301in}{3.125951in}}{\pgfqpoint{2.099237in}{3.130624in}}{\pgfqpoint{2.099237in}{3.135497in}}%
\pgfpathcurveto{\pgfqpoint{2.099237in}{3.140370in}}{\pgfqpoint{2.097301in}{3.145044in}}{\pgfqpoint{2.093855in}{3.148489in}}%
\pgfpathcurveto{\pgfqpoint{2.090410in}{3.151934in}}{\pgfqpoint{2.085736in}{3.153870in}}{\pgfqpoint{2.080863in}{3.153870in}}%
\pgfpathcurveto{\pgfqpoint{2.075991in}{3.153870in}}{\pgfqpoint{2.071317in}{3.151934in}}{\pgfqpoint{2.067872in}{3.148489in}}%
\pgfpathcurveto{\pgfqpoint{2.064426in}{3.145044in}}{\pgfqpoint{2.062490in}{3.140370in}}{\pgfqpoint{2.062490in}{3.135497in}}%
\pgfpathcurveto{\pgfqpoint{2.062490in}{3.130624in}}{\pgfqpoint{2.064426in}{3.125951in}}{\pgfqpoint{2.067872in}{3.122505in}}%
\pgfpathcurveto{\pgfqpoint{2.071317in}{3.119060in}}{\pgfqpoint{2.075991in}{3.117124in}}{\pgfqpoint{2.080863in}{3.117124in}}%
\pgfpathclose%
\pgfusepath{stroke,fill}%
\end{pgfscope}%
\begin{pgfscope}%
\pgfpathrectangle{\pgfqpoint{0.750000in}{0.500000in}}{\pgfqpoint{4.650000in}{3.020000in}}%
\pgfusepath{clip}%
\pgfsetbuttcap%
\pgfsetroundjoin%
\definecolor{currentfill}{rgb}{0.121569,0.466667,0.705882}%
\pgfsetfillcolor{currentfill}%
\pgfsetlinewidth{1.003750pt}%
\definecolor{currentstroke}{rgb}{0.121569,0.466667,0.705882}%
\pgfsetstrokecolor{currentstroke}%
\pgfsetdash{}{0pt}%
\pgfpathmoveto{\pgfqpoint{1.589212in}{0.773881in}}%
\pgfpathcurveto{\pgfqpoint{1.593723in}{0.773881in}}{\pgfqpoint{1.598050in}{0.775673in}}{\pgfqpoint{1.601240in}{0.778863in}}%
\pgfpathcurveto{\pgfqpoint{1.604430in}{0.782053in}}{\pgfqpoint{1.606223in}{0.786380in}}{\pgfqpoint{1.606223in}{0.790891in}}%
\pgfpathcurveto{\pgfqpoint{1.606223in}{0.795403in}}{\pgfqpoint{1.604430in}{0.799730in}}{\pgfqpoint{1.601240in}{0.802920in}}%
\pgfpathcurveto{\pgfqpoint{1.598050in}{0.806109in}}{\pgfqpoint{1.593723in}{0.807902in}}{\pgfqpoint{1.589212in}{0.807902in}}%
\pgfpathcurveto{\pgfqpoint{1.584701in}{0.807902in}}{\pgfqpoint{1.580374in}{0.806109in}}{\pgfqpoint{1.577184in}{0.802920in}}%
\pgfpathcurveto{\pgfqpoint{1.573994in}{0.799730in}}{\pgfqpoint{1.572202in}{0.795403in}}{\pgfqpoint{1.572202in}{0.790891in}}%
\pgfpathcurveto{\pgfqpoint{1.572202in}{0.786380in}}{\pgfqpoint{1.573994in}{0.782053in}}{\pgfqpoint{1.577184in}{0.778863in}}%
\pgfpathcurveto{\pgfqpoint{1.580374in}{0.775673in}}{\pgfqpoint{1.584701in}{0.773881in}}{\pgfqpoint{1.589212in}{0.773881in}}%
\pgfpathclose%
\pgfusepath{stroke,fill}%
\end{pgfscope}%
\begin{pgfscope}%
\pgfpathrectangle{\pgfqpoint{0.750000in}{0.500000in}}{\pgfqpoint{4.650000in}{3.020000in}}%
\pgfusepath{clip}%
\pgfsetbuttcap%
\pgfsetroundjoin%
\definecolor{currentfill}{rgb}{0.121569,0.466667,0.705882}%
\pgfsetfillcolor{currentfill}%
\pgfsetlinewidth{1.003750pt}%
\definecolor{currentstroke}{rgb}{0.121569,0.466667,0.705882}%
\pgfsetstrokecolor{currentstroke}%
\pgfsetdash{}{0pt}%
\pgfpathmoveto{\pgfqpoint{1.002120in}{0.631292in}}%
\pgfpathcurveto{\pgfqpoint{1.004724in}{0.631292in}}{\pgfqpoint{1.007222in}{0.632327in}}{\pgfqpoint{1.009064in}{0.634169in}}%
\pgfpathcurveto{\pgfqpoint{1.010906in}{0.636010in}}{\pgfqpoint{1.011941in}{0.638509in}}{\pgfqpoint{1.011941in}{0.641113in}}%
\pgfpathcurveto{\pgfqpoint{1.011941in}{0.643718in}}{\pgfqpoint{1.010906in}{0.646216in}}{\pgfqpoint{1.009064in}{0.648058in}}%
\pgfpathcurveto{\pgfqpoint{1.007222in}{0.649899in}}{\pgfqpoint{1.004724in}{0.650934in}}{\pgfqpoint{1.002120in}{0.650934in}}%
\pgfpathcurveto{\pgfqpoint{0.999515in}{0.650934in}}{\pgfqpoint{0.997017in}{0.649899in}}{\pgfqpoint{0.995175in}{0.648058in}}%
\pgfpathcurveto{\pgfqpoint{0.993333in}{0.646216in}}{\pgfqpoint{0.992299in}{0.643718in}}{\pgfqpoint{0.992299in}{0.641113in}}%
\pgfpathcurveto{\pgfqpoint{0.992299in}{0.638509in}}{\pgfqpoint{0.993333in}{0.636010in}}{\pgfqpoint{0.995175in}{0.634169in}}%
\pgfpathcurveto{\pgfqpoint{0.997017in}{0.632327in}}{\pgfqpoint{0.999515in}{0.631292in}}{\pgfqpoint{1.002120in}{0.631292in}}%
\pgfpathclose%
\pgfusepath{stroke,fill}%
\end{pgfscope}%
\begin{pgfscope}%
\pgfpathrectangle{\pgfqpoint{0.750000in}{0.500000in}}{\pgfqpoint{4.650000in}{3.020000in}}%
\pgfusepath{clip}%
\pgfsetbuttcap%
\pgfsetroundjoin%
\definecolor{currentfill}{rgb}{0.121569,0.466667,0.705882}%
\pgfsetfillcolor{currentfill}%
\pgfsetlinewidth{1.003750pt}%
\definecolor{currentstroke}{rgb}{0.121569,0.466667,0.705882}%
\pgfsetstrokecolor{currentstroke}%
\pgfsetdash{}{0pt}%
\pgfpathmoveto{\pgfqpoint{1.237885in}{0.723657in}}%
\pgfpathcurveto{\pgfqpoint{1.241075in}{0.723657in}}{\pgfqpoint{1.244135in}{0.724924in}}{\pgfqpoint{1.246390in}{0.727180in}}%
\pgfpathcurveto{\pgfqpoint{1.248646in}{0.729435in}}{\pgfqpoint{1.249913in}{0.732495in}}{\pgfqpoint{1.249913in}{0.735685in}}%
\pgfpathcurveto{\pgfqpoint{1.249913in}{0.738875in}}{\pgfqpoint{1.248646in}{0.741934in}}{\pgfqpoint{1.246390in}{0.744190in}}%
\pgfpathcurveto{\pgfqpoint{1.244135in}{0.746445in}}{\pgfqpoint{1.241075in}{0.747713in}}{\pgfqpoint{1.237885in}{0.747713in}}%
\pgfpathcurveto{\pgfqpoint{1.234695in}{0.747713in}}{\pgfqpoint{1.231636in}{0.746445in}}{\pgfqpoint{1.229380in}{0.744190in}}%
\pgfpathcurveto{\pgfqpoint{1.227124in}{0.741934in}}{\pgfqpoint{1.225857in}{0.738875in}}{\pgfqpoint{1.225857in}{0.735685in}}%
\pgfpathcurveto{\pgfqpoint{1.225857in}{0.732495in}}{\pgfqpoint{1.227124in}{0.729435in}}{\pgfqpoint{1.229380in}{0.727180in}}%
\pgfpathcurveto{\pgfqpoint{1.231636in}{0.724924in}}{\pgfqpoint{1.234695in}{0.723657in}}{\pgfqpoint{1.237885in}{0.723657in}}%
\pgfpathclose%
\pgfusepath{stroke,fill}%
\end{pgfscope}%
\begin{pgfscope}%
\pgfpathrectangle{\pgfqpoint{0.750000in}{0.500000in}}{\pgfqpoint{4.650000in}{3.020000in}}%
\pgfusepath{clip}%
\pgfsetbuttcap%
\pgfsetroundjoin%
\definecolor{currentfill}{rgb}{0.121569,0.466667,0.705882}%
\pgfsetfillcolor{currentfill}%
\pgfsetlinewidth{1.003750pt}%
\definecolor{currentstroke}{rgb}{0.121569,0.466667,0.705882}%
\pgfsetstrokecolor{currentstroke}%
\pgfsetdash{}{0pt}%
\pgfpathmoveto{\pgfqpoint{0.977872in}{0.638013in}}%
\pgfpathcurveto{\pgfqpoint{0.980477in}{0.638013in}}{\pgfqpoint{0.982975in}{0.639048in}}{\pgfqpoint{0.984817in}{0.640890in}}%
\pgfpathcurveto{\pgfqpoint{0.986659in}{0.642731in}}{\pgfqpoint{0.987693in}{0.645229in}}{\pgfqpoint{0.987693in}{0.647834in}}%
\pgfpathcurveto{\pgfqpoint{0.987693in}{0.650439in}}{\pgfqpoint{0.986659in}{0.652937in}}{\pgfqpoint{0.984817in}{0.654778in}}%
\pgfpathcurveto{\pgfqpoint{0.982975in}{0.656620in}}{\pgfqpoint{0.980477in}{0.657655in}}{\pgfqpoint{0.977872in}{0.657655in}}%
\pgfpathcurveto{\pgfqpoint{0.975268in}{0.657655in}}{\pgfqpoint{0.972770in}{0.656620in}}{\pgfqpoint{0.970928in}{0.654778in}}%
\pgfpathcurveto{\pgfqpoint{0.969086in}{0.652937in}}{\pgfqpoint{0.968051in}{0.650439in}}{\pgfqpoint{0.968051in}{0.647834in}}%
\pgfpathcurveto{\pgfqpoint{0.968051in}{0.645229in}}{\pgfqpoint{0.969086in}{0.642731in}}{\pgfqpoint{0.970928in}{0.640890in}}%
\pgfpathcurveto{\pgfqpoint{0.972770in}{0.639048in}}{\pgfqpoint{0.975268in}{0.638013in}}{\pgfqpoint{0.977872in}{0.638013in}}%
\pgfpathclose%
\pgfusepath{stroke,fill}%
\end{pgfscope}%
\begin{pgfscope}%
\pgfpathrectangle{\pgfqpoint{0.750000in}{0.500000in}}{\pgfqpoint{4.650000in}{3.020000in}}%
\pgfusepath{clip}%
\pgfsetbuttcap%
\pgfsetroundjoin%
\definecolor{currentfill}{rgb}{0.121569,0.466667,0.705882}%
\pgfsetfillcolor{currentfill}%
\pgfsetlinewidth{1.003750pt}%
\definecolor{currentstroke}{rgb}{0.121569,0.466667,0.705882}%
\pgfsetstrokecolor{currentstroke}%
\pgfsetdash{}{0pt}%
\pgfpathmoveto{\pgfqpoint{2.120588in}{0.777963in}}%
\pgfpathcurveto{\pgfqpoint{2.124271in}{0.777963in}}{\pgfqpoint{2.127804in}{0.779426in}}{\pgfqpoint{2.130409in}{0.782031in}}%
\pgfpathcurveto{\pgfqpoint{2.133013in}{0.784635in}}{\pgfqpoint{2.134477in}{0.788168in}}{\pgfqpoint{2.134477in}{0.791852in}}%
\pgfpathcurveto{\pgfqpoint{2.134477in}{0.795535in}}{\pgfqpoint{2.133013in}{0.799068in}}{\pgfqpoint{2.130409in}{0.801672in}}%
\pgfpathcurveto{\pgfqpoint{2.127804in}{0.804277in}}{\pgfqpoint{2.124271in}{0.805740in}}{\pgfqpoint{2.120588in}{0.805740in}}%
\pgfpathcurveto{\pgfqpoint{2.116904in}{0.805740in}}{\pgfqpoint{2.113371in}{0.804277in}}{\pgfqpoint{2.110767in}{0.801672in}}%
\pgfpathcurveto{\pgfqpoint{2.108162in}{0.799068in}}{\pgfqpoint{2.106699in}{0.795535in}}{\pgfqpoint{2.106699in}{0.791852in}}%
\pgfpathcurveto{\pgfqpoint{2.106699in}{0.788168in}}{\pgfqpoint{2.108162in}{0.784635in}}{\pgfqpoint{2.110767in}{0.782031in}}%
\pgfpathcurveto{\pgfqpoint{2.113371in}{0.779426in}}{\pgfqpoint{2.116904in}{0.777963in}}{\pgfqpoint{2.120588in}{0.777963in}}%
\pgfpathclose%
\pgfusepath{stroke,fill}%
\end{pgfscope}%
\begin{pgfscope}%
\pgfpathrectangle{\pgfqpoint{0.750000in}{0.500000in}}{\pgfqpoint{4.650000in}{3.020000in}}%
\pgfusepath{clip}%
\pgfsetbuttcap%
\pgfsetroundjoin%
\definecolor{currentfill}{rgb}{0.121569,0.466667,0.705882}%
\pgfsetfillcolor{currentfill}%
\pgfsetlinewidth{1.003750pt}%
\definecolor{currentstroke}{rgb}{0.121569,0.466667,0.705882}%
\pgfsetstrokecolor{currentstroke}%
\pgfsetdash{}{0pt}%
\pgfpathmoveto{\pgfqpoint{1.467460in}{0.829750in}}%
\pgfpathcurveto{\pgfqpoint{1.470650in}{0.829750in}}{\pgfqpoint{1.473710in}{0.831017in}}{\pgfqpoint{1.475965in}{0.833272in}}%
\pgfpathcurveto{\pgfqpoint{1.478221in}{0.835528in}}{\pgfqpoint{1.479488in}{0.838588in}}{\pgfqpoint{1.479488in}{0.841778in}}%
\pgfpathcurveto{\pgfqpoint{1.479488in}{0.844968in}}{\pgfqpoint{1.478221in}{0.848027in}}{\pgfqpoint{1.475965in}{0.850283in}}%
\pgfpathcurveto{\pgfqpoint{1.473710in}{0.852538in}}{\pgfqpoint{1.470650in}{0.853806in}}{\pgfqpoint{1.467460in}{0.853806in}}%
\pgfpathcurveto{\pgfqpoint{1.464270in}{0.853806in}}{\pgfqpoint{1.461211in}{0.852538in}}{\pgfqpoint{1.458955in}{0.850283in}}%
\pgfpathcurveto{\pgfqpoint{1.456699in}{0.848027in}}{\pgfqpoint{1.455432in}{0.844968in}}{\pgfqpoint{1.455432in}{0.841778in}}%
\pgfpathcurveto{\pgfqpoint{1.455432in}{0.838588in}}{\pgfqpoint{1.456699in}{0.835528in}}{\pgfqpoint{1.458955in}{0.833272in}}%
\pgfpathcurveto{\pgfqpoint{1.461211in}{0.831017in}}{\pgfqpoint{1.464270in}{0.829750in}}{\pgfqpoint{1.467460in}{0.829750in}}%
\pgfpathclose%
\pgfusepath{stroke,fill}%
\end{pgfscope}%
\begin{pgfscope}%
\pgfpathrectangle{\pgfqpoint{0.750000in}{0.500000in}}{\pgfqpoint{4.650000in}{3.020000in}}%
\pgfusepath{clip}%
\pgfsetbuttcap%
\pgfsetroundjoin%
\definecolor{currentfill}{rgb}{0.121569,0.466667,0.705882}%
\pgfsetfillcolor{currentfill}%
\pgfsetlinewidth{1.003750pt}%
\definecolor{currentstroke}{rgb}{0.121569,0.466667,0.705882}%
\pgfsetstrokecolor{currentstroke}%
\pgfsetdash{}{0pt}%
\pgfpathmoveto{\pgfqpoint{1.505637in}{0.704741in}}%
\pgfpathcurveto{\pgfqpoint{1.508241in}{0.704741in}}{\pgfqpoint{1.510739in}{0.705776in}}{\pgfqpoint{1.512581in}{0.707618in}}%
\pgfpathcurveto{\pgfqpoint{1.514423in}{0.709459in}}{\pgfqpoint{1.515458in}{0.711958in}}{\pgfqpoint{1.515458in}{0.714562in}}%
\pgfpathcurveto{\pgfqpoint{1.515458in}{0.717167in}}{\pgfqpoint{1.514423in}{0.719665in}}{\pgfqpoint{1.512581in}{0.721507in}}%
\pgfpathcurveto{\pgfqpoint{1.510739in}{0.723348in}}{\pgfqpoint{1.508241in}{0.724383in}}{\pgfqpoint{1.505637in}{0.724383in}}%
\pgfpathcurveto{\pgfqpoint{1.503032in}{0.724383in}}{\pgfqpoint{1.500534in}{0.723348in}}{\pgfqpoint{1.498692in}{0.721507in}}%
\pgfpathcurveto{\pgfqpoint{1.496850in}{0.719665in}}{\pgfqpoint{1.495816in}{0.717167in}}{\pgfqpoint{1.495816in}{0.714562in}}%
\pgfpathcurveto{\pgfqpoint{1.495816in}{0.711958in}}{\pgfqpoint{1.496850in}{0.709459in}}{\pgfqpoint{1.498692in}{0.707618in}}%
\pgfpathcurveto{\pgfqpoint{1.500534in}{0.705776in}}{\pgfqpoint{1.503032in}{0.704741in}}{\pgfqpoint{1.505637in}{0.704741in}}%
\pgfpathclose%
\pgfusepath{stroke,fill}%
\end{pgfscope}%
\begin{pgfscope}%
\pgfpathrectangle{\pgfqpoint{0.750000in}{0.500000in}}{\pgfqpoint{4.650000in}{3.020000in}}%
\pgfusepath{clip}%
\pgfsetbuttcap%
\pgfsetroundjoin%
\definecolor{currentfill}{rgb}{0.121569,0.466667,0.705882}%
\pgfsetfillcolor{currentfill}%
\pgfsetlinewidth{1.003750pt}%
\definecolor{currentstroke}{rgb}{0.121569,0.466667,0.705882}%
\pgfsetstrokecolor{currentstroke}%
\pgfsetdash{}{0pt}%
\pgfpathmoveto{\pgfqpoint{1.131094in}{0.645694in}}%
\pgfpathcurveto{\pgfqpoint{1.133699in}{0.645694in}}{\pgfqpoint{1.136197in}{0.646729in}}{\pgfqpoint{1.138039in}{0.648571in}}%
\pgfpathcurveto{\pgfqpoint{1.139880in}{0.650412in}}{\pgfqpoint{1.140915in}{0.652910in}}{\pgfqpoint{1.140915in}{0.655515in}}%
\pgfpathcurveto{\pgfqpoint{1.140915in}{0.658119in}}{\pgfqpoint{1.139880in}{0.660618in}}{\pgfqpoint{1.138039in}{0.662459in}}%
\pgfpathcurveto{\pgfqpoint{1.136197in}{0.664301in}}{\pgfqpoint{1.133699in}{0.665336in}}{\pgfqpoint{1.131094in}{0.665336in}}%
\pgfpathcurveto{\pgfqpoint{1.128490in}{0.665336in}}{\pgfqpoint{1.125992in}{0.664301in}}{\pgfqpoint{1.124150in}{0.662459in}}%
\pgfpathcurveto{\pgfqpoint{1.122308in}{0.660618in}}{\pgfqpoint{1.121273in}{0.658119in}}{\pgfqpoint{1.121273in}{0.655515in}}%
\pgfpathcurveto{\pgfqpoint{1.121273in}{0.652910in}}{\pgfqpoint{1.122308in}{0.650412in}}{\pgfqpoint{1.124150in}{0.648571in}}%
\pgfpathcurveto{\pgfqpoint{1.125992in}{0.646729in}}{\pgfqpoint{1.128490in}{0.645694in}}{\pgfqpoint{1.131094in}{0.645694in}}%
\pgfpathclose%
\pgfusepath{stroke,fill}%
\end{pgfscope}%
\begin{pgfscope}%
\pgfpathrectangle{\pgfqpoint{0.750000in}{0.500000in}}{\pgfqpoint{4.650000in}{3.020000in}}%
\pgfusepath{clip}%
\pgfsetbuttcap%
\pgfsetroundjoin%
\definecolor{currentfill}{rgb}{0.121569,0.466667,0.705882}%
\pgfsetfillcolor{currentfill}%
\pgfsetlinewidth{1.003750pt}%
\definecolor{currentstroke}{rgb}{0.121569,0.466667,0.705882}%
\pgfsetstrokecolor{currentstroke}%
\pgfsetdash{}{0pt}%
\pgfpathmoveto{\pgfqpoint{1.258005in}{0.651935in}}%
\pgfpathcurveto{\pgfqpoint{1.260610in}{0.651935in}}{\pgfqpoint{1.263108in}{0.652970in}}{\pgfqpoint{1.264950in}{0.654811in}}%
\pgfpathcurveto{\pgfqpoint{1.266791in}{0.656653in}}{\pgfqpoint{1.267826in}{0.659151in}}{\pgfqpoint{1.267826in}{0.661756in}}%
\pgfpathcurveto{\pgfqpoint{1.267826in}{0.664360in}}{\pgfqpoint{1.266791in}{0.666858in}}{\pgfqpoint{1.264950in}{0.668700in}}%
\pgfpathcurveto{\pgfqpoint{1.263108in}{0.670542in}}{\pgfqpoint{1.260610in}{0.671577in}}{\pgfqpoint{1.258005in}{0.671577in}}%
\pgfpathcurveto{\pgfqpoint{1.255401in}{0.671577in}}{\pgfqpoint{1.252903in}{0.670542in}}{\pgfqpoint{1.251061in}{0.668700in}}%
\pgfpathcurveto{\pgfqpoint{1.249219in}{0.666858in}}{\pgfqpoint{1.248184in}{0.664360in}}{\pgfqpoint{1.248184in}{0.661756in}}%
\pgfpathcurveto{\pgfqpoint{1.248184in}{0.659151in}}{\pgfqpoint{1.249219in}{0.656653in}}{\pgfqpoint{1.251061in}{0.654811in}}%
\pgfpathcurveto{\pgfqpoint{1.252903in}{0.652970in}}{\pgfqpoint{1.255401in}{0.651935in}}{\pgfqpoint{1.258005in}{0.651935in}}%
\pgfpathclose%
\pgfusepath{stroke,fill}%
\end{pgfscope}%
\begin{pgfscope}%
\pgfpathrectangle{\pgfqpoint{0.750000in}{0.500000in}}{\pgfqpoint{4.650000in}{3.020000in}}%
\pgfusepath{clip}%
\pgfsetbuttcap%
\pgfsetroundjoin%
\definecolor{currentfill}{rgb}{0.121569,0.466667,0.705882}%
\pgfsetfillcolor{currentfill}%
\pgfsetlinewidth{1.003750pt}%
\definecolor{currentstroke}{rgb}{0.121569,0.466667,0.705882}%
\pgfsetstrokecolor{currentstroke}%
\pgfsetdash{}{0pt}%
\pgfpathmoveto{\pgfqpoint{1.130062in}{0.654108in}}%
\pgfpathcurveto{\pgfqpoint{1.133746in}{0.654108in}}{\pgfqpoint{1.137279in}{0.655571in}}{\pgfqpoint{1.139883in}{0.658176in}}%
\pgfpathcurveto{\pgfqpoint{1.142488in}{0.660780in}}{\pgfqpoint{1.143951in}{0.664313in}}{\pgfqpoint{1.143951in}{0.667996in}}%
\pgfpathcurveto{\pgfqpoint{1.143951in}{0.671680in}}{\pgfqpoint{1.142488in}{0.675213in}}{\pgfqpoint{1.139883in}{0.677817in}}%
\pgfpathcurveto{\pgfqpoint{1.137279in}{0.680422in}}{\pgfqpoint{1.133746in}{0.681885in}}{\pgfqpoint{1.130062in}{0.681885in}}%
\pgfpathcurveto{\pgfqpoint{1.126379in}{0.681885in}}{\pgfqpoint{1.122846in}{0.680422in}}{\pgfqpoint{1.120242in}{0.677817in}}%
\pgfpathcurveto{\pgfqpoint{1.117637in}{0.675213in}}{\pgfqpoint{1.116174in}{0.671680in}}{\pgfqpoint{1.116174in}{0.667996in}}%
\pgfpathcurveto{\pgfqpoint{1.116174in}{0.664313in}}{\pgfqpoint{1.117637in}{0.660780in}}{\pgfqpoint{1.120242in}{0.658176in}}%
\pgfpathcurveto{\pgfqpoint{1.122846in}{0.655571in}}{\pgfqpoint{1.126379in}{0.654108in}}{\pgfqpoint{1.130062in}{0.654108in}}%
\pgfpathclose%
\pgfusepath{stroke,fill}%
\end{pgfscope}%
\begin{pgfscope}%
\pgfpathrectangle{\pgfqpoint{0.750000in}{0.500000in}}{\pgfqpoint{4.650000in}{3.020000in}}%
\pgfusepath{clip}%
\pgfsetbuttcap%
\pgfsetroundjoin%
\definecolor{currentfill}{rgb}{0.121569,0.466667,0.705882}%
\pgfsetfillcolor{currentfill}%
\pgfsetlinewidth{1.003750pt}%
\definecolor{currentstroke}{rgb}{0.121569,0.466667,0.705882}%
\pgfsetstrokecolor{currentstroke}%
\pgfsetdash{}{0pt}%
\pgfpathmoveto{\pgfqpoint{1.070218in}{0.628892in}}%
\pgfpathcurveto{\pgfqpoint{1.072823in}{0.628892in}}{\pgfqpoint{1.075321in}{0.629927in}}{\pgfqpoint{1.077163in}{0.631768in}}%
\pgfpathcurveto{\pgfqpoint{1.079004in}{0.633610in}}{\pgfqpoint{1.080039in}{0.636108in}}{\pgfqpoint{1.080039in}{0.638713in}}%
\pgfpathcurveto{\pgfqpoint{1.080039in}{0.641317in}}{\pgfqpoint{1.079004in}{0.643816in}}{\pgfqpoint{1.077163in}{0.645657in}}%
\pgfpathcurveto{\pgfqpoint{1.075321in}{0.647499in}}{\pgfqpoint{1.072823in}{0.648534in}}{\pgfqpoint{1.070218in}{0.648534in}}%
\pgfpathcurveto{\pgfqpoint{1.067614in}{0.648534in}}{\pgfqpoint{1.065115in}{0.647499in}}{\pgfqpoint{1.063274in}{0.645657in}}%
\pgfpathcurveto{\pgfqpoint{1.061432in}{0.643816in}}{\pgfqpoint{1.060397in}{0.641317in}}{\pgfqpoint{1.060397in}{0.638713in}}%
\pgfpathcurveto{\pgfqpoint{1.060397in}{0.636108in}}{\pgfqpoint{1.061432in}{0.633610in}}{\pgfqpoint{1.063274in}{0.631768in}}%
\pgfpathcurveto{\pgfqpoint{1.065115in}{0.629927in}}{\pgfqpoint{1.067614in}{0.628892in}}{\pgfqpoint{1.070218in}{0.628892in}}%
\pgfpathclose%
\pgfusepath{stroke,fill}%
\end{pgfscope}%
\begin{pgfscope}%
\pgfpathrectangle{\pgfqpoint{0.750000in}{0.500000in}}{\pgfqpoint{4.650000in}{3.020000in}}%
\pgfusepath{clip}%
\pgfsetbuttcap%
\pgfsetroundjoin%
\definecolor{currentfill}{rgb}{0.121569,0.466667,0.705882}%
\pgfsetfillcolor{currentfill}%
\pgfsetlinewidth{1.003750pt}%
\definecolor{currentstroke}{rgb}{0.121569,0.466667,0.705882}%
\pgfsetstrokecolor{currentstroke}%
\pgfsetdash{}{0pt}%
\pgfpathmoveto{\pgfqpoint{2.524536in}{0.885332in}}%
\pgfpathcurveto{\pgfqpoint{2.529409in}{0.885332in}}{\pgfqpoint{2.534083in}{0.887268in}}{\pgfqpoint{2.537528in}{0.890713in}}%
\pgfpathcurveto{\pgfqpoint{2.540974in}{0.894159in}}{\pgfqpoint{2.542910in}{0.898833in}}{\pgfqpoint{2.542910in}{0.903705in}}%
\pgfpathcurveto{\pgfqpoint{2.542910in}{0.908578in}}{\pgfqpoint{2.540974in}{0.913252in}}{\pgfqpoint{2.537528in}{0.916697in}}%
\pgfpathcurveto{\pgfqpoint{2.534083in}{0.920143in}}{\pgfqpoint{2.529409in}{0.922078in}}{\pgfqpoint{2.524536in}{0.922078in}}%
\pgfpathcurveto{\pgfqpoint{2.519664in}{0.922078in}}{\pgfqpoint{2.514990in}{0.920143in}}{\pgfqpoint{2.511544in}{0.916697in}}%
\pgfpathcurveto{\pgfqpoint{2.508099in}{0.913252in}}{\pgfqpoint{2.506163in}{0.908578in}}{\pgfqpoint{2.506163in}{0.903705in}}%
\pgfpathcurveto{\pgfqpoint{2.506163in}{0.898833in}}{\pgfqpoint{2.508099in}{0.894159in}}{\pgfqpoint{2.511544in}{0.890713in}}%
\pgfpathcurveto{\pgfqpoint{2.514990in}{0.887268in}}{\pgfqpoint{2.519664in}{0.885332in}}{\pgfqpoint{2.524536in}{0.885332in}}%
\pgfpathclose%
\pgfusepath{stroke,fill}%
\end{pgfscope}%
\begin{pgfscope}%
\pgfpathrectangle{\pgfqpoint{0.750000in}{0.500000in}}{\pgfqpoint{4.650000in}{3.020000in}}%
\pgfusepath{clip}%
\pgfsetbuttcap%
\pgfsetroundjoin%
\definecolor{currentfill}{rgb}{0.121569,0.466667,0.705882}%
\pgfsetfillcolor{currentfill}%
\pgfsetlinewidth{1.003750pt}%
\definecolor{currentstroke}{rgb}{0.121569,0.466667,0.705882}%
\pgfsetstrokecolor{currentstroke}%
\pgfsetdash{}{0pt}%
\pgfpathmoveto{\pgfqpoint{5.188636in}{1.713488in}}%
\pgfpathcurveto{\pgfqpoint{5.195277in}{1.713488in}}{\pgfqpoint{5.201646in}{1.716127in}}{\pgfqpoint{5.206341in}{1.720822in}}%
\pgfpathcurveto{\pgfqpoint{5.211037in}{1.725517in}}{\pgfqpoint{5.213675in}{1.731887in}}{\pgfqpoint{5.213675in}{1.738527in}}%
\pgfpathcurveto{\pgfqpoint{5.213675in}{1.745167in}}{\pgfqpoint{5.211037in}{1.751536in}}{\pgfqpoint{5.206341in}{1.756232in}}%
\pgfpathcurveto{\pgfqpoint{5.201646in}{1.760927in}}{\pgfqpoint{5.195277in}{1.763565in}}{\pgfqpoint{5.188636in}{1.763565in}}%
\pgfpathcurveto{\pgfqpoint{5.181996in}{1.763565in}}{\pgfqpoint{5.175627in}{1.760927in}}{\pgfqpoint{5.170931in}{1.756232in}}%
\pgfpathcurveto{\pgfqpoint{5.166236in}{1.751536in}}{\pgfqpoint{5.163598in}{1.745167in}}{\pgfqpoint{5.163598in}{1.738527in}}%
\pgfpathcurveto{\pgfqpoint{5.163598in}{1.731887in}}{\pgfqpoint{5.166236in}{1.725517in}}{\pgfqpoint{5.170931in}{1.720822in}}%
\pgfpathcurveto{\pgfqpoint{5.175627in}{1.716127in}}{\pgfqpoint{5.181996in}{1.713488in}}{\pgfqpoint{5.188636in}{1.713488in}}%
\pgfpathclose%
\pgfusepath{stroke,fill}%
\end{pgfscope}%
\begin{pgfscope}%
\pgfpathrectangle{\pgfqpoint{0.750000in}{0.500000in}}{\pgfqpoint{4.650000in}{3.020000in}}%
\pgfusepath{clip}%
\pgfsetbuttcap%
\pgfsetroundjoin%
\definecolor{currentfill}{rgb}{0.121569,0.466667,0.705882}%
\pgfsetfillcolor{currentfill}%
\pgfsetlinewidth{1.003750pt}%
\definecolor{currentstroke}{rgb}{0.121569,0.466667,0.705882}%
\pgfsetstrokecolor{currentstroke}%
\pgfsetdash{}{0pt}%
\pgfpathmoveto{\pgfqpoint{1.288443in}{0.659388in}}%
\pgfpathcurveto{\pgfqpoint{1.292127in}{0.659388in}}{\pgfqpoint{1.295660in}{0.660852in}}{\pgfqpoint{1.298264in}{0.663456in}}%
\pgfpathcurveto{\pgfqpoint{1.300869in}{0.666061in}}{\pgfqpoint{1.302332in}{0.669594in}}{\pgfqpoint{1.302332in}{0.673277in}}%
\pgfpathcurveto{\pgfqpoint{1.302332in}{0.676960in}}{\pgfqpoint{1.300869in}{0.680494in}}{\pgfqpoint{1.298264in}{0.683098in}}%
\pgfpathcurveto{\pgfqpoint{1.295660in}{0.685703in}}{\pgfqpoint{1.292127in}{0.687166in}}{\pgfqpoint{1.288443in}{0.687166in}}%
\pgfpathcurveto{\pgfqpoint{1.284760in}{0.687166in}}{\pgfqpoint{1.281227in}{0.685703in}}{\pgfqpoint{1.278622in}{0.683098in}}%
\pgfpathcurveto{\pgfqpoint{1.276018in}{0.680494in}}{\pgfqpoint{1.274554in}{0.676960in}}{\pgfqpoint{1.274554in}{0.673277in}}%
\pgfpathcurveto{\pgfqpoint{1.274554in}{0.669594in}}{\pgfqpoint{1.276018in}{0.666061in}}{\pgfqpoint{1.278622in}{0.663456in}}%
\pgfpathcurveto{\pgfqpoint{1.281227in}{0.660852in}}{\pgfqpoint{1.284760in}{0.659388in}}{\pgfqpoint{1.288443in}{0.659388in}}%
\pgfpathclose%
\pgfusepath{stroke,fill}%
\end{pgfscope}%
\begin{pgfscope}%
\pgfpathrectangle{\pgfqpoint{0.750000in}{0.500000in}}{\pgfqpoint{4.650000in}{3.020000in}}%
\pgfusepath{clip}%
\pgfsetbuttcap%
\pgfsetroundjoin%
\definecolor{currentfill}{rgb}{0.121569,0.466667,0.705882}%
\pgfsetfillcolor{currentfill}%
\pgfsetlinewidth{1.003750pt}%
\definecolor{currentstroke}{rgb}{0.121569,0.466667,0.705882}%
\pgfsetstrokecolor{currentstroke}%
\pgfsetdash{}{0pt}%
\pgfpathmoveto{\pgfqpoint{1.403489in}{0.666050in}}%
\pgfpathcurveto{\pgfqpoint{1.406679in}{0.666050in}}{\pgfqpoint{1.409738in}{0.667317in}}{\pgfqpoint{1.411994in}{0.669573in}}%
\pgfpathcurveto{\pgfqpoint{1.414249in}{0.671828in}}{\pgfqpoint{1.415517in}{0.674888in}}{\pgfqpoint{1.415517in}{0.678078in}}%
\pgfpathcurveto{\pgfqpoint{1.415517in}{0.681268in}}{\pgfqpoint{1.414249in}{0.684327in}}{\pgfqpoint{1.411994in}{0.686583in}}%
\pgfpathcurveto{\pgfqpoint{1.409738in}{0.688838in}}{\pgfqpoint{1.406679in}{0.690106in}}{\pgfqpoint{1.403489in}{0.690106in}}%
\pgfpathcurveto{\pgfqpoint{1.400299in}{0.690106in}}{\pgfqpoint{1.397239in}{0.688838in}}{\pgfqpoint{1.394984in}{0.686583in}}%
\pgfpathcurveto{\pgfqpoint{1.392728in}{0.684327in}}{\pgfqpoint{1.391461in}{0.681268in}}{\pgfqpoint{1.391461in}{0.678078in}}%
\pgfpathcurveto{\pgfqpoint{1.391461in}{0.674888in}}{\pgfqpoint{1.392728in}{0.671828in}}{\pgfqpoint{1.394984in}{0.669573in}}%
\pgfpathcurveto{\pgfqpoint{1.397239in}{0.667317in}}{\pgfqpoint{1.400299in}{0.666050in}}{\pgfqpoint{1.403489in}{0.666050in}}%
\pgfpathclose%
\pgfusepath{stroke,fill}%
\end{pgfscope}%
\begin{pgfscope}%
\pgfpathrectangle{\pgfqpoint{0.750000in}{0.500000in}}{\pgfqpoint{4.650000in}{3.020000in}}%
\pgfusepath{clip}%
\pgfsetbuttcap%
\pgfsetroundjoin%
\definecolor{currentfill}{rgb}{0.121569,0.466667,0.705882}%
\pgfsetfillcolor{currentfill}%
\pgfsetlinewidth{1.003750pt}%
\definecolor{currentstroke}{rgb}{0.121569,0.466667,0.705882}%
\pgfsetstrokecolor{currentstroke}%
\pgfsetdash{}{0pt}%
\pgfpathmoveto{\pgfqpoint{1.066607in}{0.640126in}}%
\pgfpathcurveto{\pgfqpoint{1.069797in}{0.640126in}}{\pgfqpoint{1.072857in}{0.641394in}}{\pgfqpoint{1.075112in}{0.643649in}}%
\pgfpathcurveto{\pgfqpoint{1.077368in}{0.645905in}}{\pgfqpoint{1.078635in}{0.648965in}}{\pgfqpoint{1.078635in}{0.652155in}}%
\pgfpathcurveto{\pgfqpoint{1.078635in}{0.655344in}}{\pgfqpoint{1.077368in}{0.658404in}}{\pgfqpoint{1.075112in}{0.660660in}}%
\pgfpathcurveto{\pgfqpoint{1.072857in}{0.662915in}}{\pgfqpoint{1.069797in}{0.664183in}}{\pgfqpoint{1.066607in}{0.664183in}}%
\pgfpathcurveto{\pgfqpoint{1.063417in}{0.664183in}}{\pgfqpoint{1.060357in}{0.662915in}}{\pgfqpoint{1.058102in}{0.660660in}}%
\pgfpathcurveto{\pgfqpoint{1.055846in}{0.658404in}}{\pgfqpoint{1.054579in}{0.655344in}}{\pgfqpoint{1.054579in}{0.652155in}}%
\pgfpathcurveto{\pgfqpoint{1.054579in}{0.648965in}}{\pgfqpoint{1.055846in}{0.645905in}}{\pgfqpoint{1.058102in}{0.643649in}}%
\pgfpathcurveto{\pgfqpoint{1.060357in}{0.641394in}}{\pgfqpoint{1.063417in}{0.640126in}}{\pgfqpoint{1.066607in}{0.640126in}}%
\pgfpathclose%
\pgfusepath{stroke,fill}%
\end{pgfscope}%
\begin{pgfscope}%
\pgfpathrectangle{\pgfqpoint{0.750000in}{0.500000in}}{\pgfqpoint{4.650000in}{3.020000in}}%
\pgfusepath{clip}%
\pgfsetbuttcap%
\pgfsetroundjoin%
\definecolor{currentfill}{rgb}{0.121569,0.466667,0.705882}%
\pgfsetfillcolor{currentfill}%
\pgfsetlinewidth{1.003750pt}%
\definecolor{currentstroke}{rgb}{0.121569,0.466667,0.705882}%
\pgfsetstrokecolor{currentstroke}%
\pgfsetdash{}{0pt}%
\pgfpathmoveto{\pgfqpoint{0.985611in}{0.630812in}}%
\pgfpathcurveto{\pgfqpoint{0.988215in}{0.630812in}}{\pgfqpoint{0.990714in}{0.631847in}}{\pgfqpoint{0.992555in}{0.633689in}}%
\pgfpathcurveto{\pgfqpoint{0.994397in}{0.635530in}}{\pgfqpoint{0.995432in}{0.638029in}}{\pgfqpoint{0.995432in}{0.640633in}}%
\pgfpathcurveto{\pgfqpoint{0.995432in}{0.643238in}}{\pgfqpoint{0.994397in}{0.645736in}}{\pgfqpoint{0.992555in}{0.647578in}}%
\pgfpathcurveto{\pgfqpoint{0.990714in}{0.649419in}}{\pgfqpoint{0.988215in}{0.650454in}}{\pgfqpoint{0.985611in}{0.650454in}}%
\pgfpathcurveto{\pgfqpoint{0.983006in}{0.650454in}}{\pgfqpoint{0.980508in}{0.649419in}}{\pgfqpoint{0.978666in}{0.647578in}}%
\pgfpathcurveto{\pgfqpoint{0.976825in}{0.645736in}}{\pgfqpoint{0.975790in}{0.643238in}}{\pgfqpoint{0.975790in}{0.640633in}}%
\pgfpathcurveto{\pgfqpoint{0.975790in}{0.638029in}}{\pgfqpoint{0.976825in}{0.635530in}}{\pgfqpoint{0.978666in}{0.633689in}}%
\pgfpathcurveto{\pgfqpoint{0.980508in}{0.631847in}}{\pgfqpoint{0.983006in}{0.630812in}}{\pgfqpoint{0.985611in}{0.630812in}}%
\pgfpathclose%
\pgfusepath{stroke,fill}%
\end{pgfscope}%
\begin{pgfscope}%
\pgfpathrectangle{\pgfqpoint{0.750000in}{0.500000in}}{\pgfqpoint{4.650000in}{3.020000in}}%
\pgfusepath{clip}%
\pgfsetbuttcap%
\pgfsetroundjoin%
\definecolor{currentfill}{rgb}{0.121569,0.466667,0.705882}%
\pgfsetfillcolor{currentfill}%
\pgfsetlinewidth{1.003750pt}%
\definecolor{currentstroke}{rgb}{0.121569,0.466667,0.705882}%
\pgfsetstrokecolor{currentstroke}%
\pgfsetdash{}{0pt}%
\pgfpathmoveto{\pgfqpoint{1.453531in}{0.692032in}}%
\pgfpathcurveto{\pgfqpoint{1.457214in}{0.692032in}}{\pgfqpoint{1.460747in}{0.693496in}}{\pgfqpoint{1.463352in}{0.696100in}}%
\pgfpathcurveto{\pgfqpoint{1.465956in}{0.698705in}}{\pgfqpoint{1.467420in}{0.702238in}}{\pgfqpoint{1.467420in}{0.705921in}}%
\pgfpathcurveto{\pgfqpoint{1.467420in}{0.709604in}}{\pgfqpoint{1.465956in}{0.713137in}}{\pgfqpoint{1.463352in}{0.715742in}}%
\pgfpathcurveto{\pgfqpoint{1.460747in}{0.718347in}}{\pgfqpoint{1.457214in}{0.719810in}}{\pgfqpoint{1.453531in}{0.719810in}}%
\pgfpathcurveto{\pgfqpoint{1.449847in}{0.719810in}}{\pgfqpoint{1.446314in}{0.718347in}}{\pgfqpoint{1.443710in}{0.715742in}}%
\pgfpathcurveto{\pgfqpoint{1.441105in}{0.713137in}}{\pgfqpoint{1.439642in}{0.709604in}}{\pgfqpoint{1.439642in}{0.705921in}}%
\pgfpathcurveto{\pgfqpoint{1.439642in}{0.702238in}}{\pgfqpoint{1.441105in}{0.698705in}}{\pgfqpoint{1.443710in}{0.696100in}}%
\pgfpathcurveto{\pgfqpoint{1.446314in}{0.693496in}}{\pgfqpoint{1.449847in}{0.692032in}}{\pgfqpoint{1.453531in}{0.692032in}}%
\pgfpathclose%
\pgfusepath{stroke,fill}%
\end{pgfscope}%
\begin{pgfscope}%
\pgfpathrectangle{\pgfqpoint{0.750000in}{0.500000in}}{\pgfqpoint{4.650000in}{3.020000in}}%
\pgfusepath{clip}%
\pgfsetbuttcap%
\pgfsetroundjoin%
\definecolor{currentfill}{rgb}{0.121569,0.466667,0.705882}%
\pgfsetfillcolor{currentfill}%
\pgfsetlinewidth{1.003750pt}%
\definecolor{currentstroke}{rgb}{0.121569,0.466667,0.705882}%
\pgfsetstrokecolor{currentstroke}%
\pgfsetdash{}{0pt}%
\pgfpathmoveto{\pgfqpoint{2.273294in}{0.875415in}}%
\pgfpathcurveto{\pgfqpoint{2.276977in}{0.875415in}}{\pgfqpoint{2.280510in}{0.876878in}}{\pgfqpoint{2.283115in}{0.879483in}}%
\pgfpathcurveto{\pgfqpoint{2.285719in}{0.882087in}}{\pgfqpoint{2.287183in}{0.885620in}}{\pgfqpoint{2.287183in}{0.889303in}}%
\pgfpathcurveto{\pgfqpoint{2.287183in}{0.892987in}}{\pgfqpoint{2.285719in}{0.896520in}}{\pgfqpoint{2.283115in}{0.899124in}}%
\pgfpathcurveto{\pgfqpoint{2.280510in}{0.901729in}}{\pgfqpoint{2.276977in}{0.903192in}}{\pgfqpoint{2.273294in}{0.903192in}}%
\pgfpathcurveto{\pgfqpoint{2.269610in}{0.903192in}}{\pgfqpoint{2.266077in}{0.901729in}}{\pgfqpoint{2.263473in}{0.899124in}}%
\pgfpathcurveto{\pgfqpoint{2.260868in}{0.896520in}}{\pgfqpoint{2.259405in}{0.892987in}}{\pgfqpoint{2.259405in}{0.889303in}}%
\pgfpathcurveto{\pgfqpoint{2.259405in}{0.885620in}}{\pgfqpoint{2.260868in}{0.882087in}}{\pgfqpoint{2.263473in}{0.879483in}}%
\pgfpathcurveto{\pgfqpoint{2.266077in}{0.876878in}}{\pgfqpoint{2.269610in}{0.875415in}}{\pgfqpoint{2.273294in}{0.875415in}}%
\pgfpathclose%
\pgfusepath{stroke,fill}%
\end{pgfscope}%
\begin{pgfscope}%
\pgfpathrectangle{\pgfqpoint{0.750000in}{0.500000in}}{\pgfqpoint{4.650000in}{3.020000in}}%
\pgfusepath{clip}%
\pgfsetbuttcap%
\pgfsetroundjoin%
\definecolor{currentfill}{rgb}{0.121569,0.466667,0.705882}%
\pgfsetfillcolor{currentfill}%
\pgfsetlinewidth{1.003750pt}%
\definecolor{currentstroke}{rgb}{0.121569,0.466667,0.705882}%
\pgfsetstrokecolor{currentstroke}%
\pgfsetdash{}{0pt}%
\pgfpathmoveto{\pgfqpoint{1.084663in}{0.634846in}}%
\pgfpathcurveto{\pgfqpoint{1.087853in}{0.634846in}}{\pgfqpoint{1.090913in}{0.636113in}}{\pgfqpoint{1.093169in}{0.638369in}}%
\pgfpathcurveto{\pgfqpoint{1.095424in}{0.640624in}}{\pgfqpoint{1.096692in}{0.643684in}}{\pgfqpoint{1.096692in}{0.646874in}}%
\pgfpathcurveto{\pgfqpoint{1.096692in}{0.650064in}}{\pgfqpoint{1.095424in}{0.653123in}}{\pgfqpoint{1.093169in}{0.655379in}}%
\pgfpathcurveto{\pgfqpoint{1.090913in}{0.657635in}}{\pgfqpoint{1.087853in}{0.658902in}}{\pgfqpoint{1.084663in}{0.658902in}}%
\pgfpathcurveto{\pgfqpoint{1.081473in}{0.658902in}}{\pgfqpoint{1.078414in}{0.657635in}}{\pgfqpoint{1.076158in}{0.655379in}}%
\pgfpathcurveto{\pgfqpoint{1.073903in}{0.653123in}}{\pgfqpoint{1.072635in}{0.650064in}}{\pgfqpoint{1.072635in}{0.646874in}}%
\pgfpathcurveto{\pgfqpoint{1.072635in}{0.643684in}}{\pgfqpoint{1.073903in}{0.640624in}}{\pgfqpoint{1.076158in}{0.638369in}}%
\pgfpathcurveto{\pgfqpoint{1.078414in}{0.636113in}}{\pgfqpoint{1.081473in}{0.634846in}}{\pgfqpoint{1.084663in}{0.634846in}}%
\pgfpathclose%
\pgfusepath{stroke,fill}%
\end{pgfscope}%
\begin{pgfscope}%
\pgfpathrectangle{\pgfqpoint{0.750000in}{0.500000in}}{\pgfqpoint{4.650000in}{3.020000in}}%
\pgfusepath{clip}%
\pgfsetbuttcap%
\pgfsetroundjoin%
\definecolor{currentfill}{rgb}{0.121569,0.466667,0.705882}%
\pgfsetfillcolor{currentfill}%
\pgfsetlinewidth{1.003750pt}%
\definecolor{currentstroke}{rgb}{0.121569,0.466667,0.705882}%
\pgfsetstrokecolor{currentstroke}%
\pgfsetdash{}{0pt}%
\pgfpathmoveto{\pgfqpoint{1.383885in}{0.695632in}}%
\pgfpathcurveto{\pgfqpoint{1.388396in}{0.695632in}}{\pgfqpoint{1.392723in}{0.697424in}}{\pgfqpoint{1.395913in}{0.700614in}}%
\pgfpathcurveto{\pgfqpoint{1.399103in}{0.703804in}}{\pgfqpoint{1.400895in}{0.708131in}}{\pgfqpoint{1.400895in}{0.712642in}}%
\pgfpathcurveto{\pgfqpoint{1.400895in}{0.717153in}}{\pgfqpoint{1.399103in}{0.721480in}}{\pgfqpoint{1.395913in}{0.724670in}}%
\pgfpathcurveto{\pgfqpoint{1.392723in}{0.727860in}}{\pgfqpoint{1.388396in}{0.729652in}}{\pgfqpoint{1.383885in}{0.729652in}}%
\pgfpathcurveto{\pgfqpoint{1.379373in}{0.729652in}}{\pgfqpoint{1.375046in}{0.727860in}}{\pgfqpoint{1.371856in}{0.724670in}}%
\pgfpathcurveto{\pgfqpoint{1.368667in}{0.721480in}}{\pgfqpoint{1.366874in}{0.717153in}}{\pgfqpoint{1.366874in}{0.712642in}}%
\pgfpathcurveto{\pgfqpoint{1.366874in}{0.708131in}}{\pgfqpoint{1.368667in}{0.703804in}}{\pgfqpoint{1.371856in}{0.700614in}}%
\pgfpathcurveto{\pgfqpoint{1.375046in}{0.697424in}}{\pgfqpoint{1.379373in}{0.695632in}}{\pgfqpoint{1.383885in}{0.695632in}}%
\pgfpathclose%
\pgfusepath{stroke,fill}%
\end{pgfscope}%
\begin{pgfscope}%
\pgfpathrectangle{\pgfqpoint{0.750000in}{0.500000in}}{\pgfqpoint{4.650000in}{3.020000in}}%
\pgfusepath{clip}%
\pgfsetbuttcap%
\pgfsetroundjoin%
\definecolor{currentfill}{rgb}{0.121569,0.466667,0.705882}%
\pgfsetfillcolor{currentfill}%
\pgfsetlinewidth{1.003750pt}%
\definecolor{currentstroke}{rgb}{0.121569,0.466667,0.705882}%
\pgfsetstrokecolor{currentstroke}%
\pgfsetdash{}{0pt}%
\pgfpathmoveto{\pgfqpoint{1.729021in}{0.698500in}}%
\pgfpathcurveto{\pgfqpoint{1.731625in}{0.698500in}}{\pgfqpoint{1.734123in}{0.699535in}}{\pgfqpoint{1.735965in}{0.701377in}}%
\pgfpathcurveto{\pgfqpoint{1.737807in}{0.703219in}}{\pgfqpoint{1.738842in}{0.705717in}}{\pgfqpoint{1.738842in}{0.708321in}}%
\pgfpathcurveto{\pgfqpoint{1.738842in}{0.710926in}}{\pgfqpoint{1.737807in}{0.713424in}}{\pgfqpoint{1.735965in}{0.715266in}}%
\pgfpathcurveto{\pgfqpoint{1.734123in}{0.717108in}}{\pgfqpoint{1.731625in}{0.718142in}}{\pgfqpoint{1.729021in}{0.718142in}}%
\pgfpathcurveto{\pgfqpoint{1.726416in}{0.718142in}}{\pgfqpoint{1.723918in}{0.717108in}}{\pgfqpoint{1.722076in}{0.715266in}}%
\pgfpathcurveto{\pgfqpoint{1.720235in}{0.713424in}}{\pgfqpoint{1.719200in}{0.710926in}}{\pgfqpoint{1.719200in}{0.708321in}}%
\pgfpathcurveto{\pgfqpoint{1.719200in}{0.705717in}}{\pgfqpoint{1.720235in}{0.703219in}}{\pgfqpoint{1.722076in}{0.701377in}}%
\pgfpathcurveto{\pgfqpoint{1.723918in}{0.699535in}}{\pgfqpoint{1.726416in}{0.698500in}}{\pgfqpoint{1.729021in}{0.698500in}}%
\pgfpathclose%
\pgfusepath{stroke,fill}%
\end{pgfscope}%
\begin{pgfscope}%
\pgfpathrectangle{\pgfqpoint{0.750000in}{0.500000in}}{\pgfqpoint{4.650000in}{3.020000in}}%
\pgfusepath{clip}%
\pgfsetbuttcap%
\pgfsetroundjoin%
\definecolor{currentfill}{rgb}{0.121569,0.466667,0.705882}%
\pgfsetfillcolor{currentfill}%
\pgfsetlinewidth{1.003750pt}%
\definecolor{currentstroke}{rgb}{0.121569,0.466667,0.705882}%
\pgfsetstrokecolor{currentstroke}%
\pgfsetdash{}{0pt}%
\pgfpathmoveto{\pgfqpoint{1.119744in}{0.633693in}}%
\pgfpathcurveto{\pgfqpoint{1.122349in}{0.633693in}}{\pgfqpoint{1.124847in}{0.634727in}}{\pgfqpoint{1.126689in}{0.636569in}}%
\pgfpathcurveto{\pgfqpoint{1.128531in}{0.638411in}}{\pgfqpoint{1.129565in}{0.640909in}}{\pgfqpoint{1.129565in}{0.643513in}}%
\pgfpathcurveto{\pgfqpoint{1.129565in}{0.646118in}}{\pgfqpoint{1.128531in}{0.648616in}}{\pgfqpoint{1.126689in}{0.650458in}}%
\pgfpathcurveto{\pgfqpoint{1.124847in}{0.652300in}}{\pgfqpoint{1.122349in}{0.653334in}}{\pgfqpoint{1.119744in}{0.653334in}}%
\pgfpathcurveto{\pgfqpoint{1.117140in}{0.653334in}}{\pgfqpoint{1.114642in}{0.652300in}}{\pgfqpoint{1.112800in}{0.650458in}}%
\pgfpathcurveto{\pgfqpoint{1.110958in}{0.648616in}}{\pgfqpoint{1.109924in}{0.646118in}}{\pgfqpoint{1.109924in}{0.643513in}}%
\pgfpathcurveto{\pgfqpoint{1.109924in}{0.640909in}}{\pgfqpoint{1.110958in}{0.638411in}}{\pgfqpoint{1.112800in}{0.636569in}}%
\pgfpathcurveto{\pgfqpoint{1.114642in}{0.634727in}}{\pgfqpoint{1.117140in}{0.633693in}}{\pgfqpoint{1.119744in}{0.633693in}}%
\pgfpathclose%
\pgfusepath{stroke,fill}%
\end{pgfscope}%
\begin{pgfscope}%
\pgfpathrectangle{\pgfqpoint{0.750000in}{0.500000in}}{\pgfqpoint{4.650000in}{3.020000in}}%
\pgfusepath{clip}%
\pgfsetbuttcap%
\pgfsetroundjoin%
\definecolor{currentfill}{rgb}{0.121569,0.466667,0.705882}%
\pgfsetfillcolor{currentfill}%
\pgfsetlinewidth{1.003750pt}%
\definecolor{currentstroke}{rgb}{0.121569,0.466667,0.705882}%
\pgfsetstrokecolor{currentstroke}%
\pgfsetdash{}{0pt}%
\pgfpathmoveto{\pgfqpoint{0.982000in}{0.627452in}}%
\pgfpathcurveto{\pgfqpoint{0.984604in}{0.627452in}}{\pgfqpoint{0.987102in}{0.628487in}}{\pgfqpoint{0.988944in}{0.630328in}}%
\pgfpathcurveto{\pgfqpoint{0.990786in}{0.632170in}}{\pgfqpoint{0.991821in}{0.634668in}}{\pgfqpoint{0.991821in}{0.637273in}}%
\pgfpathcurveto{\pgfqpoint{0.991821in}{0.639877in}}{\pgfqpoint{0.990786in}{0.642375in}}{\pgfqpoint{0.988944in}{0.644217in}}%
\pgfpathcurveto{\pgfqpoint{0.987102in}{0.646059in}}{\pgfqpoint{0.984604in}{0.647094in}}{\pgfqpoint{0.982000in}{0.647094in}}%
\pgfpathcurveto{\pgfqpoint{0.979395in}{0.647094in}}{\pgfqpoint{0.976897in}{0.646059in}}{\pgfqpoint{0.975055in}{0.644217in}}%
\pgfpathcurveto{\pgfqpoint{0.973213in}{0.642375in}}{\pgfqpoint{0.972179in}{0.639877in}}{\pgfqpoint{0.972179in}{0.637273in}}%
\pgfpathcurveto{\pgfqpoint{0.972179in}{0.634668in}}{\pgfqpoint{0.973213in}{0.632170in}}{\pgfqpoint{0.975055in}{0.630328in}}%
\pgfpathcurveto{\pgfqpoint{0.976897in}{0.628487in}}{\pgfqpoint{0.979395in}{0.627452in}}{\pgfqpoint{0.982000in}{0.627452in}}%
\pgfpathclose%
\pgfusepath{stroke,fill}%
\end{pgfscope}%
\begin{pgfscope}%
\pgfpathrectangle{\pgfqpoint{0.750000in}{0.500000in}}{\pgfqpoint{4.650000in}{3.020000in}}%
\pgfusepath{clip}%
\pgfsetbuttcap%
\pgfsetroundjoin%
\definecolor{currentfill}{rgb}{0.121569,0.466667,0.705882}%
\pgfsetfillcolor{currentfill}%
\pgfsetlinewidth{1.003750pt}%
\definecolor{currentstroke}{rgb}{0.121569,0.466667,0.705882}%
\pgfsetstrokecolor{currentstroke}%
\pgfsetdash{}{0pt}%
\pgfpathmoveto{\pgfqpoint{1.157405in}{0.648574in}}%
\pgfpathcurveto{\pgfqpoint{1.160010in}{0.648574in}}{\pgfqpoint{1.162508in}{0.649609in}}{\pgfqpoint{1.164350in}{0.651451in}}%
\pgfpathcurveto{\pgfqpoint{1.166191in}{0.653293in}}{\pgfqpoint{1.167226in}{0.655791in}}{\pgfqpoint{1.167226in}{0.658395in}}%
\pgfpathcurveto{\pgfqpoint{1.167226in}{0.661000in}}{\pgfqpoint{1.166191in}{0.663498in}}{\pgfqpoint{1.164350in}{0.665340in}}%
\pgfpathcurveto{\pgfqpoint{1.162508in}{0.667181in}}{\pgfqpoint{1.160010in}{0.668216in}}{\pgfqpoint{1.157405in}{0.668216in}}%
\pgfpathcurveto{\pgfqpoint{1.154801in}{0.668216in}}{\pgfqpoint{1.152302in}{0.667181in}}{\pgfqpoint{1.150461in}{0.665340in}}%
\pgfpathcurveto{\pgfqpoint{1.148619in}{0.663498in}}{\pgfqpoint{1.147584in}{0.661000in}}{\pgfqpoint{1.147584in}{0.658395in}}%
\pgfpathcurveto{\pgfqpoint{1.147584in}{0.655791in}}{\pgfqpoint{1.148619in}{0.653293in}}{\pgfqpoint{1.150461in}{0.651451in}}%
\pgfpathcurveto{\pgfqpoint{1.152302in}{0.649609in}}{\pgfqpoint{1.154801in}{0.648574in}}{\pgfqpoint{1.157405in}{0.648574in}}%
\pgfpathclose%
\pgfusepath{stroke,fill}%
\end{pgfscope}%
\begin{pgfscope}%
\pgfpathrectangle{\pgfqpoint{0.750000in}{0.500000in}}{\pgfqpoint{4.650000in}{3.020000in}}%
\pgfusepath{clip}%
\pgfsetbuttcap%
\pgfsetroundjoin%
\definecolor{currentfill}{rgb}{0.121569,0.466667,0.705882}%
\pgfsetfillcolor{currentfill}%
\pgfsetlinewidth{1.003750pt}%
\definecolor{currentstroke}{rgb}{0.121569,0.466667,0.705882}%
\pgfsetstrokecolor{currentstroke}%
\pgfsetdash{}{0pt}%
\pgfpathmoveto{\pgfqpoint{1.260585in}{0.695229in}}%
\pgfpathcurveto{\pgfqpoint{1.265457in}{0.695229in}}{\pgfqpoint{1.270131in}{0.697165in}}{\pgfqpoint{1.273577in}{0.700610in}}%
\pgfpathcurveto{\pgfqpoint{1.277022in}{0.704056in}}{\pgfqpoint{1.278958in}{0.708729in}}{\pgfqpoint{1.278958in}{0.713602in}}%
\pgfpathcurveto{\pgfqpoint{1.278958in}{0.718475in}}{\pgfqpoint{1.277022in}{0.723148in}}{\pgfqpoint{1.273577in}{0.726594in}}%
\pgfpathcurveto{\pgfqpoint{1.270131in}{0.730039in}}{\pgfqpoint{1.265457in}{0.731975in}}{\pgfqpoint{1.260585in}{0.731975in}}%
\pgfpathcurveto{\pgfqpoint{1.255712in}{0.731975in}}{\pgfqpoint{1.251038in}{0.730039in}}{\pgfqpoint{1.247593in}{0.726594in}}%
\pgfpathcurveto{\pgfqpoint{1.244147in}{0.723148in}}{\pgfqpoint{1.242212in}{0.718475in}}{\pgfqpoint{1.242212in}{0.713602in}}%
\pgfpathcurveto{\pgfqpoint{1.242212in}{0.708729in}}{\pgfqpoint{1.244147in}{0.704056in}}{\pgfqpoint{1.247593in}{0.700610in}}%
\pgfpathcurveto{\pgfqpoint{1.251038in}{0.697165in}}{\pgfqpoint{1.255712in}{0.695229in}}{\pgfqpoint{1.260585in}{0.695229in}}%
\pgfpathclose%
\pgfusepath{stroke,fill}%
\end{pgfscope}%
\begin{pgfscope}%
\pgfpathrectangle{\pgfqpoint{0.750000in}{0.500000in}}{\pgfqpoint{4.650000in}{3.020000in}}%
\pgfusepath{clip}%
\pgfsetbuttcap%
\pgfsetroundjoin%
\definecolor{currentfill}{rgb}{0.121569,0.466667,0.705882}%
\pgfsetfillcolor{currentfill}%
\pgfsetlinewidth{1.003750pt}%
\definecolor{currentstroke}{rgb}{0.121569,0.466667,0.705882}%
\pgfsetstrokecolor{currentstroke}%
\pgfsetdash{}{0pt}%
\pgfpathmoveto{\pgfqpoint{1.859027in}{0.963013in}}%
\pgfpathcurveto{\pgfqpoint{1.861632in}{0.963013in}}{\pgfqpoint{1.864130in}{0.964047in}}{\pgfqpoint{1.865972in}{0.965889in}}%
\pgfpathcurveto{\pgfqpoint{1.867813in}{0.967731in}}{\pgfqpoint{1.868848in}{0.970229in}}{\pgfqpoint{1.868848in}{0.972834in}}%
\pgfpathcurveto{\pgfqpoint{1.868848in}{0.975438in}}{\pgfqpoint{1.867813in}{0.977936in}}{\pgfqpoint{1.865972in}{0.979778in}}%
\pgfpathcurveto{\pgfqpoint{1.864130in}{0.981620in}}{\pgfqpoint{1.861632in}{0.982655in}}{\pgfqpoint{1.859027in}{0.982655in}}%
\pgfpathcurveto{\pgfqpoint{1.856423in}{0.982655in}}{\pgfqpoint{1.853924in}{0.981620in}}{\pgfqpoint{1.852083in}{0.979778in}}%
\pgfpathcurveto{\pgfqpoint{1.850241in}{0.977936in}}{\pgfqpoint{1.849206in}{0.975438in}}{\pgfqpoint{1.849206in}{0.972834in}}%
\pgfpathcurveto{\pgfqpoint{1.849206in}{0.970229in}}{\pgfqpoint{1.850241in}{0.967731in}}{\pgfqpoint{1.852083in}{0.965889in}}%
\pgfpathcurveto{\pgfqpoint{1.853924in}{0.964047in}}{\pgfqpoint{1.856423in}{0.963013in}}{\pgfqpoint{1.859027in}{0.963013in}}%
\pgfpathclose%
\pgfusepath{stroke,fill}%
\end{pgfscope}%
\begin{pgfscope}%
\pgfpathrectangle{\pgfqpoint{0.750000in}{0.500000in}}{\pgfqpoint{4.650000in}{3.020000in}}%
\pgfusepath{clip}%
\pgfsetbuttcap%
\pgfsetroundjoin%
\definecolor{currentfill}{rgb}{0.121569,0.466667,0.705882}%
\pgfsetfillcolor{currentfill}%
\pgfsetlinewidth{1.003750pt}%
\definecolor{currentstroke}{rgb}{0.121569,0.466667,0.705882}%
\pgfsetstrokecolor{currentstroke}%
\pgfsetdash{}{0pt}%
\pgfpathmoveto{\pgfqpoint{1.988518in}{0.944457in}}%
\pgfpathcurveto{\pgfqpoint{1.994897in}{0.944457in}}{\pgfqpoint{2.001017in}{0.946992in}}{\pgfqpoint{2.005528in}{0.951503in}}%
\pgfpathcurveto{\pgfqpoint{2.010039in}{0.956014in}}{\pgfqpoint{2.012574in}{0.962133in}}{\pgfqpoint{2.012574in}{0.968513in}}%
\pgfpathcurveto{\pgfqpoint{2.012574in}{0.974893in}}{\pgfqpoint{2.010039in}{0.981012in}}{\pgfqpoint{2.005528in}{0.985523in}}%
\pgfpathcurveto{\pgfqpoint{2.001017in}{0.990035in}}{\pgfqpoint{1.994897in}{0.992569in}}{\pgfqpoint{1.988518in}{0.992569in}}%
\pgfpathcurveto{\pgfqpoint{1.982138in}{0.992569in}}{\pgfqpoint{1.976019in}{0.990035in}}{\pgfqpoint{1.971507in}{0.985523in}}%
\pgfpathcurveto{\pgfqpoint{1.966996in}{0.981012in}}{\pgfqpoint{1.964461in}{0.974893in}}{\pgfqpoint{1.964461in}{0.968513in}}%
\pgfpathcurveto{\pgfqpoint{1.964461in}{0.962133in}}{\pgfqpoint{1.966996in}{0.956014in}}{\pgfqpoint{1.971507in}{0.951503in}}%
\pgfpathcurveto{\pgfqpoint{1.976019in}{0.946992in}}{\pgfqpoint{1.982138in}{0.944457in}}{\pgfqpoint{1.988518in}{0.944457in}}%
\pgfpathclose%
\pgfusepath{stroke,fill}%
\end{pgfscope}%
\begin{pgfscope}%
\pgfpathrectangle{\pgfqpoint{0.750000in}{0.500000in}}{\pgfqpoint{4.650000in}{3.020000in}}%
\pgfusepath{clip}%
\pgfsetbuttcap%
\pgfsetroundjoin%
\definecolor{currentfill}{rgb}{0.121569,0.466667,0.705882}%
\pgfsetfillcolor{currentfill}%
\pgfsetlinewidth{1.003750pt}%
\definecolor{currentstroke}{rgb}{0.121569,0.466667,0.705882}%
\pgfsetstrokecolor{currentstroke}%
\pgfsetdash{}{0pt}%
\pgfpathmoveto{\pgfqpoint{1.195582in}{0.648574in}}%
\pgfpathcurveto{\pgfqpoint{1.198186in}{0.648574in}}{\pgfqpoint{1.200684in}{0.649609in}}{\pgfqpoint{1.202526in}{0.651451in}}%
\pgfpathcurveto{\pgfqpoint{1.204368in}{0.653293in}}{\pgfqpoint{1.205403in}{0.655791in}}{\pgfqpoint{1.205403in}{0.658395in}}%
\pgfpathcurveto{\pgfqpoint{1.205403in}{0.661000in}}{\pgfqpoint{1.204368in}{0.663498in}}{\pgfqpoint{1.202526in}{0.665340in}}%
\pgfpathcurveto{\pgfqpoint{1.200684in}{0.667181in}}{\pgfqpoint{1.198186in}{0.668216in}}{\pgfqpoint{1.195582in}{0.668216in}}%
\pgfpathcurveto{\pgfqpoint{1.192977in}{0.668216in}}{\pgfqpoint{1.190479in}{0.667181in}}{\pgfqpoint{1.188637in}{0.665340in}}%
\pgfpathcurveto{\pgfqpoint{1.186795in}{0.663498in}}{\pgfqpoint{1.185761in}{0.661000in}}{\pgfqpoint{1.185761in}{0.658395in}}%
\pgfpathcurveto{\pgfqpoint{1.185761in}{0.655791in}}{\pgfqpoint{1.186795in}{0.653293in}}{\pgfqpoint{1.188637in}{0.651451in}}%
\pgfpathcurveto{\pgfqpoint{1.190479in}{0.649609in}}{\pgfqpoint{1.192977in}{0.648574in}}{\pgfqpoint{1.195582in}{0.648574in}}%
\pgfpathclose%
\pgfusepath{stroke,fill}%
\end{pgfscope}%
\begin{pgfscope}%
\pgfpathrectangle{\pgfqpoint{0.750000in}{0.500000in}}{\pgfqpoint{4.650000in}{3.020000in}}%
\pgfusepath{clip}%
\pgfsetbuttcap%
\pgfsetroundjoin%
\definecolor{currentfill}{rgb}{0.121569,0.466667,0.705882}%
\pgfsetfillcolor{currentfill}%
\pgfsetlinewidth{1.003750pt}%
\definecolor{currentstroke}{rgb}{0.121569,0.466667,0.705882}%
\pgfsetstrokecolor{currentstroke}%
\pgfsetdash{}{0pt}%
\pgfpathmoveto{\pgfqpoint{1.485001in}{0.740526in}}%
\pgfpathcurveto{\pgfqpoint{1.490210in}{0.740526in}}{\pgfqpoint{1.495206in}{0.742595in}}{\pgfqpoint{1.498890in}{0.746279in}}%
\pgfpathcurveto{\pgfqpoint{1.502573in}{0.749962in}}{\pgfqpoint{1.504643in}{0.754959in}}{\pgfqpoint{1.504643in}{0.760168in}}%
\pgfpathcurveto{\pgfqpoint{1.504643in}{0.765377in}}{\pgfqpoint{1.502573in}{0.770373in}}{\pgfqpoint{1.498890in}{0.774057in}}%
\pgfpathcurveto{\pgfqpoint{1.495206in}{0.777740in}}{\pgfqpoint{1.490210in}{0.779810in}}{\pgfqpoint{1.485001in}{0.779810in}}%
\pgfpathcurveto{\pgfqpoint{1.479792in}{0.779810in}}{\pgfqpoint{1.474795in}{0.777740in}}{\pgfqpoint{1.471112in}{0.774057in}}%
\pgfpathcurveto{\pgfqpoint{1.467428in}{0.770373in}}{\pgfqpoint{1.465359in}{0.765377in}}{\pgfqpoint{1.465359in}{0.760168in}}%
\pgfpathcurveto{\pgfqpoint{1.465359in}{0.754959in}}{\pgfqpoint{1.467428in}{0.749962in}}{\pgfqpoint{1.471112in}{0.746279in}}%
\pgfpathcurveto{\pgfqpoint{1.474795in}{0.742595in}}{\pgfqpoint{1.479792in}{0.740526in}}{\pgfqpoint{1.485001in}{0.740526in}}%
\pgfpathclose%
\pgfusepath{stroke,fill}%
\end{pgfscope}%
\begin{pgfscope}%
\pgfpathrectangle{\pgfqpoint{0.750000in}{0.500000in}}{\pgfqpoint{4.650000in}{3.020000in}}%
\pgfusepath{clip}%
\pgfsetbuttcap%
\pgfsetroundjoin%
\definecolor{currentfill}{rgb}{0.121569,0.466667,0.705882}%
\pgfsetfillcolor{currentfill}%
\pgfsetlinewidth{1.003750pt}%
\definecolor{currentstroke}{rgb}{0.121569,0.466667,0.705882}%
\pgfsetstrokecolor{currentstroke}%
\pgfsetdash{}{0pt}%
\pgfpathmoveto{\pgfqpoint{1.340033in}{0.661056in}}%
\pgfpathcurveto{\pgfqpoint{1.342638in}{0.661056in}}{\pgfqpoint{1.345136in}{0.662091in}}{\pgfqpoint{1.346978in}{0.663932in}}%
\pgfpathcurveto{\pgfqpoint{1.348819in}{0.665774in}}{\pgfqpoint{1.349854in}{0.668272in}}{\pgfqpoint{1.349854in}{0.670877in}}%
\pgfpathcurveto{\pgfqpoint{1.349854in}{0.673481in}}{\pgfqpoint{1.348819in}{0.675980in}}{\pgfqpoint{1.346978in}{0.677821in}}%
\pgfpathcurveto{\pgfqpoint{1.345136in}{0.679663in}}{\pgfqpoint{1.342638in}{0.680698in}}{\pgfqpoint{1.340033in}{0.680698in}}%
\pgfpathcurveto{\pgfqpoint{1.337429in}{0.680698in}}{\pgfqpoint{1.334930in}{0.679663in}}{\pgfqpoint{1.333089in}{0.677821in}}%
\pgfpathcurveto{\pgfqpoint{1.331247in}{0.675980in}}{\pgfqpoint{1.330212in}{0.673481in}}{\pgfqpoint{1.330212in}{0.670877in}}%
\pgfpathcurveto{\pgfqpoint{1.330212in}{0.668272in}}{\pgfqpoint{1.331247in}{0.665774in}}{\pgfqpoint{1.333089in}{0.663932in}}%
\pgfpathcurveto{\pgfqpoint{1.334930in}{0.662091in}}{\pgfqpoint{1.337429in}{0.661056in}}{\pgfqpoint{1.340033in}{0.661056in}}%
\pgfpathclose%
\pgfusepath{stroke,fill}%
\end{pgfscope}%
\begin{pgfscope}%
\pgfpathrectangle{\pgfqpoint{0.750000in}{0.500000in}}{\pgfqpoint{4.650000in}{3.020000in}}%
\pgfusepath{clip}%
\pgfsetbuttcap%
\pgfsetroundjoin%
\definecolor{currentfill}{rgb}{0.121569,0.466667,0.705882}%
\pgfsetfillcolor{currentfill}%
\pgfsetlinewidth{1.003750pt}%
\definecolor{currentstroke}{rgb}{0.121569,0.466667,0.705882}%
\pgfsetstrokecolor{currentstroke}%
\pgfsetdash{}{0pt}%
\pgfpathmoveto{\pgfqpoint{1.014501in}{0.628892in}}%
\pgfpathcurveto{\pgfqpoint{1.017106in}{0.628892in}}{\pgfqpoint{1.019604in}{0.629927in}}{\pgfqpoint{1.021446in}{0.631768in}}%
\pgfpathcurveto{\pgfqpoint{1.023287in}{0.633610in}}{\pgfqpoint{1.024322in}{0.636108in}}{\pgfqpoint{1.024322in}{0.638713in}}%
\pgfpathcurveto{\pgfqpoint{1.024322in}{0.641317in}}{\pgfqpoint{1.023287in}{0.643816in}}{\pgfqpoint{1.021446in}{0.645657in}}%
\pgfpathcurveto{\pgfqpoint{1.019604in}{0.647499in}}{\pgfqpoint{1.017106in}{0.648534in}}{\pgfqpoint{1.014501in}{0.648534in}}%
\pgfpathcurveto{\pgfqpoint{1.011897in}{0.648534in}}{\pgfqpoint{1.009398in}{0.647499in}}{\pgfqpoint{1.007557in}{0.645657in}}%
\pgfpathcurveto{\pgfqpoint{1.005715in}{0.643816in}}{\pgfqpoint{1.004680in}{0.641317in}}{\pgfqpoint{1.004680in}{0.638713in}}%
\pgfpathcurveto{\pgfqpoint{1.004680in}{0.636108in}}{\pgfqpoint{1.005715in}{0.633610in}}{\pgfqpoint{1.007557in}{0.631768in}}%
\pgfpathcurveto{\pgfqpoint{1.009398in}{0.629927in}}{\pgfqpoint{1.011897in}{0.628892in}}{\pgfqpoint{1.014501in}{0.628892in}}%
\pgfpathclose%
\pgfusepath{stroke,fill}%
\end{pgfscope}%
\begin{pgfscope}%
\pgfpathrectangle{\pgfqpoint{0.750000in}{0.500000in}}{\pgfqpoint{4.650000in}{3.020000in}}%
\pgfusepath{clip}%
\pgfsetbuttcap%
\pgfsetroundjoin%
\definecolor{currentfill}{rgb}{0.121569,0.466667,0.705882}%
\pgfsetfillcolor{currentfill}%
\pgfsetlinewidth{1.003750pt}%
\definecolor{currentstroke}{rgb}{0.121569,0.466667,0.705882}%
\pgfsetstrokecolor{currentstroke}%
\pgfsetdash{}{0pt}%
\pgfpathmoveto{\pgfqpoint{1.221892in}{0.676198in}}%
\pgfpathcurveto{\pgfqpoint{1.227101in}{0.676198in}}{\pgfqpoint{1.232098in}{0.678268in}}{\pgfqpoint{1.235781in}{0.681951in}}%
\pgfpathcurveto{\pgfqpoint{1.239465in}{0.685634in}}{\pgfqpoint{1.241534in}{0.690631in}}{\pgfqpoint{1.241534in}{0.695840in}}%
\pgfpathcurveto{\pgfqpoint{1.241534in}{0.701049in}}{\pgfqpoint{1.239465in}{0.706045in}}{\pgfqpoint{1.235781in}{0.709729in}}%
\pgfpathcurveto{\pgfqpoint{1.232098in}{0.713412in}}{\pgfqpoint{1.227101in}{0.715482in}}{\pgfqpoint{1.221892in}{0.715482in}}%
\pgfpathcurveto{\pgfqpoint{1.216683in}{0.715482in}}{\pgfqpoint{1.211687in}{0.713412in}}{\pgfqpoint{1.208004in}{0.709729in}}%
\pgfpathcurveto{\pgfqpoint{1.204320in}{0.706045in}}{\pgfqpoint{1.202251in}{0.701049in}}{\pgfqpoint{1.202251in}{0.695840in}}%
\pgfpathcurveto{\pgfqpoint{1.202251in}{0.690631in}}{\pgfqpoint{1.204320in}{0.685634in}}{\pgfqpoint{1.208004in}{0.681951in}}%
\pgfpathcurveto{\pgfqpoint{1.211687in}{0.678268in}}{\pgfqpoint{1.216683in}{0.676198in}}{\pgfqpoint{1.221892in}{0.676198in}}%
\pgfpathclose%
\pgfusepath{stroke,fill}%
\end{pgfscope}%
\begin{pgfscope}%
\pgfpathrectangle{\pgfqpoint{0.750000in}{0.500000in}}{\pgfqpoint{4.650000in}{3.020000in}}%
\pgfusepath{clip}%
\pgfsetbuttcap%
\pgfsetroundjoin%
\definecolor{currentfill}{rgb}{0.121569,0.466667,0.705882}%
\pgfsetfillcolor{currentfill}%
\pgfsetlinewidth{1.003750pt}%
\definecolor{currentstroke}{rgb}{0.121569,0.466667,0.705882}%
\pgfsetstrokecolor{currentstroke}%
\pgfsetdash{}{0pt}%
\pgfpathmoveto{\pgfqpoint{1.656795in}{0.722696in}}%
\pgfpathcurveto{\pgfqpoint{1.659985in}{0.722696in}}{\pgfqpoint{1.663044in}{0.723964in}}{\pgfqpoint{1.665300in}{0.726219in}}%
\pgfpathcurveto{\pgfqpoint{1.667556in}{0.728475in}}{\pgfqpoint{1.668823in}{0.731535in}}{\pgfqpoint{1.668823in}{0.734725in}}%
\pgfpathcurveto{\pgfqpoint{1.668823in}{0.737914in}}{\pgfqpoint{1.667556in}{0.740974in}}{\pgfqpoint{1.665300in}{0.743230in}}%
\pgfpathcurveto{\pgfqpoint{1.663044in}{0.745485in}}{\pgfqpoint{1.659985in}{0.746753in}}{\pgfqpoint{1.656795in}{0.746753in}}%
\pgfpathcurveto{\pgfqpoint{1.653605in}{0.746753in}}{\pgfqpoint{1.650545in}{0.745485in}}{\pgfqpoint{1.648290in}{0.743230in}}%
\pgfpathcurveto{\pgfqpoint{1.646034in}{0.740974in}}{\pgfqpoint{1.644767in}{0.737914in}}{\pgfqpoint{1.644767in}{0.734725in}}%
\pgfpathcurveto{\pgfqpoint{1.644767in}{0.731535in}}{\pgfqpoint{1.646034in}{0.728475in}}{\pgfqpoint{1.648290in}{0.726219in}}%
\pgfpathcurveto{\pgfqpoint{1.650545in}{0.723964in}}{\pgfqpoint{1.653605in}{0.722696in}}{\pgfqpoint{1.656795in}{0.722696in}}%
\pgfpathclose%
\pgfusepath{stroke,fill}%
\end{pgfscope}%
\begin{pgfscope}%
\pgfpathrectangle{\pgfqpoint{0.750000in}{0.500000in}}{\pgfqpoint{4.650000in}{3.020000in}}%
\pgfusepath{clip}%
\pgfsetbuttcap%
\pgfsetroundjoin%
\definecolor{currentfill}{rgb}{0.121569,0.466667,0.705882}%
\pgfsetfillcolor{currentfill}%
\pgfsetlinewidth{1.003750pt}%
\definecolor{currentstroke}{rgb}{0.121569,0.466667,0.705882}%
\pgfsetstrokecolor{currentstroke}%
\pgfsetdash{}{0pt}%
\pgfpathmoveto{\pgfqpoint{1.360153in}{0.646654in}}%
\pgfpathcurveto{\pgfqpoint{1.362758in}{0.646654in}}{\pgfqpoint{1.365256in}{0.647689in}}{\pgfqpoint{1.367098in}{0.649531in}}%
\pgfpathcurveto{\pgfqpoint{1.368939in}{0.651372in}}{\pgfqpoint{1.369974in}{0.653871in}}{\pgfqpoint{1.369974in}{0.656475in}}%
\pgfpathcurveto{\pgfqpoint{1.369974in}{0.659080in}}{\pgfqpoint{1.368939in}{0.661578in}}{\pgfqpoint{1.367098in}{0.663420in}}%
\pgfpathcurveto{\pgfqpoint{1.365256in}{0.665261in}}{\pgfqpoint{1.362758in}{0.666296in}}{\pgfqpoint{1.360153in}{0.666296in}}%
\pgfpathcurveto{\pgfqpoint{1.357549in}{0.666296in}}{\pgfqpoint{1.355050in}{0.665261in}}{\pgfqpoint{1.353209in}{0.663420in}}%
\pgfpathcurveto{\pgfqpoint{1.351367in}{0.661578in}}{\pgfqpoint{1.350332in}{0.659080in}}{\pgfqpoint{1.350332in}{0.656475in}}%
\pgfpathcurveto{\pgfqpoint{1.350332in}{0.653871in}}{\pgfqpoint{1.351367in}{0.651372in}}{\pgfqpoint{1.353209in}{0.649531in}}%
\pgfpathcurveto{\pgfqpoint{1.355050in}{0.647689in}}{\pgfqpoint{1.357549in}{0.646654in}}{\pgfqpoint{1.360153in}{0.646654in}}%
\pgfpathclose%
\pgfusepath{stroke,fill}%
\end{pgfscope}%
\begin{pgfscope}%
\pgfpathrectangle{\pgfqpoint{0.750000in}{0.500000in}}{\pgfqpoint{4.650000in}{3.020000in}}%
\pgfusepath{clip}%
\pgfsetbuttcap%
\pgfsetroundjoin%
\definecolor{currentfill}{rgb}{0.121569,0.466667,0.705882}%
\pgfsetfillcolor{currentfill}%
\pgfsetlinewidth{1.003750pt}%
\definecolor{currentstroke}{rgb}{0.121569,0.466667,0.705882}%
\pgfsetstrokecolor{currentstroke}%
\pgfsetdash{}{0pt}%
\pgfpathmoveto{\pgfqpoint{1.130062in}{0.634653in}}%
\pgfpathcurveto{\pgfqpoint{1.132667in}{0.634653in}}{\pgfqpoint{1.135165in}{0.635687in}}{\pgfqpoint{1.137007in}{0.637529in}}%
\pgfpathcurveto{\pgfqpoint{1.138849in}{0.639371in}}{\pgfqpoint{1.139883in}{0.641869in}}{\pgfqpoint{1.139883in}{0.644474in}}%
\pgfpathcurveto{\pgfqpoint{1.139883in}{0.647078in}}{\pgfqpoint{1.138849in}{0.649576in}}{\pgfqpoint{1.137007in}{0.651418in}}%
\pgfpathcurveto{\pgfqpoint{1.135165in}{0.653260in}}{\pgfqpoint{1.132667in}{0.654295in}}{\pgfqpoint{1.130062in}{0.654295in}}%
\pgfpathcurveto{\pgfqpoint{1.127458in}{0.654295in}}{\pgfqpoint{1.124960in}{0.653260in}}{\pgfqpoint{1.123118in}{0.651418in}}%
\pgfpathcurveto{\pgfqpoint{1.121276in}{0.649576in}}{\pgfqpoint{1.120242in}{0.647078in}}{\pgfqpoint{1.120242in}{0.644474in}}%
\pgfpathcurveto{\pgfqpoint{1.120242in}{0.641869in}}{\pgfqpoint{1.121276in}{0.639371in}}{\pgfqpoint{1.123118in}{0.637529in}}%
\pgfpathcurveto{\pgfqpoint{1.124960in}{0.635687in}}{\pgfqpoint{1.127458in}{0.634653in}}{\pgfqpoint{1.130062in}{0.634653in}}%
\pgfpathclose%
\pgfusepath{stroke,fill}%
\end{pgfscope}%
\begin{pgfscope}%
\pgfpathrectangle{\pgfqpoint{0.750000in}{0.500000in}}{\pgfqpoint{4.650000in}{3.020000in}}%
\pgfusepath{clip}%
\pgfsetbuttcap%
\pgfsetroundjoin%
\definecolor{currentfill}{rgb}{0.121569,0.466667,0.705882}%
\pgfsetfillcolor{currentfill}%
\pgfsetlinewidth{1.003750pt}%
\definecolor{currentstroke}{rgb}{0.121569,0.466667,0.705882}%
\pgfsetstrokecolor{currentstroke}%
\pgfsetdash{}{0pt}%
\pgfpathmoveto{\pgfqpoint{1.103236in}{0.777710in}}%
\pgfpathcurveto{\pgfqpoint{1.105840in}{0.777710in}}{\pgfqpoint{1.108338in}{0.778745in}}{\pgfqpoint{1.110180in}{0.780587in}}%
\pgfpathcurveto{\pgfqpoint{1.112022in}{0.782428in}}{\pgfqpoint{1.113057in}{0.784926in}}{\pgfqpoint{1.113057in}{0.787531in}}%
\pgfpathcurveto{\pgfqpoint{1.113057in}{0.790136in}}{\pgfqpoint{1.112022in}{0.792634in}}{\pgfqpoint{1.110180in}{0.794475in}}%
\pgfpathcurveto{\pgfqpoint{1.108338in}{0.796317in}}{\pgfqpoint{1.105840in}{0.797352in}}{\pgfqpoint{1.103236in}{0.797352in}}%
\pgfpathcurveto{\pgfqpoint{1.100631in}{0.797352in}}{\pgfqpoint{1.098133in}{0.796317in}}{\pgfqpoint{1.096291in}{0.794475in}}%
\pgfpathcurveto{\pgfqpoint{1.094450in}{0.792634in}}{\pgfqpoint{1.093415in}{0.790136in}}{\pgfqpoint{1.093415in}{0.787531in}}%
\pgfpathcurveto{\pgfqpoint{1.093415in}{0.784926in}}{\pgfqpoint{1.094450in}{0.782428in}}{\pgfqpoint{1.096291in}{0.780587in}}%
\pgfpathcurveto{\pgfqpoint{1.098133in}{0.778745in}}{\pgfqpoint{1.100631in}{0.777710in}}{\pgfqpoint{1.103236in}{0.777710in}}%
\pgfpathclose%
\pgfusepath{stroke,fill}%
\end{pgfscope}%
\begin{pgfscope}%
\pgfpathrectangle{\pgfqpoint{0.750000in}{0.500000in}}{\pgfqpoint{4.650000in}{3.020000in}}%
\pgfusepath{clip}%
\pgfsetbuttcap%
\pgfsetroundjoin%
\definecolor{currentfill}{rgb}{0.121569,0.466667,0.705882}%
\pgfsetfillcolor{currentfill}%
\pgfsetlinewidth{1.003750pt}%
\definecolor{currentstroke}{rgb}{0.121569,0.466667,0.705882}%
\pgfsetstrokecolor{currentstroke}%
\pgfsetdash{}{0pt}%
\pgfpathmoveto{\pgfqpoint{1.414323in}{0.684590in}}%
\pgfpathcurveto{\pgfqpoint{1.418834in}{0.684590in}}{\pgfqpoint{1.423161in}{0.686383in}}{\pgfqpoint{1.426351in}{0.689572in}}%
\pgfpathcurveto{\pgfqpoint{1.429541in}{0.692762in}}{\pgfqpoint{1.431333in}{0.697089in}}{\pgfqpoint{1.431333in}{0.701601in}}%
\pgfpathcurveto{\pgfqpoint{1.431333in}{0.706112in}}{\pgfqpoint{1.429541in}{0.710439in}}{\pgfqpoint{1.426351in}{0.713629in}}%
\pgfpathcurveto{\pgfqpoint{1.423161in}{0.716819in}}{\pgfqpoint{1.418834in}{0.718611in}}{\pgfqpoint{1.414323in}{0.718611in}}%
\pgfpathcurveto{\pgfqpoint{1.409811in}{0.718611in}}{\pgfqpoint{1.405484in}{0.716819in}}{\pgfqpoint{1.402294in}{0.713629in}}%
\pgfpathcurveto{\pgfqpoint{1.399105in}{0.710439in}}{\pgfqpoint{1.397312in}{0.706112in}}{\pgfqpoint{1.397312in}{0.701601in}}%
\pgfpathcurveto{\pgfqpoint{1.397312in}{0.697089in}}{\pgfqpoint{1.399105in}{0.692762in}}{\pgfqpoint{1.402294in}{0.689572in}}%
\pgfpathcurveto{\pgfqpoint{1.405484in}{0.686383in}}{\pgfqpoint{1.409811in}{0.684590in}}{\pgfqpoint{1.414323in}{0.684590in}}%
\pgfpathclose%
\pgfusepath{stroke,fill}%
\end{pgfscope}%
\begin{pgfscope}%
\pgfpathrectangle{\pgfqpoint{0.750000in}{0.500000in}}{\pgfqpoint{4.650000in}{3.020000in}}%
\pgfusepath{clip}%
\pgfsetbuttcap%
\pgfsetroundjoin%
\definecolor{currentfill}{rgb}{0.121569,0.466667,0.705882}%
\pgfsetfillcolor{currentfill}%
\pgfsetlinewidth{1.003750pt}%
\definecolor{currentstroke}{rgb}{0.121569,0.466667,0.705882}%
\pgfsetstrokecolor{currentstroke}%
\pgfsetdash{}{0pt}%
\pgfpathmoveto{\pgfqpoint{1.055773in}{0.632252in}}%
\pgfpathcurveto{\pgfqpoint{1.058378in}{0.632252in}}{\pgfqpoint{1.060876in}{0.633287in}}{\pgfqpoint{1.062718in}{0.635129in}}%
\pgfpathcurveto{\pgfqpoint{1.064559in}{0.636971in}}{\pgfqpoint{1.065594in}{0.639469in}}{\pgfqpoint{1.065594in}{0.642073in}}%
\pgfpathcurveto{\pgfqpoint{1.065594in}{0.644678in}}{\pgfqpoint{1.064559in}{0.647176in}}{\pgfqpoint{1.062718in}{0.649018in}}%
\pgfpathcurveto{\pgfqpoint{1.060876in}{0.650859in}}{\pgfqpoint{1.058378in}{0.651894in}}{\pgfqpoint{1.055773in}{0.651894in}}%
\pgfpathcurveto{\pgfqpoint{1.053169in}{0.651894in}}{\pgfqpoint{1.050670in}{0.650859in}}{\pgfqpoint{1.048829in}{0.649018in}}%
\pgfpathcurveto{\pgfqpoint{1.046987in}{0.647176in}}{\pgfqpoint{1.045952in}{0.644678in}}{\pgfqpoint{1.045952in}{0.642073in}}%
\pgfpathcurveto{\pgfqpoint{1.045952in}{0.639469in}}{\pgfqpoint{1.046987in}{0.636971in}}{\pgfqpoint{1.048829in}{0.635129in}}%
\pgfpathcurveto{\pgfqpoint{1.050670in}{0.633287in}}{\pgfqpoint{1.053169in}{0.632252in}}{\pgfqpoint{1.055773in}{0.632252in}}%
\pgfpathclose%
\pgfusepath{stroke,fill}%
\end{pgfscope}%
\begin{pgfscope}%
\pgfpathrectangle{\pgfqpoint{0.750000in}{0.500000in}}{\pgfqpoint{4.650000in}{3.020000in}}%
\pgfusepath{clip}%
\pgfsetbuttcap%
\pgfsetroundjoin%
\definecolor{currentfill}{rgb}{0.121569,0.466667,0.705882}%
\pgfsetfillcolor{currentfill}%
\pgfsetlinewidth{1.003750pt}%
\definecolor{currentstroke}{rgb}{0.121569,0.466667,0.705882}%
\pgfsetstrokecolor{currentstroke}%
\pgfsetdash{}{0pt}%
\pgfpathmoveto{\pgfqpoint{0.997992in}{0.627452in}}%
\pgfpathcurveto{\pgfqpoint{1.000597in}{0.627452in}}{\pgfqpoint{1.003095in}{0.628487in}}{\pgfqpoint{1.004937in}{0.630328in}}%
\pgfpathcurveto{\pgfqpoint{1.006779in}{0.632170in}}{\pgfqpoint{1.007813in}{0.634668in}}{\pgfqpoint{1.007813in}{0.637273in}}%
\pgfpathcurveto{\pgfqpoint{1.007813in}{0.639877in}}{\pgfqpoint{1.006779in}{0.642375in}}{\pgfqpoint{1.004937in}{0.644217in}}%
\pgfpathcurveto{\pgfqpoint{1.003095in}{0.646059in}}{\pgfqpoint{1.000597in}{0.647094in}}{\pgfqpoint{0.997992in}{0.647094in}}%
\pgfpathcurveto{\pgfqpoint{0.995388in}{0.647094in}}{\pgfqpoint{0.992890in}{0.646059in}}{\pgfqpoint{0.991048in}{0.644217in}}%
\pgfpathcurveto{\pgfqpoint{0.989206in}{0.642375in}}{\pgfqpoint{0.988172in}{0.639877in}}{\pgfqpoint{0.988172in}{0.637273in}}%
\pgfpathcurveto{\pgfqpoint{0.988172in}{0.634668in}}{\pgfqpoint{0.989206in}{0.632170in}}{\pgfqpoint{0.991048in}{0.630328in}}%
\pgfpathcurveto{\pgfqpoint{0.992890in}{0.628487in}}{\pgfqpoint{0.995388in}{0.627452in}}{\pgfqpoint{0.997992in}{0.627452in}}%
\pgfpathclose%
\pgfusepath{stroke,fill}%
\end{pgfscope}%
\begin{pgfscope}%
\pgfpathrectangle{\pgfqpoint{0.750000in}{0.500000in}}{\pgfqpoint{4.650000in}{3.020000in}}%
\pgfusepath{clip}%
\pgfsetbuttcap%
\pgfsetroundjoin%
\definecolor{currentfill}{rgb}{0.121569,0.466667,0.705882}%
\pgfsetfillcolor{currentfill}%
\pgfsetlinewidth{1.003750pt}%
\definecolor{currentstroke}{rgb}{0.121569,0.466667,0.705882}%
\pgfsetstrokecolor{currentstroke}%
\pgfsetdash{}{0pt}%
\pgfpathmoveto{\pgfqpoint{1.039780in}{0.627932in}}%
\pgfpathcurveto{\pgfqpoint{1.042385in}{0.627932in}}{\pgfqpoint{1.044883in}{0.628967in}}{\pgfqpoint{1.046725in}{0.630808in}}%
\pgfpathcurveto{\pgfqpoint{1.048566in}{0.632650in}}{\pgfqpoint{1.049601in}{0.635148in}}{\pgfqpoint{1.049601in}{0.637753in}}%
\pgfpathcurveto{\pgfqpoint{1.049601in}{0.640357in}}{\pgfqpoint{1.048566in}{0.642856in}}{\pgfqpoint{1.046725in}{0.644697in}}%
\pgfpathcurveto{\pgfqpoint{1.044883in}{0.646539in}}{\pgfqpoint{1.042385in}{0.647574in}}{\pgfqpoint{1.039780in}{0.647574in}}%
\pgfpathcurveto{\pgfqpoint{1.037176in}{0.647574in}}{\pgfqpoint{1.034677in}{0.646539in}}{\pgfqpoint{1.032836in}{0.644697in}}%
\pgfpathcurveto{\pgfqpoint{1.030994in}{0.642856in}}{\pgfqpoint{1.029959in}{0.640357in}}{\pgfqpoint{1.029959in}{0.637753in}}%
\pgfpathcurveto{\pgfqpoint{1.029959in}{0.635148in}}{\pgfqpoint{1.030994in}{0.632650in}}{\pgfqpoint{1.032836in}{0.630808in}}%
\pgfpathcurveto{\pgfqpoint{1.034677in}{0.628967in}}{\pgfqpoint{1.037176in}{0.627932in}}{\pgfqpoint{1.039780in}{0.627932in}}%
\pgfpathclose%
\pgfusepath{stroke,fill}%
\end{pgfscope}%
\begin{pgfscope}%
\pgfpathrectangle{\pgfqpoint{0.750000in}{0.500000in}}{\pgfqpoint{4.650000in}{3.020000in}}%
\pgfusepath{clip}%
\pgfsetbuttcap%
\pgfsetroundjoin%
\definecolor{currentfill}{rgb}{0.121569,0.466667,0.705882}%
\pgfsetfillcolor{currentfill}%
\pgfsetlinewidth{1.003750pt}%
\definecolor{currentstroke}{rgb}{0.121569,0.466667,0.705882}%
\pgfsetstrokecolor{currentstroke}%
\pgfsetdash{}{0pt}%
\pgfpathmoveto{\pgfqpoint{1.560838in}{0.686072in}}%
\pgfpathcurveto{\pgfqpoint{1.564956in}{0.686072in}}{\pgfqpoint{1.568906in}{0.687708in}}{\pgfqpoint{1.571818in}{0.690620in}}%
\pgfpathcurveto{\pgfqpoint{1.574730in}{0.693532in}}{\pgfqpoint{1.576366in}{0.697482in}}{\pgfqpoint{1.576366in}{0.701601in}}%
\pgfpathcurveto{\pgfqpoint{1.576366in}{0.705719in}}{\pgfqpoint{1.574730in}{0.709669in}}{\pgfqpoint{1.571818in}{0.712581in}}%
\pgfpathcurveto{\pgfqpoint{1.568906in}{0.715493in}}{\pgfqpoint{1.564956in}{0.717129in}}{\pgfqpoint{1.560838in}{0.717129in}}%
\pgfpathcurveto{\pgfqpoint{1.556720in}{0.717129in}}{\pgfqpoint{1.552770in}{0.715493in}}{\pgfqpoint{1.549858in}{0.712581in}}%
\pgfpathcurveto{\pgfqpoint{1.546946in}{0.709669in}}{\pgfqpoint{1.545310in}{0.705719in}}{\pgfqpoint{1.545310in}{0.701601in}}%
\pgfpathcurveto{\pgfqpoint{1.545310in}{0.697482in}}{\pgfqpoint{1.546946in}{0.693532in}}{\pgfqpoint{1.549858in}{0.690620in}}%
\pgfpathcurveto{\pgfqpoint{1.552770in}{0.687708in}}{\pgfqpoint{1.556720in}{0.686072in}}{\pgfqpoint{1.560838in}{0.686072in}}%
\pgfpathclose%
\pgfusepath{stroke,fill}%
\end{pgfscope}%
\begin{pgfscope}%
\pgfpathrectangle{\pgfqpoint{0.750000in}{0.500000in}}{\pgfqpoint{4.650000in}{3.020000in}}%
\pgfusepath{clip}%
\pgfsetbuttcap%
\pgfsetroundjoin%
\definecolor{currentfill}{rgb}{0.121569,0.466667,0.705882}%
\pgfsetfillcolor{currentfill}%
\pgfsetlinewidth{1.003750pt}%
\definecolor{currentstroke}{rgb}{0.121569,0.466667,0.705882}%
\pgfsetstrokecolor{currentstroke}%
\pgfsetdash{}{0pt}%
\pgfpathmoveto{\pgfqpoint{1.872956in}{0.780199in}}%
\pgfpathcurveto{\pgfqpoint{1.877829in}{0.780199in}}{\pgfqpoint{1.882503in}{0.782135in}}{\pgfqpoint{1.885948in}{0.785581in}}%
\pgfpathcurveto{\pgfqpoint{1.889394in}{0.789026in}}{\pgfqpoint{1.891330in}{0.793700in}}{\pgfqpoint{1.891330in}{0.798572in}}%
\pgfpathcurveto{\pgfqpoint{1.891330in}{0.803445in}}{\pgfqpoint{1.889394in}{0.808119in}}{\pgfqpoint{1.885948in}{0.811564in}}%
\pgfpathcurveto{\pgfqpoint{1.882503in}{0.815010in}}{\pgfqpoint{1.877829in}{0.816946in}}{\pgfqpoint{1.872956in}{0.816946in}}%
\pgfpathcurveto{\pgfqpoint{1.868084in}{0.816946in}}{\pgfqpoint{1.863410in}{0.815010in}}{\pgfqpoint{1.859965in}{0.811564in}}%
\pgfpathcurveto{\pgfqpoint{1.856519in}{0.808119in}}{\pgfqpoint{1.854583in}{0.803445in}}{\pgfqpoint{1.854583in}{0.798572in}}%
\pgfpathcurveto{\pgfqpoint{1.854583in}{0.793700in}}{\pgfqpoint{1.856519in}{0.789026in}}{\pgfqpoint{1.859965in}{0.785581in}}%
\pgfpathcurveto{\pgfqpoint{1.863410in}{0.782135in}}{\pgfqpoint{1.868084in}{0.780199in}}{\pgfqpoint{1.872956in}{0.780199in}}%
\pgfpathclose%
\pgfusepath{stroke,fill}%
\end{pgfscope}%
\begin{pgfscope}%
\pgfpathrectangle{\pgfqpoint{0.750000in}{0.500000in}}{\pgfqpoint{4.650000in}{3.020000in}}%
\pgfusepath{clip}%
\pgfsetbuttcap%
\pgfsetroundjoin%
\definecolor{currentfill}{rgb}{0.121569,0.466667,0.705882}%
\pgfsetfillcolor{currentfill}%
\pgfsetlinewidth{1.003750pt}%
\definecolor{currentstroke}{rgb}{0.121569,0.466667,0.705882}%
\pgfsetstrokecolor{currentstroke}%
\pgfsetdash{}{0pt}%
\pgfpathmoveto{\pgfqpoint{2.196425in}{0.809405in}}%
\pgfpathcurveto{\pgfqpoint{2.200936in}{0.809405in}}{\pgfqpoint{2.205263in}{0.811198in}}{\pgfqpoint{2.208453in}{0.814388in}}%
\pgfpathcurveto{\pgfqpoint{2.211643in}{0.817578in}}{\pgfqpoint{2.213435in}{0.821905in}}{\pgfqpoint{2.213435in}{0.826416in}}%
\pgfpathcurveto{\pgfqpoint{2.213435in}{0.830927in}}{\pgfqpoint{2.211643in}{0.835254in}}{\pgfqpoint{2.208453in}{0.838444in}}%
\pgfpathcurveto{\pgfqpoint{2.205263in}{0.841634in}}{\pgfqpoint{2.200936in}{0.843426in}}{\pgfqpoint{2.196425in}{0.843426in}}%
\pgfpathcurveto{\pgfqpoint{2.191914in}{0.843426in}}{\pgfqpoint{2.187587in}{0.841634in}}{\pgfqpoint{2.184397in}{0.838444in}}%
\pgfpathcurveto{\pgfqpoint{2.181207in}{0.835254in}}{\pgfqpoint{2.179414in}{0.830927in}}{\pgfqpoint{2.179414in}{0.826416in}}%
\pgfpathcurveto{\pgfqpoint{2.179414in}{0.821905in}}{\pgfqpoint{2.181207in}{0.817578in}}{\pgfqpoint{2.184397in}{0.814388in}}%
\pgfpathcurveto{\pgfqpoint{2.187587in}{0.811198in}}{\pgfqpoint{2.191914in}{0.809405in}}{\pgfqpoint{2.196425in}{0.809405in}}%
\pgfpathclose%
\pgfusepath{stroke,fill}%
\end{pgfscope}%
\begin{pgfscope}%
\pgfpathrectangle{\pgfqpoint{0.750000in}{0.500000in}}{\pgfqpoint{4.650000in}{3.020000in}}%
\pgfusepath{clip}%
\pgfsetbuttcap%
\pgfsetroundjoin%
\definecolor{currentfill}{rgb}{0.121569,0.466667,0.705882}%
\pgfsetfillcolor{currentfill}%
\pgfsetlinewidth{1.003750pt}%
\definecolor{currentstroke}{rgb}{0.121569,0.466667,0.705882}%
\pgfsetstrokecolor{currentstroke}%
\pgfsetdash{}{0pt}%
\pgfpathmoveto{\pgfqpoint{1.197645in}{0.638266in}}%
\pgfpathcurveto{\pgfqpoint{1.201329in}{0.638266in}}{\pgfqpoint{1.204862in}{0.639729in}}{\pgfqpoint{1.207466in}{0.642334in}}%
\pgfpathcurveto{\pgfqpoint{1.210071in}{0.644938in}}{\pgfqpoint{1.211534in}{0.648471in}}{\pgfqpoint{1.211534in}{0.652155in}}%
\pgfpathcurveto{\pgfqpoint{1.211534in}{0.655838in}}{\pgfqpoint{1.210071in}{0.659371in}}{\pgfqpoint{1.207466in}{0.661975in}}%
\pgfpathcurveto{\pgfqpoint{1.204862in}{0.664580in}}{\pgfqpoint{1.201329in}{0.666043in}}{\pgfqpoint{1.197645in}{0.666043in}}%
\pgfpathcurveto{\pgfqpoint{1.193962in}{0.666043in}}{\pgfqpoint{1.190429in}{0.664580in}}{\pgfqpoint{1.187824in}{0.661975in}}%
\pgfpathcurveto{\pgfqpoint{1.185220in}{0.659371in}}{\pgfqpoint{1.183756in}{0.655838in}}{\pgfqpoint{1.183756in}{0.652155in}}%
\pgfpathcurveto{\pgfqpoint{1.183756in}{0.648471in}}{\pgfqpoint{1.185220in}{0.644938in}}{\pgfqpoint{1.187824in}{0.642334in}}%
\pgfpathcurveto{\pgfqpoint{1.190429in}{0.639729in}}{\pgfqpoint{1.193962in}{0.638266in}}{\pgfqpoint{1.197645in}{0.638266in}}%
\pgfpathclose%
\pgfusepath{stroke,fill}%
\end{pgfscope}%
\begin{pgfscope}%
\pgfpathrectangle{\pgfqpoint{0.750000in}{0.500000in}}{\pgfqpoint{4.650000in}{3.020000in}}%
\pgfusepath{clip}%
\pgfsetbuttcap%
\pgfsetroundjoin%
\definecolor{currentfill}{rgb}{0.121569,0.466667,0.705882}%
\pgfsetfillcolor{currentfill}%
\pgfsetlinewidth{1.003750pt}%
\definecolor{currentstroke}{rgb}{0.121569,0.466667,0.705882}%
\pgfsetstrokecolor{currentstroke}%
\pgfsetdash{}{0pt}%
\pgfpathmoveto{\pgfqpoint{1.603657in}{0.742859in}}%
\pgfpathcurveto{\pgfqpoint{1.606847in}{0.742859in}}{\pgfqpoint{1.609907in}{0.744126in}}{\pgfqpoint{1.612163in}{0.746382in}}%
\pgfpathcurveto{\pgfqpoint{1.614418in}{0.748637in}}{\pgfqpoint{1.615685in}{0.751697in}}{\pgfqpoint{1.615685in}{0.754887in}}%
\pgfpathcurveto{\pgfqpoint{1.615685in}{0.758077in}}{\pgfqpoint{1.614418in}{0.761137in}}{\pgfqpoint{1.612163in}{0.763392in}}%
\pgfpathcurveto{\pgfqpoint{1.609907in}{0.765648in}}{\pgfqpoint{1.606847in}{0.766915in}}{\pgfqpoint{1.603657in}{0.766915in}}%
\pgfpathcurveto{\pgfqpoint{1.600467in}{0.766915in}}{\pgfqpoint{1.597408in}{0.765648in}}{\pgfqpoint{1.595152in}{0.763392in}}%
\pgfpathcurveto{\pgfqpoint{1.592897in}{0.761137in}}{\pgfqpoint{1.591629in}{0.758077in}}{\pgfqpoint{1.591629in}{0.754887in}}%
\pgfpathcurveto{\pgfqpoint{1.591629in}{0.751697in}}{\pgfqpoint{1.592897in}{0.748637in}}{\pgfqpoint{1.595152in}{0.746382in}}%
\pgfpathcurveto{\pgfqpoint{1.597408in}{0.744126in}}{\pgfqpoint{1.600467in}{0.742859in}}{\pgfqpoint{1.603657in}{0.742859in}}%
\pgfpathclose%
\pgfusepath{stroke,fill}%
\end{pgfscope}%
\begin{pgfscope}%
\pgfpathrectangle{\pgfqpoint{0.750000in}{0.500000in}}{\pgfqpoint{4.650000in}{3.020000in}}%
\pgfusepath{clip}%
\pgfsetbuttcap%
\pgfsetroundjoin%
\definecolor{currentfill}{rgb}{0.121569,0.466667,0.705882}%
\pgfsetfillcolor{currentfill}%
\pgfsetlinewidth{1.003750pt}%
\definecolor{currentstroke}{rgb}{0.121569,0.466667,0.705882}%
\pgfsetstrokecolor{currentstroke}%
\pgfsetdash{}{0pt}%
\pgfpathmoveto{\pgfqpoint{2.347583in}{1.041796in}}%
\pgfpathcurveto{\pgfqpoint{2.351701in}{1.041796in}}{\pgfqpoint{2.355651in}{1.043432in}}{\pgfqpoint{2.358563in}{1.046344in}}%
\pgfpathcurveto{\pgfqpoint{2.361475in}{1.049256in}}{\pgfqpoint{2.363111in}{1.053206in}}{\pgfqpoint{2.363111in}{1.057324in}}%
\pgfpathcurveto{\pgfqpoint{2.363111in}{1.061442in}}{\pgfqpoint{2.361475in}{1.065392in}}{\pgfqpoint{2.358563in}{1.068304in}}%
\pgfpathcurveto{\pgfqpoint{2.355651in}{1.071216in}}{\pgfqpoint{2.351701in}{1.072852in}}{\pgfqpoint{2.347583in}{1.072852in}}%
\pgfpathcurveto{\pgfqpoint{2.343465in}{1.072852in}}{\pgfqpoint{2.339515in}{1.071216in}}{\pgfqpoint{2.336603in}{1.068304in}}%
\pgfpathcurveto{\pgfqpoint{2.333691in}{1.065392in}}{\pgfqpoint{2.332055in}{1.061442in}}{\pgfqpoint{2.332055in}{1.057324in}}%
\pgfpathcurveto{\pgfqpoint{2.332055in}{1.053206in}}{\pgfqpoint{2.333691in}{1.049256in}}{\pgfqpoint{2.336603in}{1.046344in}}%
\pgfpathcurveto{\pgfqpoint{2.339515in}{1.043432in}}{\pgfqpoint{2.343465in}{1.041796in}}{\pgfqpoint{2.347583in}{1.041796in}}%
\pgfpathclose%
\pgfusepath{stroke,fill}%
\end{pgfscope}%
\begin{pgfscope}%
\pgfpathrectangle{\pgfqpoint{0.750000in}{0.500000in}}{\pgfqpoint{4.650000in}{3.020000in}}%
\pgfusepath{clip}%
\pgfsetbuttcap%
\pgfsetroundjoin%
\definecolor{currentfill}{rgb}{0.121569,0.466667,0.705882}%
\pgfsetfillcolor{currentfill}%
\pgfsetlinewidth{1.003750pt}%
\definecolor{currentstroke}{rgb}{0.121569,0.466667,0.705882}%
\pgfsetstrokecolor{currentstroke}%
\pgfsetdash{}{0pt}%
\pgfpathmoveto{\pgfqpoint{0.971682in}{0.628412in}}%
\pgfpathcurveto{\pgfqpoint{0.974286in}{0.628412in}}{\pgfqpoint{0.976784in}{0.629447in}}{\pgfqpoint{0.978626in}{0.631288in}}%
\pgfpathcurveto{\pgfqpoint{0.980468in}{0.633130in}}{\pgfqpoint{0.981503in}{0.635628in}}{\pgfqpoint{0.981503in}{0.638233in}}%
\pgfpathcurveto{\pgfqpoint{0.981503in}{0.640837in}}{\pgfqpoint{0.980468in}{0.643336in}}{\pgfqpoint{0.978626in}{0.645177in}}%
\pgfpathcurveto{\pgfqpoint{0.976784in}{0.647019in}}{\pgfqpoint{0.974286in}{0.648054in}}{\pgfqpoint{0.971682in}{0.648054in}}%
\pgfpathcurveto{\pgfqpoint{0.969077in}{0.648054in}}{\pgfqpoint{0.966579in}{0.647019in}}{\pgfqpoint{0.964737in}{0.645177in}}%
\pgfpathcurveto{\pgfqpoint{0.962895in}{0.643336in}}{\pgfqpoint{0.961861in}{0.640837in}}{\pgfqpoint{0.961861in}{0.638233in}}%
\pgfpathcurveto{\pgfqpoint{0.961861in}{0.635628in}}{\pgfqpoint{0.962895in}{0.633130in}}{\pgfqpoint{0.964737in}{0.631288in}}%
\pgfpathcurveto{\pgfqpoint{0.966579in}{0.629447in}}{\pgfqpoint{0.969077in}{0.628412in}}{\pgfqpoint{0.971682in}{0.628412in}}%
\pgfpathclose%
\pgfusepath{stroke,fill}%
\end{pgfscope}%
\begin{pgfscope}%
\pgfpathrectangle{\pgfqpoint{0.750000in}{0.500000in}}{\pgfqpoint{4.650000in}{3.020000in}}%
\pgfusepath{clip}%
\pgfsetbuttcap%
\pgfsetroundjoin%
\definecolor{currentfill}{rgb}{0.121569,0.466667,0.705882}%
\pgfsetfillcolor{currentfill}%
\pgfsetlinewidth{1.003750pt}%
\definecolor{currentstroke}{rgb}{0.121569,0.466667,0.705882}%
\pgfsetstrokecolor{currentstroke}%
\pgfsetdash{}{0pt}%
\pgfpathmoveto{\pgfqpoint{1.420513in}{0.649054in}}%
\pgfpathcurveto{\pgfqpoint{1.423118in}{0.649054in}}{\pgfqpoint{1.425616in}{0.650089in}}{\pgfqpoint{1.427458in}{0.651931in}}%
\pgfpathcurveto{\pgfqpoint{1.429299in}{0.653773in}}{\pgfqpoint{1.430334in}{0.656271in}}{\pgfqpoint{1.430334in}{0.658875in}}%
\pgfpathcurveto{\pgfqpoint{1.430334in}{0.661480in}}{\pgfqpoint{1.429299in}{0.663978in}}{\pgfqpoint{1.427458in}{0.665820in}}%
\pgfpathcurveto{\pgfqpoint{1.425616in}{0.667661in}}{\pgfqpoint{1.423118in}{0.668696in}}{\pgfqpoint{1.420513in}{0.668696in}}%
\pgfpathcurveto{\pgfqpoint{1.417909in}{0.668696in}}{\pgfqpoint{1.415411in}{0.667661in}}{\pgfqpoint{1.413569in}{0.665820in}}%
\pgfpathcurveto{\pgfqpoint{1.411727in}{0.663978in}}{\pgfqpoint{1.410692in}{0.661480in}}{\pgfqpoint{1.410692in}{0.658875in}}%
\pgfpathcurveto{\pgfqpoint{1.410692in}{0.656271in}}{\pgfqpoint{1.411727in}{0.653773in}}{\pgfqpoint{1.413569in}{0.651931in}}%
\pgfpathcurveto{\pgfqpoint{1.415411in}{0.650089in}}{\pgfqpoint{1.417909in}{0.649054in}}{\pgfqpoint{1.420513in}{0.649054in}}%
\pgfpathclose%
\pgfusepath{stroke,fill}%
\end{pgfscope}%
\begin{pgfscope}%
\pgfpathrectangle{\pgfqpoint{0.750000in}{0.500000in}}{\pgfqpoint{4.650000in}{3.020000in}}%
\pgfusepath{clip}%
\pgfsetbuttcap%
\pgfsetroundjoin%
\definecolor{currentfill}{rgb}{0.121569,0.466667,0.705882}%
\pgfsetfillcolor{currentfill}%
\pgfsetlinewidth{1.003750pt}%
\definecolor{currentstroke}{rgb}{0.121569,0.466667,0.705882}%
\pgfsetstrokecolor{currentstroke}%
\pgfsetdash{}{0pt}%
\pgfpathmoveto{\pgfqpoint{1.059384in}{0.632732in}}%
\pgfpathcurveto{\pgfqpoint{1.061989in}{0.632732in}}{\pgfqpoint{1.064487in}{0.633767in}}{\pgfqpoint{1.066329in}{0.635609in}}%
\pgfpathcurveto{\pgfqpoint{1.068170in}{0.637451in}}{\pgfqpoint{1.069205in}{0.639949in}}{\pgfqpoint{1.069205in}{0.642553in}}%
\pgfpathcurveto{\pgfqpoint{1.069205in}{0.645158in}}{\pgfqpoint{1.068170in}{0.647656in}}{\pgfqpoint{1.066329in}{0.649498in}}%
\pgfpathcurveto{\pgfqpoint{1.064487in}{0.651340in}}{\pgfqpoint{1.061989in}{0.652374in}}{\pgfqpoint{1.059384in}{0.652374in}}%
\pgfpathcurveto{\pgfqpoint{1.056780in}{0.652374in}}{\pgfqpoint{1.054282in}{0.651340in}}{\pgfqpoint{1.052440in}{0.649498in}}%
\pgfpathcurveto{\pgfqpoint{1.050598in}{0.647656in}}{\pgfqpoint{1.049563in}{0.645158in}}{\pgfqpoint{1.049563in}{0.642553in}}%
\pgfpathcurveto{\pgfqpoint{1.049563in}{0.639949in}}{\pgfqpoint{1.050598in}{0.637451in}}{\pgfqpoint{1.052440in}{0.635609in}}%
\pgfpathcurveto{\pgfqpoint{1.054282in}{0.633767in}}{\pgfqpoint{1.056780in}{0.632732in}}{\pgfqpoint{1.059384in}{0.632732in}}%
\pgfpathclose%
\pgfusepath{stroke,fill}%
\end{pgfscope}%
\begin{pgfscope}%
\pgfpathrectangle{\pgfqpoint{0.750000in}{0.500000in}}{\pgfqpoint{4.650000in}{3.020000in}}%
\pgfusepath{clip}%
\pgfsetbuttcap%
\pgfsetroundjoin%
\definecolor{currentfill}{rgb}{0.121569,0.466667,0.705882}%
\pgfsetfillcolor{currentfill}%
\pgfsetlinewidth{1.003750pt}%
\definecolor{currentstroke}{rgb}{0.121569,0.466667,0.705882}%
\pgfsetstrokecolor{currentstroke}%
\pgfsetdash{}{0pt}%
\pgfpathmoveto{\pgfqpoint{1.646477in}{0.711175in}}%
\pgfpathcurveto{\pgfqpoint{1.649667in}{0.711175in}}{\pgfqpoint{1.652726in}{0.712442in}}{\pgfqpoint{1.654982in}{0.714698in}}%
\pgfpathcurveto{\pgfqpoint{1.657238in}{0.716954in}}{\pgfqpoint{1.658505in}{0.720013in}}{\pgfqpoint{1.658505in}{0.723203in}}%
\pgfpathcurveto{\pgfqpoint{1.658505in}{0.726393in}}{\pgfqpoint{1.657238in}{0.729453in}}{\pgfqpoint{1.654982in}{0.731708in}}%
\pgfpathcurveto{\pgfqpoint{1.652726in}{0.733964in}}{\pgfqpoint{1.649667in}{0.735231in}}{\pgfqpoint{1.646477in}{0.735231in}}%
\pgfpathcurveto{\pgfqpoint{1.643287in}{0.735231in}}{\pgfqpoint{1.640227in}{0.733964in}}{\pgfqpoint{1.637972in}{0.731708in}}%
\pgfpathcurveto{\pgfqpoint{1.635716in}{0.729453in}}{\pgfqpoint{1.634449in}{0.726393in}}{\pgfqpoint{1.634449in}{0.723203in}}%
\pgfpathcurveto{\pgfqpoint{1.634449in}{0.720013in}}{\pgfqpoint{1.635716in}{0.716954in}}{\pgfqpoint{1.637972in}{0.714698in}}%
\pgfpathcurveto{\pgfqpoint{1.640227in}{0.712442in}}{\pgfqpoint{1.643287in}{0.711175in}}{\pgfqpoint{1.646477in}{0.711175in}}%
\pgfpathclose%
\pgfusepath{stroke,fill}%
\end{pgfscope}%
\begin{pgfscope}%
\pgfpathrectangle{\pgfqpoint{0.750000in}{0.500000in}}{\pgfqpoint{4.650000in}{3.020000in}}%
\pgfusepath{clip}%
\pgfsetbuttcap%
\pgfsetroundjoin%
\definecolor{currentfill}{rgb}{0.121569,0.466667,0.705882}%
\pgfsetfillcolor{currentfill}%
\pgfsetlinewidth{1.003750pt}%
\definecolor{currentstroke}{rgb}{0.121569,0.466667,0.705882}%
\pgfsetstrokecolor{currentstroke}%
\pgfsetdash{}{0pt}%
\pgfpathmoveto{\pgfqpoint{2.008122in}{0.840598in}}%
\pgfpathcurveto{\pgfqpoint{2.010726in}{0.840598in}}{\pgfqpoint{2.013225in}{0.841633in}}{\pgfqpoint{2.015066in}{0.843474in}}%
\pgfpathcurveto{\pgfqpoint{2.016908in}{0.845316in}}{\pgfqpoint{2.017943in}{0.847814in}}{\pgfqpoint{2.017943in}{0.850419in}}%
\pgfpathcurveto{\pgfqpoint{2.017943in}{0.853023in}}{\pgfqpoint{2.016908in}{0.855521in}}{\pgfqpoint{2.015066in}{0.857363in}}%
\pgfpathcurveto{\pgfqpoint{2.013225in}{0.859205in}}{\pgfqpoint{2.010726in}{0.860240in}}{\pgfqpoint{2.008122in}{0.860240in}}%
\pgfpathcurveto{\pgfqpoint{2.005517in}{0.860240in}}{\pgfqpoint{2.003019in}{0.859205in}}{\pgfqpoint{2.001177in}{0.857363in}}%
\pgfpathcurveto{\pgfqpoint{1.999336in}{0.855521in}}{\pgfqpoint{1.998301in}{0.853023in}}{\pgfqpoint{1.998301in}{0.850419in}}%
\pgfpathcurveto{\pgfqpoint{1.998301in}{0.847814in}}{\pgfqpoint{1.999336in}{0.845316in}}{\pgfqpoint{2.001177in}{0.843474in}}%
\pgfpathcurveto{\pgfqpoint{2.003019in}{0.841633in}}{\pgfqpoint{2.005517in}{0.840598in}}{\pgfqpoint{2.008122in}{0.840598in}}%
\pgfpathclose%
\pgfusepath{stroke,fill}%
\end{pgfscope}%
\begin{pgfscope}%
\pgfpathrectangle{\pgfqpoint{0.750000in}{0.500000in}}{\pgfqpoint{4.650000in}{3.020000in}}%
\pgfusepath{clip}%
\pgfsetbuttcap%
\pgfsetroundjoin%
\definecolor{currentfill}{rgb}{0.121569,0.466667,0.705882}%
\pgfsetfillcolor{currentfill}%
\pgfsetlinewidth{1.003750pt}%
\definecolor{currentstroke}{rgb}{0.121569,0.466667,0.705882}%
\pgfsetstrokecolor{currentstroke}%
\pgfsetdash{}{0pt}%
\pgfpathmoveto{\pgfqpoint{2.259880in}{0.907716in}}%
\pgfpathcurveto{\pgfqpoint{2.267474in}{0.907716in}}{\pgfqpoint{2.274757in}{0.910733in}}{\pgfqpoint{2.280127in}{0.916103in}}%
\pgfpathcurveto{\pgfqpoint{2.285496in}{0.921472in}}{\pgfqpoint{2.288513in}{0.928756in}}{\pgfqpoint{2.288513in}{0.936349in}}%
\pgfpathcurveto{\pgfqpoint{2.288513in}{0.943943in}}{\pgfqpoint{2.285496in}{0.951226in}}{\pgfqpoint{2.280127in}{0.956596in}}%
\pgfpathcurveto{\pgfqpoint{2.274757in}{0.961965in}}{\pgfqpoint{2.267474in}{0.964982in}}{\pgfqpoint{2.259880in}{0.964982in}}%
\pgfpathcurveto{\pgfqpoint{2.252287in}{0.964982in}}{\pgfqpoint{2.245003in}{0.961965in}}{\pgfqpoint{2.239634in}{0.956596in}}%
\pgfpathcurveto{\pgfqpoint{2.234265in}{0.951226in}}{\pgfqpoint{2.231248in}{0.943943in}}{\pgfqpoint{2.231248in}{0.936349in}}%
\pgfpathcurveto{\pgfqpoint{2.231248in}{0.928756in}}{\pgfqpoint{2.234265in}{0.921472in}}{\pgfqpoint{2.239634in}{0.916103in}}%
\pgfpathcurveto{\pgfqpoint{2.245003in}{0.910733in}}{\pgfqpoint{2.252287in}{0.907716in}}{\pgfqpoint{2.259880in}{0.907716in}}%
\pgfpathclose%
\pgfusepath{stroke,fill}%
\end{pgfscope}%
\begin{pgfscope}%
\pgfpathrectangle{\pgfqpoint{0.750000in}{0.500000in}}{\pgfqpoint{4.650000in}{3.020000in}}%
\pgfusepath{clip}%
\pgfsetbuttcap%
\pgfsetroundjoin%
\definecolor{currentfill}{rgb}{0.121569,0.466667,0.705882}%
\pgfsetfillcolor{currentfill}%
\pgfsetlinewidth{1.003750pt}%
\definecolor{currentstroke}{rgb}{0.121569,0.466667,0.705882}%
\pgfsetstrokecolor{currentstroke}%
\pgfsetdash{}{0pt}%
\pgfpathmoveto{\pgfqpoint{0.977356in}{0.629852in}}%
\pgfpathcurveto{\pgfqpoint{0.979961in}{0.629852in}}{\pgfqpoint{0.982459in}{0.630887in}}{\pgfqpoint{0.984301in}{0.632729in}}%
\pgfpathcurveto{\pgfqpoint{0.986143in}{0.634570in}}{\pgfqpoint{0.987177in}{0.637068in}}{\pgfqpoint{0.987177in}{0.639673in}}%
\pgfpathcurveto{\pgfqpoint{0.987177in}{0.642278in}}{\pgfqpoint{0.986143in}{0.644776in}}{\pgfqpoint{0.984301in}{0.646617in}}%
\pgfpathcurveto{\pgfqpoint{0.982459in}{0.648459in}}{\pgfqpoint{0.979961in}{0.649494in}}{\pgfqpoint{0.977356in}{0.649494in}}%
\pgfpathcurveto{\pgfqpoint{0.974752in}{0.649494in}}{\pgfqpoint{0.972254in}{0.648459in}}{\pgfqpoint{0.970412in}{0.646617in}}%
\pgfpathcurveto{\pgfqpoint{0.968570in}{0.644776in}}{\pgfqpoint{0.967536in}{0.642278in}}{\pgfqpoint{0.967536in}{0.639673in}}%
\pgfpathcurveto{\pgfqpoint{0.967536in}{0.637068in}}{\pgfqpoint{0.968570in}{0.634570in}}{\pgfqpoint{0.970412in}{0.632729in}}%
\pgfpathcurveto{\pgfqpoint{0.972254in}{0.630887in}}{\pgfqpoint{0.974752in}{0.629852in}}{\pgfqpoint{0.977356in}{0.629852in}}%
\pgfpathclose%
\pgfusepath{stroke,fill}%
\end{pgfscope}%
\begin{pgfscope}%
\pgfpathrectangle{\pgfqpoint{0.750000in}{0.500000in}}{\pgfqpoint{4.650000in}{3.020000in}}%
\pgfusepath{clip}%
\pgfsetbuttcap%
\pgfsetroundjoin%
\definecolor{currentfill}{rgb}{0.121569,0.466667,0.705882}%
\pgfsetfillcolor{currentfill}%
\pgfsetlinewidth{1.003750pt}%
\definecolor{currentstroke}{rgb}{0.121569,0.466667,0.705882}%
\pgfsetstrokecolor{currentstroke}%
\pgfsetdash{}{0pt}%
\pgfpathmoveto{\pgfqpoint{1.347256in}{0.711235in}}%
\pgfpathcurveto{\pgfqpoint{1.350939in}{0.711235in}}{\pgfqpoint{1.354472in}{0.712698in}}{\pgfqpoint{1.357077in}{0.715303in}}%
\pgfpathcurveto{\pgfqpoint{1.359681in}{0.717907in}}{\pgfqpoint{1.361145in}{0.721440in}}{\pgfqpoint{1.361145in}{0.725123in}}%
\pgfpathcurveto{\pgfqpoint{1.361145in}{0.728807in}}{\pgfqpoint{1.359681in}{0.732340in}}{\pgfqpoint{1.357077in}{0.734944in}}%
\pgfpathcurveto{\pgfqpoint{1.354472in}{0.737549in}}{\pgfqpoint{1.350939in}{0.739012in}}{\pgfqpoint{1.347256in}{0.739012in}}%
\pgfpathcurveto{\pgfqpoint{1.343572in}{0.739012in}}{\pgfqpoint{1.340039in}{0.737549in}}{\pgfqpoint{1.337435in}{0.734944in}}%
\pgfpathcurveto{\pgfqpoint{1.334830in}{0.732340in}}{\pgfqpoint{1.333367in}{0.728807in}}{\pgfqpoint{1.333367in}{0.725123in}}%
\pgfpathcurveto{\pgfqpoint{1.333367in}{0.721440in}}{\pgfqpoint{1.334830in}{0.717907in}}{\pgfqpoint{1.337435in}{0.715303in}}%
\pgfpathcurveto{\pgfqpoint{1.340039in}{0.712698in}}{\pgfqpoint{1.343572in}{0.711235in}}{\pgfqpoint{1.347256in}{0.711235in}}%
\pgfpathclose%
\pgfusepath{stroke,fill}%
\end{pgfscope}%
\begin{pgfscope}%
\pgfpathrectangle{\pgfqpoint{0.750000in}{0.500000in}}{\pgfqpoint{4.650000in}{3.020000in}}%
\pgfusepath{clip}%
\pgfsetbuttcap%
\pgfsetroundjoin%
\definecolor{currentfill}{rgb}{0.121569,0.466667,0.705882}%
\pgfsetfillcolor{currentfill}%
\pgfsetlinewidth{1.003750pt}%
\definecolor{currentstroke}{rgb}{0.121569,0.466667,0.705882}%
\pgfsetstrokecolor{currentstroke}%
\pgfsetdash{}{0pt}%
\pgfpathmoveto{\pgfqpoint{1.086211in}{0.647614in}}%
\pgfpathcurveto{\pgfqpoint{1.088816in}{0.647614in}}{\pgfqpoint{1.091314in}{0.648649in}}{\pgfqpoint{1.093156in}{0.650491in}}%
\pgfpathcurveto{\pgfqpoint{1.094997in}{0.652332in}}{\pgfqpoint{1.096032in}{0.654831in}}{\pgfqpoint{1.096032in}{0.657435in}}%
\pgfpathcurveto{\pgfqpoint{1.096032in}{0.660040in}}{\pgfqpoint{1.094997in}{0.662538in}}{\pgfqpoint{1.093156in}{0.664380in}}%
\pgfpathcurveto{\pgfqpoint{1.091314in}{0.666221in}}{\pgfqpoint{1.088816in}{0.667256in}}{\pgfqpoint{1.086211in}{0.667256in}}%
\pgfpathcurveto{\pgfqpoint{1.083607in}{0.667256in}}{\pgfqpoint{1.081108in}{0.666221in}}{\pgfqpoint{1.079267in}{0.664380in}}%
\pgfpathcurveto{\pgfqpoint{1.077425in}{0.662538in}}{\pgfqpoint{1.076390in}{0.660040in}}{\pgfqpoint{1.076390in}{0.657435in}}%
\pgfpathcurveto{\pgfqpoint{1.076390in}{0.654831in}}{\pgfqpoint{1.077425in}{0.652332in}}{\pgfqpoint{1.079267in}{0.650491in}}%
\pgfpathcurveto{\pgfqpoint{1.081108in}{0.648649in}}{\pgfqpoint{1.083607in}{0.647614in}}{\pgfqpoint{1.086211in}{0.647614in}}%
\pgfpathclose%
\pgfusepath{stroke,fill}%
\end{pgfscope}%
\begin{pgfscope}%
\pgfpathrectangle{\pgfqpoint{0.750000in}{0.500000in}}{\pgfqpoint{4.650000in}{3.020000in}}%
\pgfusepath{clip}%
\pgfsetbuttcap%
\pgfsetroundjoin%
\definecolor{currentfill}{rgb}{0.121569,0.466667,0.705882}%
\pgfsetfillcolor{currentfill}%
\pgfsetlinewidth{1.003750pt}%
\definecolor{currentstroke}{rgb}{0.121569,0.466667,0.705882}%
\pgfsetstrokecolor{currentstroke}%
\pgfsetdash{}{0pt}%
\pgfpathmoveto{\pgfqpoint{2.937771in}{0.966633in}}%
\pgfpathcurveto{\pgfqpoint{2.942980in}{0.966633in}}{\pgfqpoint{2.947976in}{0.968703in}}{\pgfqpoint{2.951660in}{0.972386in}}%
\pgfpathcurveto{\pgfqpoint{2.955343in}{0.976070in}}{\pgfqpoint{2.957413in}{0.981066in}}{\pgfqpoint{2.957413in}{0.986275in}}%
\pgfpathcurveto{\pgfqpoint{2.957413in}{0.991484in}}{\pgfqpoint{2.955343in}{0.996481in}}{\pgfqpoint{2.951660in}{1.000164in}}%
\pgfpathcurveto{\pgfqpoint{2.947976in}{1.003848in}}{\pgfqpoint{2.942980in}{1.005917in}}{\pgfqpoint{2.937771in}{1.005917in}}%
\pgfpathcurveto{\pgfqpoint{2.932562in}{1.005917in}}{\pgfqpoint{2.927565in}{1.003848in}}{\pgfqpoint{2.923882in}{1.000164in}}%
\pgfpathcurveto{\pgfqpoint{2.920199in}{0.996481in}}{\pgfqpoint{2.918129in}{0.991484in}}{\pgfqpoint{2.918129in}{0.986275in}}%
\pgfpathcurveto{\pgfqpoint{2.918129in}{0.981066in}}{\pgfqpoint{2.920199in}{0.976070in}}{\pgfqpoint{2.923882in}{0.972386in}}%
\pgfpathcurveto{\pgfqpoint{2.927565in}{0.968703in}}{\pgfqpoint{2.932562in}{0.966633in}}{\pgfqpoint{2.937771in}{0.966633in}}%
\pgfpathclose%
\pgfusepath{stroke,fill}%
\end{pgfscope}%
\begin{pgfscope}%
\pgfpathrectangle{\pgfqpoint{0.750000in}{0.500000in}}{\pgfqpoint{4.650000in}{3.020000in}}%
\pgfusepath{clip}%
\pgfsetbuttcap%
\pgfsetroundjoin%
\definecolor{currentfill}{rgb}{0.121569,0.466667,0.705882}%
\pgfsetfillcolor{currentfill}%
\pgfsetlinewidth{1.003750pt}%
\definecolor{currentstroke}{rgb}{0.121569,0.466667,0.705882}%
\pgfsetstrokecolor{currentstroke}%
\pgfsetdash{}{0pt}%
\pgfpathmoveto{\pgfqpoint{4.757345in}{1.430601in}}%
\pgfpathcurveto{\pgfqpoint{4.761856in}{1.430601in}}{\pgfqpoint{4.766183in}{1.432393in}}{\pgfqpoint{4.769373in}{1.435583in}}%
\pgfpathcurveto{\pgfqpoint{4.772563in}{1.438773in}}{\pgfqpoint{4.774356in}{1.443100in}}{\pgfqpoint{4.774356in}{1.447611in}}%
\pgfpathcurveto{\pgfqpoint{4.774356in}{1.452123in}}{\pgfqpoint{4.772563in}{1.456450in}}{\pgfqpoint{4.769373in}{1.459640in}}%
\pgfpathcurveto{\pgfqpoint{4.766183in}{1.462829in}}{\pgfqpoint{4.761856in}{1.464622in}}{\pgfqpoint{4.757345in}{1.464622in}}%
\pgfpathcurveto{\pgfqpoint{4.752834in}{1.464622in}}{\pgfqpoint{4.748507in}{1.462829in}}{\pgfqpoint{4.745317in}{1.459640in}}%
\pgfpathcurveto{\pgfqpoint{4.742127in}{1.456450in}}{\pgfqpoint{4.740335in}{1.452123in}}{\pgfqpoint{4.740335in}{1.447611in}}%
\pgfpathcurveto{\pgfqpoint{4.740335in}{1.443100in}}{\pgfqpoint{4.742127in}{1.438773in}}{\pgfqpoint{4.745317in}{1.435583in}}%
\pgfpathcurveto{\pgfqpoint{4.748507in}{1.432393in}}{\pgfqpoint{4.752834in}{1.430601in}}{\pgfqpoint{4.757345in}{1.430601in}}%
\pgfpathclose%
\pgfusepath{stroke,fill}%
\end{pgfscope}%
\begin{pgfscope}%
\pgfpathrectangle{\pgfqpoint{0.750000in}{0.500000in}}{\pgfqpoint{4.650000in}{3.020000in}}%
\pgfusepath{clip}%
\pgfsetbuttcap%
\pgfsetroundjoin%
\definecolor{currentfill}{rgb}{0.121569,0.466667,0.705882}%
\pgfsetfillcolor{currentfill}%
\pgfsetlinewidth{1.003750pt}%
\definecolor{currentstroke}{rgb}{0.121569,0.466667,0.705882}%
\pgfsetstrokecolor{currentstroke}%
\pgfsetdash{}{0pt}%
\pgfpathmoveto{\pgfqpoint{1.315786in}{0.687068in}}%
\pgfpathcurveto{\pgfqpoint{1.320659in}{0.687068in}}{\pgfqpoint{1.325332in}{0.689004in}}{\pgfqpoint{1.328778in}{0.692449in}}%
\pgfpathcurveto{\pgfqpoint{1.332223in}{0.695895in}}{\pgfqpoint{1.334159in}{0.700568in}}{\pgfqpoint{1.334159in}{0.705441in}}%
\pgfpathcurveto{\pgfqpoint{1.334159in}{0.710314in}}{\pgfqpoint{1.332223in}{0.714987in}}{\pgfqpoint{1.328778in}{0.718433in}}%
\pgfpathcurveto{\pgfqpoint{1.325332in}{0.721878in}}{\pgfqpoint{1.320659in}{0.723814in}}{\pgfqpoint{1.315786in}{0.723814in}}%
\pgfpathcurveto{\pgfqpoint{1.310913in}{0.723814in}}{\pgfqpoint{1.306240in}{0.721878in}}{\pgfqpoint{1.302794in}{0.718433in}}%
\pgfpathcurveto{\pgfqpoint{1.299349in}{0.714987in}}{\pgfqpoint{1.297413in}{0.710314in}}{\pgfqpoint{1.297413in}{0.705441in}}%
\pgfpathcurveto{\pgfqpoint{1.297413in}{0.700568in}}{\pgfqpoint{1.299349in}{0.695895in}}{\pgfqpoint{1.302794in}{0.692449in}}%
\pgfpathcurveto{\pgfqpoint{1.306240in}{0.689004in}}{\pgfqpoint{1.310913in}{0.687068in}}{\pgfqpoint{1.315786in}{0.687068in}}%
\pgfpathclose%
\pgfusepath{stroke,fill}%
\end{pgfscope}%
\begin{pgfscope}%
\pgfpathrectangle{\pgfqpoint{0.750000in}{0.500000in}}{\pgfqpoint{4.650000in}{3.020000in}}%
\pgfusepath{clip}%
\pgfsetbuttcap%
\pgfsetroundjoin%
\definecolor{currentfill}{rgb}{0.121569,0.466667,0.705882}%
\pgfsetfillcolor{currentfill}%
\pgfsetlinewidth{1.003750pt}%
\definecolor{currentstroke}{rgb}{0.121569,0.466667,0.705882}%
\pgfsetstrokecolor{currentstroke}%
\pgfsetdash{}{0pt}%
\pgfpathmoveto{\pgfqpoint{1.658343in}{0.739499in}}%
\pgfpathcurveto{\pgfqpoint{1.661532in}{0.739499in}}{\pgfqpoint{1.664592in}{0.740766in}}{\pgfqpoint{1.666848in}{0.743021in}}%
\pgfpathcurveto{\pgfqpoint{1.669103in}{0.745277in}}{\pgfqpoint{1.670371in}{0.748337in}}{\pgfqpoint{1.670371in}{0.751527in}}%
\pgfpathcurveto{\pgfqpoint{1.670371in}{0.754717in}}{\pgfqpoint{1.669103in}{0.757776in}}{\pgfqpoint{1.666848in}{0.760032in}}%
\pgfpathcurveto{\pgfqpoint{1.664592in}{0.762287in}}{\pgfqpoint{1.661532in}{0.763555in}}{\pgfqpoint{1.658343in}{0.763555in}}%
\pgfpathcurveto{\pgfqpoint{1.655153in}{0.763555in}}{\pgfqpoint{1.652093in}{0.762287in}}{\pgfqpoint{1.649837in}{0.760032in}}%
\pgfpathcurveto{\pgfqpoint{1.647582in}{0.757776in}}{\pgfqpoint{1.646314in}{0.754717in}}{\pgfqpoint{1.646314in}{0.751527in}}%
\pgfpathcurveto{\pgfqpoint{1.646314in}{0.748337in}}{\pgfqpoint{1.647582in}{0.745277in}}{\pgfqpoint{1.649837in}{0.743021in}}%
\pgfpathcurveto{\pgfqpoint{1.652093in}{0.740766in}}{\pgfqpoint{1.655153in}{0.739499in}}{\pgfqpoint{1.658343in}{0.739499in}}%
\pgfpathclose%
\pgfusepath{stroke,fill}%
\end{pgfscope}%
\begin{pgfscope}%
\pgfpathrectangle{\pgfqpoint{0.750000in}{0.500000in}}{\pgfqpoint{4.650000in}{3.020000in}}%
\pgfusepath{clip}%
\pgfsetbuttcap%
\pgfsetroundjoin%
\definecolor{currentfill}{rgb}{0.121569,0.466667,0.705882}%
\pgfsetfillcolor{currentfill}%
\pgfsetlinewidth{1.003750pt}%
\definecolor{currentstroke}{rgb}{0.121569,0.466667,0.705882}%
\pgfsetstrokecolor{currentstroke}%
\pgfsetdash{}{0pt}%
\pgfpathmoveto{\pgfqpoint{1.017081in}{0.650975in}}%
\pgfpathcurveto{\pgfqpoint{1.019685in}{0.650975in}}{\pgfqpoint{1.022183in}{0.652009in}}{\pgfqpoint{1.024025in}{0.653851in}}%
\pgfpathcurveto{\pgfqpoint{1.025867in}{0.655693in}}{\pgfqpoint{1.026902in}{0.658191in}}{\pgfqpoint{1.026902in}{0.660796in}}%
\pgfpathcurveto{\pgfqpoint{1.026902in}{0.663400in}}{\pgfqpoint{1.025867in}{0.665898in}}{\pgfqpoint{1.024025in}{0.667740in}}%
\pgfpathcurveto{\pgfqpoint{1.022183in}{0.669582in}}{\pgfqpoint{1.019685in}{0.670617in}}{\pgfqpoint{1.017081in}{0.670617in}}%
\pgfpathcurveto{\pgfqpoint{1.014476in}{0.670617in}}{\pgfqpoint{1.011978in}{0.669582in}}{\pgfqpoint{1.010136in}{0.667740in}}%
\pgfpathcurveto{\pgfqpoint{1.008295in}{0.665898in}}{\pgfqpoint{1.007260in}{0.663400in}}{\pgfqpoint{1.007260in}{0.660796in}}%
\pgfpathcurveto{\pgfqpoint{1.007260in}{0.658191in}}{\pgfqpoint{1.008295in}{0.655693in}}{\pgfqpoint{1.010136in}{0.653851in}}%
\pgfpathcurveto{\pgfqpoint{1.011978in}{0.652009in}}{\pgfqpoint{1.014476in}{0.650975in}}{\pgfqpoint{1.017081in}{0.650975in}}%
\pgfpathclose%
\pgfusepath{stroke,fill}%
\end{pgfscope}%
\begin{pgfscope}%
\pgfpathrectangle{\pgfqpoint{0.750000in}{0.500000in}}{\pgfqpoint{4.650000in}{3.020000in}}%
\pgfusepath{clip}%
\pgfsetbuttcap%
\pgfsetroundjoin%
\definecolor{currentfill}{rgb}{0.121569,0.466667,0.705882}%
\pgfsetfillcolor{currentfill}%
\pgfsetlinewidth{1.003750pt}%
\definecolor{currentstroke}{rgb}{0.121569,0.466667,0.705882}%
\pgfsetstrokecolor{currentstroke}%
\pgfsetdash{}{0pt}%
\pgfpathmoveto{\pgfqpoint{1.008826in}{0.628892in}}%
\pgfpathcurveto{\pgfqpoint{1.011431in}{0.628892in}}{\pgfqpoint{1.013929in}{0.629927in}}{\pgfqpoint{1.015771in}{0.631768in}}%
\pgfpathcurveto{\pgfqpoint{1.017612in}{0.633610in}}{\pgfqpoint{1.018647in}{0.636108in}}{\pgfqpoint{1.018647in}{0.638713in}}%
\pgfpathcurveto{\pgfqpoint{1.018647in}{0.641317in}}{\pgfqpoint{1.017612in}{0.643816in}}{\pgfqpoint{1.015771in}{0.645657in}}%
\pgfpathcurveto{\pgfqpoint{1.013929in}{0.647499in}}{\pgfqpoint{1.011431in}{0.648534in}}{\pgfqpoint{1.008826in}{0.648534in}}%
\pgfpathcurveto{\pgfqpoint{1.006222in}{0.648534in}}{\pgfqpoint{1.003724in}{0.647499in}}{\pgfqpoint{1.001882in}{0.645657in}}%
\pgfpathcurveto{\pgfqpoint{1.000040in}{0.643816in}}{\pgfqpoint{0.999005in}{0.641317in}}{\pgfqpoint{0.999005in}{0.638713in}}%
\pgfpathcurveto{\pgfqpoint{0.999005in}{0.636108in}}{\pgfqpoint{1.000040in}{0.633610in}}{\pgfqpoint{1.001882in}{0.631768in}}%
\pgfpathcurveto{\pgfqpoint{1.003724in}{0.629927in}}{\pgfqpoint{1.006222in}{0.628892in}}{\pgfqpoint{1.008826in}{0.628892in}}%
\pgfpathclose%
\pgfusepath{stroke,fill}%
\end{pgfscope}%
\begin{pgfscope}%
\pgfpathrectangle{\pgfqpoint{0.750000in}{0.500000in}}{\pgfqpoint{4.650000in}{3.020000in}}%
\pgfusepath{clip}%
\pgfsetbuttcap%
\pgfsetroundjoin%
\definecolor{currentfill}{rgb}{0.121569,0.466667,0.705882}%
\pgfsetfillcolor{currentfill}%
\pgfsetlinewidth{1.003750pt}%
\definecolor{currentstroke}{rgb}{0.121569,0.466667,0.705882}%
\pgfsetstrokecolor{currentstroke}%
\pgfsetdash{}{0pt}%
\pgfpathmoveto{\pgfqpoint{1.816208in}{0.748907in}}%
\pgfpathcurveto{\pgfqpoint{1.818812in}{0.748907in}}{\pgfqpoint{1.821310in}{0.749941in}}{\pgfqpoint{1.823152in}{0.751783in}}%
\pgfpathcurveto{\pgfqpoint{1.824994in}{0.753625in}}{\pgfqpoint{1.826028in}{0.756123in}}{\pgfqpoint{1.826028in}{0.758728in}}%
\pgfpathcurveto{\pgfqpoint{1.826028in}{0.761332in}}{\pgfqpoint{1.824994in}{0.763830in}}{\pgfqpoint{1.823152in}{0.765672in}}%
\pgfpathcurveto{\pgfqpoint{1.821310in}{0.767514in}}{\pgfqpoint{1.818812in}{0.768548in}}{\pgfqpoint{1.816208in}{0.768548in}}%
\pgfpathcurveto{\pgfqpoint{1.813603in}{0.768548in}}{\pgfqpoint{1.811105in}{0.767514in}}{\pgfqpoint{1.809263in}{0.765672in}}%
\pgfpathcurveto{\pgfqpoint{1.807421in}{0.763830in}}{\pgfqpoint{1.806387in}{0.761332in}}{\pgfqpoint{1.806387in}{0.758728in}}%
\pgfpathcurveto{\pgfqpoint{1.806387in}{0.756123in}}{\pgfqpoint{1.807421in}{0.753625in}}{\pgfqpoint{1.809263in}{0.751783in}}%
\pgfpathcurveto{\pgfqpoint{1.811105in}{0.749941in}}{\pgfqpoint{1.813603in}{0.748907in}}{\pgfqpoint{1.816208in}{0.748907in}}%
\pgfpathclose%
\pgfusepath{stroke,fill}%
\end{pgfscope}%
\begin{pgfscope}%
\pgfpathrectangle{\pgfqpoint{0.750000in}{0.500000in}}{\pgfqpoint{4.650000in}{3.020000in}}%
\pgfusepath{clip}%
\pgfsetbuttcap%
\pgfsetroundjoin%
\definecolor{currentfill}{rgb}{0.121569,0.466667,0.705882}%
\pgfsetfillcolor{currentfill}%
\pgfsetlinewidth{1.003750pt}%
\definecolor{currentstroke}{rgb}{0.121569,0.466667,0.705882}%
\pgfsetstrokecolor{currentstroke}%
\pgfsetdash{}{0pt}%
\pgfpathmoveto{\pgfqpoint{1.458690in}{0.729278in}}%
\pgfpathcurveto{\pgfqpoint{1.462808in}{0.729278in}}{\pgfqpoint{1.466758in}{0.730914in}}{\pgfqpoint{1.469670in}{0.733826in}}%
\pgfpathcurveto{\pgfqpoint{1.472582in}{0.736738in}}{\pgfqpoint{1.474218in}{0.740688in}}{\pgfqpoint{1.474218in}{0.744806in}}%
\pgfpathcurveto{\pgfqpoint{1.474218in}{0.748924in}}{\pgfqpoint{1.472582in}{0.752874in}}{\pgfqpoint{1.469670in}{0.755786in}}%
\pgfpathcurveto{\pgfqpoint{1.466758in}{0.758698in}}{\pgfqpoint{1.462808in}{0.760334in}}{\pgfqpoint{1.458690in}{0.760334in}}%
\pgfpathcurveto{\pgfqpoint{1.454572in}{0.760334in}}{\pgfqpoint{1.450622in}{0.758698in}}{\pgfqpoint{1.447710in}{0.755786in}}%
\pgfpathcurveto{\pgfqpoint{1.444798in}{0.752874in}}{\pgfqpoint{1.443162in}{0.748924in}}{\pgfqpoint{1.443162in}{0.744806in}}%
\pgfpathcurveto{\pgfqpoint{1.443162in}{0.740688in}}{\pgfqpoint{1.444798in}{0.736738in}}{\pgfqpoint{1.447710in}{0.733826in}}%
\pgfpathcurveto{\pgfqpoint{1.450622in}{0.730914in}}{\pgfqpoint{1.454572in}{0.729278in}}{\pgfqpoint{1.458690in}{0.729278in}}%
\pgfpathclose%
\pgfusepath{stroke,fill}%
\end{pgfscope}%
\begin{pgfscope}%
\pgfpathrectangle{\pgfqpoint{0.750000in}{0.500000in}}{\pgfqpoint{4.650000in}{3.020000in}}%
\pgfusepath{clip}%
\pgfsetbuttcap%
\pgfsetroundjoin%
\definecolor{currentfill}{rgb}{0.121569,0.466667,0.705882}%
\pgfsetfillcolor{currentfill}%
\pgfsetlinewidth{1.003750pt}%
\definecolor{currentstroke}{rgb}{0.121569,0.466667,0.705882}%
\pgfsetstrokecolor{currentstroke}%
\pgfsetdash{}{0pt}%
\pgfpathmoveto{\pgfqpoint{1.016565in}{0.634653in}}%
\pgfpathcurveto{\pgfqpoint{1.019169in}{0.634653in}}{\pgfqpoint{1.021668in}{0.635687in}}{\pgfqpoint{1.023509in}{0.637529in}}%
\pgfpathcurveto{\pgfqpoint{1.025351in}{0.639371in}}{\pgfqpoint{1.026386in}{0.641869in}}{\pgfqpoint{1.026386in}{0.644474in}}%
\pgfpathcurveto{\pgfqpoint{1.026386in}{0.647078in}}{\pgfqpoint{1.025351in}{0.649576in}}{\pgfqpoint{1.023509in}{0.651418in}}%
\pgfpathcurveto{\pgfqpoint{1.021668in}{0.653260in}}{\pgfqpoint{1.019169in}{0.654295in}}{\pgfqpoint{1.016565in}{0.654295in}}%
\pgfpathcurveto{\pgfqpoint{1.013960in}{0.654295in}}{\pgfqpoint{1.011462in}{0.653260in}}{\pgfqpoint{1.009620in}{0.651418in}}%
\pgfpathcurveto{\pgfqpoint{1.007779in}{0.649576in}}{\pgfqpoint{1.006744in}{0.647078in}}{\pgfqpoint{1.006744in}{0.644474in}}%
\pgfpathcurveto{\pgfqpoint{1.006744in}{0.641869in}}{\pgfqpoint{1.007779in}{0.639371in}}{\pgfqpoint{1.009620in}{0.637529in}}%
\pgfpathcurveto{\pgfqpoint{1.011462in}{0.635687in}}{\pgfqpoint{1.013960in}{0.634653in}}{\pgfqpoint{1.016565in}{0.634653in}}%
\pgfpathclose%
\pgfusepath{stroke,fill}%
\end{pgfscope}%
\begin{pgfscope}%
\pgfpathrectangle{\pgfqpoint{0.750000in}{0.500000in}}{\pgfqpoint{4.650000in}{3.020000in}}%
\pgfusepath{clip}%
\pgfsetbuttcap%
\pgfsetroundjoin%
\definecolor{currentfill}{rgb}{0.121569,0.466667,0.705882}%
\pgfsetfillcolor{currentfill}%
\pgfsetlinewidth{1.003750pt}%
\definecolor{currentstroke}{rgb}{0.121569,0.466667,0.705882}%
\pgfsetstrokecolor{currentstroke}%
\pgfsetdash{}{0pt}%
\pgfpathmoveto{\pgfqpoint{1.428252in}{0.718716in}}%
\pgfpathcurveto{\pgfqpoint{1.432370in}{0.718716in}}{\pgfqpoint{1.436320in}{0.720352in}}{\pgfqpoint{1.439232in}{0.723264in}}%
\pgfpathcurveto{\pgfqpoint{1.442144in}{0.726176in}}{\pgfqpoint{1.443780in}{0.730126in}}{\pgfqpoint{1.443780in}{0.734245in}}%
\pgfpathcurveto{\pgfqpoint{1.443780in}{0.738363in}}{\pgfqpoint{1.442144in}{0.742313in}}{\pgfqpoint{1.439232in}{0.745225in}}%
\pgfpathcurveto{\pgfqpoint{1.436320in}{0.748137in}}{\pgfqpoint{1.432370in}{0.749773in}}{\pgfqpoint{1.428252in}{0.749773in}}%
\pgfpathcurveto{\pgfqpoint{1.424134in}{0.749773in}}{\pgfqpoint{1.420184in}{0.748137in}}{\pgfqpoint{1.417272in}{0.745225in}}%
\pgfpathcurveto{\pgfqpoint{1.414360in}{0.742313in}}{\pgfqpoint{1.412724in}{0.738363in}}{\pgfqpoint{1.412724in}{0.734245in}}%
\pgfpathcurveto{\pgfqpoint{1.412724in}{0.730126in}}{\pgfqpoint{1.414360in}{0.726176in}}{\pgfqpoint{1.417272in}{0.723264in}}%
\pgfpathcurveto{\pgfqpoint{1.420184in}{0.720352in}}{\pgfqpoint{1.424134in}{0.718716in}}{\pgfqpoint{1.428252in}{0.718716in}}%
\pgfpathclose%
\pgfusepath{stroke,fill}%
\end{pgfscope}%
\begin{pgfscope}%
\pgfpathrectangle{\pgfqpoint{0.750000in}{0.500000in}}{\pgfqpoint{4.650000in}{3.020000in}}%
\pgfusepath{clip}%
\pgfsetbuttcap%
\pgfsetroundjoin%
\definecolor{currentfill}{rgb}{0.121569,0.466667,0.705882}%
\pgfsetfillcolor{currentfill}%
\pgfsetlinewidth{1.003750pt}%
\definecolor{currentstroke}{rgb}{0.121569,0.466667,0.705882}%
\pgfsetstrokecolor{currentstroke}%
\pgfsetdash{}{0pt}%
\pgfpathmoveto{\pgfqpoint{1.424641in}{1.633654in}}%
\pgfpathcurveto{\pgfqpoint{1.427245in}{1.633654in}}{\pgfqpoint{1.429743in}{1.634689in}}{\pgfqpoint{1.431585in}{1.636531in}}%
\pgfpathcurveto{\pgfqpoint{1.433427in}{1.638373in}}{\pgfqpoint{1.434461in}{1.640871in}}{\pgfqpoint{1.434461in}{1.643475in}}%
\pgfpathcurveto{\pgfqpoint{1.434461in}{1.646080in}}{\pgfqpoint{1.433427in}{1.648578in}}{\pgfqpoint{1.431585in}{1.650420in}}%
\pgfpathcurveto{\pgfqpoint{1.429743in}{1.652261in}}{\pgfqpoint{1.427245in}{1.653296in}}{\pgfqpoint{1.424641in}{1.653296in}}%
\pgfpathcurveto{\pgfqpoint{1.422036in}{1.653296in}}{\pgfqpoint{1.419538in}{1.652261in}}{\pgfqpoint{1.417696in}{1.650420in}}%
\pgfpathcurveto{\pgfqpoint{1.415854in}{1.648578in}}{\pgfqpoint{1.414820in}{1.646080in}}{\pgfqpoint{1.414820in}{1.643475in}}%
\pgfpathcurveto{\pgfqpoint{1.414820in}{1.640871in}}{\pgfqpoint{1.415854in}{1.638373in}}{\pgfqpoint{1.417696in}{1.636531in}}%
\pgfpathcurveto{\pgfqpoint{1.419538in}{1.634689in}}{\pgfqpoint{1.422036in}{1.633654in}}{\pgfqpoint{1.424641in}{1.633654in}}%
\pgfpathclose%
\pgfusepath{stroke,fill}%
\end{pgfscope}%
\begin{pgfscope}%
\pgfpathrectangle{\pgfqpoint{0.750000in}{0.500000in}}{\pgfqpoint{4.650000in}{3.020000in}}%
\pgfusepath{clip}%
\pgfsetbuttcap%
\pgfsetroundjoin%
\definecolor{currentfill}{rgb}{0.121569,0.466667,0.705882}%
\pgfsetfillcolor{currentfill}%
\pgfsetlinewidth{1.003750pt}%
\definecolor{currentstroke}{rgb}{0.121569,0.466667,0.705882}%
\pgfsetstrokecolor{currentstroke}%
\pgfsetdash{}{0pt}%
\pgfpathmoveto{\pgfqpoint{0.984063in}{0.637053in}}%
\pgfpathcurveto{\pgfqpoint{0.986668in}{0.637053in}}{\pgfqpoint{0.989166in}{0.638088in}}{\pgfqpoint{0.991008in}{0.639929in}}%
\pgfpathcurveto{\pgfqpoint{0.992849in}{0.641771in}}{\pgfqpoint{0.993884in}{0.644269in}}{\pgfqpoint{0.993884in}{0.646874in}}%
\pgfpathcurveto{\pgfqpoint{0.993884in}{0.649478in}}{\pgfqpoint{0.992849in}{0.651977in}}{\pgfqpoint{0.991008in}{0.653818in}}%
\pgfpathcurveto{\pgfqpoint{0.989166in}{0.655660in}}{\pgfqpoint{0.986668in}{0.656695in}}{\pgfqpoint{0.984063in}{0.656695in}}%
\pgfpathcurveto{\pgfqpoint{0.981459in}{0.656695in}}{\pgfqpoint{0.978960in}{0.655660in}}{\pgfqpoint{0.977119in}{0.653818in}}%
\pgfpathcurveto{\pgfqpoint{0.975277in}{0.651977in}}{\pgfqpoint{0.974242in}{0.649478in}}{\pgfqpoint{0.974242in}{0.646874in}}%
\pgfpathcurveto{\pgfqpoint{0.974242in}{0.644269in}}{\pgfqpoint{0.975277in}{0.641771in}}{\pgfqpoint{0.977119in}{0.639929in}}%
\pgfpathcurveto{\pgfqpoint{0.978960in}{0.638088in}}{\pgfqpoint{0.981459in}{0.637053in}}{\pgfqpoint{0.984063in}{0.637053in}}%
\pgfpathclose%
\pgfusepath{stroke,fill}%
\end{pgfscope}%
\begin{pgfscope}%
\pgfpathrectangle{\pgfqpoint{0.750000in}{0.500000in}}{\pgfqpoint{4.650000in}{3.020000in}}%
\pgfusepath{clip}%
\pgfsetbuttcap%
\pgfsetroundjoin%
\definecolor{currentfill}{rgb}{0.121569,0.466667,0.705882}%
\pgfsetfillcolor{currentfill}%
\pgfsetlinewidth{1.003750pt}%
\definecolor{currentstroke}{rgb}{0.121569,0.466667,0.705882}%
\pgfsetstrokecolor{currentstroke}%
\pgfsetdash{}{0pt}%
\pgfpathmoveto{\pgfqpoint{1.522145in}{0.878522in}}%
\pgfpathcurveto{\pgfqpoint{1.524750in}{0.878522in}}{\pgfqpoint{1.527248in}{0.879557in}}{\pgfqpoint{1.529090in}{0.881399in}}%
\pgfpathcurveto{\pgfqpoint{1.530931in}{0.883241in}}{\pgfqpoint{1.531966in}{0.885739in}}{\pgfqpoint{1.531966in}{0.888343in}}%
\pgfpathcurveto{\pgfqpoint{1.531966in}{0.890948in}}{\pgfqpoint{1.530931in}{0.893446in}}{\pgfqpoint{1.529090in}{0.895288in}}%
\pgfpathcurveto{\pgfqpoint{1.527248in}{0.897129in}}{\pgfqpoint{1.524750in}{0.898164in}}{\pgfqpoint{1.522145in}{0.898164in}}%
\pgfpathcurveto{\pgfqpoint{1.519541in}{0.898164in}}{\pgfqpoint{1.517043in}{0.897129in}}{\pgfqpoint{1.515201in}{0.895288in}}%
\pgfpathcurveto{\pgfqpoint{1.513359in}{0.893446in}}{\pgfqpoint{1.512324in}{0.890948in}}{\pgfqpoint{1.512324in}{0.888343in}}%
\pgfpathcurveto{\pgfqpoint{1.512324in}{0.885739in}}{\pgfqpoint{1.513359in}{0.883241in}}{\pgfqpoint{1.515201in}{0.881399in}}%
\pgfpathcurveto{\pgfqpoint{1.517043in}{0.879557in}}{\pgfqpoint{1.519541in}{0.878522in}}{\pgfqpoint{1.522145in}{0.878522in}}%
\pgfpathclose%
\pgfusepath{stroke,fill}%
\end{pgfscope}%
\begin{pgfscope}%
\pgfpathrectangle{\pgfqpoint{0.750000in}{0.500000in}}{\pgfqpoint{4.650000in}{3.020000in}}%
\pgfusepath{clip}%
\pgfsetbuttcap%
\pgfsetroundjoin%
\definecolor{currentfill}{rgb}{0.121569,0.466667,0.705882}%
\pgfsetfillcolor{currentfill}%
\pgfsetlinewidth{1.003750pt}%
\definecolor{currentstroke}{rgb}{0.121569,0.466667,0.705882}%
\pgfsetstrokecolor{currentstroke}%
\pgfsetdash{}{0pt}%
\pgfpathmoveto{\pgfqpoint{1.248203in}{0.781551in}}%
\pgfpathcurveto{\pgfqpoint{1.250808in}{0.781551in}}{\pgfqpoint{1.253306in}{0.782585in}}{\pgfqpoint{1.255148in}{0.784427in}}%
\pgfpathcurveto{\pgfqpoint{1.256989in}{0.786269in}}{\pgfqpoint{1.258024in}{0.788767in}}{\pgfqpoint{1.258024in}{0.791372in}}%
\pgfpathcurveto{\pgfqpoint{1.258024in}{0.793976in}}{\pgfqpoint{1.256989in}{0.796474in}}{\pgfqpoint{1.255148in}{0.798316in}}%
\pgfpathcurveto{\pgfqpoint{1.253306in}{0.800158in}}{\pgfqpoint{1.250808in}{0.801192in}}{\pgfqpoint{1.248203in}{0.801192in}}%
\pgfpathcurveto{\pgfqpoint{1.245599in}{0.801192in}}{\pgfqpoint{1.243100in}{0.800158in}}{\pgfqpoint{1.241259in}{0.798316in}}%
\pgfpathcurveto{\pgfqpoint{1.239417in}{0.796474in}}{\pgfqpoint{1.238382in}{0.793976in}}{\pgfqpoint{1.238382in}{0.791372in}}%
\pgfpathcurveto{\pgfqpoint{1.238382in}{0.788767in}}{\pgfqpoint{1.239417in}{0.786269in}}{\pgfqpoint{1.241259in}{0.784427in}}%
\pgfpathcurveto{\pgfqpoint{1.243100in}{0.782585in}}{\pgfqpoint{1.245599in}{0.781551in}}{\pgfqpoint{1.248203in}{0.781551in}}%
\pgfpathclose%
\pgfusepath{stroke,fill}%
\end{pgfscope}%
\begin{pgfscope}%
\pgfpathrectangle{\pgfqpoint{0.750000in}{0.500000in}}{\pgfqpoint{4.650000in}{3.020000in}}%
\pgfusepath{clip}%
\pgfsetbuttcap%
\pgfsetroundjoin%
\definecolor{currentfill}{rgb}{0.121569,0.466667,0.705882}%
\pgfsetfillcolor{currentfill}%
\pgfsetlinewidth{1.003750pt}%
\definecolor{currentstroke}{rgb}{0.121569,0.466667,0.705882}%
\pgfsetstrokecolor{currentstroke}%
\pgfsetdash{}{0pt}%
\pgfpathmoveto{\pgfqpoint{0.971682in}{0.628892in}}%
\pgfpathcurveto{\pgfqpoint{0.974286in}{0.628892in}}{\pgfqpoint{0.976784in}{0.629927in}}{\pgfqpoint{0.978626in}{0.631768in}}%
\pgfpathcurveto{\pgfqpoint{0.980468in}{0.633610in}}{\pgfqpoint{0.981503in}{0.636108in}}{\pgfqpoint{0.981503in}{0.638713in}}%
\pgfpathcurveto{\pgfqpoint{0.981503in}{0.641317in}}{\pgfqpoint{0.980468in}{0.643816in}}{\pgfqpoint{0.978626in}{0.645657in}}%
\pgfpathcurveto{\pgfqpoint{0.976784in}{0.647499in}}{\pgfqpoint{0.974286in}{0.648534in}}{\pgfqpoint{0.971682in}{0.648534in}}%
\pgfpathcurveto{\pgfqpoint{0.969077in}{0.648534in}}{\pgfqpoint{0.966579in}{0.647499in}}{\pgfqpoint{0.964737in}{0.645657in}}%
\pgfpathcurveto{\pgfqpoint{0.962895in}{0.643816in}}{\pgfqpoint{0.961861in}{0.641317in}}{\pgfqpoint{0.961861in}{0.638713in}}%
\pgfpathcurveto{\pgfqpoint{0.961861in}{0.636108in}}{\pgfqpoint{0.962895in}{0.633610in}}{\pgfqpoint{0.964737in}{0.631768in}}%
\pgfpathcurveto{\pgfqpoint{0.966579in}{0.629927in}}{\pgfqpoint{0.969077in}{0.628892in}}{\pgfqpoint{0.971682in}{0.628892in}}%
\pgfpathclose%
\pgfusepath{stroke,fill}%
\end{pgfscope}%
\begin{pgfscope}%
\pgfpathrectangle{\pgfqpoint{0.750000in}{0.500000in}}{\pgfqpoint{4.650000in}{3.020000in}}%
\pgfusepath{clip}%
\pgfsetbuttcap%
\pgfsetroundjoin%
\definecolor{currentfill}{rgb}{0.121569,0.466667,0.705882}%
\pgfsetfillcolor{currentfill}%
\pgfsetlinewidth{1.003750pt}%
\definecolor{currentstroke}{rgb}{0.121569,0.466667,0.705882}%
\pgfsetstrokecolor{currentstroke}%
\pgfsetdash{}{0pt}%
\pgfpathmoveto{\pgfqpoint{1.572188in}{2.225827in}}%
\pgfpathcurveto{\pgfqpoint{1.577397in}{2.225827in}}{\pgfqpoint{1.582393in}{2.227896in}}{\pgfqpoint{1.586076in}{2.231580in}}%
\pgfpathcurveto{\pgfqpoint{1.589760in}{2.235263in}}{\pgfqpoint{1.591829in}{2.240260in}}{\pgfqpoint{1.591829in}{2.245469in}}%
\pgfpathcurveto{\pgfqpoint{1.591829in}{2.250678in}}{\pgfqpoint{1.589760in}{2.255674in}}{\pgfqpoint{1.586076in}{2.259358in}}%
\pgfpathcurveto{\pgfqpoint{1.582393in}{2.263041in}}{\pgfqpoint{1.577397in}{2.265111in}}{\pgfqpoint{1.572188in}{2.265111in}}%
\pgfpathcurveto{\pgfqpoint{1.566978in}{2.265111in}}{\pgfqpoint{1.561982in}{2.263041in}}{\pgfqpoint{1.558299in}{2.259358in}}%
\pgfpathcurveto{\pgfqpoint{1.554615in}{2.255674in}}{\pgfqpoint{1.552546in}{2.250678in}}{\pgfqpoint{1.552546in}{2.245469in}}%
\pgfpathcurveto{\pgfqpoint{1.552546in}{2.240260in}}{\pgfqpoint{1.554615in}{2.235263in}}{\pgfqpoint{1.558299in}{2.231580in}}%
\pgfpathcurveto{\pgfqpoint{1.561982in}{2.227896in}}{\pgfqpoint{1.566978in}{2.225827in}}{\pgfqpoint{1.572188in}{2.225827in}}%
\pgfpathclose%
\pgfusepath{stroke,fill}%
\end{pgfscope}%
\begin{pgfscope}%
\pgfpathrectangle{\pgfqpoint{0.750000in}{0.500000in}}{\pgfqpoint{4.650000in}{3.020000in}}%
\pgfusepath{clip}%
\pgfsetbuttcap%
\pgfsetroundjoin%
\definecolor{currentfill}{rgb}{0.121569,0.466667,0.705882}%
\pgfsetfillcolor{currentfill}%
\pgfsetlinewidth{1.003750pt}%
\definecolor{currentstroke}{rgb}{0.121569,0.466667,0.705882}%
\pgfsetstrokecolor{currentstroke}%
\pgfsetdash{}{0pt}%
\pgfpathmoveto{\pgfqpoint{1.501509in}{0.790683in}}%
\pgfpathcurveto{\pgfqpoint{1.506021in}{0.790683in}}{\pgfqpoint{1.510348in}{0.792475in}}{\pgfqpoint{1.513538in}{0.795665in}}%
\pgfpathcurveto{\pgfqpoint{1.516727in}{0.798855in}}{\pgfqpoint{1.518520in}{0.803182in}}{\pgfqpoint{1.518520in}{0.807693in}}%
\pgfpathcurveto{\pgfqpoint{1.518520in}{0.812205in}}{\pgfqpoint{1.516727in}{0.816532in}}{\pgfqpoint{1.513538in}{0.819722in}}%
\pgfpathcurveto{\pgfqpoint{1.510348in}{0.822912in}}{\pgfqpoint{1.506021in}{0.824704in}}{\pgfqpoint{1.501509in}{0.824704in}}%
\pgfpathcurveto{\pgfqpoint{1.496998in}{0.824704in}}{\pgfqpoint{1.492671in}{0.822912in}}{\pgfqpoint{1.489481in}{0.819722in}}%
\pgfpathcurveto{\pgfqpoint{1.486291in}{0.816532in}}{\pgfqpoint{1.484499in}{0.812205in}}{\pgfqpoint{1.484499in}{0.807693in}}%
\pgfpathcurveto{\pgfqpoint{1.484499in}{0.803182in}}{\pgfqpoint{1.486291in}{0.798855in}}{\pgfqpoint{1.489481in}{0.795665in}}%
\pgfpathcurveto{\pgfqpoint{1.492671in}{0.792475in}}{\pgfqpoint{1.496998in}{0.790683in}}{\pgfqpoint{1.501509in}{0.790683in}}%
\pgfpathclose%
\pgfusepath{stroke,fill}%
\end{pgfscope}%
\begin{pgfscope}%
\pgfpathrectangle{\pgfqpoint{0.750000in}{0.500000in}}{\pgfqpoint{4.650000in}{3.020000in}}%
\pgfusepath{clip}%
\pgfsetbuttcap%
\pgfsetroundjoin%
\definecolor{currentfill}{rgb}{0.121569,0.466667,0.705882}%
\pgfsetfillcolor{currentfill}%
\pgfsetlinewidth{1.003750pt}%
\definecolor{currentstroke}{rgb}{0.121569,0.466667,0.705882}%
\pgfsetstrokecolor{currentstroke}%
\pgfsetdash{}{0pt}%
\pgfpathmoveto{\pgfqpoint{1.480873in}{0.891004in}}%
\pgfpathcurveto{\pgfqpoint{1.483478in}{0.891004in}}{\pgfqpoint{1.485976in}{0.892039in}}{\pgfqpoint{1.487818in}{0.893880in}}%
\pgfpathcurveto{\pgfqpoint{1.489660in}{0.895722in}}{\pgfqpoint{1.490694in}{0.898220in}}{\pgfqpoint{1.490694in}{0.900825in}}%
\pgfpathcurveto{\pgfqpoint{1.490694in}{0.903429in}}{\pgfqpoint{1.489660in}{0.905928in}}{\pgfqpoint{1.487818in}{0.907769in}}%
\pgfpathcurveto{\pgfqpoint{1.485976in}{0.909611in}}{\pgfqpoint{1.483478in}{0.910646in}}{\pgfqpoint{1.480873in}{0.910646in}}%
\pgfpathcurveto{\pgfqpoint{1.478269in}{0.910646in}}{\pgfqpoint{1.475771in}{0.909611in}}{\pgfqpoint{1.473929in}{0.907769in}}%
\pgfpathcurveto{\pgfqpoint{1.472087in}{0.905928in}}{\pgfqpoint{1.471053in}{0.903429in}}{\pgfqpoint{1.471053in}{0.900825in}}%
\pgfpathcurveto{\pgfqpoint{1.471053in}{0.898220in}}{\pgfqpoint{1.472087in}{0.895722in}}{\pgfqpoint{1.473929in}{0.893880in}}%
\pgfpathcurveto{\pgfqpoint{1.475771in}{0.892039in}}{\pgfqpoint{1.478269in}{0.891004in}}{\pgfqpoint{1.480873in}{0.891004in}}%
\pgfpathclose%
\pgfusepath{stroke,fill}%
\end{pgfscope}%
\begin{pgfscope}%
\pgfpathrectangle{\pgfqpoint{0.750000in}{0.500000in}}{\pgfqpoint{4.650000in}{3.020000in}}%
\pgfusepath{clip}%
\pgfsetbuttcap%
\pgfsetroundjoin%
\definecolor{currentfill}{rgb}{0.121569,0.466667,0.705882}%
\pgfsetfillcolor{currentfill}%
\pgfsetlinewidth{1.003750pt}%
\definecolor{currentstroke}{rgb}{0.121569,0.466667,0.705882}%
\pgfsetstrokecolor{currentstroke}%
\pgfsetdash{}{0pt}%
\pgfpathmoveto{\pgfqpoint{1.329715in}{0.954852in}}%
\pgfpathcurveto{\pgfqpoint{1.332320in}{0.954852in}}{\pgfqpoint{1.334818in}{0.955886in}}{\pgfqpoint{1.336660in}{0.957728in}}%
\pgfpathcurveto{\pgfqpoint{1.338501in}{0.959570in}}{\pgfqpoint{1.339536in}{0.962068in}}{\pgfqpoint{1.339536in}{0.964673in}}%
\pgfpathcurveto{\pgfqpoint{1.339536in}{0.967277in}}{\pgfqpoint{1.338501in}{0.969775in}}{\pgfqpoint{1.336660in}{0.971617in}}%
\pgfpathcurveto{\pgfqpoint{1.334818in}{0.973459in}}{\pgfqpoint{1.332320in}{0.974494in}}{\pgfqpoint{1.329715in}{0.974494in}}%
\pgfpathcurveto{\pgfqpoint{1.327111in}{0.974494in}}{\pgfqpoint{1.324612in}{0.973459in}}{\pgfqpoint{1.322771in}{0.971617in}}%
\pgfpathcurveto{\pgfqpoint{1.320929in}{0.969775in}}{\pgfqpoint{1.319894in}{0.967277in}}{\pgfqpoint{1.319894in}{0.964673in}}%
\pgfpathcurveto{\pgfqpoint{1.319894in}{0.962068in}}{\pgfqpoint{1.320929in}{0.959570in}}{\pgfqpoint{1.322771in}{0.957728in}}%
\pgfpathcurveto{\pgfqpoint{1.324612in}{0.955886in}}{\pgfqpoint{1.327111in}{0.954852in}}{\pgfqpoint{1.329715in}{0.954852in}}%
\pgfpathclose%
\pgfusepath{stroke,fill}%
\end{pgfscope}%
\begin{pgfscope}%
\pgfpathrectangle{\pgfqpoint{0.750000in}{0.500000in}}{\pgfqpoint{4.650000in}{3.020000in}}%
\pgfusepath{clip}%
\pgfsetbuttcap%
\pgfsetroundjoin%
\definecolor{currentfill}{rgb}{0.121569,0.466667,0.705882}%
\pgfsetfillcolor{currentfill}%
\pgfsetlinewidth{1.003750pt}%
\definecolor{currentstroke}{rgb}{0.121569,0.466667,0.705882}%
\pgfsetstrokecolor{currentstroke}%
\pgfsetdash{}{0pt}%
\pgfpathmoveto{\pgfqpoint{1.102204in}{0.696580in}}%
\pgfpathcurveto{\pgfqpoint{1.104808in}{0.696580in}}{\pgfqpoint{1.107307in}{0.697615in}}{\pgfqpoint{1.109148in}{0.699457in}}%
\pgfpathcurveto{\pgfqpoint{1.110990in}{0.701298in}}{\pgfqpoint{1.112025in}{0.703797in}}{\pgfqpoint{1.112025in}{0.706401in}}%
\pgfpathcurveto{\pgfqpoint{1.112025in}{0.709006in}}{\pgfqpoint{1.110990in}{0.711504in}}{\pgfqpoint{1.109148in}{0.713346in}}%
\pgfpathcurveto{\pgfqpoint{1.107307in}{0.715187in}}{\pgfqpoint{1.104808in}{0.716222in}}{\pgfqpoint{1.102204in}{0.716222in}}%
\pgfpathcurveto{\pgfqpoint{1.099599in}{0.716222in}}{\pgfqpoint{1.097101in}{0.715187in}}{\pgfqpoint{1.095259in}{0.713346in}}%
\pgfpathcurveto{\pgfqpoint{1.093418in}{0.711504in}}{\pgfqpoint{1.092383in}{0.709006in}}{\pgfqpoint{1.092383in}{0.706401in}}%
\pgfpathcurveto{\pgfqpoint{1.092383in}{0.703797in}}{\pgfqpoint{1.093418in}{0.701298in}}{\pgfqpoint{1.095259in}{0.699457in}}%
\pgfpathcurveto{\pgfqpoint{1.097101in}{0.697615in}}{\pgfqpoint{1.099599in}{0.696580in}}{\pgfqpoint{1.102204in}{0.696580in}}%
\pgfpathclose%
\pgfusepath{stroke,fill}%
\end{pgfscope}%
\begin{pgfscope}%
\pgfpathrectangle{\pgfqpoint{0.750000in}{0.500000in}}{\pgfqpoint{4.650000in}{3.020000in}}%
\pgfusepath{clip}%
\pgfsetbuttcap%
\pgfsetroundjoin%
\definecolor{currentfill}{rgb}{0.121569,0.466667,0.705882}%
\pgfsetfillcolor{currentfill}%
\pgfsetlinewidth{1.003750pt}%
\definecolor{currentstroke}{rgb}{0.121569,0.466667,0.705882}%
\pgfsetstrokecolor{currentstroke}%
\pgfsetdash{}{0pt}%
\pgfpathmoveto{\pgfqpoint{0.999024in}{0.637533in}}%
\pgfpathcurveto{\pgfqpoint{1.001629in}{0.637533in}}{\pgfqpoint{1.004127in}{0.638568in}}{\pgfqpoint{1.005969in}{0.640410in}}%
\pgfpathcurveto{\pgfqpoint{1.007810in}{0.642251in}}{\pgfqpoint{1.008845in}{0.644749in}}{\pgfqpoint{1.008845in}{0.647354in}}%
\pgfpathcurveto{\pgfqpoint{1.008845in}{0.649958in}}{\pgfqpoint{1.007810in}{0.652457in}}{\pgfqpoint{1.005969in}{0.654298in}}%
\pgfpathcurveto{\pgfqpoint{1.004127in}{0.656140in}}{\pgfqpoint{1.001629in}{0.657175in}}{\pgfqpoint{0.999024in}{0.657175in}}%
\pgfpathcurveto{\pgfqpoint{0.996420in}{0.657175in}}{\pgfqpoint{0.993921in}{0.656140in}}{\pgfqpoint{0.992080in}{0.654298in}}%
\pgfpathcurveto{\pgfqpoint{0.990238in}{0.652457in}}{\pgfqpoint{0.989203in}{0.649958in}}{\pgfqpoint{0.989203in}{0.647354in}}%
\pgfpathcurveto{\pgfqpoint{0.989203in}{0.644749in}}{\pgfqpoint{0.990238in}{0.642251in}}{\pgfqpoint{0.992080in}{0.640410in}}%
\pgfpathcurveto{\pgfqpoint{0.993921in}{0.638568in}}{\pgfqpoint{0.996420in}{0.637533in}}{\pgfqpoint{0.999024in}{0.637533in}}%
\pgfpathclose%
\pgfusepath{stroke,fill}%
\end{pgfscope}%
\begin{pgfscope}%
\pgfpathrectangle{\pgfqpoint{0.750000in}{0.500000in}}{\pgfqpoint{4.650000in}{3.020000in}}%
\pgfusepath{clip}%
\pgfsetbuttcap%
\pgfsetroundjoin%
\definecolor{currentfill}{rgb}{0.121569,0.466667,0.705882}%
\pgfsetfillcolor{currentfill}%
\pgfsetlinewidth{1.003750pt}%
\definecolor{currentstroke}{rgb}{0.121569,0.466667,0.705882}%
\pgfsetstrokecolor{currentstroke}%
\pgfsetdash{}{0pt}%
\pgfpathmoveto{\pgfqpoint{1.426188in}{0.746080in}}%
\pgfpathcurveto{\pgfqpoint{1.430306in}{0.746080in}}{\pgfqpoint{1.434256in}{0.747716in}}{\pgfqpoint{1.437168in}{0.750628in}}%
\pgfpathcurveto{\pgfqpoint{1.440080in}{0.753540in}}{\pgfqpoint{1.441716in}{0.757490in}}{\pgfqpoint{1.441716in}{0.761608in}}%
\pgfpathcurveto{\pgfqpoint{1.441716in}{0.765726in}}{\pgfqpoint{1.440080in}{0.769676in}}{\pgfqpoint{1.437168in}{0.772588in}}%
\pgfpathcurveto{\pgfqpoint{1.434256in}{0.775500in}}{\pgfqpoint{1.430306in}{0.777136in}}{\pgfqpoint{1.426188in}{0.777136in}}%
\pgfpathcurveto{\pgfqpoint{1.422070in}{0.777136in}}{\pgfqpoint{1.418120in}{0.775500in}}{\pgfqpoint{1.415208in}{0.772588in}}%
\pgfpathcurveto{\pgfqpoint{1.412296in}{0.769676in}}{\pgfqpoint{1.410660in}{0.765726in}}{\pgfqpoint{1.410660in}{0.761608in}}%
\pgfpathcurveto{\pgfqpoint{1.410660in}{0.757490in}}{\pgfqpoint{1.412296in}{0.753540in}}{\pgfqpoint{1.415208in}{0.750628in}}%
\pgfpathcurveto{\pgfqpoint{1.418120in}{0.747716in}}{\pgfqpoint{1.422070in}{0.746080in}}{\pgfqpoint{1.426188in}{0.746080in}}%
\pgfpathclose%
\pgfusepath{stroke,fill}%
\end{pgfscope}%
\begin{pgfscope}%
\pgfpathrectangle{\pgfqpoint{0.750000in}{0.500000in}}{\pgfqpoint{4.650000in}{3.020000in}}%
\pgfusepath{clip}%
\pgfsetbuttcap%
\pgfsetroundjoin%
\definecolor{currentfill}{rgb}{0.121569,0.466667,0.705882}%
\pgfsetfillcolor{currentfill}%
\pgfsetlinewidth{1.003750pt}%
\definecolor{currentstroke}{rgb}{0.121569,0.466667,0.705882}%
\pgfsetstrokecolor{currentstroke}%
\pgfsetdash{}{0pt}%
\pgfpathmoveto{\pgfqpoint{1.039780in}{0.640413in}}%
\pgfpathcurveto{\pgfqpoint{1.042385in}{0.640413in}}{\pgfqpoint{1.044883in}{0.641448in}}{\pgfqpoint{1.046725in}{0.643290in}}%
\pgfpathcurveto{\pgfqpoint{1.048566in}{0.645132in}}{\pgfqpoint{1.049601in}{0.647630in}}{\pgfqpoint{1.049601in}{0.650234in}}%
\pgfpathcurveto{\pgfqpoint{1.049601in}{0.652839in}}{\pgfqpoint{1.048566in}{0.655337in}}{\pgfqpoint{1.046725in}{0.657179in}}%
\pgfpathcurveto{\pgfqpoint{1.044883in}{0.659020in}}{\pgfqpoint{1.042385in}{0.660055in}}{\pgfqpoint{1.039780in}{0.660055in}}%
\pgfpathcurveto{\pgfqpoint{1.037176in}{0.660055in}}{\pgfqpoint{1.034677in}{0.659020in}}{\pgfqpoint{1.032836in}{0.657179in}}%
\pgfpathcurveto{\pgfqpoint{1.030994in}{0.655337in}}{\pgfqpoint{1.029959in}{0.652839in}}{\pgfqpoint{1.029959in}{0.650234in}}%
\pgfpathcurveto{\pgfqpoint{1.029959in}{0.647630in}}{\pgfqpoint{1.030994in}{0.645132in}}{\pgfqpoint{1.032836in}{0.643290in}}%
\pgfpathcurveto{\pgfqpoint{1.034677in}{0.641448in}}{\pgfqpoint{1.037176in}{0.640413in}}{\pgfqpoint{1.039780in}{0.640413in}}%
\pgfpathclose%
\pgfusepath{stroke,fill}%
\end{pgfscope}%
\begin{pgfscope}%
\pgfpathrectangle{\pgfqpoint{0.750000in}{0.500000in}}{\pgfqpoint{4.650000in}{3.020000in}}%
\pgfusepath{clip}%
\pgfsetbuttcap%
\pgfsetroundjoin%
\definecolor{currentfill}{rgb}{0.121569,0.466667,0.705882}%
\pgfsetfillcolor{currentfill}%
\pgfsetlinewidth{1.003750pt}%
\definecolor{currentstroke}{rgb}{0.121569,0.466667,0.705882}%
\pgfsetstrokecolor{currentstroke}%
\pgfsetdash{}{0pt}%
\pgfpathmoveto{\pgfqpoint{1.042876in}{0.634653in}}%
\pgfpathcurveto{\pgfqpoint{1.045480in}{0.634653in}}{\pgfqpoint{1.047978in}{0.635687in}}{\pgfqpoint{1.049820in}{0.637529in}}%
\pgfpathcurveto{\pgfqpoint{1.051662in}{0.639371in}}{\pgfqpoint{1.052697in}{0.641869in}}{\pgfqpoint{1.052697in}{0.644474in}}%
\pgfpathcurveto{\pgfqpoint{1.052697in}{0.647078in}}{\pgfqpoint{1.051662in}{0.649576in}}{\pgfqpoint{1.049820in}{0.651418in}}%
\pgfpathcurveto{\pgfqpoint{1.047978in}{0.653260in}}{\pgfqpoint{1.045480in}{0.654295in}}{\pgfqpoint{1.042876in}{0.654295in}}%
\pgfpathcurveto{\pgfqpoint{1.040271in}{0.654295in}}{\pgfqpoint{1.037773in}{0.653260in}}{\pgfqpoint{1.035931in}{0.651418in}}%
\pgfpathcurveto{\pgfqpoint{1.034089in}{0.649576in}}{\pgfqpoint{1.033055in}{0.647078in}}{\pgfqpoint{1.033055in}{0.644474in}}%
\pgfpathcurveto{\pgfqpoint{1.033055in}{0.641869in}}{\pgfqpoint{1.034089in}{0.639371in}}{\pgfqpoint{1.035931in}{0.637529in}}%
\pgfpathcurveto{\pgfqpoint{1.037773in}{0.635687in}}{\pgfqpoint{1.040271in}{0.634653in}}{\pgfqpoint{1.042876in}{0.634653in}}%
\pgfpathclose%
\pgfusepath{stroke,fill}%
\end{pgfscope}%
\begin{pgfscope}%
\pgfpathrectangle{\pgfqpoint{0.750000in}{0.500000in}}{\pgfqpoint{4.650000in}{3.020000in}}%
\pgfusepath{clip}%
\pgfsetbuttcap%
\pgfsetroundjoin%
\definecolor{currentfill}{rgb}{0.121569,0.466667,0.705882}%
\pgfsetfillcolor{currentfill}%
\pgfsetlinewidth{1.003750pt}%
\definecolor{currentstroke}{rgb}{0.121569,0.466667,0.705882}%
\pgfsetstrokecolor{currentstroke}%
\pgfsetdash{}{0pt}%
\pgfpathmoveto{\pgfqpoint{1.085695in}{0.679778in}}%
\pgfpathcurveto{\pgfqpoint{1.088300in}{0.679778in}}{\pgfqpoint{1.090798in}{0.680813in}}{\pgfqpoint{1.092640in}{0.682655in}}%
\pgfpathcurveto{\pgfqpoint{1.094481in}{0.684496in}}{\pgfqpoint{1.095516in}{0.686995in}}{\pgfqpoint{1.095516in}{0.689599in}}%
\pgfpathcurveto{\pgfqpoint{1.095516in}{0.692204in}}{\pgfqpoint{1.094481in}{0.694702in}}{\pgfqpoint{1.092640in}{0.696544in}}%
\pgfpathcurveto{\pgfqpoint{1.090798in}{0.698385in}}{\pgfqpoint{1.088300in}{0.699420in}}{\pgfqpoint{1.085695in}{0.699420in}}%
\pgfpathcurveto{\pgfqpoint{1.083091in}{0.699420in}}{\pgfqpoint{1.080592in}{0.698385in}}{\pgfqpoint{1.078751in}{0.696544in}}%
\pgfpathcurveto{\pgfqpoint{1.076909in}{0.694702in}}{\pgfqpoint{1.075874in}{0.692204in}}{\pgfqpoint{1.075874in}{0.689599in}}%
\pgfpathcurveto{\pgfqpoint{1.075874in}{0.686995in}}{\pgfqpoint{1.076909in}{0.684496in}}{\pgfqpoint{1.078751in}{0.682655in}}%
\pgfpathcurveto{\pgfqpoint{1.080592in}{0.680813in}}{\pgfqpoint{1.083091in}{0.679778in}}{\pgfqpoint{1.085695in}{0.679778in}}%
\pgfpathclose%
\pgfusepath{stroke,fill}%
\end{pgfscope}%
\begin{pgfscope}%
\pgfpathrectangle{\pgfqpoint{0.750000in}{0.500000in}}{\pgfqpoint{4.650000in}{3.020000in}}%
\pgfusepath{clip}%
\pgfsetbuttcap%
\pgfsetroundjoin%
\definecolor{currentfill}{rgb}{0.121569,0.466667,0.705882}%
\pgfsetfillcolor{currentfill}%
\pgfsetlinewidth{1.003750pt}%
\definecolor{currentstroke}{rgb}{0.121569,0.466667,0.705882}%
\pgfsetstrokecolor{currentstroke}%
\pgfsetdash{}{0pt}%
\pgfpathmoveto{\pgfqpoint{1.140380in}{0.682178in}}%
\pgfpathcurveto{\pgfqpoint{1.142985in}{0.682178in}}{\pgfqpoint{1.145483in}{0.683213in}}{\pgfqpoint{1.147325in}{0.685055in}}%
\pgfpathcurveto{\pgfqpoint{1.149167in}{0.686897in}}{\pgfqpoint{1.150201in}{0.689395in}}{\pgfqpoint{1.150201in}{0.691999in}}%
\pgfpathcurveto{\pgfqpoint{1.150201in}{0.694604in}}{\pgfqpoint{1.149167in}{0.697102in}}{\pgfqpoint{1.147325in}{0.698944in}}%
\pgfpathcurveto{\pgfqpoint{1.145483in}{0.700786in}}{\pgfqpoint{1.142985in}{0.701820in}}{\pgfqpoint{1.140380in}{0.701820in}}%
\pgfpathcurveto{\pgfqpoint{1.137776in}{0.701820in}}{\pgfqpoint{1.135278in}{0.700786in}}{\pgfqpoint{1.133436in}{0.698944in}}%
\pgfpathcurveto{\pgfqpoint{1.131594in}{0.697102in}}{\pgfqpoint{1.130560in}{0.694604in}}{\pgfqpoint{1.130560in}{0.691999in}}%
\pgfpathcurveto{\pgfqpoint{1.130560in}{0.689395in}}{\pgfqpoint{1.131594in}{0.686897in}}{\pgfqpoint{1.133436in}{0.685055in}}%
\pgfpathcurveto{\pgfqpoint{1.135278in}{0.683213in}}{\pgfqpoint{1.137776in}{0.682178in}}{\pgfqpoint{1.140380in}{0.682178in}}%
\pgfpathclose%
\pgfusepath{stroke,fill}%
\end{pgfscope}%
\begin{pgfscope}%
\pgfpathrectangle{\pgfqpoint{0.750000in}{0.500000in}}{\pgfqpoint{4.650000in}{3.020000in}}%
\pgfusepath{clip}%
\pgfsetbuttcap%
\pgfsetroundjoin%
\definecolor{currentfill}{rgb}{0.121569,0.466667,0.705882}%
\pgfsetfillcolor{currentfill}%
\pgfsetlinewidth{1.003750pt}%
\definecolor{currentstroke}{rgb}{0.121569,0.466667,0.705882}%
\pgfsetstrokecolor{currentstroke}%
\pgfsetdash{}{0pt}%
\pgfpathmoveto{\pgfqpoint{0.992833in}{1.230405in}}%
\pgfpathcurveto{\pgfqpoint{0.995438in}{1.230405in}}{\pgfqpoint{0.997936in}{1.231440in}}{\pgfqpoint{0.999778in}{1.233282in}}%
\pgfpathcurveto{\pgfqpoint{1.001620in}{1.235123in}}{\pgfqpoint{1.002654in}{1.237622in}}{\pgfqpoint{1.002654in}{1.240226in}}%
\pgfpathcurveto{\pgfqpoint{1.002654in}{1.242831in}}{\pgfqpoint{1.001620in}{1.245329in}}{\pgfqpoint{0.999778in}{1.247171in}}%
\pgfpathcurveto{\pgfqpoint{0.997936in}{1.249012in}}{\pgfqpoint{0.995438in}{1.250047in}}{\pgfqpoint{0.992833in}{1.250047in}}%
\pgfpathcurveto{\pgfqpoint{0.990229in}{1.250047in}}{\pgfqpoint{0.987731in}{1.249012in}}{\pgfqpoint{0.985889in}{1.247171in}}%
\pgfpathcurveto{\pgfqpoint{0.984047in}{1.245329in}}{\pgfqpoint{0.983013in}{1.242831in}}{\pgfqpoint{0.983013in}{1.240226in}}%
\pgfpathcurveto{\pgfqpoint{0.983013in}{1.237622in}}{\pgfqpoint{0.984047in}{1.235123in}}{\pgfqpoint{0.985889in}{1.233282in}}%
\pgfpathcurveto{\pgfqpoint{0.987731in}{1.231440in}}{\pgfqpoint{0.990229in}{1.230405in}}{\pgfqpoint{0.992833in}{1.230405in}}%
\pgfpathclose%
\pgfusepath{stroke,fill}%
\end{pgfscope}%
\begin{pgfscope}%
\pgfpathrectangle{\pgfqpoint{0.750000in}{0.500000in}}{\pgfqpoint{4.650000in}{3.020000in}}%
\pgfusepath{clip}%
\pgfsetbuttcap%
\pgfsetroundjoin%
\definecolor{currentfill}{rgb}{0.121569,0.466667,0.705882}%
\pgfsetfillcolor{currentfill}%
\pgfsetlinewidth{1.003750pt}%
\definecolor{currentstroke}{rgb}{0.121569,0.466667,0.705882}%
\pgfsetstrokecolor{currentstroke}%
\pgfsetdash{}{0pt}%
\pgfpathmoveto{\pgfqpoint{0.977356in}{0.630812in}}%
\pgfpathcurveto{\pgfqpoint{0.979961in}{0.630812in}}{\pgfqpoint{0.982459in}{0.631847in}}{\pgfqpoint{0.984301in}{0.633689in}}%
\pgfpathcurveto{\pgfqpoint{0.986143in}{0.635530in}}{\pgfqpoint{0.987177in}{0.638029in}}{\pgfqpoint{0.987177in}{0.640633in}}%
\pgfpathcurveto{\pgfqpoint{0.987177in}{0.643238in}}{\pgfqpoint{0.986143in}{0.645736in}}{\pgfqpoint{0.984301in}{0.647578in}}%
\pgfpathcurveto{\pgfqpoint{0.982459in}{0.649419in}}{\pgfqpoint{0.979961in}{0.650454in}}{\pgfqpoint{0.977356in}{0.650454in}}%
\pgfpathcurveto{\pgfqpoint{0.974752in}{0.650454in}}{\pgfqpoint{0.972254in}{0.649419in}}{\pgfqpoint{0.970412in}{0.647578in}}%
\pgfpathcurveto{\pgfqpoint{0.968570in}{0.645736in}}{\pgfqpoint{0.967536in}{0.643238in}}{\pgfqpoint{0.967536in}{0.640633in}}%
\pgfpathcurveto{\pgfqpoint{0.967536in}{0.638029in}}{\pgfqpoint{0.968570in}{0.635530in}}{\pgfqpoint{0.970412in}{0.633689in}}%
\pgfpathcurveto{\pgfqpoint{0.972254in}{0.631847in}}{\pgfqpoint{0.974752in}{0.630812in}}{\pgfqpoint{0.977356in}{0.630812in}}%
\pgfpathclose%
\pgfusepath{stroke,fill}%
\end{pgfscope}%
\begin{pgfscope}%
\pgfpathrectangle{\pgfqpoint{0.750000in}{0.500000in}}{\pgfqpoint{4.650000in}{3.020000in}}%
\pgfusepath{clip}%
\pgfsetbuttcap%
\pgfsetroundjoin%
\definecolor{currentfill}{rgb}{0.121569,0.466667,0.705882}%
\pgfsetfillcolor{currentfill}%
\pgfsetlinewidth{1.003750pt}%
\definecolor{currentstroke}{rgb}{0.121569,0.466667,0.705882}%
\pgfsetstrokecolor{currentstroke}%
\pgfsetdash{}{0pt}%
\pgfpathmoveto{\pgfqpoint{1.246656in}{0.722023in}}%
\pgfpathcurveto{\pgfqpoint{1.249260in}{0.722023in}}{\pgfqpoint{1.251758in}{0.723058in}}{\pgfqpoint{1.253600in}{0.724900in}}%
\pgfpathcurveto{\pgfqpoint{1.255442in}{0.726741in}}{\pgfqpoint{1.256476in}{0.729240in}}{\pgfqpoint{1.256476in}{0.731844in}}%
\pgfpathcurveto{\pgfqpoint{1.256476in}{0.734449in}}{\pgfqpoint{1.255442in}{0.736947in}}{\pgfqpoint{1.253600in}{0.738789in}}%
\pgfpathcurveto{\pgfqpoint{1.251758in}{0.740630in}}{\pgfqpoint{1.249260in}{0.741665in}}{\pgfqpoint{1.246656in}{0.741665in}}%
\pgfpathcurveto{\pgfqpoint{1.244051in}{0.741665in}}{\pgfqpoint{1.241553in}{0.740630in}}{\pgfqpoint{1.239711in}{0.738789in}}%
\pgfpathcurveto{\pgfqpoint{1.237869in}{0.736947in}}{\pgfqpoint{1.236835in}{0.734449in}}{\pgfqpoint{1.236835in}{0.731844in}}%
\pgfpathcurveto{\pgfqpoint{1.236835in}{0.729240in}}{\pgfqpoint{1.237869in}{0.726741in}}{\pgfqpoint{1.239711in}{0.724900in}}%
\pgfpathcurveto{\pgfqpoint{1.241553in}{0.723058in}}{\pgfqpoint{1.244051in}{0.722023in}}{\pgfqpoint{1.246656in}{0.722023in}}%
\pgfpathclose%
\pgfusepath{stroke,fill}%
\end{pgfscope}%
\begin{pgfscope}%
\pgfpathrectangle{\pgfqpoint{0.750000in}{0.500000in}}{\pgfqpoint{4.650000in}{3.020000in}}%
\pgfusepath{clip}%
\pgfsetbuttcap%
\pgfsetroundjoin%
\definecolor{currentfill}{rgb}{0.121569,0.466667,0.705882}%
\pgfsetfillcolor{currentfill}%
\pgfsetlinewidth{1.003750pt}%
\definecolor{currentstroke}{rgb}{0.121569,0.466667,0.705882}%
\pgfsetstrokecolor{currentstroke}%
\pgfsetdash{}{0pt}%
\pgfpathmoveto{\pgfqpoint{1.241497in}{1.083760in}}%
\pgfpathcurveto{\pgfqpoint{1.245180in}{1.083760in}}{\pgfqpoint{1.248713in}{1.085223in}}{\pgfqpoint{1.251317in}{1.087828in}}%
\pgfpathcurveto{\pgfqpoint{1.253922in}{1.090432in}}{\pgfqpoint{1.255385in}{1.093965in}}{\pgfqpoint{1.255385in}{1.097649in}}%
\pgfpathcurveto{\pgfqpoint{1.255385in}{1.101332in}}{\pgfqpoint{1.253922in}{1.104865in}}{\pgfqpoint{1.251317in}{1.107470in}}%
\pgfpathcurveto{\pgfqpoint{1.248713in}{1.110074in}}{\pgfqpoint{1.245180in}{1.111538in}}{\pgfqpoint{1.241497in}{1.111538in}}%
\pgfpathcurveto{\pgfqpoint{1.237813in}{1.111538in}}{\pgfqpoint{1.234280in}{1.110074in}}{\pgfqpoint{1.231676in}{1.107470in}}%
\pgfpathcurveto{\pgfqpoint{1.229071in}{1.104865in}}{\pgfqpoint{1.227608in}{1.101332in}}{\pgfqpoint{1.227608in}{1.097649in}}%
\pgfpathcurveto{\pgfqpoint{1.227608in}{1.093965in}}{\pgfqpoint{1.229071in}{1.090432in}}{\pgfqpoint{1.231676in}{1.087828in}}%
\pgfpathcurveto{\pgfqpoint{1.234280in}{1.085223in}}{\pgfqpoint{1.237813in}{1.083760in}}{\pgfqpoint{1.241497in}{1.083760in}}%
\pgfpathclose%
\pgfusepath{stroke,fill}%
\end{pgfscope}%
\begin{pgfscope}%
\pgfpathrectangle{\pgfqpoint{0.750000in}{0.500000in}}{\pgfqpoint{4.650000in}{3.020000in}}%
\pgfusepath{clip}%
\pgfsetbuttcap%
\pgfsetroundjoin%
\definecolor{currentfill}{rgb}{0.121569,0.466667,0.705882}%
\pgfsetfillcolor{currentfill}%
\pgfsetlinewidth{1.003750pt}%
\definecolor{currentstroke}{rgb}{0.121569,0.466667,0.705882}%
\pgfsetstrokecolor{currentstroke}%
\pgfsetdash{}{0pt}%
\pgfpathmoveto{\pgfqpoint{1.272966in}{0.859513in}}%
\pgfpathcurveto{\pgfqpoint{1.276156in}{0.859513in}}{\pgfqpoint{1.279216in}{0.860781in}}{\pgfqpoint{1.281472in}{0.863036in}}%
\pgfpathcurveto{\pgfqpoint{1.283727in}{0.865292in}}{\pgfqpoint{1.284994in}{0.868351in}}{\pgfqpoint{1.284994in}{0.871541in}}%
\pgfpathcurveto{\pgfqpoint{1.284994in}{0.874731in}}{\pgfqpoint{1.283727in}{0.877791in}}{\pgfqpoint{1.281472in}{0.880046in}}%
\pgfpathcurveto{\pgfqpoint{1.279216in}{0.882302in}}{\pgfqpoint{1.276156in}{0.883569in}}{\pgfqpoint{1.272966in}{0.883569in}}%
\pgfpathcurveto{\pgfqpoint{1.269776in}{0.883569in}}{\pgfqpoint{1.266717in}{0.882302in}}{\pgfqpoint{1.264461in}{0.880046in}}%
\pgfpathcurveto{\pgfqpoint{1.262206in}{0.877791in}}{\pgfqpoint{1.260938in}{0.874731in}}{\pgfqpoint{1.260938in}{0.871541in}}%
\pgfpathcurveto{\pgfqpoint{1.260938in}{0.868351in}}{\pgfqpoint{1.262206in}{0.865292in}}{\pgfqpoint{1.264461in}{0.863036in}}%
\pgfpathcurveto{\pgfqpoint{1.266717in}{0.860781in}}{\pgfqpoint{1.269776in}{0.859513in}}{\pgfqpoint{1.272966in}{0.859513in}}%
\pgfpathclose%
\pgfusepath{stroke,fill}%
\end{pgfscope}%
\begin{pgfscope}%
\pgfpathrectangle{\pgfqpoint{0.750000in}{0.500000in}}{\pgfqpoint{4.650000in}{3.020000in}}%
\pgfusepath{clip}%
\pgfsetbuttcap%
\pgfsetroundjoin%
\definecolor{currentfill}{rgb}{0.121569,0.466667,0.705882}%
\pgfsetfillcolor{currentfill}%
\pgfsetlinewidth{1.003750pt}%
\definecolor{currentstroke}{rgb}{0.121569,0.466667,0.705882}%
\pgfsetstrokecolor{currentstroke}%
\pgfsetdash{}{0pt}%
\pgfpathmoveto{\pgfqpoint{1.631000in}{3.345330in}}%
\pgfpathcurveto{\pgfqpoint{1.640918in}{3.345330in}}{\pgfqpoint{1.650431in}{3.349271in}}{\pgfqpoint{1.657444in}{3.356284in}}%
\pgfpathcurveto{\pgfqpoint{1.664457in}{3.363297in}}{\pgfqpoint{1.668397in}{3.372809in}}{\pgfqpoint{1.668397in}{3.382727in}}%
\pgfpathcurveto{\pgfqpoint{1.668397in}{3.392645in}}{\pgfqpoint{1.664457in}{3.402158in}}{\pgfqpoint{1.657444in}{3.409171in}}%
\pgfpathcurveto{\pgfqpoint{1.650431in}{3.416184in}}{\pgfqpoint{1.640918in}{3.420124in}}{\pgfqpoint{1.631000in}{3.420124in}}%
\pgfpathcurveto{\pgfqpoint{1.621082in}{3.420124in}}{\pgfqpoint{1.611569in}{3.416184in}}{\pgfqpoint{1.604556in}{3.409171in}}%
\pgfpathcurveto{\pgfqpoint{1.597543in}{3.402158in}}{\pgfqpoint{1.593603in}{3.392645in}}{\pgfqpoint{1.593603in}{3.382727in}}%
\pgfpathcurveto{\pgfqpoint{1.593603in}{3.372809in}}{\pgfqpoint{1.597543in}{3.363297in}}{\pgfqpoint{1.604556in}{3.356284in}}%
\pgfpathcurveto{\pgfqpoint{1.611569in}{3.349271in}}{\pgfqpoint{1.621082in}{3.345330in}}{\pgfqpoint{1.631000in}{3.345330in}}%
\pgfpathclose%
\pgfusepath{stroke,fill}%
\end{pgfscope}%
\begin{pgfscope}%
\pgfpathrectangle{\pgfqpoint{0.750000in}{0.500000in}}{\pgfqpoint{4.650000in}{3.020000in}}%
\pgfusepath{clip}%
\pgfsetbuttcap%
\pgfsetroundjoin%
\definecolor{currentfill}{rgb}{0.121569,0.466667,0.705882}%
\pgfsetfillcolor{currentfill}%
\pgfsetlinewidth{1.003750pt}%
\definecolor{currentstroke}{rgb}{0.121569,0.466667,0.705882}%
\pgfsetstrokecolor{currentstroke}%
\pgfsetdash{}{0pt}%
\pgfpathmoveto{\pgfqpoint{2.116976in}{2.830553in}}%
\pgfpathcurveto{\pgfqpoint{2.129736in}{2.830553in}}{\pgfqpoint{2.141975in}{2.835623in}}{\pgfqpoint{2.150997in}{2.844645in}}%
\pgfpathcurveto{\pgfqpoint{2.160019in}{2.853668in}}{\pgfqpoint{2.165089in}{2.865906in}}{\pgfqpoint{2.165089in}{2.878666in}}%
\pgfpathcurveto{\pgfqpoint{2.165089in}{2.891425in}}{\pgfqpoint{2.160019in}{2.903664in}}{\pgfqpoint{2.150997in}{2.912687in}}%
\pgfpathcurveto{\pgfqpoint{2.141975in}{2.921709in}}{\pgfqpoint{2.129736in}{2.926778in}}{\pgfqpoint{2.116976in}{2.926778in}}%
\pgfpathcurveto{\pgfqpoint{2.104217in}{2.926778in}}{\pgfqpoint{2.091978in}{2.921709in}}{\pgfqpoint{2.082956in}{2.912687in}}%
\pgfpathcurveto{\pgfqpoint{2.073933in}{2.903664in}}{\pgfqpoint{2.068864in}{2.891425in}}{\pgfqpoint{2.068864in}{2.878666in}}%
\pgfpathcurveto{\pgfqpoint{2.068864in}{2.865906in}}{\pgfqpoint{2.073933in}{2.853668in}}{\pgfqpoint{2.082956in}{2.844645in}}%
\pgfpathcurveto{\pgfqpoint{2.091978in}{2.835623in}}{\pgfqpoint{2.104217in}{2.830553in}}{\pgfqpoint{2.116976in}{2.830553in}}%
\pgfpathclose%
\pgfusepath{stroke,fill}%
\end{pgfscope}%
\begin{pgfscope}%
\pgfpathrectangle{\pgfqpoint{0.750000in}{0.500000in}}{\pgfqpoint{4.650000in}{3.020000in}}%
\pgfusepath{clip}%
\pgfsetbuttcap%
\pgfsetroundjoin%
\definecolor{currentfill}{rgb}{0.121569,0.466667,0.705882}%
\pgfsetfillcolor{currentfill}%
\pgfsetlinewidth{1.003750pt}%
\definecolor{currentstroke}{rgb}{0.121569,0.466667,0.705882}%
\pgfsetstrokecolor{currentstroke}%
\pgfsetdash{}{0pt}%
\pgfpathmoveto{\pgfqpoint{1.230147in}{1.796874in}}%
\pgfpathcurveto{\pgfqpoint{1.232751in}{1.796874in}}{\pgfqpoint{1.235250in}{1.797909in}}{\pgfqpoint{1.237091in}{1.799751in}}%
\pgfpathcurveto{\pgfqpoint{1.238933in}{1.801592in}}{\pgfqpoint{1.239968in}{1.804091in}}{\pgfqpoint{1.239968in}{1.806695in}}%
\pgfpathcurveto{\pgfqpoint{1.239968in}{1.809300in}}{\pgfqpoint{1.238933in}{1.811798in}}{\pgfqpoint{1.237091in}{1.813640in}}%
\pgfpathcurveto{\pgfqpoint{1.235250in}{1.815481in}}{\pgfqpoint{1.232751in}{1.816516in}}{\pgfqpoint{1.230147in}{1.816516in}}%
\pgfpathcurveto{\pgfqpoint{1.227542in}{1.816516in}}{\pgfqpoint{1.225044in}{1.815481in}}{\pgfqpoint{1.223202in}{1.813640in}}%
\pgfpathcurveto{\pgfqpoint{1.221361in}{1.811798in}}{\pgfqpoint{1.220326in}{1.809300in}}{\pgfqpoint{1.220326in}{1.806695in}}%
\pgfpathcurveto{\pgfqpoint{1.220326in}{1.804091in}}{\pgfqpoint{1.221361in}{1.801592in}}{\pgfqpoint{1.223202in}{1.799751in}}%
\pgfpathcurveto{\pgfqpoint{1.225044in}{1.797909in}}{\pgfqpoint{1.227542in}{1.796874in}}{\pgfqpoint{1.230147in}{1.796874in}}%
\pgfpathclose%
\pgfusepath{stroke,fill}%
\end{pgfscope}%
\begin{pgfscope}%
\pgfpathrectangle{\pgfqpoint{0.750000in}{0.500000in}}{\pgfqpoint{4.650000in}{3.020000in}}%
\pgfusepath{clip}%
\pgfsetbuttcap%
\pgfsetroundjoin%
\definecolor{currentfill}{rgb}{0.121569,0.466667,0.705882}%
\pgfsetfillcolor{currentfill}%
\pgfsetlinewidth{1.003750pt}%
\definecolor{currentstroke}{rgb}{0.121569,0.466667,0.705882}%
\pgfsetstrokecolor{currentstroke}%
\pgfsetdash{}{0pt}%
\pgfpathmoveto{\pgfqpoint{1.096529in}{0.675938in}}%
\pgfpathcurveto{\pgfqpoint{1.099134in}{0.675938in}}{\pgfqpoint{1.101632in}{0.676973in}}{\pgfqpoint{1.103474in}{0.678814in}}%
\pgfpathcurveto{\pgfqpoint{1.105315in}{0.680656in}}{\pgfqpoint{1.106350in}{0.683154in}}{\pgfqpoint{1.106350in}{0.685759in}}%
\pgfpathcurveto{\pgfqpoint{1.106350in}{0.688363in}}{\pgfqpoint{1.105315in}{0.690861in}}{\pgfqpoint{1.103474in}{0.692703in}}%
\pgfpathcurveto{\pgfqpoint{1.101632in}{0.694545in}}{\pgfqpoint{1.099134in}{0.695580in}}{\pgfqpoint{1.096529in}{0.695580in}}%
\pgfpathcurveto{\pgfqpoint{1.093925in}{0.695580in}}{\pgfqpoint{1.091426in}{0.694545in}}{\pgfqpoint{1.089585in}{0.692703in}}%
\pgfpathcurveto{\pgfqpoint{1.087743in}{0.690861in}}{\pgfqpoint{1.086708in}{0.688363in}}{\pgfqpoint{1.086708in}{0.685759in}}%
\pgfpathcurveto{\pgfqpoint{1.086708in}{0.683154in}}{\pgfqpoint{1.087743in}{0.680656in}}{\pgfqpoint{1.089585in}{0.678814in}}%
\pgfpathcurveto{\pgfqpoint{1.091426in}{0.676973in}}{\pgfqpoint{1.093925in}{0.675938in}}{\pgfqpoint{1.096529in}{0.675938in}}%
\pgfpathclose%
\pgfusepath{stroke,fill}%
\end{pgfscope}%
\begin{pgfscope}%
\pgfpathrectangle{\pgfqpoint{0.750000in}{0.500000in}}{\pgfqpoint{4.650000in}{3.020000in}}%
\pgfusepath{clip}%
\pgfsetbuttcap%
\pgfsetroundjoin%
\definecolor{currentfill}{rgb}{0.121569,0.466667,0.705882}%
\pgfsetfillcolor{currentfill}%
\pgfsetlinewidth{1.003750pt}%
\definecolor{currentstroke}{rgb}{0.121569,0.466667,0.705882}%
\pgfsetstrokecolor{currentstroke}%
\pgfsetdash{}{0pt}%
\pgfpathmoveto{\pgfqpoint{0.972198in}{0.888124in}}%
\pgfpathcurveto{\pgfqpoint{0.974802in}{0.888124in}}{\pgfqpoint{0.977300in}{0.889158in}}{\pgfqpoint{0.979142in}{0.891000in}}%
\pgfpathcurveto{\pgfqpoint{0.980984in}{0.892842in}}{\pgfqpoint{0.982018in}{0.895340in}}{\pgfqpoint{0.982018in}{0.897944in}}%
\pgfpathcurveto{\pgfqpoint{0.982018in}{0.900549in}}{\pgfqpoint{0.980984in}{0.903047in}}{\pgfqpoint{0.979142in}{0.904889in}}%
\pgfpathcurveto{\pgfqpoint{0.977300in}{0.906731in}}{\pgfqpoint{0.974802in}{0.907765in}}{\pgfqpoint{0.972198in}{0.907765in}}%
\pgfpathcurveto{\pgfqpoint{0.969593in}{0.907765in}}{\pgfqpoint{0.967095in}{0.906731in}}{\pgfqpoint{0.965253in}{0.904889in}}%
\pgfpathcurveto{\pgfqpoint{0.963411in}{0.903047in}}{\pgfqpoint{0.962377in}{0.900549in}}{\pgfqpoint{0.962377in}{0.897944in}}%
\pgfpathcurveto{\pgfqpoint{0.962377in}{0.895340in}}{\pgfqpoint{0.963411in}{0.892842in}}{\pgfqpoint{0.965253in}{0.891000in}}%
\pgfpathcurveto{\pgfqpoint{0.967095in}{0.889158in}}{\pgfqpoint{0.969593in}{0.888124in}}{\pgfqpoint{0.972198in}{0.888124in}}%
\pgfpathclose%
\pgfusepath{stroke,fill}%
\end{pgfscope}%
\begin{pgfscope}%
\pgfpathrectangle{\pgfqpoint{0.750000in}{0.500000in}}{\pgfqpoint{4.650000in}{3.020000in}}%
\pgfusepath{clip}%
\pgfsetbuttcap%
\pgfsetroundjoin%
\definecolor{currentfill}{rgb}{0.121569,0.466667,0.705882}%
\pgfsetfillcolor{currentfill}%
\pgfsetlinewidth{1.003750pt}%
\definecolor{currentstroke}{rgb}{0.121569,0.466667,0.705882}%
\pgfsetstrokecolor{currentstroke}%
\pgfsetdash{}{0pt}%
\pgfpathmoveto{\pgfqpoint{1.391623in}{1.283464in}}%
\pgfpathcurveto{\pgfqpoint{1.395306in}{1.283464in}}{\pgfqpoint{1.398839in}{1.284928in}}{\pgfqpoint{1.401444in}{1.287532in}}%
\pgfpathcurveto{\pgfqpoint{1.404048in}{1.290137in}}{\pgfqpoint{1.405512in}{1.293670in}}{\pgfqpoint{1.405512in}{1.297353in}}%
\pgfpathcurveto{\pgfqpoint{1.405512in}{1.301037in}}{\pgfqpoint{1.404048in}{1.304570in}}{\pgfqpoint{1.401444in}{1.307174in}}%
\pgfpathcurveto{\pgfqpoint{1.398839in}{1.309779in}}{\pgfqpoint{1.395306in}{1.311242in}}{\pgfqpoint{1.391623in}{1.311242in}}%
\pgfpathcurveto{\pgfqpoint{1.387940in}{1.311242in}}{\pgfqpoint{1.384407in}{1.309779in}}{\pgfqpoint{1.381802in}{1.307174in}}%
\pgfpathcurveto{\pgfqpoint{1.379198in}{1.304570in}}{\pgfqpoint{1.377734in}{1.301037in}}{\pgfqpoint{1.377734in}{1.297353in}}%
\pgfpathcurveto{\pgfqpoint{1.377734in}{1.293670in}}{\pgfqpoint{1.379198in}{1.290137in}}{\pgfqpoint{1.381802in}{1.287532in}}%
\pgfpathcurveto{\pgfqpoint{1.384407in}{1.284928in}}{\pgfqpoint{1.387940in}{1.283464in}}{\pgfqpoint{1.391623in}{1.283464in}}%
\pgfpathclose%
\pgfusepath{stroke,fill}%
\end{pgfscope}%
\begin{pgfscope}%
\pgfpathrectangle{\pgfqpoint{0.750000in}{0.500000in}}{\pgfqpoint{4.650000in}{3.020000in}}%
\pgfusepath{clip}%
\pgfsetbuttcap%
\pgfsetroundjoin%
\definecolor{currentfill}{rgb}{0.121569,0.466667,0.705882}%
\pgfsetfillcolor{currentfill}%
\pgfsetlinewidth{1.003750pt}%
\definecolor{currentstroke}{rgb}{0.121569,0.466667,0.705882}%
\pgfsetstrokecolor{currentstroke}%
\pgfsetdash{}{0pt}%
\pgfpathmoveto{\pgfqpoint{1.145539in}{0.731818in}}%
\pgfpathcurveto{\pgfqpoint{1.148729in}{0.731818in}}{\pgfqpoint{1.151789in}{0.733085in}}{\pgfqpoint{1.154045in}{0.735341in}}%
\pgfpathcurveto{\pgfqpoint{1.156300in}{0.737596in}}{\pgfqpoint{1.157568in}{0.740656in}}{\pgfqpoint{1.157568in}{0.743846in}}%
\pgfpathcurveto{\pgfqpoint{1.157568in}{0.747036in}}{\pgfqpoint{1.156300in}{0.750095in}}{\pgfqpoint{1.154045in}{0.752351in}}%
\pgfpathcurveto{\pgfqpoint{1.151789in}{0.754606in}}{\pgfqpoint{1.148729in}{0.755874in}}{\pgfqpoint{1.145539in}{0.755874in}}%
\pgfpathcurveto{\pgfqpoint{1.142350in}{0.755874in}}{\pgfqpoint{1.139290in}{0.754606in}}{\pgfqpoint{1.137034in}{0.752351in}}%
\pgfpathcurveto{\pgfqpoint{1.134779in}{0.750095in}}{\pgfqpoint{1.133511in}{0.747036in}}{\pgfqpoint{1.133511in}{0.743846in}}%
\pgfpathcurveto{\pgfqpoint{1.133511in}{0.740656in}}{\pgfqpoint{1.134779in}{0.737596in}}{\pgfqpoint{1.137034in}{0.735341in}}%
\pgfpathcurveto{\pgfqpoint{1.139290in}{0.733085in}}{\pgfqpoint{1.142350in}{0.731818in}}{\pgfqpoint{1.145539in}{0.731818in}}%
\pgfpathclose%
\pgfusepath{stroke,fill}%
\end{pgfscope}%
\begin{pgfscope}%
\pgfpathrectangle{\pgfqpoint{0.750000in}{0.500000in}}{\pgfqpoint{4.650000in}{3.020000in}}%
\pgfusepath{clip}%
\pgfsetbuttcap%
\pgfsetroundjoin%
\definecolor{currentfill}{rgb}{0.121569,0.466667,0.705882}%
\pgfsetfillcolor{currentfill}%
\pgfsetlinewidth{1.003750pt}%
\definecolor{currentstroke}{rgb}{0.121569,0.466667,0.705882}%
\pgfsetstrokecolor{currentstroke}%
\pgfsetdash{}{0pt}%
\pgfpathmoveto{\pgfqpoint{1.038748in}{0.791152in}}%
\pgfpathcurveto{\pgfqpoint{1.041353in}{0.791152in}}{\pgfqpoint{1.043851in}{0.792187in}}{\pgfqpoint{1.045693in}{0.794028in}}%
\pgfpathcurveto{\pgfqpoint{1.047535in}{0.795870in}}{\pgfqpoint{1.048569in}{0.798368in}}{\pgfqpoint{1.048569in}{0.800973in}}%
\pgfpathcurveto{\pgfqpoint{1.048569in}{0.803577in}}{\pgfqpoint{1.047535in}{0.806075in}}{\pgfqpoint{1.045693in}{0.807917in}}%
\pgfpathcurveto{\pgfqpoint{1.043851in}{0.809759in}}{\pgfqpoint{1.041353in}{0.810794in}}{\pgfqpoint{1.038748in}{0.810794in}}%
\pgfpathcurveto{\pgfqpoint{1.036144in}{0.810794in}}{\pgfqpoint{1.033646in}{0.809759in}}{\pgfqpoint{1.031804in}{0.807917in}}%
\pgfpathcurveto{\pgfqpoint{1.029962in}{0.806075in}}{\pgfqpoint{1.028927in}{0.803577in}}{\pgfqpoint{1.028927in}{0.800973in}}%
\pgfpathcurveto{\pgfqpoint{1.028927in}{0.798368in}}{\pgfqpoint{1.029962in}{0.795870in}}{\pgfqpoint{1.031804in}{0.794028in}}%
\pgfpathcurveto{\pgfqpoint{1.033646in}{0.792187in}}{\pgfqpoint{1.036144in}{0.791152in}}{\pgfqpoint{1.038748in}{0.791152in}}%
\pgfpathclose%
\pgfusepath{stroke,fill}%
\end{pgfscope}%
\begin{pgfscope}%
\pgfpathrectangle{\pgfqpoint{0.750000in}{0.500000in}}{\pgfqpoint{4.650000in}{3.020000in}}%
\pgfusepath{clip}%
\pgfsetbuttcap%
\pgfsetroundjoin%
\definecolor{currentfill}{rgb}{0.121569,0.466667,0.705882}%
\pgfsetfillcolor{currentfill}%
\pgfsetlinewidth{1.003750pt}%
\definecolor{currentstroke}{rgb}{0.121569,0.466667,0.705882}%
\pgfsetstrokecolor{currentstroke}%
\pgfsetdash{}{0pt}%
\pgfpathmoveto{\pgfqpoint{1.052678in}{1.282732in}}%
\pgfpathcurveto{\pgfqpoint{1.055282in}{1.282732in}}{\pgfqpoint{1.057780in}{1.283766in}}{\pgfqpoint{1.059622in}{1.285608in}}%
\pgfpathcurveto{\pgfqpoint{1.061464in}{1.287450in}}{\pgfqpoint{1.062499in}{1.289948in}}{\pgfqpoint{1.062499in}{1.292553in}}%
\pgfpathcurveto{\pgfqpoint{1.062499in}{1.295157in}}{\pgfqpoint{1.061464in}{1.297655in}}{\pgfqpoint{1.059622in}{1.299497in}}%
\pgfpathcurveto{\pgfqpoint{1.057780in}{1.301339in}}{\pgfqpoint{1.055282in}{1.302374in}}{\pgfqpoint{1.052678in}{1.302374in}}%
\pgfpathcurveto{\pgfqpoint{1.050073in}{1.302374in}}{\pgfqpoint{1.047575in}{1.301339in}}{\pgfqpoint{1.045733in}{1.299497in}}%
\pgfpathcurveto{\pgfqpoint{1.043892in}{1.297655in}}{\pgfqpoint{1.042857in}{1.295157in}}{\pgfqpoint{1.042857in}{1.292553in}}%
\pgfpathcurveto{\pgfqpoint{1.042857in}{1.289948in}}{\pgfqpoint{1.043892in}{1.287450in}}{\pgfqpoint{1.045733in}{1.285608in}}%
\pgfpathcurveto{\pgfqpoint{1.047575in}{1.283766in}}{\pgfqpoint{1.050073in}{1.282732in}}{\pgfqpoint{1.052678in}{1.282732in}}%
\pgfpathclose%
\pgfusepath{stroke,fill}%
\end{pgfscope}%
\begin{pgfscope}%
\pgfpathrectangle{\pgfqpoint{0.750000in}{0.500000in}}{\pgfqpoint{4.650000in}{3.020000in}}%
\pgfusepath{clip}%
\pgfsetbuttcap%
\pgfsetroundjoin%
\definecolor{currentfill}{rgb}{0.121569,0.466667,0.705882}%
\pgfsetfillcolor{currentfill}%
\pgfsetlinewidth{1.003750pt}%
\definecolor{currentstroke}{rgb}{0.121569,0.466667,0.705882}%
\pgfsetstrokecolor{currentstroke}%
\pgfsetdash{}{0pt}%
\pgfpathmoveto{\pgfqpoint{1.045971in}{0.726344in}}%
\pgfpathcurveto{\pgfqpoint{1.048576in}{0.726344in}}{\pgfqpoint{1.051074in}{0.727379in}}{\pgfqpoint{1.052915in}{0.729220in}}%
\pgfpathcurveto{\pgfqpoint{1.054757in}{0.731062in}}{\pgfqpoint{1.055792in}{0.733560in}}{\pgfqpoint{1.055792in}{0.736165in}}%
\pgfpathcurveto{\pgfqpoint{1.055792in}{0.738769in}}{\pgfqpoint{1.054757in}{0.741268in}}{\pgfqpoint{1.052915in}{0.743109in}}%
\pgfpathcurveto{\pgfqpoint{1.051074in}{0.744951in}}{\pgfqpoint{1.048576in}{0.745986in}}{\pgfqpoint{1.045971in}{0.745986in}}%
\pgfpathcurveto{\pgfqpoint{1.043366in}{0.745986in}}{\pgfqpoint{1.040868in}{0.744951in}}{\pgfqpoint{1.039027in}{0.743109in}}%
\pgfpathcurveto{\pgfqpoint{1.037185in}{0.741268in}}{\pgfqpoint{1.036150in}{0.738769in}}{\pgfqpoint{1.036150in}{0.736165in}}%
\pgfpathcurveto{\pgfqpoint{1.036150in}{0.733560in}}{\pgfqpoint{1.037185in}{0.731062in}}{\pgfqpoint{1.039027in}{0.729220in}}%
\pgfpathcurveto{\pgfqpoint{1.040868in}{0.727379in}}{\pgfqpoint{1.043366in}{0.726344in}}{\pgfqpoint{1.045971in}{0.726344in}}%
\pgfpathclose%
\pgfusepath{stroke,fill}%
\end{pgfscope}%
\begin{pgfscope}%
\pgfpathrectangle{\pgfqpoint{0.750000in}{0.500000in}}{\pgfqpoint{4.650000in}{3.020000in}}%
\pgfusepath{clip}%
\pgfsetbuttcap%
\pgfsetroundjoin%
\definecolor{currentfill}{rgb}{0.121569,0.466667,0.705882}%
\pgfsetfillcolor{currentfill}%
\pgfsetlinewidth{1.003750pt}%
\definecolor{currentstroke}{rgb}{0.121569,0.466667,0.705882}%
\pgfsetstrokecolor{currentstroke}%
\pgfsetdash{}{0pt}%
\pgfpathmoveto{\pgfqpoint{1.133158in}{0.732105in}}%
\pgfpathcurveto{\pgfqpoint{1.135762in}{0.732105in}}{\pgfqpoint{1.138261in}{0.733139in}}{\pgfqpoint{1.140102in}{0.734981in}}%
\pgfpathcurveto{\pgfqpoint{1.141944in}{0.736823in}}{\pgfqpoint{1.142979in}{0.739321in}}{\pgfqpoint{1.142979in}{0.741925in}}%
\pgfpathcurveto{\pgfqpoint{1.142979in}{0.744530in}}{\pgfqpoint{1.141944in}{0.747028in}}{\pgfqpoint{1.140102in}{0.748870in}}%
\pgfpathcurveto{\pgfqpoint{1.138261in}{0.750712in}}{\pgfqpoint{1.135762in}{0.751746in}}{\pgfqpoint{1.133158in}{0.751746in}}%
\pgfpathcurveto{\pgfqpoint{1.130553in}{0.751746in}}{\pgfqpoint{1.128055in}{0.750712in}}{\pgfqpoint{1.126213in}{0.748870in}}%
\pgfpathcurveto{\pgfqpoint{1.124372in}{0.747028in}}{\pgfqpoint{1.123337in}{0.744530in}}{\pgfqpoint{1.123337in}{0.741925in}}%
\pgfpathcurveto{\pgfqpoint{1.123337in}{0.739321in}}{\pgfqpoint{1.124372in}{0.736823in}}{\pgfqpoint{1.126213in}{0.734981in}}%
\pgfpathcurveto{\pgfqpoint{1.128055in}{0.733139in}}{\pgfqpoint{1.130553in}{0.732105in}}{\pgfqpoint{1.133158in}{0.732105in}}%
\pgfpathclose%
\pgfusepath{stroke,fill}%
\end{pgfscope}%
\begin{pgfscope}%
\pgfpathrectangle{\pgfqpoint{0.750000in}{0.500000in}}{\pgfqpoint{4.650000in}{3.020000in}}%
\pgfusepath{clip}%
\pgfsetbuttcap%
\pgfsetroundjoin%
\definecolor{currentfill}{rgb}{0.121569,0.466667,0.705882}%
\pgfsetfillcolor{currentfill}%
\pgfsetlinewidth{1.003750pt}%
\definecolor{currentstroke}{rgb}{0.121569,0.466667,0.705882}%
\pgfsetstrokecolor{currentstroke}%
\pgfsetdash{}{0pt}%
\pgfpathmoveto{\pgfqpoint{1.059900in}{0.689859in}}%
\pgfpathcurveto{\pgfqpoint{1.062505in}{0.689859in}}{\pgfqpoint{1.065003in}{0.690894in}}{\pgfqpoint{1.066845in}{0.692736in}}%
\pgfpathcurveto{\pgfqpoint{1.068686in}{0.694578in}}{\pgfqpoint{1.069721in}{0.697076in}}{\pgfqpoint{1.069721in}{0.699680in}}%
\pgfpathcurveto{\pgfqpoint{1.069721in}{0.702285in}}{\pgfqpoint{1.068686in}{0.704783in}}{\pgfqpoint{1.066845in}{0.706625in}}%
\pgfpathcurveto{\pgfqpoint{1.065003in}{0.708466in}}{\pgfqpoint{1.062505in}{0.709501in}}{\pgfqpoint{1.059900in}{0.709501in}}%
\pgfpathcurveto{\pgfqpoint{1.057296in}{0.709501in}}{\pgfqpoint{1.054798in}{0.708466in}}{\pgfqpoint{1.052956in}{0.706625in}}%
\pgfpathcurveto{\pgfqpoint{1.051114in}{0.704783in}}{\pgfqpoint{1.050079in}{0.702285in}}{\pgfqpoint{1.050079in}{0.699680in}}%
\pgfpathcurveto{\pgfqpoint{1.050079in}{0.697076in}}{\pgfqpoint{1.051114in}{0.694578in}}{\pgfqpoint{1.052956in}{0.692736in}}%
\pgfpathcurveto{\pgfqpoint{1.054798in}{0.690894in}}{\pgfqpoint{1.057296in}{0.689859in}}{\pgfqpoint{1.059900in}{0.689859in}}%
\pgfpathclose%
\pgfusepath{stroke,fill}%
\end{pgfscope}%
\begin{pgfscope}%
\pgfpathrectangle{\pgfqpoint{0.750000in}{0.500000in}}{\pgfqpoint{4.650000in}{3.020000in}}%
\pgfusepath{clip}%
\pgfsetbuttcap%
\pgfsetroundjoin%
\definecolor{currentfill}{rgb}{0.121569,0.466667,0.705882}%
\pgfsetfillcolor{currentfill}%
\pgfsetlinewidth{1.003750pt}%
\definecolor{currentstroke}{rgb}{0.121569,0.466667,0.705882}%
\pgfsetstrokecolor{currentstroke}%
\pgfsetdash{}{0pt}%
\pgfpathmoveto{\pgfqpoint{1.005215in}{0.645214in}}%
\pgfpathcurveto{\pgfqpoint{1.007820in}{0.645214in}}{\pgfqpoint{1.010318in}{0.646249in}}{\pgfqpoint{1.012159in}{0.648090in}}%
\pgfpathcurveto{\pgfqpoint{1.014001in}{0.649932in}}{\pgfqpoint{1.015036in}{0.652430in}}{\pgfqpoint{1.015036in}{0.655035in}}%
\pgfpathcurveto{\pgfqpoint{1.015036in}{0.657639in}}{\pgfqpoint{1.014001in}{0.660138in}}{\pgfqpoint{1.012159in}{0.661979in}}%
\pgfpathcurveto{\pgfqpoint{1.010318in}{0.663821in}}{\pgfqpoint{1.007820in}{0.664856in}}{\pgfqpoint{1.005215in}{0.664856in}}%
\pgfpathcurveto{\pgfqpoint{1.002610in}{0.664856in}}{\pgfqpoint{1.000112in}{0.663821in}}{\pgfqpoint{0.998271in}{0.661979in}}%
\pgfpathcurveto{\pgfqpoint{0.996429in}{0.660138in}}{\pgfqpoint{0.995394in}{0.657639in}}{\pgfqpoint{0.995394in}{0.655035in}}%
\pgfpathcurveto{\pgfqpoint{0.995394in}{0.652430in}}{\pgfqpoint{0.996429in}{0.649932in}}{\pgfqpoint{0.998271in}{0.648090in}}%
\pgfpathcurveto{\pgfqpoint{1.000112in}{0.646249in}}{\pgfqpoint{1.002610in}{0.645214in}}{\pgfqpoint{1.005215in}{0.645214in}}%
\pgfpathclose%
\pgfusepath{stroke,fill}%
\end{pgfscope}%
\begin{pgfscope}%
\pgfpathrectangle{\pgfqpoint{0.750000in}{0.500000in}}{\pgfqpoint{4.650000in}{3.020000in}}%
\pgfusepath{clip}%
\pgfsetbuttcap%
\pgfsetroundjoin%
\definecolor{currentfill}{rgb}{0.121569,0.466667,0.705882}%
\pgfsetfillcolor{currentfill}%
\pgfsetlinewidth{1.003750pt}%
\definecolor{currentstroke}{rgb}{0.121569,0.466667,0.705882}%
\pgfsetstrokecolor{currentstroke}%
\pgfsetdash{}{0pt}%
\pgfpathmoveto{\pgfqpoint{1.015533in}{0.646174in}}%
\pgfpathcurveto{\pgfqpoint{1.018138in}{0.646174in}}{\pgfqpoint{1.020636in}{0.647209in}}{\pgfqpoint{1.022477in}{0.649051in}}%
\pgfpathcurveto{\pgfqpoint{1.024319in}{0.650892in}}{\pgfqpoint{1.025354in}{0.653390in}}{\pgfqpoint{1.025354in}{0.655995in}}%
\pgfpathcurveto{\pgfqpoint{1.025354in}{0.658600in}}{\pgfqpoint{1.024319in}{0.661098in}}{\pgfqpoint{1.022477in}{0.662939in}}%
\pgfpathcurveto{\pgfqpoint{1.020636in}{0.664781in}}{\pgfqpoint{1.018138in}{0.665816in}}{\pgfqpoint{1.015533in}{0.665816in}}%
\pgfpathcurveto{\pgfqpoint{1.012928in}{0.665816in}}{\pgfqpoint{1.010430in}{0.664781in}}{\pgfqpoint{1.008589in}{0.662939in}}%
\pgfpathcurveto{\pgfqpoint{1.006747in}{0.661098in}}{\pgfqpoint{1.005712in}{0.658600in}}{\pgfqpoint{1.005712in}{0.655995in}}%
\pgfpathcurveto{\pgfqpoint{1.005712in}{0.653390in}}{\pgfqpoint{1.006747in}{0.650892in}}{\pgfqpoint{1.008589in}{0.649051in}}%
\pgfpathcurveto{\pgfqpoint{1.010430in}{0.647209in}}{\pgfqpoint{1.012928in}{0.646174in}}{\pgfqpoint{1.015533in}{0.646174in}}%
\pgfpathclose%
\pgfusepath{stroke,fill}%
\end{pgfscope}%
\begin{pgfscope}%
\pgfpathrectangle{\pgfqpoint{0.750000in}{0.500000in}}{\pgfqpoint{4.650000in}{3.020000in}}%
\pgfusepath{clip}%
\pgfsetbuttcap%
\pgfsetroundjoin%
\definecolor{currentfill}{rgb}{0.121569,0.466667,0.705882}%
\pgfsetfillcolor{currentfill}%
\pgfsetlinewidth{1.003750pt}%
\definecolor{currentstroke}{rgb}{0.121569,0.466667,0.705882}%
\pgfsetstrokecolor{currentstroke}%
\pgfsetdash{}{0pt}%
\pgfpathmoveto{\pgfqpoint{1.013469in}{0.656735in}}%
\pgfpathcurveto{\pgfqpoint{1.016074in}{0.656735in}}{\pgfqpoint{1.018572in}{0.657770in}}{\pgfqpoint{1.020414in}{0.659612in}}%
\pgfpathcurveto{\pgfqpoint{1.022256in}{0.661454in}}{\pgfqpoint{1.023290in}{0.663952in}}{\pgfqpoint{1.023290in}{0.666556in}}%
\pgfpathcurveto{\pgfqpoint{1.023290in}{0.669161in}}{\pgfqpoint{1.022256in}{0.671659in}}{\pgfqpoint{1.020414in}{0.673501in}}%
\pgfpathcurveto{\pgfqpoint{1.018572in}{0.675342in}}{\pgfqpoint{1.016074in}{0.676377in}}{\pgfqpoint{1.013469in}{0.676377in}}%
\pgfpathcurveto{\pgfqpoint{1.010865in}{0.676377in}}{\pgfqpoint{1.008367in}{0.675342in}}{\pgfqpoint{1.006525in}{0.673501in}}%
\pgfpathcurveto{\pgfqpoint{1.004683in}{0.671659in}}{\pgfqpoint{1.003648in}{0.669161in}}{\pgfqpoint{1.003648in}{0.666556in}}%
\pgfpathcurveto{\pgfqpoint{1.003648in}{0.663952in}}{\pgfqpoint{1.004683in}{0.661454in}}{\pgfqpoint{1.006525in}{0.659612in}}%
\pgfpathcurveto{\pgfqpoint{1.008367in}{0.657770in}}{\pgfqpoint{1.010865in}{0.656735in}}{\pgfqpoint{1.013469in}{0.656735in}}%
\pgfpathclose%
\pgfusepath{stroke,fill}%
\end{pgfscope}%
\begin{pgfscope}%
\pgfpathrectangle{\pgfqpoint{0.750000in}{0.500000in}}{\pgfqpoint{4.650000in}{3.020000in}}%
\pgfusepath{clip}%
\pgfsetbuttcap%
\pgfsetroundjoin%
\definecolor{currentfill}{rgb}{0.121569,0.466667,0.705882}%
\pgfsetfillcolor{currentfill}%
\pgfsetlinewidth{1.003750pt}%
\definecolor{currentstroke}{rgb}{0.121569,0.466667,0.705882}%
\pgfsetstrokecolor{currentstroke}%
\pgfsetdash{}{0pt}%
\pgfpathmoveto{\pgfqpoint{1.044939in}{1.151676in}}%
\pgfpathcurveto{\pgfqpoint{1.047544in}{1.151676in}}{\pgfqpoint{1.050042in}{1.152710in}}{\pgfqpoint{1.051884in}{1.154552in}}%
\pgfpathcurveto{\pgfqpoint{1.053725in}{1.156394in}}{\pgfqpoint{1.054760in}{1.158892in}}{\pgfqpoint{1.054760in}{1.161497in}}%
\pgfpathcurveto{\pgfqpoint{1.054760in}{1.164101in}}{\pgfqpoint{1.053725in}{1.166599in}}{\pgfqpoint{1.051884in}{1.168441in}}%
\pgfpathcurveto{\pgfqpoint{1.050042in}{1.170283in}}{\pgfqpoint{1.047544in}{1.171318in}}{\pgfqpoint{1.044939in}{1.171318in}}%
\pgfpathcurveto{\pgfqpoint{1.042335in}{1.171318in}}{\pgfqpoint{1.039836in}{1.170283in}}{\pgfqpoint{1.037995in}{1.168441in}}%
\pgfpathcurveto{\pgfqpoint{1.036153in}{1.166599in}}{\pgfqpoint{1.035118in}{1.164101in}}{\pgfqpoint{1.035118in}{1.161497in}}%
\pgfpathcurveto{\pgfqpoint{1.035118in}{1.158892in}}{\pgfqpoint{1.036153in}{1.156394in}}{\pgfqpoint{1.037995in}{1.154552in}}%
\pgfpathcurveto{\pgfqpoint{1.039836in}{1.152710in}}{\pgfqpoint{1.042335in}{1.151676in}}{\pgfqpoint{1.044939in}{1.151676in}}%
\pgfpathclose%
\pgfusepath{stroke,fill}%
\end{pgfscope}%
\begin{pgfscope}%
\pgfpathrectangle{\pgfqpoint{0.750000in}{0.500000in}}{\pgfqpoint{4.650000in}{3.020000in}}%
\pgfusepath{clip}%
\pgfsetbuttcap%
\pgfsetroundjoin%
\definecolor{currentfill}{rgb}{0.121569,0.466667,0.705882}%
\pgfsetfillcolor{currentfill}%
\pgfsetlinewidth{1.003750pt}%
\definecolor{currentstroke}{rgb}{0.121569,0.466667,0.705882}%
\pgfsetstrokecolor{currentstroke}%
\pgfsetdash{}{0pt}%
\pgfpathmoveto{\pgfqpoint{1.264196in}{0.951491in}}%
\pgfpathcurveto{\pgfqpoint{1.266801in}{0.951491in}}{\pgfqpoint{1.269299in}{0.952526in}}{\pgfqpoint{1.271141in}{0.954368in}}%
\pgfpathcurveto{\pgfqpoint{1.272982in}{0.956209in}}{\pgfqpoint{1.274017in}{0.958708in}}{\pgfqpoint{1.274017in}{0.961312in}}%
\pgfpathcurveto{\pgfqpoint{1.274017in}{0.963917in}}{\pgfqpoint{1.272982in}{0.966415in}}{\pgfqpoint{1.271141in}{0.968257in}}%
\pgfpathcurveto{\pgfqpoint{1.269299in}{0.970098in}}{\pgfqpoint{1.266801in}{0.971133in}}{\pgfqpoint{1.264196in}{0.971133in}}%
\pgfpathcurveto{\pgfqpoint{1.261592in}{0.971133in}}{\pgfqpoint{1.259093in}{0.970098in}}{\pgfqpoint{1.257252in}{0.968257in}}%
\pgfpathcurveto{\pgfqpoint{1.255410in}{0.966415in}}{\pgfqpoint{1.254375in}{0.963917in}}{\pgfqpoint{1.254375in}{0.961312in}}%
\pgfpathcurveto{\pgfqpoint{1.254375in}{0.958708in}}{\pgfqpoint{1.255410in}{0.956209in}}{\pgfqpoint{1.257252in}{0.954368in}}%
\pgfpathcurveto{\pgfqpoint{1.259093in}{0.952526in}}{\pgfqpoint{1.261592in}{0.951491in}}{\pgfqpoint{1.264196in}{0.951491in}}%
\pgfpathclose%
\pgfusepath{stroke,fill}%
\end{pgfscope}%
\begin{pgfscope}%
\pgfpathrectangle{\pgfqpoint{0.750000in}{0.500000in}}{\pgfqpoint{4.650000in}{3.020000in}}%
\pgfusepath{clip}%
\pgfsetbuttcap%
\pgfsetroundjoin%
\definecolor{currentfill}{rgb}{0.121569,0.466667,0.705882}%
\pgfsetfillcolor{currentfill}%
\pgfsetlinewidth{1.003750pt}%
\definecolor{currentstroke}{rgb}{0.121569,0.466667,0.705882}%
\pgfsetstrokecolor{currentstroke}%
\pgfsetdash{}{0pt}%
\pgfpathmoveto{\pgfqpoint{1.276578in}{0.746026in}}%
\pgfpathcurveto{\pgfqpoint{1.279182in}{0.746026in}}{\pgfqpoint{1.281680in}{0.747061in}}{\pgfqpoint{1.283522in}{0.748903in}}%
\pgfpathcurveto{\pgfqpoint{1.285364in}{0.750744in}}{\pgfqpoint{1.286399in}{0.753243in}}{\pgfqpoint{1.286399in}{0.755847in}}%
\pgfpathcurveto{\pgfqpoint{1.286399in}{0.758452in}}{\pgfqpoint{1.285364in}{0.760950in}}{\pgfqpoint{1.283522in}{0.762792in}}%
\pgfpathcurveto{\pgfqpoint{1.281680in}{0.764633in}}{\pgfqpoint{1.279182in}{0.765668in}}{\pgfqpoint{1.276578in}{0.765668in}}%
\pgfpathcurveto{\pgfqpoint{1.273973in}{0.765668in}}{\pgfqpoint{1.271475in}{0.764633in}}{\pgfqpoint{1.269633in}{0.762792in}}%
\pgfpathcurveto{\pgfqpoint{1.267792in}{0.760950in}}{\pgfqpoint{1.266757in}{0.758452in}}{\pgfqpoint{1.266757in}{0.755847in}}%
\pgfpathcurveto{\pgfqpoint{1.266757in}{0.753243in}}{\pgfqpoint{1.267792in}{0.750744in}}{\pgfqpoint{1.269633in}{0.748903in}}%
\pgfpathcurveto{\pgfqpoint{1.271475in}{0.747061in}}{\pgfqpoint{1.273973in}{0.746026in}}{\pgfqpoint{1.276578in}{0.746026in}}%
\pgfpathclose%
\pgfusepath{stroke,fill}%
\end{pgfscope}%
\begin{pgfscope}%
\pgfpathrectangle{\pgfqpoint{0.750000in}{0.500000in}}{\pgfqpoint{4.650000in}{3.020000in}}%
\pgfusepath{clip}%
\pgfsetbuttcap%
\pgfsetroundjoin%
\definecolor{currentfill}{rgb}{0.121569,0.466667,0.705882}%
\pgfsetfillcolor{currentfill}%
\pgfsetlinewidth{1.003750pt}%
\definecolor{currentstroke}{rgb}{0.121569,0.466667,0.705882}%
\pgfsetstrokecolor{currentstroke}%
\pgfsetdash{}{0pt}%
\pgfpathmoveto{\pgfqpoint{1.273998in}{0.794992in}}%
\pgfpathcurveto{\pgfqpoint{1.276603in}{0.794992in}}{\pgfqpoint{1.279101in}{0.796027in}}{\pgfqpoint{1.280943in}{0.797869in}}%
\pgfpathcurveto{\pgfqpoint{1.282784in}{0.799710in}}{\pgfqpoint{1.283819in}{0.802209in}}{\pgfqpoint{1.283819in}{0.804813in}}%
\pgfpathcurveto{\pgfqpoint{1.283819in}{0.807418in}}{\pgfqpoint{1.282784in}{0.809916in}}{\pgfqpoint{1.280943in}{0.811758in}}%
\pgfpathcurveto{\pgfqpoint{1.279101in}{0.813599in}}{\pgfqpoint{1.276603in}{0.814634in}}{\pgfqpoint{1.273998in}{0.814634in}}%
\pgfpathcurveto{\pgfqpoint{1.271394in}{0.814634in}}{\pgfqpoint{1.268895in}{0.813599in}}{\pgfqpoint{1.267054in}{0.811758in}}%
\pgfpathcurveto{\pgfqpoint{1.265212in}{0.809916in}}{\pgfqpoint{1.264177in}{0.807418in}}{\pgfqpoint{1.264177in}{0.804813in}}%
\pgfpathcurveto{\pgfqpoint{1.264177in}{0.802209in}}{\pgfqpoint{1.265212in}{0.799710in}}{\pgfqpoint{1.267054in}{0.797869in}}%
\pgfpathcurveto{\pgfqpoint{1.268895in}{0.796027in}}{\pgfqpoint{1.271394in}{0.794992in}}{\pgfqpoint{1.273998in}{0.794992in}}%
\pgfpathclose%
\pgfusepath{stroke,fill}%
\end{pgfscope}%
\begin{pgfscope}%
\pgfpathrectangle{\pgfqpoint{0.750000in}{0.500000in}}{\pgfqpoint{4.650000in}{3.020000in}}%
\pgfusepath{clip}%
\pgfsetbuttcap%
\pgfsetroundjoin%
\definecolor{currentfill}{rgb}{0.121569,0.466667,0.705882}%
\pgfsetfillcolor{currentfill}%
\pgfsetlinewidth{1.003750pt}%
\definecolor{currentstroke}{rgb}{0.121569,0.466667,0.705882}%
\pgfsetstrokecolor{currentstroke}%
\pgfsetdash{}{0pt}%
\pgfpathmoveto{\pgfqpoint{1.077441in}{0.684099in}}%
\pgfpathcurveto{\pgfqpoint{1.080045in}{0.684099in}}{\pgfqpoint{1.082544in}{0.685133in}}{\pgfqpoint{1.084385in}{0.686975in}}%
\pgfpathcurveto{\pgfqpoint{1.086227in}{0.688817in}}{\pgfqpoint{1.087262in}{0.691315in}}{\pgfqpoint{1.087262in}{0.693920in}}%
\pgfpathcurveto{\pgfqpoint{1.087262in}{0.696524in}}{\pgfqpoint{1.086227in}{0.699022in}}{\pgfqpoint{1.084385in}{0.700864in}}%
\pgfpathcurveto{\pgfqpoint{1.082544in}{0.702706in}}{\pgfqpoint{1.080045in}{0.703741in}}{\pgfqpoint{1.077441in}{0.703741in}}%
\pgfpathcurveto{\pgfqpoint{1.074836in}{0.703741in}}{\pgfqpoint{1.072338in}{0.702706in}}{\pgfqpoint{1.070496in}{0.700864in}}%
\pgfpathcurveto{\pgfqpoint{1.068655in}{0.699022in}}{\pgfqpoint{1.067620in}{0.696524in}}{\pgfqpoint{1.067620in}{0.693920in}}%
\pgfpathcurveto{\pgfqpoint{1.067620in}{0.691315in}}{\pgfqpoint{1.068655in}{0.688817in}}{\pgfqpoint{1.070496in}{0.686975in}}%
\pgfpathcurveto{\pgfqpoint{1.072338in}{0.685133in}}{\pgfqpoint{1.074836in}{0.684099in}}{\pgfqpoint{1.077441in}{0.684099in}}%
\pgfpathclose%
\pgfusepath{stroke,fill}%
\end{pgfscope}%
\begin{pgfscope}%
\pgfpathrectangle{\pgfqpoint{0.750000in}{0.500000in}}{\pgfqpoint{4.650000in}{3.020000in}}%
\pgfusepath{clip}%
\pgfsetbuttcap%
\pgfsetroundjoin%
\definecolor{currentfill}{rgb}{0.121569,0.466667,0.705882}%
\pgfsetfillcolor{currentfill}%
\pgfsetlinewidth{1.003750pt}%
\definecolor{currentstroke}{rgb}{0.121569,0.466667,0.705882}%
\pgfsetstrokecolor{currentstroke}%
\pgfsetdash{}{0pt}%
\pgfpathmoveto{\pgfqpoint{1.093950in}{0.695620in}}%
\pgfpathcurveto{\pgfqpoint{1.096554in}{0.695620in}}{\pgfqpoint{1.099052in}{0.696655in}}{\pgfqpoint{1.100894in}{0.698497in}}%
\pgfpathcurveto{\pgfqpoint{1.102736in}{0.700338in}}{\pgfqpoint{1.103770in}{0.702836in}}{\pgfqpoint{1.103770in}{0.705441in}}%
\pgfpathcurveto{\pgfqpoint{1.103770in}{0.708046in}}{\pgfqpoint{1.102736in}{0.710544in}}{\pgfqpoint{1.100894in}{0.712385in}}%
\pgfpathcurveto{\pgfqpoint{1.099052in}{0.714227in}}{\pgfqpoint{1.096554in}{0.715262in}}{\pgfqpoint{1.093950in}{0.715262in}}%
\pgfpathcurveto{\pgfqpoint{1.091345in}{0.715262in}}{\pgfqpoint{1.088847in}{0.714227in}}{\pgfqpoint{1.087005in}{0.712385in}}%
\pgfpathcurveto{\pgfqpoint{1.085163in}{0.710544in}}{\pgfqpoint{1.084129in}{0.708046in}}{\pgfqpoint{1.084129in}{0.705441in}}%
\pgfpathcurveto{\pgfqpoint{1.084129in}{0.702836in}}{\pgfqpoint{1.085163in}{0.700338in}}{\pgfqpoint{1.087005in}{0.698497in}}%
\pgfpathcurveto{\pgfqpoint{1.088847in}{0.696655in}}{\pgfqpoint{1.091345in}{0.695620in}}{\pgfqpoint{1.093950in}{0.695620in}}%
\pgfpathclose%
\pgfusepath{stroke,fill}%
\end{pgfscope}%
\begin{pgfscope}%
\pgfpathrectangle{\pgfqpoint{0.750000in}{0.500000in}}{\pgfqpoint{4.650000in}{3.020000in}}%
\pgfusepath{clip}%
\pgfsetbuttcap%
\pgfsetroundjoin%
\definecolor{currentfill}{rgb}{0.121569,0.466667,0.705882}%
\pgfsetfillcolor{currentfill}%
\pgfsetlinewidth{1.003750pt}%
\definecolor{currentstroke}{rgb}{0.121569,0.466667,0.705882}%
\pgfsetstrokecolor{currentstroke}%
\pgfsetdash{}{0pt}%
\pgfpathmoveto{\pgfqpoint{1.031010in}{0.657695in}}%
\pgfpathcurveto{\pgfqpoint{1.033614in}{0.657695in}}{\pgfqpoint{1.036113in}{0.658730in}}{\pgfqpoint{1.037954in}{0.660572in}}%
\pgfpathcurveto{\pgfqpoint{1.039796in}{0.662414in}}{\pgfqpoint{1.040831in}{0.664912in}}{\pgfqpoint{1.040831in}{0.667516in}}%
\pgfpathcurveto{\pgfqpoint{1.040831in}{0.670121in}}{\pgfqpoint{1.039796in}{0.672619in}}{\pgfqpoint{1.037954in}{0.674461in}}%
\pgfpathcurveto{\pgfqpoint{1.036113in}{0.676303in}}{\pgfqpoint{1.033614in}{0.677337in}}{\pgfqpoint{1.031010in}{0.677337in}}%
\pgfpathcurveto{\pgfqpoint{1.028405in}{0.677337in}}{\pgfqpoint{1.025907in}{0.676303in}}{\pgfqpoint{1.024065in}{0.674461in}}%
\pgfpathcurveto{\pgfqpoint{1.022224in}{0.672619in}}{\pgfqpoint{1.021189in}{0.670121in}}{\pgfqpoint{1.021189in}{0.667516in}}%
\pgfpathcurveto{\pgfqpoint{1.021189in}{0.664912in}}{\pgfqpoint{1.022224in}{0.662414in}}{\pgfqpoint{1.024065in}{0.660572in}}%
\pgfpathcurveto{\pgfqpoint{1.025907in}{0.658730in}}{\pgfqpoint{1.028405in}{0.657695in}}{\pgfqpoint{1.031010in}{0.657695in}}%
\pgfpathclose%
\pgfusepath{stroke,fill}%
\end{pgfscope}%
\begin{pgfscope}%
\pgfpathrectangle{\pgfqpoint{0.750000in}{0.500000in}}{\pgfqpoint{4.650000in}{3.020000in}}%
\pgfusepath{clip}%
\pgfsetbuttcap%
\pgfsetroundjoin%
\definecolor{currentfill}{rgb}{0.121569,0.466667,0.705882}%
\pgfsetfillcolor{currentfill}%
\pgfsetlinewidth{1.003750pt}%
\definecolor{currentstroke}{rgb}{0.121569,0.466667,0.705882}%
\pgfsetstrokecolor{currentstroke}%
\pgfsetdash{}{0pt}%
\pgfpathmoveto{\pgfqpoint{1.032042in}{0.647134in}}%
\pgfpathcurveto{\pgfqpoint{1.034646in}{0.647134in}}{\pgfqpoint{1.037144in}{0.648169in}}{\pgfqpoint{1.038986in}{0.650011in}}%
\pgfpathcurveto{\pgfqpoint{1.040828in}{0.651852in}}{\pgfqpoint{1.041863in}{0.654351in}}{\pgfqpoint{1.041863in}{0.656955in}}%
\pgfpathcurveto{\pgfqpoint{1.041863in}{0.659560in}}{\pgfqpoint{1.040828in}{0.662058in}}{\pgfqpoint{1.038986in}{0.663900in}}%
\pgfpathcurveto{\pgfqpoint{1.037144in}{0.665741in}}{\pgfqpoint{1.034646in}{0.666776in}}{\pgfqpoint{1.032042in}{0.666776in}}%
\pgfpathcurveto{\pgfqpoint{1.029437in}{0.666776in}}{\pgfqpoint{1.026939in}{0.665741in}}{\pgfqpoint{1.025097in}{0.663900in}}%
\pgfpathcurveto{\pgfqpoint{1.023256in}{0.662058in}}{\pgfqpoint{1.022221in}{0.659560in}}{\pgfqpoint{1.022221in}{0.656955in}}%
\pgfpathcurveto{\pgfqpoint{1.022221in}{0.654351in}}{\pgfqpoint{1.023256in}{0.651852in}}{\pgfqpoint{1.025097in}{0.650011in}}%
\pgfpathcurveto{\pgfqpoint{1.026939in}{0.648169in}}{\pgfqpoint{1.029437in}{0.647134in}}{\pgfqpoint{1.032042in}{0.647134in}}%
\pgfpathclose%
\pgfusepath{stroke,fill}%
\end{pgfscope}%
\begin{pgfscope}%
\pgfpathrectangle{\pgfqpoint{0.750000in}{0.500000in}}{\pgfqpoint{4.650000in}{3.020000in}}%
\pgfusepath{clip}%
\pgfsetbuttcap%
\pgfsetroundjoin%
\definecolor{currentfill}{rgb}{0.121569,0.466667,0.705882}%
\pgfsetfillcolor{currentfill}%
\pgfsetlinewidth{1.003750pt}%
\definecolor{currentstroke}{rgb}{0.121569,0.466667,0.705882}%
\pgfsetstrokecolor{currentstroke}%
\pgfsetdash{}{0pt}%
\pgfpathmoveto{\pgfqpoint{1.299277in}{0.812754in}}%
\pgfpathcurveto{\pgfqpoint{1.301882in}{0.812754in}}{\pgfqpoint{1.304380in}{0.813789in}}{\pgfqpoint{1.306222in}{0.815631in}}%
\pgfpathcurveto{\pgfqpoint{1.308063in}{0.817473in}}{\pgfqpoint{1.309098in}{0.819971in}}{\pgfqpoint{1.309098in}{0.822575in}}%
\pgfpathcurveto{\pgfqpoint{1.309098in}{0.825180in}}{\pgfqpoint{1.308063in}{0.827678in}}{\pgfqpoint{1.306222in}{0.829520in}}%
\pgfpathcurveto{\pgfqpoint{1.304380in}{0.831361in}}{\pgfqpoint{1.301882in}{0.832396in}}{\pgfqpoint{1.299277in}{0.832396in}}%
\pgfpathcurveto{\pgfqpoint{1.296673in}{0.832396in}}{\pgfqpoint{1.294174in}{0.831361in}}{\pgfqpoint{1.292333in}{0.829520in}}%
\pgfpathcurveto{\pgfqpoint{1.290491in}{0.827678in}}{\pgfqpoint{1.289456in}{0.825180in}}{\pgfqpoint{1.289456in}{0.822575in}}%
\pgfpathcurveto{\pgfqpoint{1.289456in}{0.819971in}}{\pgfqpoint{1.290491in}{0.817473in}}{\pgfqpoint{1.292333in}{0.815631in}}%
\pgfpathcurveto{\pgfqpoint{1.294174in}{0.813789in}}{\pgfqpoint{1.296673in}{0.812754in}}{\pgfqpoint{1.299277in}{0.812754in}}%
\pgfpathclose%
\pgfusepath{stroke,fill}%
\end{pgfscope}%
\begin{pgfscope}%
\pgfpathrectangle{\pgfqpoint{0.750000in}{0.500000in}}{\pgfqpoint{4.650000in}{3.020000in}}%
\pgfusepath{clip}%
\pgfsetbuttcap%
\pgfsetroundjoin%
\definecolor{currentfill}{rgb}{0.121569,0.466667,0.705882}%
\pgfsetfillcolor{currentfill}%
\pgfsetlinewidth{1.003750pt}%
\definecolor{currentstroke}{rgb}{0.121569,0.466667,0.705882}%
\pgfsetstrokecolor{currentstroke}%
\pgfsetdash{}{0pt}%
\pgfpathmoveto{\pgfqpoint{1.034105in}{0.990856in}}%
\pgfpathcurveto{\pgfqpoint{1.036710in}{0.990856in}}{\pgfqpoint{1.039208in}{0.991891in}}{\pgfqpoint{1.041050in}{0.993733in}}%
\pgfpathcurveto{\pgfqpoint{1.042891in}{0.995574in}}{\pgfqpoint{1.043926in}{0.998072in}}{\pgfqpoint{1.043926in}{1.000677in}}%
\pgfpathcurveto{\pgfqpoint{1.043926in}{1.003282in}}{\pgfqpoint{1.042891in}{1.005780in}}{\pgfqpoint{1.041050in}{1.007621in}}%
\pgfpathcurveto{\pgfqpoint{1.039208in}{1.009463in}}{\pgfqpoint{1.036710in}{1.010498in}}{\pgfqpoint{1.034105in}{1.010498in}}%
\pgfpathcurveto{\pgfqpoint{1.031501in}{1.010498in}}{\pgfqpoint{1.029003in}{1.009463in}}{\pgfqpoint{1.027161in}{1.007621in}}%
\pgfpathcurveto{\pgfqpoint{1.025319in}{1.005780in}}{\pgfqpoint{1.024284in}{1.003282in}}{\pgfqpoint{1.024284in}{1.000677in}}%
\pgfpathcurveto{\pgfqpoint{1.024284in}{0.998072in}}{\pgfqpoint{1.025319in}{0.995574in}}{\pgfqpoint{1.027161in}{0.993733in}}%
\pgfpathcurveto{\pgfqpoint{1.029003in}{0.991891in}}{\pgfqpoint{1.031501in}{0.990856in}}{\pgfqpoint{1.034105in}{0.990856in}}%
\pgfpathclose%
\pgfusepath{stroke,fill}%
\end{pgfscope}%
\begin{pgfscope}%
\pgfpathrectangle{\pgfqpoint{0.750000in}{0.500000in}}{\pgfqpoint{4.650000in}{3.020000in}}%
\pgfusepath{clip}%
\pgfsetbuttcap%
\pgfsetroundjoin%
\definecolor{currentfill}{rgb}{0.121569,0.466667,0.705882}%
\pgfsetfillcolor{currentfill}%
\pgfsetlinewidth{1.003750pt}%
\definecolor{currentstroke}{rgb}{0.121569,0.466667,0.705882}%
\pgfsetstrokecolor{currentstroke}%
\pgfsetdash{}{0pt}%
\pgfpathmoveto{\pgfqpoint{2.402784in}{2.164707in}}%
\pgfpathcurveto{\pgfqpoint{2.412872in}{2.164707in}}{\pgfqpoint{2.422547in}{2.168715in}}{\pgfqpoint{2.429680in}{2.175848in}}%
\pgfpathcurveto{\pgfqpoint{2.436813in}{2.182981in}}{\pgfqpoint{2.440820in}{2.192656in}}{\pgfqpoint{2.440820in}{2.202743in}}%
\pgfpathcurveto{\pgfqpoint{2.440820in}{2.212831in}}{\pgfqpoint{2.436813in}{2.222506in}}{\pgfqpoint{2.429680in}{2.229639in}}%
\pgfpathcurveto{\pgfqpoint{2.422547in}{2.236772in}}{\pgfqpoint{2.412872in}{2.240780in}}{\pgfqpoint{2.402784in}{2.240780in}}%
\pgfpathcurveto{\pgfqpoint{2.392697in}{2.240780in}}{\pgfqpoint{2.383021in}{2.236772in}}{\pgfqpoint{2.375888in}{2.229639in}}%
\pgfpathcurveto{\pgfqpoint{2.368756in}{2.222506in}}{\pgfqpoint{2.364748in}{2.212831in}}{\pgfqpoint{2.364748in}{2.202743in}}%
\pgfpathcurveto{\pgfqpoint{2.364748in}{2.192656in}}{\pgfqpoint{2.368756in}{2.182981in}}{\pgfqpoint{2.375888in}{2.175848in}}%
\pgfpathcurveto{\pgfqpoint{2.383021in}{2.168715in}}{\pgfqpoint{2.392697in}{2.164707in}}{\pgfqpoint{2.402784in}{2.164707in}}%
\pgfpathclose%
\pgfusepath{stroke,fill}%
\end{pgfscope}%
\begin{pgfscope}%
\pgfpathrectangle{\pgfqpoint{0.750000in}{0.500000in}}{\pgfqpoint{4.650000in}{3.020000in}}%
\pgfusepath{clip}%
\pgfsetbuttcap%
\pgfsetroundjoin%
\definecolor{currentfill}{rgb}{0.121569,0.466667,0.705882}%
\pgfsetfillcolor{currentfill}%
\pgfsetlinewidth{1.003750pt}%
\definecolor{currentstroke}{rgb}{0.121569,0.466667,0.705882}%
\pgfsetstrokecolor{currentstroke}%
\pgfsetdash{}{0pt}%
\pgfpathmoveto{\pgfqpoint{1.097045in}{0.674691in}}%
\pgfpathcurveto{\pgfqpoint{1.100235in}{0.674691in}}{\pgfqpoint{1.103295in}{0.675958in}}{\pgfqpoint{1.105550in}{0.678214in}}%
\pgfpathcurveto{\pgfqpoint{1.107806in}{0.680469in}}{\pgfqpoint{1.109073in}{0.683529in}}{\pgfqpoint{1.109073in}{0.686719in}}%
\pgfpathcurveto{\pgfqpoint{1.109073in}{0.689909in}}{\pgfqpoint{1.107806in}{0.692968in}}{\pgfqpoint{1.105550in}{0.695224in}}%
\pgfpathcurveto{\pgfqpoint{1.103295in}{0.697480in}}{\pgfqpoint{1.100235in}{0.698747in}}{\pgfqpoint{1.097045in}{0.698747in}}%
\pgfpathcurveto{\pgfqpoint{1.093855in}{0.698747in}}{\pgfqpoint{1.090795in}{0.697480in}}{\pgfqpoint{1.088540in}{0.695224in}}%
\pgfpathcurveto{\pgfqpoint{1.086284in}{0.692968in}}{\pgfqpoint{1.085017in}{0.689909in}}{\pgfqpoint{1.085017in}{0.686719in}}%
\pgfpathcurveto{\pgfqpoint{1.085017in}{0.683529in}}{\pgfqpoint{1.086284in}{0.680469in}}{\pgfqpoint{1.088540in}{0.678214in}}%
\pgfpathcurveto{\pgfqpoint{1.090795in}{0.675958in}}{\pgfqpoint{1.093855in}{0.674691in}}{\pgfqpoint{1.097045in}{0.674691in}}%
\pgfpathclose%
\pgfusepath{stroke,fill}%
\end{pgfscope}%
\begin{pgfscope}%
\pgfpathrectangle{\pgfqpoint{0.750000in}{0.500000in}}{\pgfqpoint{4.650000in}{3.020000in}}%
\pgfusepath{clip}%
\pgfsetbuttcap%
\pgfsetroundjoin%
\definecolor{currentfill}{rgb}{0.121569,0.466667,0.705882}%
\pgfsetfillcolor{currentfill}%
\pgfsetlinewidth{1.003750pt}%
\definecolor{currentstroke}{rgb}{0.121569,0.466667,0.705882}%
\pgfsetstrokecolor{currentstroke}%
\pgfsetdash{}{0pt}%
\pgfpathmoveto{\pgfqpoint{1.551036in}{0.838678in}}%
\pgfpathcurveto{\pgfqpoint{1.553640in}{0.838678in}}{\pgfqpoint{1.556138in}{0.839712in}}{\pgfqpoint{1.557980in}{0.841554in}}%
\pgfpathcurveto{\pgfqpoint{1.559822in}{0.843396in}}{\pgfqpoint{1.560857in}{0.845894in}}{\pgfqpoint{1.560857in}{0.848498in}}%
\pgfpathcurveto{\pgfqpoint{1.560857in}{0.851103in}}{\pgfqpoint{1.559822in}{0.853601in}}{\pgfqpoint{1.557980in}{0.855443in}}%
\pgfpathcurveto{\pgfqpoint{1.556138in}{0.857285in}}{\pgfqpoint{1.553640in}{0.858319in}}{\pgfqpoint{1.551036in}{0.858319in}}%
\pgfpathcurveto{\pgfqpoint{1.548431in}{0.858319in}}{\pgfqpoint{1.545933in}{0.857285in}}{\pgfqpoint{1.544091in}{0.855443in}}%
\pgfpathcurveto{\pgfqpoint{1.542250in}{0.853601in}}{\pgfqpoint{1.541215in}{0.851103in}}{\pgfqpoint{1.541215in}{0.848498in}}%
\pgfpathcurveto{\pgfqpoint{1.541215in}{0.845894in}}{\pgfqpoint{1.542250in}{0.843396in}}{\pgfqpoint{1.544091in}{0.841554in}}%
\pgfpathcurveto{\pgfqpoint{1.545933in}{0.839712in}}{\pgfqpoint{1.548431in}{0.838678in}}{\pgfqpoint{1.551036in}{0.838678in}}%
\pgfpathclose%
\pgfusepath{stroke,fill}%
\end{pgfscope}%
\begin{pgfscope}%
\pgfpathrectangle{\pgfqpoint{0.750000in}{0.500000in}}{\pgfqpoint{4.650000in}{3.020000in}}%
\pgfusepath{clip}%
\pgfsetbuttcap%
\pgfsetroundjoin%
\definecolor{currentfill}{rgb}{0.121569,0.466667,0.705882}%
\pgfsetfillcolor{currentfill}%
\pgfsetlinewidth{1.003750pt}%
\definecolor{currentstroke}{rgb}{0.121569,0.466667,0.705882}%
\pgfsetstrokecolor{currentstroke}%
\pgfsetdash{}{0pt}%
\pgfpathmoveto{\pgfqpoint{0.961364in}{0.631292in}}%
\pgfpathcurveto{\pgfqpoint{0.963968in}{0.631292in}}{\pgfqpoint{0.966466in}{0.632327in}}{\pgfqpoint{0.968308in}{0.634169in}}%
\pgfpathcurveto{\pgfqpoint{0.970150in}{0.636010in}}{\pgfqpoint{0.971185in}{0.638509in}}{\pgfqpoint{0.971185in}{0.641113in}}%
\pgfpathcurveto{\pgfqpoint{0.971185in}{0.643718in}}{\pgfqpoint{0.970150in}{0.646216in}}{\pgfqpoint{0.968308in}{0.648058in}}%
\pgfpathcurveto{\pgfqpoint{0.966466in}{0.649899in}}{\pgfqpoint{0.963968in}{0.650934in}}{\pgfqpoint{0.961364in}{0.650934in}}%
\pgfpathcurveto{\pgfqpoint{0.958759in}{0.650934in}}{\pgfqpoint{0.956261in}{0.649899in}}{\pgfqpoint{0.954419in}{0.648058in}}%
\pgfpathcurveto{\pgfqpoint{0.952578in}{0.646216in}}{\pgfqpoint{0.951543in}{0.643718in}}{\pgfqpoint{0.951543in}{0.641113in}}%
\pgfpathcurveto{\pgfqpoint{0.951543in}{0.638509in}}{\pgfqpoint{0.952578in}{0.636010in}}{\pgfqpoint{0.954419in}{0.634169in}}%
\pgfpathcurveto{\pgfqpoint{0.956261in}{0.632327in}}{\pgfqpoint{0.958759in}{0.631292in}}{\pgfqpoint{0.961364in}{0.631292in}}%
\pgfpathclose%
\pgfusepath{stroke,fill}%
\end{pgfscope}%
\begin{pgfscope}%
\pgfpathrectangle{\pgfqpoint{0.750000in}{0.500000in}}{\pgfqpoint{4.650000in}{3.020000in}}%
\pgfusepath{clip}%
\pgfsetbuttcap%
\pgfsetroundjoin%
\definecolor{currentfill}{rgb}{0.121569,0.466667,0.705882}%
\pgfsetfillcolor{currentfill}%
\pgfsetlinewidth{1.003750pt}%
\definecolor{currentstroke}{rgb}{0.121569,0.466667,0.705882}%
\pgfsetstrokecolor{currentstroke}%
\pgfsetdash{}{0pt}%
\pgfpathmoveto{\pgfqpoint{1.330231in}{1.403487in}}%
\pgfpathcurveto{\pgfqpoint{1.335440in}{1.403487in}}{\pgfqpoint{1.340437in}{1.405556in}}{\pgfqpoint{1.344120in}{1.409240in}}%
\pgfpathcurveto{\pgfqpoint{1.347803in}{1.412923in}}{\pgfqpoint{1.349873in}{1.417919in}}{\pgfqpoint{1.349873in}{1.423128in}}%
\pgfpathcurveto{\pgfqpoint{1.349873in}{1.428338in}}{\pgfqpoint{1.347803in}{1.433334in}}{\pgfqpoint{1.344120in}{1.437017in}}%
\pgfpathcurveto{\pgfqpoint{1.340437in}{1.440701in}}{\pgfqpoint{1.335440in}{1.442770in}}{\pgfqpoint{1.330231in}{1.442770in}}%
\pgfpathcurveto{\pgfqpoint{1.325022in}{1.442770in}}{\pgfqpoint{1.320026in}{1.440701in}}{\pgfqpoint{1.316342in}{1.437017in}}%
\pgfpathcurveto{\pgfqpoint{1.312659in}{1.433334in}}{\pgfqpoint{1.310589in}{1.428338in}}{\pgfqpoint{1.310589in}{1.423128in}}%
\pgfpathcurveto{\pgfqpoint{1.310589in}{1.417919in}}{\pgfqpoint{1.312659in}{1.412923in}}{\pgfqpoint{1.316342in}{1.409240in}}%
\pgfpathcurveto{\pgfqpoint{1.320026in}{1.405556in}}{\pgfqpoint{1.325022in}{1.403487in}}{\pgfqpoint{1.330231in}{1.403487in}}%
\pgfpathclose%
\pgfusepath{stroke,fill}%
\end{pgfscope}%
\begin{pgfscope}%
\pgfpathrectangle{\pgfqpoint{0.750000in}{0.500000in}}{\pgfqpoint{4.650000in}{3.020000in}}%
\pgfusepath{clip}%
\pgfsetbuttcap%
\pgfsetroundjoin%
\definecolor{currentfill}{rgb}{0.121569,0.466667,0.705882}%
\pgfsetfillcolor{currentfill}%
\pgfsetlinewidth{1.003750pt}%
\definecolor{currentstroke}{rgb}{0.121569,0.466667,0.705882}%
\pgfsetstrokecolor{currentstroke}%
\pgfsetdash{}{0pt}%
\pgfpathmoveto{\pgfqpoint{1.009342in}{0.852119in}}%
\pgfpathcurveto{\pgfqpoint{1.011947in}{0.852119in}}{\pgfqpoint{1.014445in}{0.853154in}}{\pgfqpoint{1.016287in}{0.854996in}}%
\pgfpathcurveto{\pgfqpoint{1.018128in}{0.856837in}}{\pgfqpoint{1.019163in}{0.859336in}}{\pgfqpoint{1.019163in}{0.861940in}}%
\pgfpathcurveto{\pgfqpoint{1.019163in}{0.864545in}}{\pgfqpoint{1.018128in}{0.867043in}}{\pgfqpoint{1.016287in}{0.868885in}}%
\pgfpathcurveto{\pgfqpoint{1.014445in}{0.870726in}}{\pgfqpoint{1.011947in}{0.871761in}}{\pgfqpoint{1.009342in}{0.871761in}}%
\pgfpathcurveto{\pgfqpoint{1.006738in}{0.871761in}}{\pgfqpoint{1.004239in}{0.870726in}}{\pgfqpoint{1.002398in}{0.868885in}}%
\pgfpathcurveto{\pgfqpoint{1.000556in}{0.867043in}}{\pgfqpoint{0.999521in}{0.864545in}}{\pgfqpoint{0.999521in}{0.861940in}}%
\pgfpathcurveto{\pgfqpoint{0.999521in}{0.859336in}}{\pgfqpoint{1.000556in}{0.856837in}}{\pgfqpoint{1.002398in}{0.854996in}}%
\pgfpathcurveto{\pgfqpoint{1.004239in}{0.853154in}}{\pgfqpoint{1.006738in}{0.852119in}}{\pgfqpoint{1.009342in}{0.852119in}}%
\pgfpathclose%
\pgfusepath{stroke,fill}%
\end{pgfscope}%
\begin{pgfscope}%
\pgfpathrectangle{\pgfqpoint{0.750000in}{0.500000in}}{\pgfqpoint{4.650000in}{3.020000in}}%
\pgfusepath{clip}%
\pgfsetbuttcap%
\pgfsetroundjoin%
\definecolor{currentfill}{rgb}{0.121569,0.466667,0.705882}%
\pgfsetfillcolor{currentfill}%
\pgfsetlinewidth{1.003750pt}%
\definecolor{currentstroke}{rgb}{0.121569,0.466667,0.705882}%
\pgfsetstrokecolor{currentstroke}%
\pgfsetdash{}{0pt}%
\pgfpathmoveto{\pgfqpoint{1.094465in}{0.651455in}}%
\pgfpathcurveto{\pgfqpoint{1.097070in}{0.651455in}}{\pgfqpoint{1.099568in}{0.652490in}}{\pgfqpoint{1.101410in}{0.654331in}}%
\pgfpathcurveto{\pgfqpoint{1.103252in}{0.656173in}}{\pgfqpoint{1.104286in}{0.658671in}}{\pgfqpoint{1.104286in}{0.661276in}}%
\pgfpathcurveto{\pgfqpoint{1.104286in}{0.663880in}}{\pgfqpoint{1.103252in}{0.666378in}}{\pgfqpoint{1.101410in}{0.668220in}}%
\pgfpathcurveto{\pgfqpoint{1.099568in}{0.670062in}}{\pgfqpoint{1.097070in}{0.671097in}}{\pgfqpoint{1.094465in}{0.671097in}}%
\pgfpathcurveto{\pgfqpoint{1.091861in}{0.671097in}}{\pgfqpoint{1.089363in}{0.670062in}}{\pgfqpoint{1.087521in}{0.668220in}}%
\pgfpathcurveto{\pgfqpoint{1.085679in}{0.666378in}}{\pgfqpoint{1.084645in}{0.663880in}}{\pgfqpoint{1.084645in}{0.661276in}}%
\pgfpathcurveto{\pgfqpoint{1.084645in}{0.658671in}}{\pgfqpoint{1.085679in}{0.656173in}}{\pgfqpoint{1.087521in}{0.654331in}}%
\pgfpathcurveto{\pgfqpoint{1.089363in}{0.652490in}}{\pgfqpoint{1.091861in}{0.651455in}}{\pgfqpoint{1.094465in}{0.651455in}}%
\pgfpathclose%
\pgfusepath{stroke,fill}%
\end{pgfscope}%
\begin{pgfscope}%
\pgfpathrectangle{\pgfqpoint{0.750000in}{0.500000in}}{\pgfqpoint{4.650000in}{3.020000in}}%
\pgfusepath{clip}%
\pgfsetbuttcap%
\pgfsetroundjoin%
\definecolor{currentfill}{rgb}{0.121569,0.466667,0.705882}%
\pgfsetfillcolor{currentfill}%
\pgfsetlinewidth{1.003750pt}%
\definecolor{currentstroke}{rgb}{0.121569,0.466667,0.705882}%
\pgfsetstrokecolor{currentstroke}%
\pgfsetdash{}{0pt}%
\pgfpathmoveto{\pgfqpoint{0.988706in}{0.646654in}}%
\pgfpathcurveto{\pgfqpoint{0.991311in}{0.646654in}}{\pgfqpoint{0.993809in}{0.647689in}}{\pgfqpoint{0.995651in}{0.649531in}}%
\pgfpathcurveto{\pgfqpoint{0.997492in}{0.651372in}}{\pgfqpoint{0.998527in}{0.653871in}}{\pgfqpoint{0.998527in}{0.656475in}}%
\pgfpathcurveto{\pgfqpoint{0.998527in}{0.659080in}}{\pgfqpoint{0.997492in}{0.661578in}}{\pgfqpoint{0.995651in}{0.663420in}}%
\pgfpathcurveto{\pgfqpoint{0.993809in}{0.665261in}}{\pgfqpoint{0.991311in}{0.666296in}}{\pgfqpoint{0.988706in}{0.666296in}}%
\pgfpathcurveto{\pgfqpoint{0.986102in}{0.666296in}}{\pgfqpoint{0.983604in}{0.665261in}}{\pgfqpoint{0.981762in}{0.663420in}}%
\pgfpathcurveto{\pgfqpoint{0.979920in}{0.661578in}}{\pgfqpoint{0.978885in}{0.659080in}}{\pgfqpoint{0.978885in}{0.656475in}}%
\pgfpathcurveto{\pgfqpoint{0.978885in}{0.653871in}}{\pgfqpoint{0.979920in}{0.651372in}}{\pgfqpoint{0.981762in}{0.649531in}}%
\pgfpathcurveto{\pgfqpoint{0.983604in}{0.647689in}}{\pgfqpoint{0.986102in}{0.646654in}}{\pgfqpoint{0.988706in}{0.646654in}}%
\pgfpathclose%
\pgfusepath{stroke,fill}%
\end{pgfscope}%
\begin{pgfscope}%
\pgfpathrectangle{\pgfqpoint{0.750000in}{0.500000in}}{\pgfqpoint{4.650000in}{3.020000in}}%
\pgfusepath{clip}%
\pgfsetbuttcap%
\pgfsetroundjoin%
\definecolor{currentfill}{rgb}{0.121569,0.466667,0.705882}%
\pgfsetfillcolor{currentfill}%
\pgfsetlinewidth{1.003750pt}%
\definecolor{currentstroke}{rgb}{0.121569,0.466667,0.705882}%
\pgfsetstrokecolor{currentstroke}%
\pgfsetdash{}{0pt}%
\pgfpathmoveto{\pgfqpoint{0.975293in}{0.832917in}}%
\pgfpathcurveto{\pgfqpoint{0.977897in}{0.832917in}}{\pgfqpoint{0.980396in}{0.833952in}}{\pgfqpoint{0.982237in}{0.835793in}}%
\pgfpathcurveto{\pgfqpoint{0.984079in}{0.837635in}}{\pgfqpoint{0.985114in}{0.840133in}}{\pgfqpoint{0.985114in}{0.842738in}}%
\pgfpathcurveto{\pgfqpoint{0.985114in}{0.845342in}}{\pgfqpoint{0.984079in}{0.847841in}}{\pgfqpoint{0.982237in}{0.849682in}}%
\pgfpathcurveto{\pgfqpoint{0.980396in}{0.851524in}}{\pgfqpoint{0.977897in}{0.852559in}}{\pgfqpoint{0.975293in}{0.852559in}}%
\pgfpathcurveto{\pgfqpoint{0.972688in}{0.852559in}}{\pgfqpoint{0.970190in}{0.851524in}}{\pgfqpoint{0.968348in}{0.849682in}}%
\pgfpathcurveto{\pgfqpoint{0.966507in}{0.847841in}}{\pgfqpoint{0.965472in}{0.845342in}}{\pgfqpoint{0.965472in}{0.842738in}}%
\pgfpathcurveto{\pgfqpoint{0.965472in}{0.840133in}}{\pgfqpoint{0.966507in}{0.837635in}}{\pgfqpoint{0.968348in}{0.835793in}}%
\pgfpathcurveto{\pgfqpoint{0.970190in}{0.833952in}}{\pgfqpoint{0.972688in}{0.832917in}}{\pgfqpoint{0.975293in}{0.832917in}}%
\pgfpathclose%
\pgfusepath{stroke,fill}%
\end{pgfscope}%
\begin{pgfscope}%
\pgfpathrectangle{\pgfqpoint{0.750000in}{0.500000in}}{\pgfqpoint{4.650000in}{3.020000in}}%
\pgfusepath{clip}%
\pgfsetbuttcap%
\pgfsetroundjoin%
\definecolor{currentfill}{rgb}{0.121569,0.466667,0.705882}%
\pgfsetfillcolor{currentfill}%
\pgfsetlinewidth{1.003750pt}%
\definecolor{currentstroke}{rgb}{0.121569,0.466667,0.705882}%
\pgfsetstrokecolor{currentstroke}%
\pgfsetdash{}{0pt}%
\pgfpathmoveto{\pgfqpoint{1.513891in}{1.752709in}}%
\pgfpathcurveto{\pgfqpoint{1.516496in}{1.752709in}}{\pgfqpoint{1.518994in}{1.753744in}}{\pgfqpoint{1.520835in}{1.755585in}}%
\pgfpathcurveto{\pgfqpoint{1.522677in}{1.757427in}}{\pgfqpoint{1.523712in}{1.759925in}}{\pgfqpoint{1.523712in}{1.762530in}}%
\pgfpathcurveto{\pgfqpoint{1.523712in}{1.765134in}}{\pgfqpoint{1.522677in}{1.767633in}}{\pgfqpoint{1.520835in}{1.769474in}}%
\pgfpathcurveto{\pgfqpoint{1.518994in}{1.771316in}}{\pgfqpoint{1.516496in}{1.772351in}}{\pgfqpoint{1.513891in}{1.772351in}}%
\pgfpathcurveto{\pgfqpoint{1.511286in}{1.772351in}}{\pgfqpoint{1.508788in}{1.771316in}}{\pgfqpoint{1.506947in}{1.769474in}}%
\pgfpathcurveto{\pgfqpoint{1.505105in}{1.767633in}}{\pgfqpoint{1.504070in}{1.765134in}}{\pgfqpoint{1.504070in}{1.762530in}}%
\pgfpathcurveto{\pgfqpoint{1.504070in}{1.759925in}}{\pgfqpoint{1.505105in}{1.757427in}}{\pgfqpoint{1.506947in}{1.755585in}}%
\pgfpathcurveto{\pgfqpoint{1.508788in}{1.753744in}}{\pgfqpoint{1.511286in}{1.752709in}}{\pgfqpoint{1.513891in}{1.752709in}}%
\pgfpathclose%
\pgfusepath{stroke,fill}%
\end{pgfscope}%
\begin{pgfscope}%
\pgfpathrectangle{\pgfqpoint{0.750000in}{0.500000in}}{\pgfqpoint{4.650000in}{3.020000in}}%
\pgfusepath{clip}%
\pgfsetbuttcap%
\pgfsetroundjoin%
\definecolor{currentfill}{rgb}{0.121569,0.466667,0.705882}%
\pgfsetfillcolor{currentfill}%
\pgfsetlinewidth{1.003750pt}%
\definecolor{currentstroke}{rgb}{0.121569,0.466667,0.705882}%
\pgfsetstrokecolor{currentstroke}%
\pgfsetdash{}{0pt}%
\pgfpathmoveto{\pgfqpoint{1.157921in}{0.665856in}}%
\pgfpathcurveto{\pgfqpoint{1.160526in}{0.665856in}}{\pgfqpoint{1.163024in}{0.666891in}}{\pgfqpoint{1.164865in}{0.668733in}}%
\pgfpathcurveto{\pgfqpoint{1.166707in}{0.670575in}}{\pgfqpoint{1.167742in}{0.673073in}}{\pgfqpoint{1.167742in}{0.675677in}}%
\pgfpathcurveto{\pgfqpoint{1.167742in}{0.678282in}}{\pgfqpoint{1.166707in}{0.680780in}}{\pgfqpoint{1.164865in}{0.682622in}}%
\pgfpathcurveto{\pgfqpoint{1.163024in}{0.684464in}}{\pgfqpoint{1.160526in}{0.685498in}}{\pgfqpoint{1.157921in}{0.685498in}}%
\pgfpathcurveto{\pgfqpoint{1.155316in}{0.685498in}}{\pgfqpoint{1.152818in}{0.684464in}}{\pgfqpoint{1.150977in}{0.682622in}}%
\pgfpathcurveto{\pgfqpoint{1.149135in}{0.680780in}}{\pgfqpoint{1.148100in}{0.678282in}}{\pgfqpoint{1.148100in}{0.675677in}}%
\pgfpathcurveto{\pgfqpoint{1.148100in}{0.673073in}}{\pgfqpoint{1.149135in}{0.670575in}}{\pgfqpoint{1.150977in}{0.668733in}}%
\pgfpathcurveto{\pgfqpoint{1.152818in}{0.666891in}}{\pgfqpoint{1.155316in}{0.665856in}}{\pgfqpoint{1.157921in}{0.665856in}}%
\pgfpathclose%
\pgfusepath{stroke,fill}%
\end{pgfscope}%
\begin{pgfscope}%
\pgfpathrectangle{\pgfqpoint{0.750000in}{0.500000in}}{\pgfqpoint{4.650000in}{3.020000in}}%
\pgfusepath{clip}%
\pgfsetbuttcap%
\pgfsetroundjoin%
\definecolor{currentfill}{rgb}{0.121569,0.466667,0.705882}%
\pgfsetfillcolor{currentfill}%
\pgfsetlinewidth{1.003750pt}%
\definecolor{currentstroke}{rgb}{0.121569,0.466667,0.705882}%
\pgfsetstrokecolor{currentstroke}%
\pgfsetdash{}{0pt}%
\pgfpathmoveto{\pgfqpoint{1.390591in}{1.044623in}}%
\pgfpathcurveto{\pgfqpoint{1.393196in}{1.044623in}}{\pgfqpoint{1.395694in}{1.045657in}}{\pgfqpoint{1.397536in}{1.047499in}}%
\pgfpathcurveto{\pgfqpoint{1.399377in}{1.049341in}}{\pgfqpoint{1.400412in}{1.051839in}}{\pgfqpoint{1.400412in}{1.054444in}}%
\pgfpathcurveto{\pgfqpoint{1.400412in}{1.057048in}}{\pgfqpoint{1.399377in}{1.059546in}}{\pgfqpoint{1.397536in}{1.061388in}}%
\pgfpathcurveto{\pgfqpoint{1.395694in}{1.063230in}}{\pgfqpoint{1.393196in}{1.064264in}}{\pgfqpoint{1.390591in}{1.064264in}}%
\pgfpathcurveto{\pgfqpoint{1.387987in}{1.064264in}}{\pgfqpoint{1.385488in}{1.063230in}}{\pgfqpoint{1.383647in}{1.061388in}}%
\pgfpathcurveto{\pgfqpoint{1.381805in}{1.059546in}}{\pgfqpoint{1.380770in}{1.057048in}}{\pgfqpoint{1.380770in}{1.054444in}}%
\pgfpathcurveto{\pgfqpoint{1.380770in}{1.051839in}}{\pgfqpoint{1.381805in}{1.049341in}}{\pgfqpoint{1.383647in}{1.047499in}}%
\pgfpathcurveto{\pgfqpoint{1.385488in}{1.045657in}}{\pgfqpoint{1.387987in}{1.044623in}}{\pgfqpoint{1.390591in}{1.044623in}}%
\pgfpathclose%
\pgfusepath{stroke,fill}%
\end{pgfscope}%
\begin{pgfscope}%
\pgfpathrectangle{\pgfqpoint{0.750000in}{0.500000in}}{\pgfqpoint{4.650000in}{3.020000in}}%
\pgfusepath{clip}%
\pgfsetbuttcap%
\pgfsetroundjoin%
\definecolor{currentfill}{rgb}{0.121569,0.466667,0.705882}%
\pgfsetfillcolor{currentfill}%
\pgfsetlinewidth{1.003750pt}%
\definecolor{currentstroke}{rgb}{0.121569,0.466667,0.705882}%
\pgfsetstrokecolor{currentstroke}%
\pgfsetdash{}{0pt}%
\pgfpathmoveto{\pgfqpoint{1.170303in}{0.926528in}}%
\pgfpathcurveto{\pgfqpoint{1.172907in}{0.926528in}}{\pgfqpoint{1.175405in}{0.927563in}}{\pgfqpoint{1.177247in}{0.929405in}}%
\pgfpathcurveto{\pgfqpoint{1.179089in}{0.931246in}}{\pgfqpoint{1.180123in}{0.933745in}}{\pgfqpoint{1.180123in}{0.936349in}}%
\pgfpathcurveto{\pgfqpoint{1.180123in}{0.938954in}}{\pgfqpoint{1.179089in}{0.941452in}}{\pgfqpoint{1.177247in}{0.943294in}}%
\pgfpathcurveto{\pgfqpoint{1.175405in}{0.945135in}}{\pgfqpoint{1.172907in}{0.946170in}}{\pgfqpoint{1.170303in}{0.946170in}}%
\pgfpathcurveto{\pgfqpoint{1.167698in}{0.946170in}}{\pgfqpoint{1.165200in}{0.945135in}}{\pgfqpoint{1.163358in}{0.943294in}}%
\pgfpathcurveto{\pgfqpoint{1.161516in}{0.941452in}}{\pgfqpoint{1.160482in}{0.938954in}}{\pgfqpoint{1.160482in}{0.936349in}}%
\pgfpathcurveto{\pgfqpoint{1.160482in}{0.933745in}}{\pgfqpoint{1.161516in}{0.931246in}}{\pgfqpoint{1.163358in}{0.929405in}}%
\pgfpathcurveto{\pgfqpoint{1.165200in}{0.927563in}}{\pgfqpoint{1.167698in}{0.926528in}}{\pgfqpoint{1.170303in}{0.926528in}}%
\pgfpathclose%
\pgfusepath{stroke,fill}%
\end{pgfscope}%
\begin{pgfscope}%
\pgfpathrectangle{\pgfqpoint{0.750000in}{0.500000in}}{\pgfqpoint{4.650000in}{3.020000in}}%
\pgfusepath{clip}%
\pgfsetbuttcap%
\pgfsetroundjoin%
\definecolor{currentfill}{rgb}{0.121569,0.466667,0.705882}%
\pgfsetfillcolor{currentfill}%
\pgfsetlinewidth{1.003750pt}%
\definecolor{currentstroke}{rgb}{0.121569,0.466667,0.705882}%
\pgfsetstrokecolor{currentstroke}%
\pgfsetdash{}{0pt}%
\pgfpathmoveto{\pgfqpoint{1.056289in}{0.643774in}}%
\pgfpathcurveto{\pgfqpoint{1.058894in}{0.643774in}}{\pgfqpoint{1.061392in}{0.644809in}}{\pgfqpoint{1.063233in}{0.646650in}}%
\pgfpathcurveto{\pgfqpoint{1.065075in}{0.648492in}}{\pgfqpoint{1.066110in}{0.650990in}}{\pgfqpoint{1.066110in}{0.653595in}}%
\pgfpathcurveto{\pgfqpoint{1.066110in}{0.656199in}}{\pgfqpoint{1.065075in}{0.658697in}}{\pgfqpoint{1.063233in}{0.660539in}}%
\pgfpathcurveto{\pgfqpoint{1.061392in}{0.662381in}}{\pgfqpoint{1.058894in}{0.663416in}}{\pgfqpoint{1.056289in}{0.663416in}}%
\pgfpathcurveto{\pgfqpoint{1.053684in}{0.663416in}}{\pgfqpoint{1.051186in}{0.662381in}}{\pgfqpoint{1.049345in}{0.660539in}}%
\pgfpathcurveto{\pgfqpoint{1.047503in}{0.658697in}}{\pgfqpoint{1.046468in}{0.656199in}}{\pgfqpoint{1.046468in}{0.653595in}}%
\pgfpathcurveto{\pgfqpoint{1.046468in}{0.650990in}}{\pgfqpoint{1.047503in}{0.648492in}}{\pgfqpoint{1.049345in}{0.646650in}}%
\pgfpathcurveto{\pgfqpoint{1.051186in}{0.644809in}}{\pgfqpoint{1.053684in}{0.643774in}}{\pgfqpoint{1.056289in}{0.643774in}}%
\pgfpathclose%
\pgfusepath{stroke,fill}%
\end{pgfscope}%
\begin{pgfscope}%
\pgfpathrectangle{\pgfqpoint{0.750000in}{0.500000in}}{\pgfqpoint{4.650000in}{3.020000in}}%
\pgfusepath{clip}%
\pgfsetbuttcap%
\pgfsetroundjoin%
\definecolor{currentfill}{rgb}{0.121569,0.466667,0.705882}%
\pgfsetfillcolor{currentfill}%
\pgfsetlinewidth{1.003750pt}%
\definecolor{currentstroke}{rgb}{0.121569,0.466667,0.705882}%
\pgfsetstrokecolor{currentstroke}%
\pgfsetdash{}{0pt}%
\pgfpathmoveto{\pgfqpoint{1.268839in}{0.704454in}}%
\pgfpathcurveto{\pgfqpoint{1.272029in}{0.704454in}}{\pgfqpoint{1.275089in}{0.705722in}}{\pgfqpoint{1.277344in}{0.707977in}}%
\pgfpathcurveto{\pgfqpoint{1.279600in}{0.710233in}}{\pgfqpoint{1.280867in}{0.713292in}}{\pgfqpoint{1.280867in}{0.716482in}}%
\pgfpathcurveto{\pgfqpoint{1.280867in}{0.719672in}}{\pgfqpoint{1.279600in}{0.722732in}}{\pgfqpoint{1.277344in}{0.724988in}}%
\pgfpathcurveto{\pgfqpoint{1.275089in}{0.727243in}}{\pgfqpoint{1.272029in}{0.728511in}}{\pgfqpoint{1.268839in}{0.728511in}}%
\pgfpathcurveto{\pgfqpoint{1.265649in}{0.728511in}}{\pgfqpoint{1.262590in}{0.727243in}}{\pgfqpoint{1.260334in}{0.724988in}}%
\pgfpathcurveto{\pgfqpoint{1.258078in}{0.722732in}}{\pgfqpoint{1.256811in}{0.719672in}}{\pgfqpoint{1.256811in}{0.716482in}}%
\pgfpathcurveto{\pgfqpoint{1.256811in}{0.713292in}}{\pgfqpoint{1.258078in}{0.710233in}}{\pgfqpoint{1.260334in}{0.707977in}}%
\pgfpathcurveto{\pgfqpoint{1.262590in}{0.705722in}}{\pgfqpoint{1.265649in}{0.704454in}}{\pgfqpoint{1.268839in}{0.704454in}}%
\pgfpathclose%
\pgfusepath{stroke,fill}%
\end{pgfscope}%
\begin{pgfscope}%
\pgfpathrectangle{\pgfqpoint{0.750000in}{0.500000in}}{\pgfqpoint{4.650000in}{3.020000in}}%
\pgfusepath{clip}%
\pgfsetbuttcap%
\pgfsetroundjoin%
\definecolor{currentfill}{rgb}{0.121569,0.466667,0.705882}%
\pgfsetfillcolor{currentfill}%
\pgfsetlinewidth{1.003750pt}%
\definecolor{currentstroke}{rgb}{0.121569,0.466667,0.705882}%
\pgfsetstrokecolor{currentstroke}%
\pgfsetdash{}{0pt}%
\pgfpathmoveto{\pgfqpoint{0.963427in}{0.631292in}}%
\pgfpathcurveto{\pgfqpoint{0.966032in}{0.631292in}}{\pgfqpoint{0.968530in}{0.632327in}}{\pgfqpoint{0.970372in}{0.634169in}}%
\pgfpathcurveto{\pgfqpoint{0.972213in}{0.636010in}}{\pgfqpoint{0.973248in}{0.638509in}}{\pgfqpoint{0.973248in}{0.641113in}}%
\pgfpathcurveto{\pgfqpoint{0.973248in}{0.643718in}}{\pgfqpoint{0.972213in}{0.646216in}}{\pgfqpoint{0.970372in}{0.648058in}}%
\pgfpathcurveto{\pgfqpoint{0.968530in}{0.649899in}}{\pgfqpoint{0.966032in}{0.650934in}}{\pgfqpoint{0.963427in}{0.650934in}}%
\pgfpathcurveto{\pgfqpoint{0.960823in}{0.650934in}}{\pgfqpoint{0.958324in}{0.649899in}}{\pgfqpoint{0.956483in}{0.648058in}}%
\pgfpathcurveto{\pgfqpoint{0.954641in}{0.646216in}}{\pgfqpoint{0.953606in}{0.643718in}}{\pgfqpoint{0.953606in}{0.641113in}}%
\pgfpathcurveto{\pgfqpoint{0.953606in}{0.638509in}}{\pgfqpoint{0.954641in}{0.636010in}}{\pgfqpoint{0.956483in}{0.634169in}}%
\pgfpathcurveto{\pgfqpoint{0.958324in}{0.632327in}}{\pgfqpoint{0.960823in}{0.631292in}}{\pgfqpoint{0.963427in}{0.631292in}}%
\pgfpathclose%
\pgfusepath{stroke,fill}%
\end{pgfscope}%
\begin{pgfscope}%
\pgfpathrectangle{\pgfqpoint{0.750000in}{0.500000in}}{\pgfqpoint{4.650000in}{3.020000in}}%
\pgfusepath{clip}%
\pgfsetbuttcap%
\pgfsetroundjoin%
\definecolor{currentfill}{rgb}{0.121569,0.466667,0.705882}%
\pgfsetfillcolor{currentfill}%
\pgfsetlinewidth{1.003750pt}%
\definecolor{currentstroke}{rgb}{0.121569,0.466667,0.705882}%
\pgfsetstrokecolor{currentstroke}%
\pgfsetdash{}{0pt}%
\pgfpathmoveto{\pgfqpoint{1.180621in}{0.670657in}}%
\pgfpathcurveto{\pgfqpoint{1.183225in}{0.670657in}}{\pgfqpoint{1.185723in}{0.671692in}}{\pgfqpoint{1.187565in}{0.673534in}}%
\pgfpathcurveto{\pgfqpoint{1.189407in}{0.675375in}}{\pgfqpoint{1.190441in}{0.677873in}}{\pgfqpoint{1.190441in}{0.680478in}}%
\pgfpathcurveto{\pgfqpoint{1.190441in}{0.683083in}}{\pgfqpoint{1.189407in}{0.685581in}}{\pgfqpoint{1.187565in}{0.687422in}}%
\pgfpathcurveto{\pgfqpoint{1.185723in}{0.689264in}}{\pgfqpoint{1.183225in}{0.690299in}}{\pgfqpoint{1.180621in}{0.690299in}}%
\pgfpathcurveto{\pgfqpoint{1.178016in}{0.690299in}}{\pgfqpoint{1.175518in}{0.689264in}}{\pgfqpoint{1.173676in}{0.687422in}}%
\pgfpathcurveto{\pgfqpoint{1.171834in}{0.685581in}}{\pgfqpoint{1.170800in}{0.683083in}}{\pgfqpoint{1.170800in}{0.680478in}}%
\pgfpathcurveto{\pgfqpoint{1.170800in}{0.677873in}}{\pgfqpoint{1.171834in}{0.675375in}}{\pgfqpoint{1.173676in}{0.673534in}}%
\pgfpathcurveto{\pgfqpoint{1.175518in}{0.671692in}}{\pgfqpoint{1.178016in}{0.670657in}}{\pgfqpoint{1.180621in}{0.670657in}}%
\pgfpathclose%
\pgfusepath{stroke,fill}%
\end{pgfscope}%
\begin{pgfscope}%
\pgfpathrectangle{\pgfqpoint{0.750000in}{0.500000in}}{\pgfqpoint{4.650000in}{3.020000in}}%
\pgfusepath{clip}%
\pgfsetbuttcap%
\pgfsetroundjoin%
\definecolor{currentfill}{rgb}{0.121569,0.466667,0.705882}%
\pgfsetfillcolor{currentfill}%
\pgfsetlinewidth{1.003750pt}%
\definecolor{currentstroke}{rgb}{0.121569,0.466667,0.705882}%
\pgfsetstrokecolor{currentstroke}%
\pgfsetdash{}{0pt}%
\pgfpathmoveto{\pgfqpoint{1.169787in}{0.700421in}}%
\pgfpathcurveto{\pgfqpoint{1.172391in}{0.700421in}}{\pgfqpoint{1.174889in}{0.701455in}}{\pgfqpoint{1.176731in}{0.703297in}}%
\pgfpathcurveto{\pgfqpoint{1.178573in}{0.705139in}}{\pgfqpoint{1.179608in}{0.707637in}}{\pgfqpoint{1.179608in}{0.710242in}}%
\pgfpathcurveto{\pgfqpoint{1.179608in}{0.712846in}}{\pgfqpoint{1.178573in}{0.715344in}}{\pgfqpoint{1.176731in}{0.717186in}}%
\pgfpathcurveto{\pgfqpoint{1.174889in}{0.719028in}}{\pgfqpoint{1.172391in}{0.720063in}}{\pgfqpoint{1.169787in}{0.720063in}}%
\pgfpathcurveto{\pgfqpoint{1.167182in}{0.720063in}}{\pgfqpoint{1.164684in}{0.719028in}}{\pgfqpoint{1.162842in}{0.717186in}}%
\pgfpathcurveto{\pgfqpoint{1.161001in}{0.715344in}}{\pgfqpoint{1.159966in}{0.712846in}}{\pgfqpoint{1.159966in}{0.710242in}}%
\pgfpathcurveto{\pgfqpoint{1.159966in}{0.707637in}}{\pgfqpoint{1.161001in}{0.705139in}}{\pgfqpoint{1.162842in}{0.703297in}}%
\pgfpathcurveto{\pgfqpoint{1.164684in}{0.701455in}}{\pgfqpoint{1.167182in}{0.700421in}}{\pgfqpoint{1.169787in}{0.700421in}}%
\pgfpathclose%
\pgfusepath{stroke,fill}%
\end{pgfscope}%
\begin{pgfscope}%
\pgfpathrectangle{\pgfqpoint{0.750000in}{0.500000in}}{\pgfqpoint{4.650000in}{3.020000in}}%
\pgfusepath{clip}%
\pgfsetbuttcap%
\pgfsetroundjoin%
\definecolor{currentfill}{rgb}{0.121569,0.466667,0.705882}%
\pgfsetfillcolor{currentfill}%
\pgfsetlinewidth{1.003750pt}%
\definecolor{currentstroke}{rgb}{0.121569,0.466667,0.705882}%
\pgfsetstrokecolor{currentstroke}%
\pgfsetdash{}{0pt}%
\pgfpathmoveto{\pgfqpoint{1.161016in}{0.737385in}}%
\pgfpathcurveto{\pgfqpoint{1.163621in}{0.737385in}}{\pgfqpoint{1.166119in}{0.738420in}}{\pgfqpoint{1.167961in}{0.740262in}}%
\pgfpathcurveto{\pgfqpoint{1.169803in}{0.742103in}}{\pgfqpoint{1.170837in}{0.744602in}}{\pgfqpoint{1.170837in}{0.747206in}}%
\pgfpathcurveto{\pgfqpoint{1.170837in}{0.749811in}}{\pgfqpoint{1.169803in}{0.752309in}}{\pgfqpoint{1.167961in}{0.754151in}}%
\pgfpathcurveto{\pgfqpoint{1.166119in}{0.755992in}}{\pgfqpoint{1.163621in}{0.757027in}}{\pgfqpoint{1.161016in}{0.757027in}}%
\pgfpathcurveto{\pgfqpoint{1.158412in}{0.757027in}}{\pgfqpoint{1.155914in}{0.755992in}}{\pgfqpoint{1.154072in}{0.754151in}}%
\pgfpathcurveto{\pgfqpoint{1.152230in}{0.752309in}}{\pgfqpoint{1.151195in}{0.749811in}}{\pgfqpoint{1.151195in}{0.747206in}}%
\pgfpathcurveto{\pgfqpoint{1.151195in}{0.744602in}}{\pgfqpoint{1.152230in}{0.742103in}}{\pgfqpoint{1.154072in}{0.740262in}}%
\pgfpathcurveto{\pgfqpoint{1.155914in}{0.738420in}}{\pgfqpoint{1.158412in}{0.737385in}}{\pgfqpoint{1.161016in}{0.737385in}}%
\pgfpathclose%
\pgfusepath{stroke,fill}%
\end{pgfscope}%
\begin{pgfscope}%
\pgfpathrectangle{\pgfqpoint{0.750000in}{0.500000in}}{\pgfqpoint{4.650000in}{3.020000in}}%
\pgfusepath{clip}%
\pgfsetbuttcap%
\pgfsetroundjoin%
\definecolor{currentfill}{rgb}{0.121569,0.466667,0.705882}%
\pgfsetfillcolor{currentfill}%
\pgfsetlinewidth{1.003750pt}%
\definecolor{currentstroke}{rgb}{0.121569,0.466667,0.705882}%
\pgfsetstrokecolor{currentstroke}%
\pgfsetdash{}{0pt}%
\pgfpathmoveto{\pgfqpoint{1.170818in}{0.780303in}}%
\pgfpathcurveto{\pgfqpoint{1.174008in}{0.780303in}}{\pgfqpoint{1.177068in}{0.781571in}}{\pgfqpoint{1.179324in}{0.783826in}}%
\pgfpathcurveto{\pgfqpoint{1.181579in}{0.786082in}}{\pgfqpoint{1.182847in}{0.789142in}}{\pgfqpoint{1.182847in}{0.792332in}}%
\pgfpathcurveto{\pgfqpoint{1.182847in}{0.795522in}}{\pgfqpoint{1.181579in}{0.798581in}}{\pgfqpoint{1.179324in}{0.800837in}}%
\pgfpathcurveto{\pgfqpoint{1.177068in}{0.803092in}}{\pgfqpoint{1.174008in}{0.804360in}}{\pgfqpoint{1.170818in}{0.804360in}}%
\pgfpathcurveto{\pgfqpoint{1.167629in}{0.804360in}}{\pgfqpoint{1.164569in}{0.803092in}}{\pgfqpoint{1.162313in}{0.800837in}}%
\pgfpathcurveto{\pgfqpoint{1.160058in}{0.798581in}}{\pgfqpoint{1.158790in}{0.795522in}}{\pgfqpoint{1.158790in}{0.792332in}}%
\pgfpathcurveto{\pgfqpoint{1.158790in}{0.789142in}}{\pgfqpoint{1.160058in}{0.786082in}}{\pgfqpoint{1.162313in}{0.783826in}}%
\pgfpathcurveto{\pgfqpoint{1.164569in}{0.781571in}}{\pgfqpoint{1.167629in}{0.780303in}}{\pgfqpoint{1.170818in}{0.780303in}}%
\pgfpathclose%
\pgfusepath{stroke,fill}%
\end{pgfscope}%
\begin{pgfscope}%
\pgfpathrectangle{\pgfqpoint{0.750000in}{0.500000in}}{\pgfqpoint{4.650000in}{3.020000in}}%
\pgfusepath{clip}%
\pgfsetbuttcap%
\pgfsetroundjoin%
\definecolor{currentfill}{rgb}{0.121569,0.466667,0.705882}%
\pgfsetfillcolor{currentfill}%
\pgfsetlinewidth{1.003750pt}%
\definecolor{currentstroke}{rgb}{0.121569,0.466667,0.705882}%
\pgfsetstrokecolor{currentstroke}%
\pgfsetdash{}{0pt}%
\pgfpathmoveto{\pgfqpoint{1.232210in}{0.721063in}}%
\pgfpathcurveto{\pgfqpoint{1.234815in}{0.721063in}}{\pgfqpoint{1.237313in}{0.722098in}}{\pgfqpoint{1.239155in}{0.723940in}}%
\pgfpathcurveto{\pgfqpoint{1.240997in}{0.725781in}}{\pgfqpoint{1.242031in}{0.728280in}}{\pgfqpoint{1.242031in}{0.730884in}}%
\pgfpathcurveto{\pgfqpoint{1.242031in}{0.733489in}}{\pgfqpoint{1.240997in}{0.735987in}}{\pgfqpoint{1.239155in}{0.737829in}}%
\pgfpathcurveto{\pgfqpoint{1.237313in}{0.739670in}}{\pgfqpoint{1.234815in}{0.740705in}}{\pgfqpoint{1.232210in}{0.740705in}}%
\pgfpathcurveto{\pgfqpoint{1.229606in}{0.740705in}}{\pgfqpoint{1.227108in}{0.739670in}}{\pgfqpoint{1.225266in}{0.737829in}}%
\pgfpathcurveto{\pgfqpoint{1.223424in}{0.735987in}}{\pgfqpoint{1.222389in}{0.733489in}}{\pgfqpoint{1.222389in}{0.730884in}}%
\pgfpathcurveto{\pgfqpoint{1.222389in}{0.728280in}}{\pgfqpoint{1.223424in}{0.725781in}}{\pgfqpoint{1.225266in}{0.723940in}}%
\pgfpathcurveto{\pgfqpoint{1.227108in}{0.722098in}}{\pgfqpoint{1.229606in}{0.721063in}}{\pgfqpoint{1.232210in}{0.721063in}}%
\pgfpathclose%
\pgfusepath{stroke,fill}%
\end{pgfscope}%
\begin{pgfscope}%
\pgfpathrectangle{\pgfqpoint{0.750000in}{0.500000in}}{\pgfqpoint{4.650000in}{3.020000in}}%
\pgfusepath{clip}%
\pgfsetbuttcap%
\pgfsetroundjoin%
\definecolor{currentfill}{rgb}{0.121569,0.466667,0.705882}%
\pgfsetfillcolor{currentfill}%
\pgfsetlinewidth{1.003750pt}%
\definecolor{currentstroke}{rgb}{0.121569,0.466667,0.705882}%
\pgfsetstrokecolor{currentstroke}%
\pgfsetdash{}{0pt}%
\pgfpathmoveto{\pgfqpoint{1.029462in}{0.656735in}}%
\pgfpathcurveto{\pgfqpoint{1.032067in}{0.656735in}}{\pgfqpoint{1.034565in}{0.657770in}}{\pgfqpoint{1.036407in}{0.659612in}}%
\pgfpathcurveto{\pgfqpoint{1.038248in}{0.661454in}}{\pgfqpoint{1.039283in}{0.663952in}}{\pgfqpoint{1.039283in}{0.666556in}}%
\pgfpathcurveto{\pgfqpoint{1.039283in}{0.669161in}}{\pgfqpoint{1.038248in}{0.671659in}}{\pgfqpoint{1.036407in}{0.673501in}}%
\pgfpathcurveto{\pgfqpoint{1.034565in}{0.675342in}}{\pgfqpoint{1.032067in}{0.676377in}}{\pgfqpoint{1.029462in}{0.676377in}}%
\pgfpathcurveto{\pgfqpoint{1.026858in}{0.676377in}}{\pgfqpoint{1.024359in}{0.675342in}}{\pgfqpoint{1.022518in}{0.673501in}}%
\pgfpathcurveto{\pgfqpoint{1.020676in}{0.671659in}}{\pgfqpoint{1.019641in}{0.669161in}}{\pgfqpoint{1.019641in}{0.666556in}}%
\pgfpathcurveto{\pgfqpoint{1.019641in}{0.663952in}}{\pgfqpoint{1.020676in}{0.661454in}}{\pgfqpoint{1.022518in}{0.659612in}}%
\pgfpathcurveto{\pgfqpoint{1.024359in}{0.657770in}}{\pgfqpoint{1.026858in}{0.656735in}}{\pgfqpoint{1.029462in}{0.656735in}}%
\pgfpathclose%
\pgfusepath{stroke,fill}%
\end{pgfscope}%
\begin{pgfscope}%
\pgfpathrectangle{\pgfqpoint{0.750000in}{0.500000in}}{\pgfqpoint{4.650000in}{3.020000in}}%
\pgfusepath{clip}%
\pgfsetbuttcap%
\pgfsetroundjoin%
\definecolor{currentfill}{rgb}{0.121569,0.466667,0.705882}%
\pgfsetfillcolor{currentfill}%
\pgfsetlinewidth{1.003750pt}%
\definecolor{currentstroke}{rgb}{0.121569,0.466667,0.705882}%
\pgfsetstrokecolor{currentstroke}%
\pgfsetdash{}{0pt}%
\pgfpathmoveto{\pgfqpoint{1.745529in}{2.359314in}}%
\pgfpathcurveto{\pgfqpoint{1.754168in}{2.359314in}}{\pgfqpoint{1.762453in}{2.362746in}}{\pgfqpoint{1.768562in}{2.368854in}}%
\pgfpathcurveto{\pgfqpoint{1.774670in}{2.374963in}}{\pgfqpoint{1.778102in}{2.383248in}}{\pgfqpoint{1.778102in}{2.391887in}}%
\pgfpathcurveto{\pgfqpoint{1.778102in}{2.400525in}}{\pgfqpoint{1.774670in}{2.408810in}}{\pgfqpoint{1.768562in}{2.414919in}}%
\pgfpathcurveto{\pgfqpoint{1.762453in}{2.421027in}}{\pgfqpoint{1.754168in}{2.424459in}}{\pgfqpoint{1.745529in}{2.424459in}}%
\pgfpathcurveto{\pgfqpoint{1.736891in}{2.424459in}}{\pgfqpoint{1.728606in}{2.421027in}}{\pgfqpoint{1.722497in}{2.414919in}}%
\pgfpathcurveto{\pgfqpoint{1.716389in}{2.408810in}}{\pgfqpoint{1.712957in}{2.400525in}}{\pgfqpoint{1.712957in}{2.391887in}}%
\pgfpathcurveto{\pgfqpoint{1.712957in}{2.383248in}}{\pgfqpoint{1.716389in}{2.374963in}}{\pgfqpoint{1.722497in}{2.368854in}}%
\pgfpathcurveto{\pgfqpoint{1.728606in}{2.362746in}}{\pgfqpoint{1.736891in}{2.359314in}}{\pgfqpoint{1.745529in}{2.359314in}}%
\pgfpathclose%
\pgfusepath{stroke,fill}%
\end{pgfscope}%
\begin{pgfscope}%
\pgfpathrectangle{\pgfqpoint{0.750000in}{0.500000in}}{\pgfqpoint{4.650000in}{3.020000in}}%
\pgfusepath{clip}%
\pgfsetbuttcap%
\pgfsetroundjoin%
\definecolor{currentfill}{rgb}{0.121569,0.466667,0.705882}%
\pgfsetfillcolor{currentfill}%
\pgfsetlinewidth{1.003750pt}%
\definecolor{currentstroke}{rgb}{0.121569,0.466667,0.705882}%
\pgfsetstrokecolor{currentstroke}%
\pgfsetdash{}{0pt}%
\pgfpathmoveto{\pgfqpoint{1.677947in}{2.564299in}}%
\pgfpathcurveto{\pgfqpoint{1.686585in}{2.564299in}}{\pgfqpoint{1.694871in}{2.567731in}}{\pgfqpoint{1.700979in}{2.573839in}}%
\pgfpathcurveto{\pgfqpoint{1.707087in}{2.579948in}}{\pgfqpoint{1.710519in}{2.588233in}}{\pgfqpoint{1.710519in}{2.596872in}}%
\pgfpathcurveto{\pgfqpoint{1.710519in}{2.605510in}}{\pgfqpoint{1.707087in}{2.613795in}}{\pgfqpoint{1.700979in}{2.619904in}}%
\pgfpathcurveto{\pgfqpoint{1.694871in}{2.626012in}}{\pgfqpoint{1.686585in}{2.629444in}}{\pgfqpoint{1.677947in}{2.629444in}}%
\pgfpathcurveto{\pgfqpoint{1.669308in}{2.629444in}}{\pgfqpoint{1.661023in}{2.626012in}}{\pgfqpoint{1.654915in}{2.619904in}}%
\pgfpathcurveto{\pgfqpoint{1.648806in}{2.613795in}}{\pgfqpoint{1.645374in}{2.605510in}}{\pgfqpoint{1.645374in}{2.596872in}}%
\pgfpathcurveto{\pgfqpoint{1.645374in}{2.588233in}}{\pgfqpoint{1.648806in}{2.579948in}}{\pgfqpoint{1.654915in}{2.573839in}}%
\pgfpathcurveto{\pgfqpoint{1.661023in}{2.567731in}}{\pgfqpoint{1.669308in}{2.564299in}}{\pgfqpoint{1.677947in}{2.564299in}}%
\pgfpathclose%
\pgfusepath{stroke,fill}%
\end{pgfscope}%
\begin{pgfscope}%
\pgfpathrectangle{\pgfqpoint{0.750000in}{0.500000in}}{\pgfqpoint{4.650000in}{3.020000in}}%
\pgfusepath{clip}%
\pgfsetbuttcap%
\pgfsetroundjoin%
\definecolor{currentfill}{rgb}{0.121569,0.466667,0.705882}%
\pgfsetfillcolor{currentfill}%
\pgfsetlinewidth{1.003750pt}%
\definecolor{currentstroke}{rgb}{0.121569,0.466667,0.705882}%
\pgfsetstrokecolor{currentstroke}%
\pgfsetdash{}{0pt}%
\pgfpathmoveto{\pgfqpoint{1.227051in}{0.761730in}}%
\pgfpathcurveto{\pgfqpoint{1.232875in}{0.761730in}}{\pgfqpoint{1.238462in}{0.764044in}}{\pgfqpoint{1.242580in}{0.768162in}}%
\pgfpathcurveto{\pgfqpoint{1.246698in}{0.772280in}}{\pgfqpoint{1.249012in}{0.777867in}}{\pgfqpoint{1.249012in}{0.783691in}}%
\pgfpathcurveto{\pgfqpoint{1.249012in}{0.789514in}}{\pgfqpoint{1.246698in}{0.795101in}}{\pgfqpoint{1.242580in}{0.799219in}}%
\pgfpathcurveto{\pgfqpoint{1.238462in}{0.803337in}}{\pgfqpoint{1.232875in}{0.805651in}}{\pgfqpoint{1.227051in}{0.805651in}}%
\pgfpathcurveto{\pgfqpoint{1.221227in}{0.805651in}}{\pgfqpoint{1.215641in}{0.803337in}}{\pgfqpoint{1.211523in}{0.799219in}}%
\pgfpathcurveto{\pgfqpoint{1.207405in}{0.795101in}}{\pgfqpoint{1.205091in}{0.789514in}}{\pgfqpoint{1.205091in}{0.783691in}}%
\pgfpathcurveto{\pgfqpoint{1.205091in}{0.777867in}}{\pgfqpoint{1.207405in}{0.772280in}}{\pgfqpoint{1.211523in}{0.768162in}}%
\pgfpathcurveto{\pgfqpoint{1.215641in}{0.764044in}}{\pgfqpoint{1.221227in}{0.761730in}}{\pgfqpoint{1.227051in}{0.761730in}}%
\pgfpathclose%
\pgfusepath{stroke,fill}%
\end{pgfscope}%
\begin{pgfscope}%
\pgfpathrectangle{\pgfqpoint{0.750000in}{0.500000in}}{\pgfqpoint{4.650000in}{3.020000in}}%
\pgfusepath{clip}%
\pgfsetbuttcap%
\pgfsetroundjoin%
\definecolor{currentfill}{rgb}{0.121569,0.466667,0.705882}%
\pgfsetfillcolor{currentfill}%
\pgfsetlinewidth{1.003750pt}%
\definecolor{currentstroke}{rgb}{0.121569,0.466667,0.705882}%
\pgfsetstrokecolor{currentstroke}%
\pgfsetdash{}{0pt}%
\pgfpathmoveto{\pgfqpoint{1.064027in}{0.778863in}}%
\pgfpathcurveto{\pgfqpoint{1.067217in}{0.778863in}}{\pgfqpoint{1.070277in}{0.780131in}}{\pgfqpoint{1.072533in}{0.782386in}}%
\pgfpathcurveto{\pgfqpoint{1.074788in}{0.784642in}}{\pgfqpoint{1.076056in}{0.787702in}}{\pgfqpoint{1.076056in}{0.790891in}}%
\pgfpathcurveto{\pgfqpoint{1.076056in}{0.794081in}}{\pgfqpoint{1.074788in}{0.797141in}}{\pgfqpoint{1.072533in}{0.799397in}}%
\pgfpathcurveto{\pgfqpoint{1.070277in}{0.801652in}}{\pgfqpoint{1.067217in}{0.802920in}}{\pgfqpoint{1.064027in}{0.802920in}}%
\pgfpathcurveto{\pgfqpoint{1.060838in}{0.802920in}}{\pgfqpoint{1.057778in}{0.801652in}}{\pgfqpoint{1.055522in}{0.799397in}}%
\pgfpathcurveto{\pgfqpoint{1.053267in}{0.797141in}}{\pgfqpoint{1.051999in}{0.794081in}}{\pgfqpoint{1.051999in}{0.790891in}}%
\pgfpathcurveto{\pgfqpoint{1.051999in}{0.787702in}}{\pgfqpoint{1.053267in}{0.784642in}}{\pgfqpoint{1.055522in}{0.782386in}}%
\pgfpathcurveto{\pgfqpoint{1.057778in}{0.780131in}}{\pgfqpoint{1.060838in}{0.778863in}}{\pgfqpoint{1.064027in}{0.778863in}}%
\pgfpathclose%
\pgfusepath{stroke,fill}%
\end{pgfscope}%
\begin{pgfscope}%
\pgfpathrectangle{\pgfqpoint{0.750000in}{0.500000in}}{\pgfqpoint{4.650000in}{3.020000in}}%
\pgfusepath{clip}%
\pgfsetbuttcap%
\pgfsetroundjoin%
\definecolor{currentfill}{rgb}{0.121569,0.466667,0.705882}%
\pgfsetfillcolor{currentfill}%
\pgfsetlinewidth{1.003750pt}%
\definecolor{currentstroke}{rgb}{0.121569,0.466667,0.705882}%
\pgfsetstrokecolor{currentstroke}%
\pgfsetdash{}{0pt}%
\pgfpathmoveto{\pgfqpoint{1.284316in}{0.858073in}}%
\pgfpathcurveto{\pgfqpoint{1.292344in}{0.858073in}}{\pgfqpoint{1.300044in}{0.861263in}}{\pgfqpoint{1.305720in}{0.866939in}}%
\pgfpathcurveto{\pgfqpoint{1.311397in}{0.872616in}}{\pgfqpoint{1.314586in}{0.880316in}}{\pgfqpoint{1.314586in}{0.888343in}}%
\pgfpathcurveto{\pgfqpoint{1.314586in}{0.896371in}}{\pgfqpoint{1.311397in}{0.904071in}}{\pgfqpoint{1.305720in}{0.909748in}}%
\pgfpathcurveto{\pgfqpoint{1.300044in}{0.915424in}}{\pgfqpoint{1.292344in}{0.918613in}}{\pgfqpoint{1.284316in}{0.918613in}}%
\pgfpathcurveto{\pgfqpoint{1.276288in}{0.918613in}}{\pgfqpoint{1.268588in}{0.915424in}}{\pgfqpoint{1.262912in}{0.909748in}}%
\pgfpathcurveto{\pgfqpoint{1.257235in}{0.904071in}}{\pgfqpoint{1.254046in}{0.896371in}}{\pgfqpoint{1.254046in}{0.888343in}}%
\pgfpathcurveto{\pgfqpoint{1.254046in}{0.880316in}}{\pgfqpoint{1.257235in}{0.872616in}}{\pgfqpoint{1.262912in}{0.866939in}}%
\pgfpathcurveto{\pgfqpoint{1.268588in}{0.861263in}}{\pgfqpoint{1.276288in}{0.858073in}}{\pgfqpoint{1.284316in}{0.858073in}}%
\pgfpathclose%
\pgfusepath{stroke,fill}%
\end{pgfscope}%
\begin{pgfscope}%
\pgfpathrectangle{\pgfqpoint{0.750000in}{0.500000in}}{\pgfqpoint{4.650000in}{3.020000in}}%
\pgfusepath{clip}%
\pgfsetbuttcap%
\pgfsetroundjoin%
\definecolor{currentfill}{rgb}{0.121569,0.466667,0.705882}%
\pgfsetfillcolor{currentfill}%
\pgfsetlinewidth{1.003750pt}%
\definecolor{currentstroke}{rgb}{0.121569,0.466667,0.705882}%
\pgfsetstrokecolor{currentstroke}%
\pgfsetdash{}{0pt}%
\pgfpathmoveto{\pgfqpoint{1.560322in}{0.812947in}}%
\pgfpathcurveto{\pgfqpoint{1.563512in}{0.812947in}}{\pgfqpoint{1.566571in}{0.814215in}}{\pgfqpoint{1.568827in}{0.816470in}}%
\pgfpathcurveto{\pgfqpoint{1.571083in}{0.818726in}}{\pgfqpoint{1.572350in}{0.821786in}}{\pgfqpoint{1.572350in}{0.824976in}}%
\pgfpathcurveto{\pgfqpoint{1.572350in}{0.828165in}}{\pgfqpoint{1.571083in}{0.831225in}}{\pgfqpoint{1.568827in}{0.833481in}}%
\pgfpathcurveto{\pgfqpoint{1.566571in}{0.835736in}}{\pgfqpoint{1.563512in}{0.837004in}}{\pgfqpoint{1.560322in}{0.837004in}}%
\pgfpathcurveto{\pgfqpoint{1.557132in}{0.837004in}}{\pgfqpoint{1.554072in}{0.835736in}}{\pgfqpoint{1.551817in}{0.833481in}}%
\pgfpathcurveto{\pgfqpoint{1.549561in}{0.831225in}}{\pgfqpoint{1.548294in}{0.828165in}}{\pgfqpoint{1.548294in}{0.824976in}}%
\pgfpathcurveto{\pgfqpoint{1.548294in}{0.821786in}}{\pgfqpoint{1.549561in}{0.818726in}}{\pgfqpoint{1.551817in}{0.816470in}}%
\pgfpathcurveto{\pgfqpoint{1.554072in}{0.814215in}}{\pgfqpoint{1.557132in}{0.812947in}}{\pgfqpoint{1.560322in}{0.812947in}}%
\pgfpathclose%
\pgfusepath{stroke,fill}%
\end{pgfscope}%
\begin{pgfscope}%
\pgfpathrectangle{\pgfqpoint{0.750000in}{0.500000in}}{\pgfqpoint{4.650000in}{3.020000in}}%
\pgfusepath{clip}%
\pgfsetbuttcap%
\pgfsetroundjoin%
\definecolor{currentfill}{rgb}{0.121569,0.466667,0.705882}%
\pgfsetfillcolor{currentfill}%
\pgfsetlinewidth{1.003750pt}%
\definecolor{currentstroke}{rgb}{0.121569,0.466667,0.705882}%
\pgfsetstrokecolor{currentstroke}%
\pgfsetdash{}{0pt}%
\pgfpathmoveto{\pgfqpoint{1.447856in}{1.101789in}}%
\pgfpathcurveto{\pgfqpoint{1.456688in}{1.101789in}}{\pgfqpoint{1.465160in}{1.105298in}}{\pgfqpoint{1.471406in}{1.111544in}}%
\pgfpathcurveto{\pgfqpoint{1.477651in}{1.117789in}}{\pgfqpoint{1.481160in}{1.126261in}}{\pgfqpoint{1.481160in}{1.135093in}}%
\pgfpathcurveto{\pgfqpoint{1.481160in}{1.143926in}}{\pgfqpoint{1.477651in}{1.152398in}}{\pgfqpoint{1.471406in}{1.158643in}}%
\pgfpathcurveto{\pgfqpoint{1.465160in}{1.164889in}}{\pgfqpoint{1.456688in}{1.168398in}}{\pgfqpoint{1.447856in}{1.168398in}}%
\pgfpathcurveto{\pgfqpoint{1.439024in}{1.168398in}}{\pgfqpoint{1.430552in}{1.164889in}}{\pgfqpoint{1.424306in}{1.158643in}}%
\pgfpathcurveto{\pgfqpoint{1.418061in}{1.152398in}}{\pgfqpoint{1.414552in}{1.143926in}}{\pgfqpoint{1.414552in}{1.135093in}}%
\pgfpathcurveto{\pgfqpoint{1.414552in}{1.126261in}}{\pgfqpoint{1.418061in}{1.117789in}}{\pgfqpoint{1.424306in}{1.111544in}}%
\pgfpathcurveto{\pgfqpoint{1.430552in}{1.105298in}}{\pgfqpoint{1.439024in}{1.101789in}}{\pgfqpoint{1.447856in}{1.101789in}}%
\pgfpathclose%
\pgfusepath{stroke,fill}%
\end{pgfscope}%
\begin{pgfscope}%
\pgfpathrectangle{\pgfqpoint{0.750000in}{0.500000in}}{\pgfqpoint{4.650000in}{3.020000in}}%
\pgfusepath{clip}%
\pgfsetbuttcap%
\pgfsetroundjoin%
\definecolor{currentfill}{rgb}{0.121569,0.466667,0.705882}%
\pgfsetfillcolor{currentfill}%
\pgfsetlinewidth{1.003750pt}%
\definecolor{currentstroke}{rgb}{0.121569,0.466667,0.705882}%
\pgfsetstrokecolor{currentstroke}%
\pgfsetdash{}{0pt}%
\pgfpathmoveto{\pgfqpoint{0.990770in}{0.642843in}}%
\pgfpathcurveto{\pgfqpoint{0.996295in}{0.642843in}}{\pgfqpoint{1.001594in}{0.645038in}}{\pgfqpoint{1.005501in}{0.648945in}}%
\pgfpathcurveto{\pgfqpoint{1.009408in}{0.652851in}}{\pgfqpoint{1.011603in}{0.658151in}}{\pgfqpoint{1.011603in}{0.663676in}}%
\pgfpathcurveto{\pgfqpoint{1.011603in}{0.669201in}}{\pgfqpoint{1.009408in}{0.674501in}}{\pgfqpoint{1.005501in}{0.678407in}}%
\pgfpathcurveto{\pgfqpoint{1.001594in}{0.682314in}}{\pgfqpoint{0.996295in}{0.684509in}}{\pgfqpoint{0.990770in}{0.684509in}}%
\pgfpathcurveto{\pgfqpoint{0.985245in}{0.684509in}}{\pgfqpoint{0.979945in}{0.682314in}}{\pgfqpoint{0.976038in}{0.678407in}}%
\pgfpathcurveto{\pgfqpoint{0.972132in}{0.674501in}}{\pgfqpoint{0.969937in}{0.669201in}}{\pgfqpoint{0.969937in}{0.663676in}}%
\pgfpathcurveto{\pgfqpoint{0.969937in}{0.658151in}}{\pgfqpoint{0.972132in}{0.652851in}}{\pgfqpoint{0.976038in}{0.648945in}}%
\pgfpathcurveto{\pgfqpoint{0.979945in}{0.645038in}}{\pgfqpoint{0.985245in}{0.642843in}}{\pgfqpoint{0.990770in}{0.642843in}}%
\pgfpathclose%
\pgfusepath{stroke,fill}%
\end{pgfscope}%
\begin{pgfscope}%
\pgfpathrectangle{\pgfqpoint{0.750000in}{0.500000in}}{\pgfqpoint{4.650000in}{3.020000in}}%
\pgfusepath{clip}%
\pgfsetbuttcap%
\pgfsetroundjoin%
\definecolor{currentfill}{rgb}{0.121569,0.466667,0.705882}%
\pgfsetfillcolor{currentfill}%
\pgfsetlinewidth{1.003750pt}%
\definecolor{currentstroke}{rgb}{0.121569,0.466667,0.705882}%
\pgfsetstrokecolor{currentstroke}%
\pgfsetdash{}{0pt}%
\pgfpathmoveto{\pgfqpoint{1.321977in}{0.810688in}}%
\pgfpathcurveto{\pgfqpoint{1.331368in}{0.810688in}}{\pgfqpoint{1.340375in}{0.814419in}}{\pgfqpoint{1.347015in}{0.821060in}}%
\pgfpathcurveto{\pgfqpoint{1.353656in}{0.827700in}}{\pgfqpoint{1.357387in}{0.836707in}}{\pgfqpoint{1.357387in}{0.846098in}}%
\pgfpathcurveto{\pgfqpoint{1.357387in}{0.855489in}}{\pgfqpoint{1.353656in}{0.864496in}}{\pgfqpoint{1.347015in}{0.871137in}}%
\pgfpathcurveto{\pgfqpoint{1.340375in}{0.877777in}}{\pgfqpoint{1.331368in}{0.881508in}}{\pgfqpoint{1.321977in}{0.881508in}}%
\pgfpathcurveto{\pgfqpoint{1.312586in}{0.881508in}}{\pgfqpoint{1.303578in}{0.877777in}}{\pgfqpoint{1.296938in}{0.871137in}}%
\pgfpathcurveto{\pgfqpoint{1.290298in}{0.864496in}}{\pgfqpoint{1.286567in}{0.855489in}}{\pgfqpoint{1.286567in}{0.846098in}}%
\pgfpathcurveto{\pgfqpoint{1.286567in}{0.836707in}}{\pgfqpoint{1.290298in}{0.827700in}}{\pgfqpoint{1.296938in}{0.821060in}}%
\pgfpathcurveto{\pgfqpoint{1.303578in}{0.814419in}}{\pgfqpoint{1.312586in}{0.810688in}}{\pgfqpoint{1.321977in}{0.810688in}}%
\pgfpathclose%
\pgfusepath{stroke,fill}%
\end{pgfscope}%
\begin{pgfscope}%
\pgfpathrectangle{\pgfqpoint{0.750000in}{0.500000in}}{\pgfqpoint{4.650000in}{3.020000in}}%
\pgfusepath{clip}%
\pgfsetbuttcap%
\pgfsetroundjoin%
\definecolor{currentfill}{rgb}{0.121569,0.466667,0.705882}%
\pgfsetfillcolor{currentfill}%
\pgfsetlinewidth{1.003750pt}%
\definecolor{currentstroke}{rgb}{0.121569,0.466667,0.705882}%
\pgfsetstrokecolor{currentstroke}%
\pgfsetdash{}{0pt}%
\pgfpathmoveto{\pgfqpoint{1.650088in}{1.025099in}}%
\pgfpathcurveto{\pgfqpoint{1.656979in}{1.025099in}}{\pgfqpoint{1.663589in}{1.027837in}}{\pgfqpoint{1.668461in}{1.032710in}}%
\pgfpathcurveto{\pgfqpoint{1.673334in}{1.037583in}}{\pgfqpoint{1.676072in}{1.044192in}}{\pgfqpoint{1.676072in}{1.051083in}}%
\pgfpathcurveto{\pgfqpoint{1.676072in}{1.057974in}}{\pgfqpoint{1.673334in}{1.064584in}}{\pgfqpoint{1.668461in}{1.069456in}}%
\pgfpathcurveto{\pgfqpoint{1.663589in}{1.074329in}}{\pgfqpoint{1.656979in}{1.077067in}}{\pgfqpoint{1.650088in}{1.077067in}}%
\pgfpathcurveto{\pgfqpoint{1.643197in}{1.077067in}}{\pgfqpoint{1.636588in}{1.074329in}}{\pgfqpoint{1.631715in}{1.069456in}}%
\pgfpathcurveto{\pgfqpoint{1.626842in}{1.064584in}}{\pgfqpoint{1.624104in}{1.057974in}}{\pgfqpoint{1.624104in}{1.051083in}}%
\pgfpathcurveto{\pgfqpoint{1.624104in}{1.044192in}}{\pgfqpoint{1.626842in}{1.037583in}}{\pgfqpoint{1.631715in}{1.032710in}}%
\pgfpathcurveto{\pgfqpoint{1.636588in}{1.027837in}}{\pgfqpoint{1.643197in}{1.025099in}}{\pgfqpoint{1.650088in}{1.025099in}}%
\pgfpathclose%
\pgfusepath{stroke,fill}%
\end{pgfscope}%
\begin{pgfscope}%
\pgfpathrectangle{\pgfqpoint{0.750000in}{0.500000in}}{\pgfqpoint{4.650000in}{3.020000in}}%
\pgfusepath{clip}%
\pgfsetbuttcap%
\pgfsetroundjoin%
\definecolor{currentfill}{rgb}{0.121569,0.466667,0.705882}%
\pgfsetfillcolor{currentfill}%
\pgfsetlinewidth{1.003750pt}%
\definecolor{currentstroke}{rgb}{0.121569,0.466667,0.705882}%
\pgfsetstrokecolor{currentstroke}%
\pgfsetdash{}{0pt}%
\pgfpathmoveto{\pgfqpoint{1.094465in}{0.653534in}}%
\pgfpathcurveto{\pgfqpoint{1.101356in}{0.653534in}}{\pgfqpoint{1.107966in}{0.656272in}}{\pgfqpoint{1.112839in}{0.661145in}}%
\pgfpathcurveto{\pgfqpoint{1.117711in}{0.666017in}}{\pgfqpoint{1.120449in}{0.672627in}}{\pgfqpoint{1.120449in}{0.679518in}}%
\pgfpathcurveto{\pgfqpoint{1.120449in}{0.686409in}}{\pgfqpoint{1.117711in}{0.693018in}}{\pgfqpoint{1.112839in}{0.697891in}}%
\pgfpathcurveto{\pgfqpoint{1.107966in}{0.702764in}}{\pgfqpoint{1.101356in}{0.705502in}}{\pgfqpoint{1.094465in}{0.705502in}}%
\pgfpathcurveto{\pgfqpoint{1.087574in}{0.705502in}}{\pgfqpoint{1.080965in}{0.702764in}}{\pgfqpoint{1.076092in}{0.697891in}}%
\pgfpathcurveto{\pgfqpoint{1.071220in}{0.693018in}}{\pgfqpoint{1.068482in}{0.686409in}}{\pgfqpoint{1.068482in}{0.679518in}}%
\pgfpathcurveto{\pgfqpoint{1.068482in}{0.672627in}}{\pgfqpoint{1.071220in}{0.666017in}}{\pgfqpoint{1.076092in}{0.661145in}}%
\pgfpathcurveto{\pgfqpoint{1.080965in}{0.656272in}}{\pgfqpoint{1.087574in}{0.653534in}}{\pgfqpoint{1.094465in}{0.653534in}}%
\pgfpathclose%
\pgfusepath{stroke,fill}%
\end{pgfscope}%
\begin{pgfscope}%
\pgfpathrectangle{\pgfqpoint{0.750000in}{0.500000in}}{\pgfqpoint{4.650000in}{3.020000in}}%
\pgfusepath{clip}%
\pgfsetbuttcap%
\pgfsetroundjoin%
\definecolor{currentfill}{rgb}{0.121569,0.466667,0.705882}%
\pgfsetfillcolor{currentfill}%
\pgfsetlinewidth{1.003750pt}%
\definecolor{currentstroke}{rgb}{0.121569,0.466667,0.705882}%
\pgfsetstrokecolor{currentstroke}%
\pgfsetdash{}{0pt}%
\pgfpathmoveto{\pgfqpoint{1.691876in}{0.998730in}}%
\pgfpathcurveto{\pgfqpoint{1.699904in}{0.998730in}}{\pgfqpoint{1.707604in}{1.001920in}}{\pgfqpoint{1.713280in}{1.007596in}}%
\pgfpathcurveto{\pgfqpoint{1.718957in}{1.013273in}}{\pgfqpoint{1.722146in}{1.020973in}}{\pgfqpoint{1.722146in}{1.029000in}}%
\pgfpathcurveto{\pgfqpoint{1.722146in}{1.037028in}}{\pgfqpoint{1.718957in}{1.044728in}}{\pgfqpoint{1.713280in}{1.050405in}}%
\pgfpathcurveto{\pgfqpoint{1.707604in}{1.056081in}}{\pgfqpoint{1.699904in}{1.059271in}}{\pgfqpoint{1.691876in}{1.059271in}}%
\pgfpathcurveto{\pgfqpoint{1.683848in}{1.059271in}}{\pgfqpoint{1.676148in}{1.056081in}}{\pgfqpoint{1.670472in}{1.050405in}}%
\pgfpathcurveto{\pgfqpoint{1.664795in}{1.044728in}}{\pgfqpoint{1.661606in}{1.037028in}}{\pgfqpoint{1.661606in}{1.029000in}}%
\pgfpathcurveto{\pgfqpoint{1.661606in}{1.020973in}}{\pgfqpoint{1.664795in}{1.013273in}}{\pgfqpoint{1.670472in}{1.007596in}}%
\pgfpathcurveto{\pgfqpoint{1.676148in}{1.001920in}}{\pgfqpoint{1.683848in}{0.998730in}}{\pgfqpoint{1.691876in}{0.998730in}}%
\pgfpathclose%
\pgfusepath{stroke,fill}%
\end{pgfscope}%
\begin{pgfscope}%
\pgfpathrectangle{\pgfqpoint{0.750000in}{0.500000in}}{\pgfqpoint{4.650000in}{3.020000in}}%
\pgfusepath{clip}%
\pgfsetbuttcap%
\pgfsetroundjoin%
\definecolor{currentfill}{rgb}{0.121569,0.466667,0.705882}%
\pgfsetfillcolor{currentfill}%
\pgfsetlinewidth{1.003750pt}%
\definecolor{currentstroke}{rgb}{0.121569,0.466667,0.705882}%
\pgfsetstrokecolor{currentstroke}%
\pgfsetdash{}{0pt}%
\pgfpathmoveto{\pgfqpoint{1.122840in}{0.852312in}}%
\pgfpathcurveto{\pgfqpoint{1.126030in}{0.852312in}}{\pgfqpoint{1.129089in}{0.853580in}}{\pgfqpoint{1.131345in}{0.855835in}}%
\pgfpathcurveto{\pgfqpoint{1.133601in}{0.858091in}}{\pgfqpoint{1.134868in}{0.861150in}}{\pgfqpoint{1.134868in}{0.864340in}}%
\pgfpathcurveto{\pgfqpoint{1.134868in}{0.867530in}}{\pgfqpoint{1.133601in}{0.870590in}}{\pgfqpoint{1.131345in}{0.872846in}}%
\pgfpathcurveto{\pgfqpoint{1.129089in}{0.875101in}}{\pgfqpoint{1.126030in}{0.876369in}}{\pgfqpoint{1.122840in}{0.876369in}}%
\pgfpathcurveto{\pgfqpoint{1.119650in}{0.876369in}}{\pgfqpoint{1.116590in}{0.875101in}}{\pgfqpoint{1.114335in}{0.872846in}}%
\pgfpathcurveto{\pgfqpoint{1.112079in}{0.870590in}}{\pgfqpoint{1.110812in}{0.867530in}}{\pgfqpoint{1.110812in}{0.864340in}}%
\pgfpathcurveto{\pgfqpoint{1.110812in}{0.861150in}}{\pgfqpoint{1.112079in}{0.858091in}}{\pgfqpoint{1.114335in}{0.855835in}}%
\pgfpathcurveto{\pgfqpoint{1.116590in}{0.853580in}}{\pgfqpoint{1.119650in}{0.852312in}}{\pgfqpoint{1.122840in}{0.852312in}}%
\pgfpathclose%
\pgfusepath{stroke,fill}%
\end{pgfscope}%
\begin{pgfscope}%
\pgfpathrectangle{\pgfqpoint{0.750000in}{0.500000in}}{\pgfqpoint{4.650000in}{3.020000in}}%
\pgfusepath{clip}%
\pgfsetbuttcap%
\pgfsetroundjoin%
\definecolor{currentfill}{rgb}{0.121569,0.466667,0.705882}%
\pgfsetfillcolor{currentfill}%
\pgfsetlinewidth{1.003750pt}%
\definecolor{currentstroke}{rgb}{0.121569,0.466667,0.705882}%
\pgfsetstrokecolor{currentstroke}%
\pgfsetdash{}{0pt}%
\pgfpathmoveto{\pgfqpoint{1.275030in}{1.537356in}}%
\pgfpathcurveto{\pgfqpoint{1.278220in}{1.537356in}}{\pgfqpoint{1.281280in}{1.538623in}}{\pgfqpoint{1.283535in}{1.540879in}}%
\pgfpathcurveto{\pgfqpoint{1.285791in}{1.543134in}}{\pgfqpoint{1.287058in}{1.546194in}}{\pgfqpoint{1.287058in}{1.549384in}}%
\pgfpathcurveto{\pgfqpoint{1.287058in}{1.552574in}}{\pgfqpoint{1.285791in}{1.555633in}}{\pgfqpoint{1.283535in}{1.557889in}}%
\pgfpathcurveto{\pgfqpoint{1.281280in}{1.560145in}}{\pgfqpoint{1.278220in}{1.561412in}}{\pgfqpoint{1.275030in}{1.561412in}}%
\pgfpathcurveto{\pgfqpoint{1.271840in}{1.561412in}}{\pgfqpoint{1.268780in}{1.560145in}}{\pgfqpoint{1.266525in}{1.557889in}}%
\pgfpathcurveto{\pgfqpoint{1.264269in}{1.555633in}}{\pgfqpoint{1.263002in}{1.552574in}}{\pgfqpoint{1.263002in}{1.549384in}}%
\pgfpathcurveto{\pgfqpoint{1.263002in}{1.546194in}}{\pgfqpoint{1.264269in}{1.543134in}}{\pgfqpoint{1.266525in}{1.540879in}}%
\pgfpathcurveto{\pgfqpoint{1.268780in}{1.538623in}}{\pgfqpoint{1.271840in}{1.537356in}}{\pgfqpoint{1.275030in}{1.537356in}}%
\pgfpathclose%
\pgfusepath{stroke,fill}%
\end{pgfscope}%
\begin{pgfscope}%
\pgfpathrectangle{\pgfqpoint{0.750000in}{0.500000in}}{\pgfqpoint{4.650000in}{3.020000in}}%
\pgfusepath{clip}%
\pgfsetbuttcap%
\pgfsetroundjoin%
\definecolor{currentfill}{rgb}{0.121569,0.466667,0.705882}%
\pgfsetfillcolor{currentfill}%
\pgfsetlinewidth{1.003750pt}%
\definecolor{currentstroke}{rgb}{0.121569,0.466667,0.705882}%
\pgfsetstrokecolor{currentstroke}%
\pgfsetdash{}{0pt}%
\pgfpathmoveto{\pgfqpoint{1.010374in}{1.157916in}}%
\pgfpathcurveto{\pgfqpoint{1.012979in}{1.157916in}}{\pgfqpoint{1.015477in}{1.158951in}}{\pgfqpoint{1.017318in}{1.160793in}}%
\pgfpathcurveto{\pgfqpoint{1.019160in}{1.162635in}}{\pgfqpoint{1.020195in}{1.165133in}}{\pgfqpoint{1.020195in}{1.167737in}}%
\pgfpathcurveto{\pgfqpoint{1.020195in}{1.170342in}}{\pgfqpoint{1.019160in}{1.172840in}}{\pgfqpoint{1.017318in}{1.174682in}}%
\pgfpathcurveto{\pgfqpoint{1.015477in}{1.176523in}}{\pgfqpoint{1.012979in}{1.177558in}}{\pgfqpoint{1.010374in}{1.177558in}}%
\pgfpathcurveto{\pgfqpoint{1.007769in}{1.177558in}}{\pgfqpoint{1.005271in}{1.176523in}}{\pgfqpoint{1.003430in}{1.174682in}}%
\pgfpathcurveto{\pgfqpoint{1.001588in}{1.172840in}}{\pgfqpoint{1.000553in}{1.170342in}}{\pgfqpoint{1.000553in}{1.167737in}}%
\pgfpathcurveto{\pgfqpoint{1.000553in}{1.165133in}}{\pgfqpoint{1.001588in}{1.162635in}}{\pgfqpoint{1.003430in}{1.160793in}}%
\pgfpathcurveto{\pgfqpoint{1.005271in}{1.158951in}}{\pgfqpoint{1.007769in}{1.157916in}}{\pgfqpoint{1.010374in}{1.157916in}}%
\pgfpathclose%
\pgfusepath{stroke,fill}%
\end{pgfscope}%
\begin{pgfscope}%
\pgfpathrectangle{\pgfqpoint{0.750000in}{0.500000in}}{\pgfqpoint{4.650000in}{3.020000in}}%
\pgfusepath{clip}%
\pgfsetbuttcap%
\pgfsetroundjoin%
\definecolor{currentfill}{rgb}{0.121569,0.466667,0.705882}%
\pgfsetfillcolor{currentfill}%
\pgfsetlinewidth{1.003750pt}%
\definecolor{currentstroke}{rgb}{0.121569,0.466667,0.705882}%
\pgfsetstrokecolor{currentstroke}%
\pgfsetdash{}{0pt}%
\pgfpathmoveto{\pgfqpoint{1.023271in}{0.678051in}}%
\pgfpathcurveto{\pgfqpoint{1.026461in}{0.678051in}}{\pgfqpoint{1.029521in}{0.679318in}}{\pgfqpoint{1.031777in}{0.681574in}}%
\pgfpathcurveto{\pgfqpoint{1.034032in}{0.683830in}}{\pgfqpoint{1.035300in}{0.686889in}}{\pgfqpoint{1.035300in}{0.690079in}}%
\pgfpathcurveto{\pgfqpoint{1.035300in}{0.693269in}}{\pgfqpoint{1.034032in}{0.696329in}}{\pgfqpoint{1.031777in}{0.698584in}}%
\pgfpathcurveto{\pgfqpoint{1.029521in}{0.700840in}}{\pgfqpoint{1.026461in}{0.702107in}}{\pgfqpoint{1.023271in}{0.702107in}}%
\pgfpathcurveto{\pgfqpoint{1.020082in}{0.702107in}}{\pgfqpoint{1.017022in}{0.700840in}}{\pgfqpoint{1.014766in}{0.698584in}}%
\pgfpathcurveto{\pgfqpoint{1.012511in}{0.696329in}}{\pgfqpoint{1.011243in}{0.693269in}}{\pgfqpoint{1.011243in}{0.690079in}}%
\pgfpathcurveto{\pgfqpoint{1.011243in}{0.686889in}}{\pgfqpoint{1.012511in}{0.683830in}}{\pgfqpoint{1.014766in}{0.681574in}}%
\pgfpathcurveto{\pgfqpoint{1.017022in}{0.679318in}}{\pgfqpoint{1.020082in}{0.678051in}}{\pgfqpoint{1.023271in}{0.678051in}}%
\pgfpathclose%
\pgfusepath{stroke,fill}%
\end{pgfscope}%
\begin{pgfscope}%
\pgfpathrectangle{\pgfqpoint{0.750000in}{0.500000in}}{\pgfqpoint{4.650000in}{3.020000in}}%
\pgfusepath{clip}%
\pgfsetbuttcap%
\pgfsetroundjoin%
\definecolor{currentfill}{rgb}{0.121569,0.466667,0.705882}%
\pgfsetfillcolor{currentfill}%
\pgfsetlinewidth{1.003750pt}%
\definecolor{currentstroke}{rgb}{0.121569,0.466667,0.705882}%
\pgfsetstrokecolor{currentstroke}%
\pgfsetdash{}{0pt}%
\pgfpathmoveto{\pgfqpoint{1.125935in}{1.577499in}}%
\pgfpathcurveto{\pgfqpoint{1.130446in}{1.577499in}}{\pgfqpoint{1.134774in}{1.579291in}}{\pgfqpoint{1.137963in}{1.582481in}}%
\pgfpathcurveto{\pgfqpoint{1.141153in}{1.585671in}}{\pgfqpoint{1.142946in}{1.589998in}}{\pgfqpoint{1.142946in}{1.594509in}}%
\pgfpathcurveto{\pgfqpoint{1.142946in}{1.599021in}}{\pgfqpoint{1.141153in}{1.603348in}}{\pgfqpoint{1.137963in}{1.606538in}}%
\pgfpathcurveto{\pgfqpoint{1.134774in}{1.609727in}}{\pgfqpoint{1.130446in}{1.611520in}}{\pgfqpoint{1.125935in}{1.611520in}}%
\pgfpathcurveto{\pgfqpoint{1.121424in}{1.611520in}}{\pgfqpoint{1.117097in}{1.609727in}}{\pgfqpoint{1.113907in}{1.606538in}}%
\pgfpathcurveto{\pgfqpoint{1.110717in}{1.603348in}}{\pgfqpoint{1.108925in}{1.599021in}}{\pgfqpoint{1.108925in}{1.594509in}}%
\pgfpathcurveto{\pgfqpoint{1.108925in}{1.589998in}}{\pgfqpoint{1.110717in}{1.585671in}}{\pgfqpoint{1.113907in}{1.582481in}}%
\pgfpathcurveto{\pgfqpoint{1.117097in}{1.579291in}}{\pgfqpoint{1.121424in}{1.577499in}}{\pgfqpoint{1.125935in}{1.577499in}}%
\pgfpathclose%
\pgfusepath{stroke,fill}%
\end{pgfscope}%
\begin{pgfscope}%
\pgfpathrectangle{\pgfqpoint{0.750000in}{0.500000in}}{\pgfqpoint{4.650000in}{3.020000in}}%
\pgfusepath{clip}%
\pgfsetbuttcap%
\pgfsetroundjoin%
\definecolor{currentfill}{rgb}{0.121569,0.466667,0.705882}%
\pgfsetfillcolor{currentfill}%
\pgfsetlinewidth{1.003750pt}%
\definecolor{currentstroke}{rgb}{0.121569,0.466667,0.705882}%
\pgfsetstrokecolor{currentstroke}%
\pgfsetdash{}{0pt}%
\pgfpathmoveto{\pgfqpoint{1.138833in}{0.874255in}}%
\pgfpathcurveto{\pgfqpoint{1.142951in}{0.874255in}}{\pgfqpoint{1.146901in}{0.875891in}}{\pgfqpoint{1.149813in}{0.878803in}}%
\pgfpathcurveto{\pgfqpoint{1.152725in}{0.881715in}}{\pgfqpoint{1.154361in}{0.885665in}}{\pgfqpoint{1.154361in}{0.889783in}}%
\pgfpathcurveto{\pgfqpoint{1.154361in}{0.893902in}}{\pgfqpoint{1.152725in}{0.897852in}}{\pgfqpoint{1.149813in}{0.900764in}}%
\pgfpathcurveto{\pgfqpoint{1.146901in}{0.903676in}}{\pgfqpoint{1.142951in}{0.905312in}}{\pgfqpoint{1.138833in}{0.905312in}}%
\pgfpathcurveto{\pgfqpoint{1.134715in}{0.905312in}}{\pgfqpoint{1.130765in}{0.903676in}}{\pgfqpoint{1.127853in}{0.900764in}}%
\pgfpathcurveto{\pgfqpoint{1.124941in}{0.897852in}}{\pgfqpoint{1.123304in}{0.893902in}}{\pgfqpoint{1.123304in}{0.889783in}}%
\pgfpathcurveto{\pgfqpoint{1.123304in}{0.885665in}}{\pgfqpoint{1.124941in}{0.881715in}}{\pgfqpoint{1.127853in}{0.878803in}}%
\pgfpathcurveto{\pgfqpoint{1.130765in}{0.875891in}}{\pgfqpoint{1.134715in}{0.874255in}}{\pgfqpoint{1.138833in}{0.874255in}}%
\pgfpathclose%
\pgfusepath{stroke,fill}%
\end{pgfscope}%
\begin{pgfscope}%
\pgfpathrectangle{\pgfqpoint{0.750000in}{0.500000in}}{\pgfqpoint{4.650000in}{3.020000in}}%
\pgfusepath{clip}%
\pgfsetbuttcap%
\pgfsetroundjoin%
\definecolor{currentfill}{rgb}{0.121569,0.466667,0.705882}%
\pgfsetfillcolor{currentfill}%
\pgfsetlinewidth{1.003750pt}%
\definecolor{currentstroke}{rgb}{0.121569,0.466667,0.705882}%
\pgfsetstrokecolor{currentstroke}%
\pgfsetdash{}{0pt}%
\pgfpathmoveto{\pgfqpoint{1.036169in}{0.668737in}}%
\pgfpathcurveto{\pgfqpoint{1.038773in}{0.668737in}}{\pgfqpoint{1.041272in}{0.669772in}}{\pgfqpoint{1.043113in}{0.671613in}}%
\pgfpathcurveto{\pgfqpoint{1.044955in}{0.673455in}}{\pgfqpoint{1.045990in}{0.675953in}}{\pgfqpoint{1.045990in}{0.678558in}}%
\pgfpathcurveto{\pgfqpoint{1.045990in}{0.681162in}}{\pgfqpoint{1.044955in}{0.683661in}}{\pgfqpoint{1.043113in}{0.685502in}}%
\pgfpathcurveto{\pgfqpoint{1.041272in}{0.687344in}}{\pgfqpoint{1.038773in}{0.688379in}}{\pgfqpoint{1.036169in}{0.688379in}}%
\pgfpathcurveto{\pgfqpoint{1.033564in}{0.688379in}}{\pgfqpoint{1.031066in}{0.687344in}}{\pgfqpoint{1.029224in}{0.685502in}}%
\pgfpathcurveto{\pgfqpoint{1.027383in}{0.683661in}}{\pgfqpoint{1.026348in}{0.681162in}}{\pgfqpoint{1.026348in}{0.678558in}}%
\pgfpathcurveto{\pgfqpoint{1.026348in}{0.675953in}}{\pgfqpoint{1.027383in}{0.673455in}}{\pgfqpoint{1.029224in}{0.671613in}}%
\pgfpathcurveto{\pgfqpoint{1.031066in}{0.669772in}}{\pgfqpoint{1.033564in}{0.668737in}}{\pgfqpoint{1.036169in}{0.668737in}}%
\pgfpathclose%
\pgfusepath{stroke,fill}%
\end{pgfscope}%
\begin{pgfscope}%
\pgfpathrectangle{\pgfqpoint{0.750000in}{0.500000in}}{\pgfqpoint{4.650000in}{3.020000in}}%
\pgfusepath{clip}%
\pgfsetbuttcap%
\pgfsetroundjoin%
\definecolor{currentfill}{rgb}{0.121569,0.466667,0.705882}%
\pgfsetfillcolor{currentfill}%
\pgfsetlinewidth{1.003750pt}%
\definecolor{currentstroke}{rgb}{0.121569,0.466667,0.705882}%
\pgfsetstrokecolor{currentstroke}%
\pgfsetdash{}{0pt}%
\pgfpathmoveto{\pgfqpoint{1.039264in}{0.673790in}}%
\pgfpathcurveto{\pgfqpoint{1.042948in}{0.673790in}}{\pgfqpoint{1.046481in}{0.675253in}}{\pgfqpoint{1.049085in}{0.677858in}}%
\pgfpathcurveto{\pgfqpoint{1.051690in}{0.680462in}}{\pgfqpoint{1.053153in}{0.683995in}}{\pgfqpoint{1.053153in}{0.687679in}}%
\pgfpathcurveto{\pgfqpoint{1.053153in}{0.691362in}}{\pgfqpoint{1.051690in}{0.694895in}}{\pgfqpoint{1.049085in}{0.697500in}}%
\pgfpathcurveto{\pgfqpoint{1.046481in}{0.700104in}}{\pgfqpoint{1.042948in}{0.701568in}}{\pgfqpoint{1.039264in}{0.701568in}}%
\pgfpathcurveto{\pgfqpoint{1.035581in}{0.701568in}}{\pgfqpoint{1.032048in}{0.700104in}}{\pgfqpoint{1.029443in}{0.697500in}}%
\pgfpathcurveto{\pgfqpoint{1.026839in}{0.694895in}}{\pgfqpoint{1.025375in}{0.691362in}}{\pgfqpoint{1.025375in}{0.687679in}}%
\pgfpathcurveto{\pgfqpoint{1.025375in}{0.683995in}}{\pgfqpoint{1.026839in}{0.680462in}}{\pgfqpoint{1.029443in}{0.677858in}}%
\pgfpathcurveto{\pgfqpoint{1.032048in}{0.675253in}}{\pgfqpoint{1.035581in}{0.673790in}}{\pgfqpoint{1.039264in}{0.673790in}}%
\pgfpathclose%
\pgfusepath{stroke,fill}%
\end{pgfscope}%
\begin{pgfscope}%
\pgfpathrectangle{\pgfqpoint{0.750000in}{0.500000in}}{\pgfqpoint{4.650000in}{3.020000in}}%
\pgfusepath{clip}%
\pgfsetbuttcap%
\pgfsetroundjoin%
\definecolor{currentfill}{rgb}{0.121569,0.466667,0.705882}%
\pgfsetfillcolor{currentfill}%
\pgfsetlinewidth{1.003750pt}%
\definecolor{currentstroke}{rgb}{0.121569,0.466667,0.705882}%
\pgfsetstrokecolor{currentstroke}%
\pgfsetdash{}{0pt}%
\pgfpathmoveto{\pgfqpoint{1.125935in}{1.045394in}}%
\pgfpathcurveto{\pgfqpoint{1.134574in}{1.045394in}}{\pgfqpoint{1.142859in}{1.048826in}}{\pgfqpoint{1.148967in}{1.054934in}}%
\pgfpathcurveto{\pgfqpoint{1.155076in}{1.061042in}}{\pgfqpoint{1.158508in}{1.069328in}}{\pgfqpoint{1.158508in}{1.077966in}}%
\pgfpathcurveto{\pgfqpoint{1.158508in}{1.086605in}}{\pgfqpoint{1.155076in}{1.094890in}}{\pgfqpoint{1.148967in}{1.100999in}}%
\pgfpathcurveto{\pgfqpoint{1.142859in}{1.107107in}}{\pgfqpoint{1.134574in}{1.110539in}}{\pgfqpoint{1.125935in}{1.110539in}}%
\pgfpathcurveto{\pgfqpoint{1.117297in}{1.110539in}}{\pgfqpoint{1.109011in}{1.107107in}}{\pgfqpoint{1.102903in}{1.100999in}}%
\pgfpathcurveto{\pgfqpoint{1.096795in}{1.094890in}}{\pgfqpoint{1.093363in}{1.086605in}}{\pgfqpoint{1.093363in}{1.077966in}}%
\pgfpathcurveto{\pgfqpoint{1.093363in}{1.069328in}}{\pgfqpoint{1.096795in}{1.061042in}}{\pgfqpoint{1.102903in}{1.054934in}}%
\pgfpathcurveto{\pgfqpoint{1.109011in}{1.048826in}}{\pgfqpoint{1.117297in}{1.045394in}}{\pgfqpoint{1.125935in}{1.045394in}}%
\pgfpathclose%
\pgfusepath{stroke,fill}%
\end{pgfscope}%
\begin{pgfscope}%
\pgfpathrectangle{\pgfqpoint{0.750000in}{0.500000in}}{\pgfqpoint{4.650000in}{3.020000in}}%
\pgfusepath{clip}%
\pgfsetbuttcap%
\pgfsetroundjoin%
\definecolor{currentfill}{rgb}{0.121569,0.466667,0.705882}%
\pgfsetfillcolor{currentfill}%
\pgfsetlinewidth{1.003750pt}%
\definecolor{currentstroke}{rgb}{0.121569,0.466667,0.705882}%
\pgfsetstrokecolor{currentstroke}%
\pgfsetdash{}{0pt}%
\pgfpathmoveto{\pgfqpoint{1.042360in}{0.666337in}}%
\pgfpathcurveto{\pgfqpoint{1.044964in}{0.666337in}}{\pgfqpoint{1.047462in}{0.667371in}}{\pgfqpoint{1.049304in}{0.669213in}}%
\pgfpathcurveto{\pgfqpoint{1.051146in}{0.671055in}}{\pgfqpoint{1.052181in}{0.673553in}}{\pgfqpoint{1.052181in}{0.676157in}}%
\pgfpathcurveto{\pgfqpoint{1.052181in}{0.678762in}}{\pgfqpoint{1.051146in}{0.681260in}}{\pgfqpoint{1.049304in}{0.683102in}}%
\pgfpathcurveto{\pgfqpoint{1.047462in}{0.684944in}}{\pgfqpoint{1.044964in}{0.685978in}}{\pgfqpoint{1.042360in}{0.685978in}}%
\pgfpathcurveto{\pgfqpoint{1.039755in}{0.685978in}}{\pgfqpoint{1.037257in}{0.684944in}}{\pgfqpoint{1.035415in}{0.683102in}}%
\pgfpathcurveto{\pgfqpoint{1.033574in}{0.681260in}}{\pgfqpoint{1.032539in}{0.678762in}}{\pgfqpoint{1.032539in}{0.676157in}}%
\pgfpathcurveto{\pgfqpoint{1.032539in}{0.673553in}}{\pgfqpoint{1.033574in}{0.671055in}}{\pgfqpoint{1.035415in}{0.669213in}}%
\pgfpathcurveto{\pgfqpoint{1.037257in}{0.667371in}}{\pgfqpoint{1.039755in}{0.666337in}}{\pgfqpoint{1.042360in}{0.666337in}}%
\pgfpathclose%
\pgfusepath{stroke,fill}%
\end{pgfscope}%
\begin{pgfscope}%
\pgfpathrectangle{\pgfqpoint{0.750000in}{0.500000in}}{\pgfqpoint{4.650000in}{3.020000in}}%
\pgfusepath{clip}%
\pgfsetbuttcap%
\pgfsetroundjoin%
\definecolor{currentfill}{rgb}{0.121569,0.466667,0.705882}%
\pgfsetfillcolor{currentfill}%
\pgfsetlinewidth{1.003750pt}%
\definecolor{currentstroke}{rgb}{0.121569,0.466667,0.705882}%
\pgfsetstrokecolor{currentstroke}%
\pgfsetdash{}{0pt}%
\pgfpathmoveto{\pgfqpoint{1.109942in}{0.688928in}}%
\pgfpathcurveto{\pgfqpoint{1.115467in}{0.688928in}}{\pgfqpoint{1.120767in}{0.691123in}}{\pgfqpoint{1.124674in}{0.695030in}}%
\pgfpathcurveto{\pgfqpoint{1.128581in}{0.698937in}}{\pgfqpoint{1.130776in}{0.704236in}}{\pgfqpoint{1.130776in}{0.709762in}}%
\pgfpathcurveto{\pgfqpoint{1.130776in}{0.715287in}}{\pgfqpoint{1.128581in}{0.720586in}}{\pgfqpoint{1.124674in}{0.724493in}}%
\pgfpathcurveto{\pgfqpoint{1.120767in}{0.728400in}}{\pgfqpoint{1.115467in}{0.730595in}}{\pgfqpoint{1.109942in}{0.730595in}}%
\pgfpathcurveto{\pgfqpoint{1.104417in}{0.730595in}}{\pgfqpoint{1.099118in}{0.728400in}}{\pgfqpoint{1.095211in}{0.724493in}}%
\pgfpathcurveto{\pgfqpoint{1.091304in}{0.720586in}}{\pgfqpoint{1.089109in}{0.715287in}}{\pgfqpoint{1.089109in}{0.709762in}}%
\pgfpathcurveto{\pgfqpoint{1.089109in}{0.704236in}}{\pgfqpoint{1.091304in}{0.698937in}}{\pgfqpoint{1.095211in}{0.695030in}}%
\pgfpathcurveto{\pgfqpoint{1.099118in}{0.691123in}}{\pgfqpoint{1.104417in}{0.688928in}}{\pgfqpoint{1.109942in}{0.688928in}}%
\pgfpathclose%
\pgfusepath{stroke,fill}%
\end{pgfscope}%
\begin{pgfscope}%
\pgfpathrectangle{\pgfqpoint{0.750000in}{0.500000in}}{\pgfqpoint{4.650000in}{3.020000in}}%
\pgfusepath{clip}%
\pgfsetbuttcap%
\pgfsetroundjoin%
\definecolor{currentfill}{rgb}{0.121569,0.466667,0.705882}%
\pgfsetfillcolor{currentfill}%
\pgfsetlinewidth{1.003750pt}%
\definecolor{currentstroke}{rgb}{0.121569,0.466667,0.705882}%
\pgfsetstrokecolor{currentstroke}%
\pgfsetdash{}{0pt}%
\pgfpathmoveto{\pgfqpoint{1.000056in}{0.641854in}}%
\pgfpathcurveto{\pgfqpoint{1.002661in}{0.641854in}}{\pgfqpoint{1.005159in}{0.642888in}}{\pgfqpoint{1.007000in}{0.644730in}}%
\pgfpathcurveto{\pgfqpoint{1.008842in}{0.646572in}}{\pgfqpoint{1.009877in}{0.649070in}}{\pgfqpoint{1.009877in}{0.651674in}}%
\pgfpathcurveto{\pgfqpoint{1.009877in}{0.654279in}}{\pgfqpoint{1.008842in}{0.656777in}}{\pgfqpoint{1.007000in}{0.658619in}}%
\pgfpathcurveto{\pgfqpoint{1.005159in}{0.660461in}}{\pgfqpoint{1.002661in}{0.661495in}}{\pgfqpoint{1.000056in}{0.661495in}}%
\pgfpathcurveto{\pgfqpoint{0.997451in}{0.661495in}}{\pgfqpoint{0.994953in}{0.660461in}}{\pgfqpoint{0.993112in}{0.658619in}}%
\pgfpathcurveto{\pgfqpoint{0.991270in}{0.656777in}}{\pgfqpoint{0.990235in}{0.654279in}}{\pgfqpoint{0.990235in}{0.651674in}}%
\pgfpathcurveto{\pgfqpoint{0.990235in}{0.649070in}}{\pgfqpoint{0.991270in}{0.646572in}}{\pgfqpoint{0.993112in}{0.644730in}}%
\pgfpathcurveto{\pgfqpoint{0.994953in}{0.642888in}}{\pgfqpoint{0.997451in}{0.641854in}}{\pgfqpoint{1.000056in}{0.641854in}}%
\pgfpathclose%
\pgfusepath{stroke,fill}%
\end{pgfscope}%
\begin{pgfscope}%
\pgfpathrectangle{\pgfqpoint{0.750000in}{0.500000in}}{\pgfqpoint{4.650000in}{3.020000in}}%
\pgfusepath{clip}%
\pgfsetbuttcap%
\pgfsetroundjoin%
\definecolor{currentfill}{rgb}{0.121569,0.466667,0.705882}%
\pgfsetfillcolor{currentfill}%
\pgfsetlinewidth{1.003750pt}%
\definecolor{currentstroke}{rgb}{0.121569,0.466667,0.705882}%
\pgfsetstrokecolor{currentstroke}%
\pgfsetdash{}{0pt}%
\pgfpathmoveto{\pgfqpoint{1.024303in}{0.652581in}}%
\pgfpathcurveto{\pgfqpoint{1.030683in}{0.652581in}}{\pgfqpoint{1.036802in}{0.655116in}}{\pgfqpoint{1.041314in}{0.659627in}}%
\pgfpathcurveto{\pgfqpoint{1.045825in}{0.664138in}}{\pgfqpoint{1.048360in}{0.670258in}}{\pgfqpoint{1.048360in}{0.676638in}}%
\pgfpathcurveto{\pgfqpoint{1.048360in}{0.683017in}}{\pgfqpoint{1.045825in}{0.689137in}}{\pgfqpoint{1.041314in}{0.693648in}}%
\pgfpathcurveto{\pgfqpoint{1.036802in}{0.698159in}}{\pgfqpoint{1.030683in}{0.700694in}}{\pgfqpoint{1.024303in}{0.700694in}}%
\pgfpathcurveto{\pgfqpoint{1.017923in}{0.700694in}}{\pgfqpoint{1.011804in}{0.698159in}}{\pgfqpoint{1.007293in}{0.693648in}}%
\pgfpathcurveto{\pgfqpoint{1.002782in}{0.689137in}}{\pgfqpoint{1.000247in}{0.683017in}}{\pgfqpoint{1.000247in}{0.676638in}}%
\pgfpathcurveto{\pgfqpoint{1.000247in}{0.670258in}}{\pgfqpoint{1.002782in}{0.664138in}}{\pgfqpoint{1.007293in}{0.659627in}}%
\pgfpathcurveto{\pgfqpoint{1.011804in}{0.655116in}}{\pgfqpoint{1.017923in}{0.652581in}}{\pgfqpoint{1.024303in}{0.652581in}}%
\pgfpathclose%
\pgfusepath{stroke,fill}%
\end{pgfscope}%
\begin{pgfscope}%
\pgfpathrectangle{\pgfqpoint{0.750000in}{0.500000in}}{\pgfqpoint{4.650000in}{3.020000in}}%
\pgfusepath{clip}%
\pgfsetbuttcap%
\pgfsetroundjoin%
\definecolor{currentfill}{rgb}{0.121569,0.466667,0.705882}%
\pgfsetfillcolor{currentfill}%
\pgfsetlinewidth{1.003750pt}%
\definecolor{currentstroke}{rgb}{0.121569,0.466667,0.705882}%
\pgfsetstrokecolor{currentstroke}%
\pgfsetdash{}{0pt}%
\pgfpathmoveto{\pgfqpoint{1.350351in}{1.009329in}}%
\pgfpathcurveto{\pgfqpoint{1.356459in}{1.009329in}}{\pgfqpoint{1.362318in}{1.011756in}}{\pgfqpoint{1.366637in}{1.016075in}}%
\pgfpathcurveto{\pgfqpoint{1.370956in}{1.020394in}}{\pgfqpoint{1.373383in}{1.026253in}}{\pgfqpoint{1.373383in}{1.032361in}}%
\pgfpathcurveto{\pgfqpoint{1.373383in}{1.038469in}}{\pgfqpoint{1.370956in}{1.044328in}}{\pgfqpoint{1.366637in}{1.048647in}}%
\pgfpathcurveto{\pgfqpoint{1.362318in}{1.052966in}}{\pgfqpoint{1.356459in}{1.055393in}}{\pgfqpoint{1.350351in}{1.055393in}}%
\pgfpathcurveto{\pgfqpoint{1.344243in}{1.055393in}}{\pgfqpoint{1.338384in}{1.052966in}}{\pgfqpoint{1.334065in}{1.048647in}}%
\pgfpathcurveto{\pgfqpoint{1.329746in}{1.044328in}}{\pgfqpoint{1.327319in}{1.038469in}}{\pgfqpoint{1.327319in}{1.032361in}}%
\pgfpathcurveto{\pgfqpoint{1.327319in}{1.026253in}}{\pgfqpoint{1.329746in}{1.020394in}}{\pgfqpoint{1.334065in}{1.016075in}}%
\pgfpathcurveto{\pgfqpoint{1.338384in}{1.011756in}}{\pgfqpoint{1.344243in}{1.009329in}}{\pgfqpoint{1.350351in}{1.009329in}}%
\pgfpathclose%
\pgfusepath{stroke,fill}%
\end{pgfscope}%
\begin{pgfscope}%
\pgfpathrectangle{\pgfqpoint{0.750000in}{0.500000in}}{\pgfqpoint{4.650000in}{3.020000in}}%
\pgfusepath{clip}%
\pgfsetbuttcap%
\pgfsetroundjoin%
\definecolor{currentfill}{rgb}{0.121569,0.466667,0.705882}%
\pgfsetfillcolor{currentfill}%
\pgfsetlinewidth{1.003750pt}%
\definecolor{currentstroke}{rgb}{0.121569,0.466667,0.705882}%
\pgfsetstrokecolor{currentstroke}%
\pgfsetdash{}{0pt}%
\pgfpathmoveto{\pgfqpoint{1.080020in}{1.317009in}}%
\pgfpathcurveto{\pgfqpoint{1.083210in}{1.317009in}}{\pgfqpoint{1.086270in}{1.318276in}}{\pgfqpoint{1.088525in}{1.320532in}}%
\pgfpathcurveto{\pgfqpoint{1.090781in}{1.322787in}}{\pgfqpoint{1.092048in}{1.325847in}}{\pgfqpoint{1.092048in}{1.329037in}}%
\pgfpathcurveto{\pgfqpoint{1.092048in}{1.332227in}}{\pgfqpoint{1.090781in}{1.335287in}}{\pgfqpoint{1.088525in}{1.337542in}}%
\pgfpathcurveto{\pgfqpoint{1.086270in}{1.339798in}}{\pgfqpoint{1.083210in}{1.341065in}}{\pgfqpoint{1.080020in}{1.341065in}}%
\pgfpathcurveto{\pgfqpoint{1.076830in}{1.341065in}}{\pgfqpoint{1.073771in}{1.339798in}}{\pgfqpoint{1.071515in}{1.337542in}}%
\pgfpathcurveto{\pgfqpoint{1.069260in}{1.335287in}}{\pgfqpoint{1.067992in}{1.332227in}}{\pgfqpoint{1.067992in}{1.329037in}}%
\pgfpathcurveto{\pgfqpoint{1.067992in}{1.325847in}}{\pgfqpoint{1.069260in}{1.322787in}}{\pgfqpoint{1.071515in}{1.320532in}}%
\pgfpathcurveto{\pgfqpoint{1.073771in}{1.318276in}}{\pgfqpoint{1.076830in}{1.317009in}}{\pgfqpoint{1.080020in}{1.317009in}}%
\pgfpathclose%
\pgfusepath{stroke,fill}%
\end{pgfscope}%
\begin{pgfscope}%
\pgfpathrectangle{\pgfqpoint{0.750000in}{0.500000in}}{\pgfqpoint{4.650000in}{3.020000in}}%
\pgfusepath{clip}%
\pgfsetbuttcap%
\pgfsetroundjoin%
\definecolor{currentfill}{rgb}{0.121569,0.466667,0.705882}%
\pgfsetfillcolor{currentfill}%
\pgfsetlinewidth{1.003750pt}%
\definecolor{currentstroke}{rgb}{0.121569,0.466667,0.705882}%
\pgfsetstrokecolor{currentstroke}%
\pgfsetdash{}{0pt}%
\pgfpathmoveto{\pgfqpoint{1.027399in}{0.661789in}}%
\pgfpathcurveto{\pgfqpoint{1.031082in}{0.661789in}}{\pgfqpoint{1.034615in}{0.663252in}}{\pgfqpoint{1.037220in}{0.665856in}}%
\pgfpathcurveto{\pgfqpoint{1.039824in}{0.668461in}}{\pgfqpoint{1.041288in}{0.671994in}}{\pgfqpoint{1.041288in}{0.675677in}}%
\pgfpathcurveto{\pgfqpoint{1.041288in}{0.679361in}}{\pgfqpoint{1.039824in}{0.682894in}}{\pgfqpoint{1.037220in}{0.685498in}}%
\pgfpathcurveto{\pgfqpoint{1.034615in}{0.688103in}}{\pgfqpoint{1.031082in}{0.689566in}}{\pgfqpoint{1.027399in}{0.689566in}}%
\pgfpathcurveto{\pgfqpoint{1.023715in}{0.689566in}}{\pgfqpoint{1.020182in}{0.688103in}}{\pgfqpoint{1.017578in}{0.685498in}}%
\pgfpathcurveto{\pgfqpoint{1.014973in}{0.682894in}}{\pgfqpoint{1.013510in}{0.679361in}}{\pgfqpoint{1.013510in}{0.675677in}}%
\pgfpathcurveto{\pgfqpoint{1.013510in}{0.671994in}}{\pgfqpoint{1.014973in}{0.668461in}}{\pgfqpoint{1.017578in}{0.665856in}}%
\pgfpathcurveto{\pgfqpoint{1.020182in}{0.663252in}}{\pgfqpoint{1.023715in}{0.661789in}}{\pgfqpoint{1.027399in}{0.661789in}}%
\pgfpathclose%
\pgfusepath{stroke,fill}%
\end{pgfscope}%
\begin{pgfscope}%
\pgfsetbuttcap%
\pgfsetroundjoin%
\definecolor{currentfill}{rgb}{0.000000,0.000000,0.000000}%
\pgfsetfillcolor{currentfill}%
\pgfsetlinewidth{0.803000pt}%
\definecolor{currentstroke}{rgb}{0.000000,0.000000,0.000000}%
\pgfsetstrokecolor{currentstroke}%
\pgfsetdash{}{0pt}%
\pgfsys@defobject{currentmarker}{\pgfqpoint{0.000000in}{-0.048611in}}{\pgfqpoint{0.000000in}{0.000000in}}{%
\pgfpathmoveto{\pgfqpoint{0.000000in}{0.000000in}}%
\pgfpathlineto{\pgfqpoint{0.000000in}{-0.048611in}}%
\pgfusepath{stroke,fill}%
}%
\begin{pgfscope}%
\pgfsys@transformshift{0.958784in}{0.500000in}%
\pgfsys@useobject{currentmarker}{}%
\end{pgfscope}%
\end{pgfscope}%
\begin{pgfscope}%
\definecolor{textcolor}{rgb}{0.000000,0.000000,0.000000}%
\pgfsetstrokecolor{textcolor}%
\pgfsetfillcolor{textcolor}%
\pgftext[x=0.958784in,y=0.402778in,,top]{\color{textcolor}\rmfamily\fontsize{10.000000}{12.000000}\selectfont \(\displaystyle {0}\)}%
\end{pgfscope}%
\begin{pgfscope}%
\pgfsetbuttcap%
\pgfsetroundjoin%
\definecolor{currentfill}{rgb}{0.000000,0.000000,0.000000}%
\pgfsetfillcolor{currentfill}%
\pgfsetlinewidth{0.803000pt}%
\definecolor{currentstroke}{rgb}{0.000000,0.000000,0.000000}%
\pgfsetstrokecolor{currentstroke}%
\pgfsetdash{}{0pt}%
\pgfsys@defobject{currentmarker}{\pgfqpoint{0.000000in}{-0.048611in}}{\pgfqpoint{0.000000in}{0.000000in}}{%
\pgfpathmoveto{\pgfqpoint{0.000000in}{0.000000in}}%
\pgfpathlineto{\pgfqpoint{0.000000in}{-0.048611in}}%
\pgfusepath{stroke,fill}%
}%
\begin{pgfscope}%
\pgfsys@transformshift{1.990581in}{0.500000in}%
\pgfsys@useobject{currentmarker}{}%
\end{pgfscope}%
\end{pgfscope}%
\begin{pgfscope}%
\definecolor{textcolor}{rgb}{0.000000,0.000000,0.000000}%
\pgfsetstrokecolor{textcolor}%
\pgfsetfillcolor{textcolor}%
\pgftext[x=1.990581in,y=0.402778in,,top]{\color{textcolor}\rmfamily\fontsize{10.000000}{12.000000}\selectfont \(\displaystyle {2000}\)}%
\end{pgfscope}%
\begin{pgfscope}%
\pgfsetbuttcap%
\pgfsetroundjoin%
\definecolor{currentfill}{rgb}{0.000000,0.000000,0.000000}%
\pgfsetfillcolor{currentfill}%
\pgfsetlinewidth{0.803000pt}%
\definecolor{currentstroke}{rgb}{0.000000,0.000000,0.000000}%
\pgfsetstrokecolor{currentstroke}%
\pgfsetdash{}{0pt}%
\pgfsys@defobject{currentmarker}{\pgfqpoint{0.000000in}{-0.048611in}}{\pgfqpoint{0.000000in}{0.000000in}}{%
\pgfpathmoveto{\pgfqpoint{0.000000in}{0.000000in}}%
\pgfpathlineto{\pgfqpoint{0.000000in}{-0.048611in}}%
\pgfusepath{stroke,fill}%
}%
\begin{pgfscope}%
\pgfsys@transformshift{3.022378in}{0.500000in}%
\pgfsys@useobject{currentmarker}{}%
\end{pgfscope}%
\end{pgfscope}%
\begin{pgfscope}%
\definecolor{textcolor}{rgb}{0.000000,0.000000,0.000000}%
\pgfsetstrokecolor{textcolor}%
\pgfsetfillcolor{textcolor}%
\pgftext[x=3.022378in,y=0.402778in,,top]{\color{textcolor}\rmfamily\fontsize{10.000000}{12.000000}\selectfont \(\displaystyle {4000}\)}%
\end{pgfscope}%
\begin{pgfscope}%
\pgfsetbuttcap%
\pgfsetroundjoin%
\definecolor{currentfill}{rgb}{0.000000,0.000000,0.000000}%
\pgfsetfillcolor{currentfill}%
\pgfsetlinewidth{0.803000pt}%
\definecolor{currentstroke}{rgb}{0.000000,0.000000,0.000000}%
\pgfsetstrokecolor{currentstroke}%
\pgfsetdash{}{0pt}%
\pgfsys@defobject{currentmarker}{\pgfqpoint{0.000000in}{-0.048611in}}{\pgfqpoint{0.000000in}{0.000000in}}{%
\pgfpathmoveto{\pgfqpoint{0.000000in}{0.000000in}}%
\pgfpathlineto{\pgfqpoint{0.000000in}{-0.048611in}}%
\pgfusepath{stroke,fill}%
}%
\begin{pgfscope}%
\pgfsys@transformshift{4.054175in}{0.500000in}%
\pgfsys@useobject{currentmarker}{}%
\end{pgfscope}%
\end{pgfscope}%
\begin{pgfscope}%
\definecolor{textcolor}{rgb}{0.000000,0.000000,0.000000}%
\pgfsetstrokecolor{textcolor}%
\pgfsetfillcolor{textcolor}%
\pgftext[x=4.054175in,y=0.402778in,,top]{\color{textcolor}\rmfamily\fontsize{10.000000}{12.000000}\selectfont \(\displaystyle {6000}\)}%
\end{pgfscope}%
\begin{pgfscope}%
\pgfsetbuttcap%
\pgfsetroundjoin%
\definecolor{currentfill}{rgb}{0.000000,0.000000,0.000000}%
\pgfsetfillcolor{currentfill}%
\pgfsetlinewidth{0.803000pt}%
\definecolor{currentstroke}{rgb}{0.000000,0.000000,0.000000}%
\pgfsetstrokecolor{currentstroke}%
\pgfsetdash{}{0pt}%
\pgfsys@defobject{currentmarker}{\pgfqpoint{0.000000in}{-0.048611in}}{\pgfqpoint{0.000000in}{0.000000in}}{%
\pgfpathmoveto{\pgfqpoint{0.000000in}{0.000000in}}%
\pgfpathlineto{\pgfqpoint{0.000000in}{-0.048611in}}%
\pgfusepath{stroke,fill}%
}%
\begin{pgfscope}%
\pgfsys@transformshift{5.085973in}{0.500000in}%
\pgfsys@useobject{currentmarker}{}%
\end{pgfscope}%
\end{pgfscope}%
\begin{pgfscope}%
\definecolor{textcolor}{rgb}{0.000000,0.000000,0.000000}%
\pgfsetstrokecolor{textcolor}%
\pgfsetfillcolor{textcolor}%
\pgftext[x=5.085973in,y=0.402778in,,top]{\color{textcolor}\rmfamily\fontsize{10.000000}{12.000000}\selectfont \(\displaystyle {8000}\)}%
\end{pgfscope}%
\begin{pgfscope}%
\definecolor{textcolor}{rgb}{0.000000,0.000000,0.000000}%
\pgfsetstrokecolor{textcolor}%
\pgfsetfillcolor{textcolor}%
\pgftext[x=3.075000in,y=0.223766in,,top]{\color{textcolor}\rmfamily\fontsize{10.000000}{12.000000}\selectfont Number of lines}%
\end{pgfscope}%
\begin{pgfscope}%
\pgfsetbuttcap%
\pgfsetroundjoin%
\definecolor{currentfill}{rgb}{0.000000,0.000000,0.000000}%
\pgfsetfillcolor{currentfill}%
\pgfsetlinewidth{0.803000pt}%
\definecolor{currentstroke}{rgb}{0.000000,0.000000,0.000000}%
\pgfsetstrokecolor{currentstroke}%
\pgfsetdash{}{0pt}%
\pgfsys@defobject{currentmarker}{\pgfqpoint{-0.048611in}{0.000000in}}{\pgfqpoint{-0.000000in}{0.000000in}}{%
\pgfpathmoveto{\pgfqpoint{-0.000000in}{0.000000in}}%
\pgfpathlineto{\pgfqpoint{-0.048611in}{0.000000in}}%
\pgfusepath{stroke,fill}%
}%
\begin{pgfscope}%
\pgfsys@transformshift{0.750000in}{0.635833in}%
\pgfsys@useobject{currentmarker}{}%
\end{pgfscope}%
\end{pgfscope}%
\begin{pgfscope}%
\definecolor{textcolor}{rgb}{0.000000,0.000000,0.000000}%
\pgfsetstrokecolor{textcolor}%
\pgfsetfillcolor{textcolor}%
\pgftext[x=0.583333in, y=0.587607in, left, base]{\color{textcolor}\rmfamily\fontsize{10.000000}{12.000000}\selectfont \(\displaystyle {0}\)}%
\end{pgfscope}%
\begin{pgfscope}%
\pgfsetbuttcap%
\pgfsetroundjoin%
\definecolor{currentfill}{rgb}{0.000000,0.000000,0.000000}%
\pgfsetfillcolor{currentfill}%
\pgfsetlinewidth{0.803000pt}%
\definecolor{currentstroke}{rgb}{0.000000,0.000000,0.000000}%
\pgfsetstrokecolor{currentstroke}%
\pgfsetdash{}{0pt}%
\pgfsys@defobject{currentmarker}{\pgfqpoint{-0.048611in}{0.000000in}}{\pgfqpoint{-0.000000in}{0.000000in}}{%
\pgfpathmoveto{\pgfqpoint{-0.000000in}{0.000000in}}%
\pgfpathlineto{\pgfqpoint{-0.048611in}{0.000000in}}%
\pgfusepath{stroke,fill}%
}%
\begin{pgfscope}%
\pgfsys@transformshift{0.750000in}{1.115891in}%
\pgfsys@useobject{currentmarker}{}%
\end{pgfscope}%
\end{pgfscope}%
\begin{pgfscope}%
\definecolor{textcolor}{rgb}{0.000000,0.000000,0.000000}%
\pgfsetstrokecolor{textcolor}%
\pgfsetfillcolor{textcolor}%
\pgftext[x=0.444444in, y=1.067666in, left, base]{\color{textcolor}\rmfamily\fontsize{10.000000}{12.000000}\selectfont \(\displaystyle {100}\)}%
\end{pgfscope}%
\begin{pgfscope}%
\pgfsetbuttcap%
\pgfsetroundjoin%
\definecolor{currentfill}{rgb}{0.000000,0.000000,0.000000}%
\pgfsetfillcolor{currentfill}%
\pgfsetlinewidth{0.803000pt}%
\definecolor{currentstroke}{rgb}{0.000000,0.000000,0.000000}%
\pgfsetstrokecolor{currentstroke}%
\pgfsetdash{}{0pt}%
\pgfsys@defobject{currentmarker}{\pgfqpoint{-0.048611in}{0.000000in}}{\pgfqpoint{-0.000000in}{0.000000in}}{%
\pgfpathmoveto{\pgfqpoint{-0.000000in}{0.000000in}}%
\pgfpathlineto{\pgfqpoint{-0.048611in}{0.000000in}}%
\pgfusepath{stroke,fill}%
}%
\begin{pgfscope}%
\pgfsys@transformshift{0.750000in}{1.595950in}%
\pgfsys@useobject{currentmarker}{}%
\end{pgfscope}%
\end{pgfscope}%
\begin{pgfscope}%
\definecolor{textcolor}{rgb}{0.000000,0.000000,0.000000}%
\pgfsetstrokecolor{textcolor}%
\pgfsetfillcolor{textcolor}%
\pgftext[x=0.444444in, y=1.547724in, left, base]{\color{textcolor}\rmfamily\fontsize{10.000000}{12.000000}\selectfont \(\displaystyle {200}\)}%
\end{pgfscope}%
\begin{pgfscope}%
\pgfsetbuttcap%
\pgfsetroundjoin%
\definecolor{currentfill}{rgb}{0.000000,0.000000,0.000000}%
\pgfsetfillcolor{currentfill}%
\pgfsetlinewidth{0.803000pt}%
\definecolor{currentstroke}{rgb}{0.000000,0.000000,0.000000}%
\pgfsetstrokecolor{currentstroke}%
\pgfsetdash{}{0pt}%
\pgfsys@defobject{currentmarker}{\pgfqpoint{-0.048611in}{0.000000in}}{\pgfqpoint{-0.000000in}{0.000000in}}{%
\pgfpathmoveto{\pgfqpoint{-0.000000in}{0.000000in}}%
\pgfpathlineto{\pgfqpoint{-0.048611in}{0.000000in}}%
\pgfusepath{stroke,fill}%
}%
\begin{pgfscope}%
\pgfsys@transformshift{0.750000in}{2.076008in}%
\pgfsys@useobject{currentmarker}{}%
\end{pgfscope}%
\end{pgfscope}%
\begin{pgfscope}%
\definecolor{textcolor}{rgb}{0.000000,0.000000,0.000000}%
\pgfsetstrokecolor{textcolor}%
\pgfsetfillcolor{textcolor}%
\pgftext[x=0.444444in, y=2.027783in, left, base]{\color{textcolor}\rmfamily\fontsize{10.000000}{12.000000}\selectfont \(\displaystyle {300}\)}%
\end{pgfscope}%
\begin{pgfscope}%
\pgfsetbuttcap%
\pgfsetroundjoin%
\definecolor{currentfill}{rgb}{0.000000,0.000000,0.000000}%
\pgfsetfillcolor{currentfill}%
\pgfsetlinewidth{0.803000pt}%
\definecolor{currentstroke}{rgb}{0.000000,0.000000,0.000000}%
\pgfsetstrokecolor{currentstroke}%
\pgfsetdash{}{0pt}%
\pgfsys@defobject{currentmarker}{\pgfqpoint{-0.048611in}{0.000000in}}{\pgfqpoint{-0.000000in}{0.000000in}}{%
\pgfpathmoveto{\pgfqpoint{-0.000000in}{0.000000in}}%
\pgfpathlineto{\pgfqpoint{-0.048611in}{0.000000in}}%
\pgfusepath{stroke,fill}%
}%
\begin{pgfscope}%
\pgfsys@transformshift{0.750000in}{2.556067in}%
\pgfsys@useobject{currentmarker}{}%
\end{pgfscope}%
\end{pgfscope}%
\begin{pgfscope}%
\definecolor{textcolor}{rgb}{0.000000,0.000000,0.000000}%
\pgfsetstrokecolor{textcolor}%
\pgfsetfillcolor{textcolor}%
\pgftext[x=0.444444in, y=2.507841in, left, base]{\color{textcolor}\rmfamily\fontsize{10.000000}{12.000000}\selectfont \(\displaystyle {400}\)}%
\end{pgfscope}%
\begin{pgfscope}%
\pgfsetbuttcap%
\pgfsetroundjoin%
\definecolor{currentfill}{rgb}{0.000000,0.000000,0.000000}%
\pgfsetfillcolor{currentfill}%
\pgfsetlinewidth{0.803000pt}%
\definecolor{currentstroke}{rgb}{0.000000,0.000000,0.000000}%
\pgfsetstrokecolor{currentstroke}%
\pgfsetdash{}{0pt}%
\pgfsys@defobject{currentmarker}{\pgfqpoint{-0.048611in}{0.000000in}}{\pgfqpoint{-0.000000in}{0.000000in}}{%
\pgfpathmoveto{\pgfqpoint{-0.000000in}{0.000000in}}%
\pgfpathlineto{\pgfqpoint{-0.048611in}{0.000000in}}%
\pgfusepath{stroke,fill}%
}%
\begin{pgfscope}%
\pgfsys@transformshift{0.750000in}{3.036125in}%
\pgfsys@useobject{currentmarker}{}%
\end{pgfscope}%
\end{pgfscope}%
\begin{pgfscope}%
\definecolor{textcolor}{rgb}{0.000000,0.000000,0.000000}%
\pgfsetstrokecolor{textcolor}%
\pgfsetfillcolor{textcolor}%
\pgftext[x=0.444444in, y=2.987900in, left, base]{\color{textcolor}\rmfamily\fontsize{10.000000}{12.000000}\selectfont \(\displaystyle {500}\)}%
\end{pgfscope}%
\begin{pgfscope}%
\pgfsetbuttcap%
\pgfsetroundjoin%
\definecolor{currentfill}{rgb}{0.000000,0.000000,0.000000}%
\pgfsetfillcolor{currentfill}%
\pgfsetlinewidth{0.803000pt}%
\definecolor{currentstroke}{rgb}{0.000000,0.000000,0.000000}%
\pgfsetstrokecolor{currentstroke}%
\pgfsetdash{}{0pt}%
\pgfsys@defobject{currentmarker}{\pgfqpoint{-0.048611in}{0.000000in}}{\pgfqpoint{-0.000000in}{0.000000in}}{%
\pgfpathmoveto{\pgfqpoint{-0.000000in}{0.000000in}}%
\pgfpathlineto{\pgfqpoint{-0.048611in}{0.000000in}}%
\pgfusepath{stroke,fill}%
}%
\begin{pgfscope}%
\pgfsys@transformshift{0.750000in}{3.516184in}%
\pgfsys@useobject{currentmarker}{}%
\end{pgfscope}%
\end{pgfscope}%
\begin{pgfscope}%
\definecolor{textcolor}{rgb}{0.000000,0.000000,0.000000}%
\pgfsetstrokecolor{textcolor}%
\pgfsetfillcolor{textcolor}%
\pgftext[x=0.444444in, y=3.467958in, left, base]{\color{textcolor}\rmfamily\fontsize{10.000000}{12.000000}\selectfont \(\displaystyle {600}\)}%
\end{pgfscope}%
\begin{pgfscope}%
\definecolor{textcolor}{rgb}{0.000000,0.000000,0.000000}%
\pgfsetstrokecolor{textcolor}%
\pgfsetfillcolor{textcolor}%
\pgftext[x=0.388888in,y=2.010000in,,bottom,rotate=90.000000]{\color{textcolor}\rmfamily\fontsize{10.000000}{12.000000}\selectfont Time (ms)}%
\end{pgfscope}%
\begin{pgfscope}%
\pgfsetrectcap%
\pgfsetmiterjoin%
\pgfsetlinewidth{0.803000pt}%
\definecolor{currentstroke}{rgb}{0.000000,0.000000,0.000000}%
\pgfsetstrokecolor{currentstroke}%
\pgfsetdash{}{0pt}%
\pgfpathmoveto{\pgfqpoint{0.750000in}{0.500000in}}%
\pgfpathlineto{\pgfqpoint{0.750000in}{3.520000in}}%
\pgfusepath{stroke}%
\end{pgfscope}%
\begin{pgfscope}%
\pgfsetrectcap%
\pgfsetmiterjoin%
\pgfsetlinewidth{0.803000pt}%
\definecolor{currentstroke}{rgb}{0.000000,0.000000,0.000000}%
\pgfsetstrokecolor{currentstroke}%
\pgfsetdash{}{0pt}%
\pgfpathmoveto{\pgfqpoint{5.400000in}{0.500000in}}%
\pgfpathlineto{\pgfqpoint{5.400000in}{3.520000in}}%
\pgfusepath{stroke}%
\end{pgfscope}%
\begin{pgfscope}%
\pgfsetrectcap%
\pgfsetmiterjoin%
\pgfsetlinewidth{0.803000pt}%
\definecolor{currentstroke}{rgb}{0.000000,0.000000,0.000000}%
\pgfsetstrokecolor{currentstroke}%
\pgfsetdash{}{0pt}%
\pgfpathmoveto{\pgfqpoint{0.750000in}{0.500000in}}%
\pgfpathlineto{\pgfqpoint{5.400000in}{0.500000in}}%
\pgfusepath{stroke}%
\end{pgfscope}%
\begin{pgfscope}%
\pgfsetrectcap%
\pgfsetmiterjoin%
\pgfsetlinewidth{0.803000pt}%
\definecolor{currentstroke}{rgb}{0.000000,0.000000,0.000000}%
\pgfsetstrokecolor{currentstroke}%
\pgfsetdash{}{0pt}%
\pgfpathmoveto{\pgfqpoint{0.750000in}{3.520000in}}%
\pgfpathlineto{\pgfqpoint{5.400000in}{3.520000in}}%
\pgfusepath{stroke}%
\end{pgfscope}%
\begin{pgfscope}%
\pgfsetbuttcap%
\pgfsetmiterjoin%
\definecolor{currentfill}{rgb}{1.000000,1.000000,1.000000}%
\pgfsetfillcolor{currentfill}%
\pgfsetfillopacity{0.800000}%
\pgfsetlinewidth{1.003750pt}%
\definecolor{currentstroke}{rgb}{0.800000,0.800000,0.800000}%
\pgfsetstrokecolor{currentstroke}%
\pgfsetstrokeopacity{0.800000}%
\pgfsetdash{}{0pt}%
\pgfpathmoveto{\pgfqpoint{0.847222in}{2.634969in}}%
\pgfpathlineto{\pgfqpoint{1.430556in}{2.634969in}}%
\pgfpathquadraticcurveto{\pgfqpoint{1.458334in}{2.634969in}}{\pgfqpoint{1.458334in}{2.662747in}}%
\pgfpathlineto{\pgfqpoint{1.458334in}{3.422778in}}%
\pgfpathquadraticcurveto{\pgfqpoint{1.458334in}{3.450556in}}{\pgfqpoint{1.430556in}{3.450556in}}%
\pgfpathlineto{\pgfqpoint{0.847222in}{3.450556in}}%
\pgfpathquadraticcurveto{\pgfqpoint{0.819444in}{3.450556in}}{\pgfqpoint{0.819444in}{3.422778in}}%
\pgfpathlineto{\pgfqpoint{0.819444in}{2.662747in}}%
\pgfpathquadraticcurveto{\pgfqpoint{0.819444in}{2.634969in}}{\pgfqpoint{0.847222in}{2.634969in}}%
\pgfpathclose%
\pgfusepath{stroke,fill}%
\end{pgfscope}%
\begin{pgfscope}%
\definecolor{textcolor}{rgb}{0.000000,0.000000,0.000000}%
\pgfsetstrokecolor{textcolor}%
\pgfsetfillcolor{textcolor}%
\pgftext[x=0.904128in,y=3.298549in,left,base]{\color{textcolor}\rmfamily\fontsize{10.000000}{12.000000}\selectfont \# Lints}%
\end{pgfscope}%
\begin{pgfscope}%
\pgfsetbuttcap%
\pgfsetroundjoin%
\definecolor{currentfill}{rgb}{0.121569,0.466667,0.705882}%
\pgfsetfillcolor{currentfill}%
\pgfsetlinewidth{1.003750pt}%
\definecolor{currentstroke}{rgb}{0.121569,0.466667,0.705882}%
\pgfsetstrokecolor{currentstroke}%
\pgfsetdash{}{0pt}%
\pgfsys@defobject{currentmarker}{\pgfqpoint{-0.009821in}{-0.009821in}}{\pgfqpoint{0.009821in}{0.009821in}}{%
\pgfpathmoveto{\pgfqpoint{0.000000in}{-0.009821in}}%
\pgfpathcurveto{\pgfqpoint{0.002605in}{-0.009821in}}{\pgfqpoint{0.005103in}{-0.008786in}}{\pgfqpoint{0.006944in}{-0.006944in}}%
\pgfpathcurveto{\pgfqpoint{0.008786in}{-0.005103in}}{\pgfqpoint{0.009821in}{-0.002605in}}{\pgfqpoint{0.009821in}{0.000000in}}%
\pgfpathcurveto{\pgfqpoint{0.009821in}{0.002605in}}{\pgfqpoint{0.008786in}{0.005103in}}{\pgfqpoint{0.006944in}{0.006944in}}%
\pgfpathcurveto{\pgfqpoint{0.005103in}{0.008786in}}{\pgfqpoint{0.002605in}{0.009821in}}{\pgfqpoint{0.000000in}{0.009821in}}%
\pgfpathcurveto{\pgfqpoint{-0.002605in}{0.009821in}}{\pgfqpoint{-0.005103in}{0.008786in}}{\pgfqpoint{-0.006944in}{0.006944in}}%
\pgfpathcurveto{\pgfqpoint{-0.008786in}{0.005103in}}{\pgfqpoint{-0.009821in}{0.002605in}}{\pgfqpoint{-0.009821in}{0.000000in}}%
\pgfpathcurveto{\pgfqpoint{-0.009821in}{-0.002605in}}{\pgfqpoint{-0.008786in}{-0.005103in}}{\pgfqpoint{-0.006944in}{-0.006944in}}%
\pgfpathcurveto{\pgfqpoint{-0.005103in}{-0.008786in}}{\pgfqpoint{-0.002605in}{-0.009821in}}{\pgfqpoint{0.000000in}{-0.009821in}}%
\pgfpathclose%
\pgfusepath{stroke,fill}%
}%
\begin{pgfscope}%
\pgfsys@transformshift{1.013889in}{3.153488in}%
\pgfsys@useobject{currentmarker}{}%
\end{pgfscope}%
\end{pgfscope}%
\begin{pgfscope}%
\definecolor{textcolor}{rgb}{0.000000,0.000000,0.000000}%
\pgfsetstrokecolor{textcolor}%
\pgfsetfillcolor{textcolor}%
\pgftext[x=1.263889in,y=3.104877in,left,base]{\color{textcolor}\rmfamily\fontsize{10.000000}{12.000000}\selectfont \(\displaystyle {0}\)}%
\end{pgfscope}%
\begin{pgfscope}%
\pgfsetbuttcap%
\pgfsetroundjoin%
\definecolor{currentfill}{rgb}{0.121569,0.466667,0.705882}%
\pgfsetfillcolor{currentfill}%
\pgfsetlinewidth{1.003750pt}%
\definecolor{currentstroke}{rgb}{0.121569,0.466667,0.705882}%
\pgfsetstrokecolor{currentstroke}%
\pgfsetdash{}{0pt}%
\pgfsys@defobject{currentmarker}{\pgfqpoint{-0.032572in}{-0.032572in}}{\pgfqpoint{0.032572in}{0.032572in}}{%
\pgfpathmoveto{\pgfqpoint{0.000000in}{-0.032572in}}%
\pgfpathcurveto{\pgfqpoint{0.008638in}{-0.032572in}}{\pgfqpoint{0.016924in}{-0.029140in}}{\pgfqpoint{0.023032in}{-0.023032in}}%
\pgfpathcurveto{\pgfqpoint{0.029140in}{-0.016924in}}{\pgfqpoint{0.032572in}{-0.008638in}}{\pgfqpoint{0.032572in}{0.000000in}}%
\pgfpathcurveto{\pgfqpoint{0.032572in}{0.008638in}}{\pgfqpoint{0.029140in}{0.016924in}}{\pgfqpoint{0.023032in}{0.023032in}}%
\pgfpathcurveto{\pgfqpoint{0.016924in}{0.029140in}}{\pgfqpoint{0.008638in}{0.032572in}}{\pgfqpoint{0.000000in}{0.032572in}}%
\pgfpathcurveto{\pgfqpoint{-0.008638in}{0.032572in}}{\pgfqpoint{-0.016924in}{0.029140in}}{\pgfqpoint{-0.023032in}{0.023032in}}%
\pgfpathcurveto{\pgfqpoint{-0.029140in}{0.016924in}}{\pgfqpoint{-0.032572in}{0.008638in}}{\pgfqpoint{-0.032572in}{0.000000in}}%
\pgfpathcurveto{\pgfqpoint{-0.032572in}{-0.008638in}}{\pgfqpoint{-0.029140in}{-0.016924in}}{\pgfqpoint{-0.023032in}{-0.023032in}}%
\pgfpathcurveto{\pgfqpoint{-0.016924in}{-0.029140in}}{\pgfqpoint{-0.008638in}{-0.032572in}}{\pgfqpoint{0.000000in}{-0.032572in}}%
\pgfpathclose%
\pgfusepath{stroke,fill}%
}%
\begin{pgfscope}%
\pgfsys@transformshift{1.013889in}{2.959815in}%
\pgfsys@useobject{currentmarker}{}%
\end{pgfscope}%
\end{pgfscope}%
\begin{pgfscope}%
\definecolor{textcolor}{rgb}{0.000000,0.000000,0.000000}%
\pgfsetstrokecolor{textcolor}%
\pgfsetfillcolor{textcolor}%
\pgftext[x=1.263889in,y=2.911204in,left,base]{\color{textcolor}\rmfamily\fontsize{10.000000}{12.000000}\selectfont \(\displaystyle {20}\)}%
\end{pgfscope}%
\begin{pgfscope}%
\pgfsetbuttcap%
\pgfsetroundjoin%
\definecolor{currentfill}{rgb}{0.121569,0.466667,0.705882}%
\pgfsetfillcolor{currentfill}%
\pgfsetlinewidth{1.003750pt}%
\definecolor{currentstroke}{rgb}{0.121569,0.466667,0.705882}%
\pgfsetstrokecolor{currentstroke}%
\pgfsetdash{}{0pt}%
\pgfsys@defobject{currentmarker}{\pgfqpoint{-0.045005in}{-0.045005in}}{\pgfqpoint{0.045005in}{0.045005in}}{%
\pgfpathmoveto{\pgfqpoint{0.000000in}{-0.045005in}}%
\pgfpathcurveto{\pgfqpoint{0.011936in}{-0.045005in}}{\pgfqpoint{0.023384in}{-0.040263in}}{\pgfqpoint{0.031823in}{-0.031823in}}%
\pgfpathcurveto{\pgfqpoint{0.040263in}{-0.023384in}}{\pgfqpoint{0.045005in}{-0.011936in}}{\pgfqpoint{0.045005in}{0.000000in}}%
\pgfpathcurveto{\pgfqpoint{0.045005in}{0.011936in}}{\pgfqpoint{0.040263in}{0.023384in}}{\pgfqpoint{0.031823in}{0.031823in}}%
\pgfpathcurveto{\pgfqpoint{0.023384in}{0.040263in}}{\pgfqpoint{0.011936in}{0.045005in}}{\pgfqpoint{0.000000in}{0.045005in}}%
\pgfpathcurveto{\pgfqpoint{-0.011936in}{0.045005in}}{\pgfqpoint{-0.023384in}{0.040263in}}{\pgfqpoint{-0.031823in}{0.031823in}}%
\pgfpathcurveto{\pgfqpoint{-0.040263in}{0.023384in}}{\pgfqpoint{-0.045005in}{0.011936in}}{\pgfqpoint{-0.045005in}{0.000000in}}%
\pgfpathcurveto{\pgfqpoint{-0.045005in}{-0.011936in}}{\pgfqpoint{-0.040263in}{-0.023384in}}{\pgfqpoint{-0.031823in}{-0.031823in}}%
\pgfpathcurveto{\pgfqpoint{-0.023384in}{-0.040263in}}{\pgfqpoint{-0.011936in}{-0.045005in}}{\pgfqpoint{0.000000in}{-0.045005in}}%
\pgfpathclose%
\pgfusepath{stroke,fill}%
}%
\begin{pgfscope}%
\pgfsys@transformshift{1.013889in}{2.766142in}%
\pgfsys@useobject{currentmarker}{}%
\end{pgfscope}%
\end{pgfscope}%
\begin{pgfscope}%
\definecolor{textcolor}{rgb}{0.000000,0.000000,0.000000}%
\pgfsetstrokecolor{textcolor}%
\pgfsetfillcolor{textcolor}%
\pgftext[x=1.263889in,y=2.717531in,left,base]{\color{textcolor}\rmfamily\fontsize{10.000000}{12.000000}\selectfont \(\displaystyle {40}\)}%
\end{pgfscope}%
\end{pgfpicture}%
\makeatother%
\endgroup%

    \caption[Linting time (All theories)]{Time taken to lint a theory depending on its
    length and the number of lints} \label{fig:timing}
\end{figure}

\begin{figure}
    \centering
    %% Creator: Matplotlib, PGF backend
%%
%% To include the figure in your LaTeX document, write
%%   \input{<filename>.pgf}
%%
%% Make sure the required packages are loaded in your preamble
%%   \usepackage{pgf}
%%
%% Figures using additional raster images can only be included by \input if
%% they are in the same directory as the main LaTeX file. For loading figures
%% from other directories you can use the `import` package
%%   \usepackage{import}
%%
%% and then include the figures with
%%   \import{<path to file>}{<filename>.pgf}
%%
%% Matplotlib used the following preamble
%%
\begingroup%
\makeatletter%
\begin{pgfpicture}%
\pgfpathrectangle{\pgfpointorigin}{\pgfqpoint{6.000000in}{4.000000in}}%
\pgfusepath{use as bounding box, clip}%
\begin{pgfscope}%
\pgfsetbuttcap%
\pgfsetmiterjoin%
\pgfsetlinewidth{0.000000pt}%
\definecolor{currentstroke}{rgb}{1.000000,1.000000,1.000000}%
\pgfsetstrokecolor{currentstroke}%
\pgfsetstrokeopacity{0.000000}%
\pgfsetdash{}{0pt}%
\pgfpathmoveto{\pgfqpoint{0.000000in}{0.000000in}}%
\pgfpathlineto{\pgfqpoint{6.000000in}{0.000000in}}%
\pgfpathlineto{\pgfqpoint{6.000000in}{4.000000in}}%
\pgfpathlineto{\pgfqpoint{0.000000in}{4.000000in}}%
\pgfpathclose%
\pgfusepath{}%
\end{pgfscope}%
\begin{pgfscope}%
\pgfsetbuttcap%
\pgfsetmiterjoin%
\definecolor{currentfill}{rgb}{1.000000,1.000000,1.000000}%
\pgfsetfillcolor{currentfill}%
\pgfsetlinewidth{0.000000pt}%
\definecolor{currentstroke}{rgb}{0.000000,0.000000,0.000000}%
\pgfsetstrokecolor{currentstroke}%
\pgfsetstrokeopacity{0.000000}%
\pgfsetdash{}{0pt}%
\pgfpathmoveto{\pgfqpoint{0.750000in}{0.500000in}}%
\pgfpathlineto{\pgfqpoint{5.400000in}{0.500000in}}%
\pgfpathlineto{\pgfqpoint{5.400000in}{3.520000in}}%
\pgfpathlineto{\pgfqpoint{0.750000in}{3.520000in}}%
\pgfpathclose%
\pgfusepath{fill}%
\end{pgfscope}%
\begin{pgfscope}%
\pgfpathrectangle{\pgfqpoint{0.750000in}{0.500000in}}{\pgfqpoint{4.650000in}{3.020000in}}%
\pgfusepath{clip}%
\pgfsetbuttcap%
\pgfsetroundjoin%
\definecolor{currentfill}{rgb}{0.121569,0.466667,0.705882}%
\pgfsetfillcolor{currentfill}%
\pgfsetlinewidth{1.003750pt}%
\definecolor{currentstroke}{rgb}{0.121569,0.466667,0.705882}%
\pgfsetstrokecolor{currentstroke}%
\pgfsetdash{}{0pt}%
\pgfpathmoveto{\pgfqpoint{1.578082in}{1.831930in}}%
\pgfpathcurveto{\pgfqpoint{1.582593in}{1.831930in}}{\pgfqpoint{1.586920in}{1.833722in}}{\pgfqpoint{1.590110in}{1.836912in}}%
\pgfpathcurveto{\pgfqpoint{1.593300in}{1.840102in}}{\pgfqpoint{1.595093in}{1.844429in}}{\pgfqpoint{1.595093in}{1.848940in}}%
\pgfpathcurveto{\pgfqpoint{1.595093in}{1.853452in}}{\pgfqpoint{1.593300in}{1.857779in}}{\pgfqpoint{1.590110in}{1.860969in}}%
\pgfpathcurveto{\pgfqpoint{1.586920in}{1.864158in}}{\pgfqpoint{1.582593in}{1.865951in}}{\pgfqpoint{1.578082in}{1.865951in}}%
\pgfpathcurveto{\pgfqpoint{1.573571in}{1.865951in}}{\pgfqpoint{1.569244in}{1.864158in}}{\pgfqpoint{1.566054in}{1.860969in}}%
\pgfpathcurveto{\pgfqpoint{1.562864in}{1.857779in}}{\pgfqpoint{1.561072in}{1.853452in}}{\pgfqpoint{1.561072in}{1.848940in}}%
\pgfpathcurveto{\pgfqpoint{1.561072in}{1.844429in}}{\pgfqpoint{1.562864in}{1.840102in}}{\pgfqpoint{1.566054in}{1.836912in}}%
\pgfpathcurveto{\pgfqpoint{1.569244in}{1.833722in}}{\pgfqpoint{1.573571in}{1.831930in}}{\pgfqpoint{1.578082in}{1.831930in}}%
\pgfpathclose%
\pgfusepath{stroke,fill}%
\end{pgfscope}%
\begin{pgfscope}%
\pgfpathrectangle{\pgfqpoint{0.750000in}{0.500000in}}{\pgfqpoint{4.650000in}{3.020000in}}%
\pgfusepath{clip}%
\pgfsetbuttcap%
\pgfsetroundjoin%
\definecolor{currentfill}{rgb}{0.121569,0.466667,0.705882}%
\pgfsetfillcolor{currentfill}%
\pgfsetlinewidth{1.003750pt}%
\definecolor{currentstroke}{rgb}{0.121569,0.466667,0.705882}%
\pgfsetstrokecolor{currentstroke}%
\pgfsetdash{}{0pt}%
\pgfpathmoveto{\pgfqpoint{1.056912in}{0.635806in}}%
\pgfpathcurveto{\pgfqpoint{1.060101in}{0.635806in}}{\pgfqpoint{1.063161in}{0.637073in}}{\pgfqpoint{1.065417in}{0.639329in}}%
\pgfpathcurveto{\pgfqpoint{1.067672in}{0.641584in}}{\pgfqpoint{1.068940in}{0.644644in}}{\pgfqpoint{1.068940in}{0.647834in}}%
\pgfpathcurveto{\pgfqpoint{1.068940in}{0.651024in}}{\pgfqpoint{1.067672in}{0.654084in}}{\pgfqpoint{1.065417in}{0.656339in}}%
\pgfpathcurveto{\pgfqpoint{1.063161in}{0.658595in}}{\pgfqpoint{1.060101in}{0.659862in}}{\pgfqpoint{1.056912in}{0.659862in}}%
\pgfpathcurveto{\pgfqpoint{1.053722in}{0.659862in}}{\pgfqpoint{1.050662in}{0.658595in}}{\pgfqpoint{1.048406in}{0.656339in}}%
\pgfpathcurveto{\pgfqpoint{1.046151in}{0.654084in}}{\pgfqpoint{1.044883in}{0.651024in}}{\pgfqpoint{1.044883in}{0.647834in}}%
\pgfpathcurveto{\pgfqpoint{1.044883in}{0.644644in}}{\pgfqpoint{1.046151in}{0.641584in}}{\pgfqpoint{1.048406in}{0.639329in}}%
\pgfpathcurveto{\pgfqpoint{1.050662in}{0.637073in}}{\pgfqpoint{1.053722in}{0.635806in}}{\pgfqpoint{1.056912in}{0.635806in}}%
\pgfpathclose%
\pgfusepath{stroke,fill}%
\end{pgfscope}%
\begin{pgfscope}%
\pgfpathrectangle{\pgfqpoint{0.750000in}{0.500000in}}{\pgfqpoint{4.650000in}{3.020000in}}%
\pgfusepath{clip}%
\pgfsetbuttcap%
\pgfsetroundjoin%
\definecolor{currentfill}{rgb}{0.121569,0.466667,0.705882}%
\pgfsetfillcolor{currentfill}%
\pgfsetlinewidth{1.003750pt}%
\definecolor{currentstroke}{rgb}{0.121569,0.466667,0.705882}%
\pgfsetstrokecolor{currentstroke}%
\pgfsetdash{}{0pt}%
\pgfpathmoveto{\pgfqpoint{3.722120in}{1.907175in}}%
\pgfpathcurveto{\pgfqpoint{3.730356in}{1.907175in}}{\pgfqpoint{3.738257in}{1.910447in}}{\pgfqpoint{3.744080in}{1.916271in}}%
\pgfpathcurveto{\pgfqpoint{3.749904in}{1.922095in}}{\pgfqpoint{3.753177in}{1.929995in}}{\pgfqpoint{3.753177in}{1.938231in}}%
\pgfpathcurveto{\pgfqpoint{3.753177in}{1.946468in}}{\pgfqpoint{3.749904in}{1.954368in}}{\pgfqpoint{3.744080in}{1.960192in}}%
\pgfpathcurveto{\pgfqpoint{3.738257in}{1.966015in}}{\pgfqpoint{3.730356in}{1.969288in}}{\pgfqpoint{3.722120in}{1.969288in}}%
\pgfpathcurveto{\pgfqpoint{3.713884in}{1.969288in}}{\pgfqpoint{3.705984in}{1.966015in}}{\pgfqpoint{3.700160in}{1.960192in}}%
\pgfpathcurveto{\pgfqpoint{3.694336in}{1.954368in}}{\pgfqpoint{3.691064in}{1.946468in}}{\pgfqpoint{3.691064in}{1.938231in}}%
\pgfpathcurveto{\pgfqpoint{3.691064in}{1.929995in}}{\pgfqpoint{3.694336in}{1.922095in}}{\pgfqpoint{3.700160in}{1.916271in}}%
\pgfpathcurveto{\pgfqpoint{3.705984in}{1.910447in}}{\pgfqpoint{3.713884in}{1.907175in}}{\pgfqpoint{3.722120in}{1.907175in}}%
\pgfpathclose%
\pgfusepath{stroke,fill}%
\end{pgfscope}%
\begin{pgfscope}%
\pgfpathrectangle{\pgfqpoint{0.750000in}{0.500000in}}{\pgfqpoint{4.650000in}{3.020000in}}%
\pgfusepath{clip}%
\pgfsetbuttcap%
\pgfsetroundjoin%
\definecolor{currentfill}{rgb}{0.121569,0.466667,0.705882}%
\pgfsetfillcolor{currentfill}%
\pgfsetlinewidth{1.003750pt}%
\definecolor{currentstroke}{rgb}{0.121569,0.466667,0.705882}%
\pgfsetstrokecolor{currentstroke}%
\pgfsetdash{}{0pt}%
\pgfpathmoveto{\pgfqpoint{2.941812in}{1.099036in}}%
\pgfpathcurveto{\pgfqpoint{2.948192in}{1.099036in}}{\pgfqpoint{2.954311in}{1.101570in}}{\pgfqpoint{2.958822in}{1.106082in}}%
\pgfpathcurveto{\pgfqpoint{2.963333in}{1.110593in}}{\pgfqpoint{2.965868in}{1.116712in}}{\pgfqpoint{2.965868in}{1.123092in}}%
\pgfpathcurveto{\pgfqpoint{2.965868in}{1.129472in}}{\pgfqpoint{2.963333in}{1.135591in}}{\pgfqpoint{2.958822in}{1.140102in}}%
\pgfpathcurveto{\pgfqpoint{2.954311in}{1.144613in}}{\pgfqpoint{2.948192in}{1.147148in}}{\pgfqpoint{2.941812in}{1.147148in}}%
\pgfpathcurveto{\pgfqpoint{2.935432in}{1.147148in}}{\pgfqpoint{2.929313in}{1.144613in}}{\pgfqpoint{2.924802in}{1.140102in}}%
\pgfpathcurveto{\pgfqpoint{2.920290in}{1.135591in}}{\pgfqpoint{2.917756in}{1.129472in}}{\pgfqpoint{2.917756in}{1.123092in}}%
\pgfpathcurveto{\pgfqpoint{2.917756in}{1.116712in}}{\pgfqpoint{2.920290in}{1.110593in}}{\pgfqpoint{2.924802in}{1.106082in}}%
\pgfpathcurveto{\pgfqpoint{2.929313in}{1.101570in}}{\pgfqpoint{2.935432in}{1.099036in}}{\pgfqpoint{2.941812in}{1.099036in}}%
\pgfpathclose%
\pgfusepath{stroke,fill}%
\end{pgfscope}%
\begin{pgfscope}%
\pgfpathrectangle{\pgfqpoint{0.750000in}{0.500000in}}{\pgfqpoint{4.650000in}{3.020000in}}%
\pgfusepath{clip}%
\pgfsetbuttcap%
\pgfsetroundjoin%
\definecolor{currentfill}{rgb}{0.121569,0.466667,0.705882}%
\pgfsetfillcolor{currentfill}%
\pgfsetlinewidth{1.003750pt}%
\definecolor{currentstroke}{rgb}{0.121569,0.466667,0.705882}%
\pgfsetstrokecolor{currentstroke}%
\pgfsetdash{}{0pt}%
\pgfpathmoveto{\pgfqpoint{1.151012in}{0.637053in}}%
\pgfpathcurveto{\pgfqpoint{1.153616in}{0.637053in}}{\pgfqpoint{1.156115in}{0.638088in}}{\pgfqpoint{1.157956in}{0.639929in}}%
\pgfpathcurveto{\pgfqpoint{1.159798in}{0.641771in}}{\pgfqpoint{1.160833in}{0.644269in}}{\pgfqpoint{1.160833in}{0.646874in}}%
\pgfpathcurveto{\pgfqpoint{1.160833in}{0.649478in}}{\pgfqpoint{1.159798in}{0.651977in}}{\pgfqpoint{1.157956in}{0.653818in}}%
\pgfpathcurveto{\pgfqpoint{1.156115in}{0.655660in}}{\pgfqpoint{1.153616in}{0.656695in}}{\pgfqpoint{1.151012in}{0.656695in}}%
\pgfpathcurveto{\pgfqpoint{1.148407in}{0.656695in}}{\pgfqpoint{1.145909in}{0.655660in}}{\pgfqpoint{1.144067in}{0.653818in}}%
\pgfpathcurveto{\pgfqpoint{1.142226in}{0.651977in}}{\pgfqpoint{1.141191in}{0.649478in}}{\pgfqpoint{1.141191in}{0.646874in}}%
\pgfpathcurveto{\pgfqpoint{1.141191in}{0.644269in}}{\pgfqpoint{1.142226in}{0.641771in}}{\pgfqpoint{1.144067in}{0.639929in}}%
\pgfpathcurveto{\pgfqpoint{1.145909in}{0.638088in}}{\pgfqpoint{1.148407in}{0.637053in}}{\pgfqpoint{1.151012in}{0.637053in}}%
\pgfpathclose%
\pgfusepath{stroke,fill}%
\end{pgfscope}%
\begin{pgfscope}%
\pgfpathrectangle{\pgfqpoint{0.750000in}{0.500000in}}{\pgfqpoint{4.650000in}{3.020000in}}%
\pgfusepath{clip}%
\pgfsetbuttcap%
\pgfsetroundjoin%
\definecolor{currentfill}{rgb}{0.121569,0.466667,0.705882}%
\pgfsetfillcolor{currentfill}%
\pgfsetlinewidth{1.003750pt}%
\definecolor{currentstroke}{rgb}{0.121569,0.466667,0.705882}%
\pgfsetstrokecolor{currentstroke}%
\pgfsetdash{}{0pt}%
\pgfpathmoveto{\pgfqpoint{2.236784in}{0.856931in}}%
\pgfpathcurveto{\pgfqpoint{2.241295in}{0.856931in}}{\pgfqpoint{2.245622in}{0.858724in}}{\pgfqpoint{2.248812in}{0.861913in}}%
\pgfpathcurveto{\pgfqpoint{2.252002in}{0.865103in}}{\pgfqpoint{2.253794in}{0.869430in}}{\pgfqpoint{2.253794in}{0.873942in}}%
\pgfpathcurveto{\pgfqpoint{2.253794in}{0.878453in}}{\pgfqpoint{2.252002in}{0.882780in}}{\pgfqpoint{2.248812in}{0.885970in}}%
\pgfpathcurveto{\pgfqpoint{2.245622in}{0.889160in}}{\pgfqpoint{2.241295in}{0.890952in}}{\pgfqpoint{2.236784in}{0.890952in}}%
\pgfpathcurveto{\pgfqpoint{2.232273in}{0.890952in}}{\pgfqpoint{2.227946in}{0.889160in}}{\pgfqpoint{2.224756in}{0.885970in}}%
\pgfpathcurveto{\pgfqpoint{2.221566in}{0.882780in}}{\pgfqpoint{2.219774in}{0.878453in}}{\pgfqpoint{2.219774in}{0.873942in}}%
\pgfpathcurveto{\pgfqpoint{2.219774in}{0.869430in}}{\pgfqpoint{2.221566in}{0.865103in}}{\pgfqpoint{2.224756in}{0.861913in}}%
\pgfpathcurveto{\pgfqpoint{2.227946in}{0.858724in}}{\pgfqpoint{2.232273in}{0.856931in}}{\pgfqpoint{2.236784in}{0.856931in}}%
\pgfpathclose%
\pgfusepath{stroke,fill}%
\end{pgfscope}%
\begin{pgfscope}%
\pgfpathrectangle{\pgfqpoint{0.750000in}{0.500000in}}{\pgfqpoint{4.650000in}{3.020000in}}%
\pgfusepath{clip}%
\pgfsetbuttcap%
\pgfsetroundjoin%
\definecolor{currentfill}{rgb}{0.121569,0.466667,0.705882}%
\pgfsetfillcolor{currentfill}%
\pgfsetlinewidth{1.003750pt}%
\definecolor{currentstroke}{rgb}{0.121569,0.466667,0.705882}%
\pgfsetstrokecolor{currentstroke}%
\pgfsetdash{}{0pt}%
\pgfpathmoveto{\pgfqpoint{4.046404in}{1.003426in}}%
\pgfpathcurveto{\pgfqpoint{4.051277in}{1.003426in}}{\pgfqpoint{4.055950in}{1.005362in}}{\pgfqpoint{4.059396in}{1.008808in}}%
\pgfpathcurveto{\pgfqpoint{4.062841in}{1.012253in}}{\pgfqpoint{4.064777in}{1.016927in}}{\pgfqpoint{4.064777in}{1.021800in}}%
\pgfpathcurveto{\pgfqpoint{4.064777in}{1.026672in}}{\pgfqpoint{4.062841in}{1.031346in}}{\pgfqpoint{4.059396in}{1.034791in}}%
\pgfpathcurveto{\pgfqpoint{4.055950in}{1.038237in}}{\pgfqpoint{4.051277in}{1.040173in}}{\pgfqpoint{4.046404in}{1.040173in}}%
\pgfpathcurveto{\pgfqpoint{4.041531in}{1.040173in}}{\pgfqpoint{4.036858in}{1.038237in}}{\pgfqpoint{4.033412in}{1.034791in}}%
\pgfpathcurveto{\pgfqpoint{4.029967in}{1.031346in}}{\pgfqpoint{4.028031in}{1.026672in}}{\pgfqpoint{4.028031in}{1.021800in}}%
\pgfpathcurveto{\pgfqpoint{4.028031in}{1.016927in}}{\pgfqpoint{4.029967in}{1.012253in}}{\pgfqpoint{4.033412in}{1.008808in}}%
\pgfpathcurveto{\pgfqpoint{4.036858in}{1.005362in}}{\pgfqpoint{4.041531in}{1.003426in}}{\pgfqpoint{4.046404in}{1.003426in}}%
\pgfpathclose%
\pgfusepath{stroke,fill}%
\end{pgfscope}%
\begin{pgfscope}%
\pgfpathrectangle{\pgfqpoint{0.750000in}{0.500000in}}{\pgfqpoint{4.650000in}{3.020000in}}%
\pgfusepath{clip}%
\pgfsetbuttcap%
\pgfsetroundjoin%
\definecolor{currentfill}{rgb}{0.121569,0.466667,0.705882}%
\pgfsetfillcolor{currentfill}%
\pgfsetlinewidth{1.003750pt}%
\definecolor{currentstroke}{rgb}{0.121569,0.466667,0.705882}%
\pgfsetstrokecolor{currentstroke}%
\pgfsetdash{}{0pt}%
\pgfpathmoveto{\pgfqpoint{1.381196in}{0.676131in}}%
\pgfpathcurveto{\pgfqpoint{1.384385in}{0.676131in}}{\pgfqpoint{1.387445in}{0.677398in}}{\pgfqpoint{1.389701in}{0.679654in}}%
\pgfpathcurveto{\pgfqpoint{1.391956in}{0.681909in}}{\pgfqpoint{1.393224in}{0.684969in}}{\pgfqpoint{1.393224in}{0.688159in}}%
\pgfpathcurveto{\pgfqpoint{1.393224in}{0.691349in}}{\pgfqpoint{1.391956in}{0.694409in}}{\pgfqpoint{1.389701in}{0.696664in}}%
\pgfpathcurveto{\pgfqpoint{1.387445in}{0.698920in}}{\pgfqpoint{1.384385in}{0.700187in}}{\pgfqpoint{1.381196in}{0.700187in}}%
\pgfpathcurveto{\pgfqpoint{1.378006in}{0.700187in}}{\pgfqpoint{1.374946in}{0.698920in}}{\pgfqpoint{1.372690in}{0.696664in}}%
\pgfpathcurveto{\pgfqpoint{1.370435in}{0.694409in}}{\pgfqpoint{1.369167in}{0.691349in}}{\pgfqpoint{1.369167in}{0.688159in}}%
\pgfpathcurveto{\pgfqpoint{1.369167in}{0.684969in}}{\pgfqpoint{1.370435in}{0.681909in}}{\pgfqpoint{1.372690in}{0.679654in}}%
\pgfpathcurveto{\pgfqpoint{1.374946in}{0.677398in}}{\pgfqpoint{1.378006in}{0.676131in}}{\pgfqpoint{1.381196in}{0.676131in}}%
\pgfpathclose%
\pgfusepath{stroke,fill}%
\end{pgfscope}%
\begin{pgfscope}%
\pgfpathrectangle{\pgfqpoint{0.750000in}{0.500000in}}{\pgfqpoint{4.650000in}{3.020000in}}%
\pgfusepath{clip}%
\pgfsetbuttcap%
\pgfsetroundjoin%
\definecolor{currentfill}{rgb}{0.121569,0.466667,0.705882}%
\pgfsetfillcolor{currentfill}%
\pgfsetlinewidth{1.003750pt}%
\definecolor{currentstroke}{rgb}{0.121569,0.466667,0.705882}%
\pgfsetstrokecolor{currentstroke}%
\pgfsetdash{}{0pt}%
\pgfpathmoveto{\pgfqpoint{1.607036in}{1.968747in}}%
\pgfpathcurveto{\pgfqpoint{1.611547in}{1.968747in}}{\pgfqpoint{1.615874in}{1.970539in}}{\pgfqpoint{1.619064in}{1.973729in}}%
\pgfpathcurveto{\pgfqpoint{1.622254in}{1.976919in}}{\pgfqpoint{1.624046in}{1.981246in}}{\pgfqpoint{1.624046in}{1.985757in}}%
\pgfpathcurveto{\pgfqpoint{1.624046in}{1.990268in}}{\pgfqpoint{1.622254in}{1.994595in}}{\pgfqpoint{1.619064in}{1.997785in}}%
\pgfpathcurveto{\pgfqpoint{1.615874in}{2.000975in}}{\pgfqpoint{1.611547in}{2.002767in}}{\pgfqpoint{1.607036in}{2.002767in}}%
\pgfpathcurveto{\pgfqpoint{1.602525in}{2.002767in}}{\pgfqpoint{1.598198in}{2.000975in}}{\pgfqpoint{1.595008in}{1.997785in}}%
\pgfpathcurveto{\pgfqpoint{1.591818in}{1.994595in}}{\pgfqpoint{1.590026in}{1.990268in}}{\pgfqpoint{1.590026in}{1.985757in}}%
\pgfpathcurveto{\pgfqpoint{1.590026in}{1.981246in}}{\pgfqpoint{1.591818in}{1.976919in}}{\pgfqpoint{1.595008in}{1.973729in}}%
\pgfpathcurveto{\pgfqpoint{1.598198in}{1.970539in}}{\pgfqpoint{1.602525in}{1.968747in}}{\pgfqpoint{1.607036in}{1.968747in}}%
\pgfpathclose%
\pgfusepath{stroke,fill}%
\end{pgfscope}%
\begin{pgfscope}%
\pgfpathrectangle{\pgfqpoint{0.750000in}{0.500000in}}{\pgfqpoint{4.650000in}{3.020000in}}%
\pgfusepath{clip}%
\pgfsetbuttcap%
\pgfsetroundjoin%
\definecolor{currentfill}{rgb}{0.121569,0.466667,0.705882}%
\pgfsetfillcolor{currentfill}%
\pgfsetlinewidth{1.003750pt}%
\definecolor{currentstroke}{rgb}{0.121569,0.466667,0.705882}%
\pgfsetstrokecolor{currentstroke}%
\pgfsetdash{}{0pt}%
\pgfpathmoveto{\pgfqpoint{1.688107in}{1.045671in}}%
\pgfpathcurveto{\pgfqpoint{1.692980in}{1.045671in}}{\pgfqpoint{1.697653in}{1.047607in}}{\pgfqpoint{1.701099in}{1.051053in}}%
\pgfpathcurveto{\pgfqpoint{1.704544in}{1.054498in}}{\pgfqpoint{1.706480in}{1.059172in}}{\pgfqpoint{1.706480in}{1.064045in}}%
\pgfpathcurveto{\pgfqpoint{1.706480in}{1.068917in}}{\pgfqpoint{1.704544in}{1.073591in}}{\pgfqpoint{1.701099in}{1.077037in}}%
\pgfpathcurveto{\pgfqpoint{1.697653in}{1.080482in}}{\pgfqpoint{1.692980in}{1.082418in}}{\pgfqpoint{1.688107in}{1.082418in}}%
\pgfpathcurveto{\pgfqpoint{1.683234in}{1.082418in}}{\pgfqpoint{1.678561in}{1.080482in}}{\pgfqpoint{1.675115in}{1.077037in}}%
\pgfpathcurveto{\pgfqpoint{1.671670in}{1.073591in}}{\pgfqpoint{1.669734in}{1.068917in}}{\pgfqpoint{1.669734in}{1.064045in}}%
\pgfpathcurveto{\pgfqpoint{1.669734in}{1.059172in}}{\pgfqpoint{1.671670in}{1.054498in}}{\pgfqpoint{1.675115in}{1.051053in}}%
\pgfpathcurveto{\pgfqpoint{1.678561in}{1.047607in}}{\pgfqpoint{1.683234in}{1.045671in}}{\pgfqpoint{1.688107in}{1.045671in}}%
\pgfpathclose%
\pgfusepath{stroke,fill}%
\end{pgfscope}%
\begin{pgfscope}%
\pgfpathrectangle{\pgfqpoint{0.750000in}{0.500000in}}{\pgfqpoint{4.650000in}{3.020000in}}%
\pgfusepath{clip}%
\pgfsetbuttcap%
\pgfsetroundjoin%
\definecolor{currentfill}{rgb}{0.121569,0.466667,0.705882}%
\pgfsetfillcolor{currentfill}%
\pgfsetlinewidth{1.003750pt}%
\definecolor{currentstroke}{rgb}{0.121569,0.466667,0.705882}%
\pgfsetstrokecolor{currentstroke}%
\pgfsetdash{}{0pt}%
\pgfpathmoveto{\pgfqpoint{2.286006in}{1.088162in}}%
\pgfpathcurveto{\pgfqpoint{2.295396in}{1.088162in}}{\pgfqpoint{2.304404in}{1.091893in}}{\pgfqpoint{2.311044in}{1.098533in}}%
\pgfpathcurveto{\pgfqpoint{2.317684in}{1.105174in}}{\pgfqpoint{2.321415in}{1.114181in}}{\pgfqpoint{2.321415in}{1.123572in}}%
\pgfpathcurveto{\pgfqpoint{2.321415in}{1.132963in}}{\pgfqpoint{2.317684in}{1.141970in}}{\pgfqpoint{2.311044in}{1.148611in}}%
\pgfpathcurveto{\pgfqpoint{2.304404in}{1.155251in}}{\pgfqpoint{2.295396in}{1.158982in}}{\pgfqpoint{2.286006in}{1.158982in}}%
\pgfpathcurveto{\pgfqpoint{2.276615in}{1.158982in}}{\pgfqpoint{2.267607in}{1.155251in}}{\pgfqpoint{2.260967in}{1.148611in}}%
\pgfpathcurveto{\pgfqpoint{2.254327in}{1.141970in}}{\pgfqpoint{2.250596in}{1.132963in}}{\pgfqpoint{2.250596in}{1.123572in}}%
\pgfpathcurveto{\pgfqpoint{2.250596in}{1.114181in}}{\pgfqpoint{2.254327in}{1.105174in}}{\pgfqpoint{2.260967in}{1.098533in}}%
\pgfpathcurveto{\pgfqpoint{2.267607in}{1.091893in}}{\pgfqpoint{2.276615in}{1.088162in}}{\pgfqpoint{2.286006in}{1.088162in}}%
\pgfpathclose%
\pgfusepath{stroke,fill}%
\end{pgfscope}%
\begin{pgfscope}%
\pgfpathrectangle{\pgfqpoint{0.750000in}{0.500000in}}{\pgfqpoint{4.650000in}{3.020000in}}%
\pgfusepath{clip}%
\pgfsetbuttcap%
\pgfsetroundjoin%
\definecolor{currentfill}{rgb}{0.121569,0.466667,0.705882}%
\pgfsetfillcolor{currentfill}%
\pgfsetlinewidth{1.003750pt}%
\definecolor{currentstroke}{rgb}{0.121569,0.466667,0.705882}%
\pgfsetstrokecolor{currentstroke}%
\pgfsetdash{}{0pt}%
\pgfpathmoveto{\pgfqpoint{1.882098in}{1.167803in}}%
\pgfpathcurveto{\pgfqpoint{1.889465in}{1.167803in}}{\pgfqpoint{1.896531in}{1.170730in}}{\pgfqpoint{1.901740in}{1.175939in}}%
\pgfpathcurveto{\pgfqpoint{1.906949in}{1.181148in}}{\pgfqpoint{1.909876in}{1.188214in}}{\pgfqpoint{1.909876in}{1.195581in}}%
\pgfpathcurveto{\pgfqpoint{1.909876in}{1.202948in}}{\pgfqpoint{1.906949in}{1.210014in}}{\pgfqpoint{1.901740in}{1.215223in}}%
\pgfpathcurveto{\pgfqpoint{1.896531in}{1.220432in}}{\pgfqpoint{1.889465in}{1.223359in}}{\pgfqpoint{1.882098in}{1.223359in}}%
\pgfpathcurveto{\pgfqpoint{1.874732in}{1.223359in}}{\pgfqpoint{1.867666in}{1.220432in}}{\pgfqpoint{1.862457in}{1.215223in}}%
\pgfpathcurveto{\pgfqpoint{1.857247in}{1.210014in}}{\pgfqpoint{1.854321in}{1.202948in}}{\pgfqpoint{1.854321in}{1.195581in}}%
\pgfpathcurveto{\pgfqpoint{1.854321in}{1.188214in}}{\pgfqpoint{1.857247in}{1.181148in}}{\pgfqpoint{1.862457in}{1.175939in}}%
\pgfpathcurveto{\pgfqpoint{1.867666in}{1.170730in}}{\pgfqpoint{1.874732in}{1.167803in}}{\pgfqpoint{1.882098in}{1.167803in}}%
\pgfpathclose%
\pgfusepath{stroke,fill}%
\end{pgfscope}%
\begin{pgfscope}%
\pgfpathrectangle{\pgfqpoint{0.750000in}{0.500000in}}{\pgfqpoint{4.650000in}{3.020000in}}%
\pgfusepath{clip}%
\pgfsetbuttcap%
\pgfsetroundjoin%
\definecolor{currentfill}{rgb}{0.121569,0.466667,0.705882}%
\pgfsetfillcolor{currentfill}%
\pgfsetlinewidth{1.003750pt}%
\definecolor{currentstroke}{rgb}{0.121569,0.466667,0.705882}%
\pgfsetstrokecolor{currentstroke}%
\pgfsetdash{}{0pt}%
\pgfpathmoveto{\pgfqpoint{2.309169in}{1.043703in}}%
\pgfpathcurveto{\pgfqpoint{2.316982in}{1.043703in}}{\pgfqpoint{2.324477in}{1.046807in}}{\pgfqpoint{2.330002in}{1.052333in}}%
\pgfpathcurveto{\pgfqpoint{2.335527in}{1.057858in}}{\pgfqpoint{2.338632in}{1.065352in}}{\pgfqpoint{2.338632in}{1.073166in}}%
\pgfpathcurveto{\pgfqpoint{2.338632in}{1.080979in}}{\pgfqpoint{2.335527in}{1.088474in}}{\pgfqpoint{2.330002in}{1.093999in}}%
\pgfpathcurveto{\pgfqpoint{2.324477in}{1.099524in}}{\pgfqpoint{2.316982in}{1.102629in}}{\pgfqpoint{2.309169in}{1.102629in}}%
\pgfpathcurveto{\pgfqpoint{2.301355in}{1.102629in}}{\pgfqpoint{2.293860in}{1.099524in}}{\pgfqpoint{2.288335in}{1.093999in}}%
\pgfpathcurveto{\pgfqpoint{2.282810in}{1.088474in}}{\pgfqpoint{2.279706in}{1.080979in}}{\pgfqpoint{2.279706in}{1.073166in}}%
\pgfpathcurveto{\pgfqpoint{2.279706in}{1.065352in}}{\pgfqpoint{2.282810in}{1.057858in}}{\pgfqpoint{2.288335in}{1.052333in}}%
\pgfpathcurveto{\pgfqpoint{2.293860in}{1.046807in}}{\pgfqpoint{2.301355in}{1.043703in}}{\pgfqpoint{2.309169in}{1.043703in}}%
\pgfpathclose%
\pgfusepath{stroke,fill}%
\end{pgfscope}%
\begin{pgfscope}%
\pgfpathrectangle{\pgfqpoint{0.750000in}{0.500000in}}{\pgfqpoint{4.650000in}{3.020000in}}%
\pgfusepath{clip}%
\pgfsetbuttcap%
\pgfsetroundjoin%
\definecolor{currentfill}{rgb}{0.121569,0.466667,0.705882}%
\pgfsetfillcolor{currentfill}%
\pgfsetlinewidth{1.003750pt}%
\definecolor{currentstroke}{rgb}{0.121569,0.466667,0.705882}%
\pgfsetstrokecolor{currentstroke}%
\pgfsetdash{}{0pt}%
\pgfpathmoveto{\pgfqpoint{1.436208in}{0.871529in}}%
\pgfpathcurveto{\pgfqpoint{1.443341in}{0.871529in}}{\pgfqpoint{1.450182in}{0.874363in}}{\pgfqpoint{1.455226in}{0.879406in}}%
\pgfpathcurveto{\pgfqpoint{1.460270in}{0.884450in}}{\pgfqpoint{1.463104in}{0.891292in}}{\pgfqpoint{1.463104in}{0.898425in}}%
\pgfpathcurveto{\pgfqpoint{1.463104in}{0.905557in}}{\pgfqpoint{1.460270in}{0.912399in}}{\pgfqpoint{1.455226in}{0.917443in}}%
\pgfpathcurveto{\pgfqpoint{1.450182in}{0.922486in}}{\pgfqpoint{1.443341in}{0.925320in}}{\pgfqpoint{1.436208in}{0.925320in}}%
\pgfpathcurveto{\pgfqpoint{1.429075in}{0.925320in}}{\pgfqpoint{1.422233in}{0.922486in}}{\pgfqpoint{1.417190in}{0.917443in}}%
\pgfpathcurveto{\pgfqpoint{1.412146in}{0.912399in}}{\pgfqpoint{1.409312in}{0.905557in}}{\pgfqpoint{1.409312in}{0.898425in}}%
\pgfpathcurveto{\pgfqpoint{1.409312in}{0.891292in}}{\pgfqpoint{1.412146in}{0.884450in}}{\pgfqpoint{1.417190in}{0.879406in}}%
\pgfpathcurveto{\pgfqpoint{1.422233in}{0.874363in}}{\pgfqpoint{1.429075in}{0.871529in}}{\pgfqpoint{1.436208in}{0.871529in}}%
\pgfpathclose%
\pgfusepath{stroke,fill}%
\end{pgfscope}%
\begin{pgfscope}%
\pgfpathrectangle{\pgfqpoint{0.750000in}{0.500000in}}{\pgfqpoint{4.650000in}{3.020000in}}%
\pgfusepath{clip}%
\pgfsetbuttcap%
\pgfsetroundjoin%
\definecolor{currentfill}{rgb}{0.121569,0.466667,0.705882}%
\pgfsetfillcolor{currentfill}%
\pgfsetlinewidth{1.003750pt}%
\definecolor{currentstroke}{rgb}{0.121569,0.466667,0.705882}%
\pgfsetstrokecolor{currentstroke}%
\pgfsetdash{}{0pt}%
\pgfpathmoveto{\pgfqpoint{2.044240in}{0.923807in}}%
\pgfpathcurveto{\pgfqpoint{2.051131in}{0.923807in}}{\pgfqpoint{2.057741in}{0.926545in}}{\pgfqpoint{2.062614in}{0.931418in}}%
\pgfpathcurveto{\pgfqpoint{2.067486in}{0.936290in}}{\pgfqpoint{2.070224in}{0.942900in}}{\pgfqpoint{2.070224in}{0.949791in}}%
\pgfpathcurveto{\pgfqpoint{2.070224in}{0.956682in}}{\pgfqpoint{2.067486in}{0.963291in}}{\pgfqpoint{2.062614in}{0.968164in}}%
\pgfpathcurveto{\pgfqpoint{2.057741in}{0.973037in}}{\pgfqpoint{2.051131in}{0.975775in}}{\pgfqpoint{2.044240in}{0.975775in}}%
\pgfpathcurveto{\pgfqpoint{2.037349in}{0.975775in}}{\pgfqpoint{2.030740in}{0.973037in}}{\pgfqpoint{2.025867in}{0.968164in}}%
\pgfpathcurveto{\pgfqpoint{2.020994in}{0.963291in}}{\pgfqpoint{2.018257in}{0.956682in}}{\pgfqpoint{2.018257in}{0.949791in}}%
\pgfpathcurveto{\pgfqpoint{2.018257in}{0.942900in}}{\pgfqpoint{2.020994in}{0.936290in}}{\pgfqpoint{2.025867in}{0.931418in}}%
\pgfpathcurveto{\pgfqpoint{2.030740in}{0.926545in}}{\pgfqpoint{2.037349in}{0.923807in}}{\pgfqpoint{2.044240in}{0.923807in}}%
\pgfpathclose%
\pgfusepath{stroke,fill}%
\end{pgfscope}%
\begin{pgfscope}%
\pgfpathrectangle{\pgfqpoint{0.750000in}{0.500000in}}{\pgfqpoint{4.650000in}{3.020000in}}%
\pgfusepath{clip}%
\pgfsetbuttcap%
\pgfsetroundjoin%
\definecolor{currentfill}{rgb}{0.121569,0.466667,0.705882}%
\pgfsetfillcolor{currentfill}%
\pgfsetlinewidth{1.003750pt}%
\definecolor{currentstroke}{rgb}{0.121569,0.466667,0.705882}%
\pgfsetstrokecolor{currentstroke}%
\pgfsetdash{}{0pt}%
\pgfpathmoveto{\pgfqpoint{1.214710in}{0.641373in}}%
\pgfpathcurveto{\pgfqpoint{1.217315in}{0.641373in}}{\pgfqpoint{1.219813in}{0.642408in}}{\pgfqpoint{1.221655in}{0.644250in}}%
\pgfpathcurveto{\pgfqpoint{1.223497in}{0.646092in}}{\pgfqpoint{1.224531in}{0.648590in}}{\pgfqpoint{1.224531in}{0.651194in}}%
\pgfpathcurveto{\pgfqpoint{1.224531in}{0.653799in}}{\pgfqpoint{1.223497in}{0.656297in}}{\pgfqpoint{1.221655in}{0.658139in}}%
\pgfpathcurveto{\pgfqpoint{1.219813in}{0.659981in}}{\pgfqpoint{1.217315in}{0.661015in}}{\pgfqpoint{1.214710in}{0.661015in}}%
\pgfpathcurveto{\pgfqpoint{1.212106in}{0.661015in}}{\pgfqpoint{1.209608in}{0.659981in}}{\pgfqpoint{1.207766in}{0.658139in}}%
\pgfpathcurveto{\pgfqpoint{1.205924in}{0.656297in}}{\pgfqpoint{1.204890in}{0.653799in}}{\pgfqpoint{1.204890in}{0.651194in}}%
\pgfpathcurveto{\pgfqpoint{1.204890in}{0.648590in}}{\pgfqpoint{1.205924in}{0.646092in}}{\pgfqpoint{1.207766in}{0.644250in}}%
\pgfpathcurveto{\pgfqpoint{1.209608in}{0.642408in}}{\pgfqpoint{1.212106in}{0.641373in}}{\pgfqpoint{1.214710in}{0.641373in}}%
\pgfpathclose%
\pgfusepath{stroke,fill}%
\end{pgfscope}%
\begin{pgfscope}%
\pgfpathrectangle{\pgfqpoint{0.750000in}{0.500000in}}{\pgfqpoint{4.650000in}{3.020000in}}%
\pgfusepath{clip}%
\pgfsetbuttcap%
\pgfsetroundjoin%
\definecolor{currentfill}{rgb}{0.121569,0.466667,0.705882}%
\pgfsetfillcolor{currentfill}%
\pgfsetlinewidth{1.003750pt}%
\definecolor{currentstroke}{rgb}{0.121569,0.466667,0.705882}%
\pgfsetstrokecolor{currentstroke}%
\pgfsetdash{}{0pt}%
\pgfpathmoveto{\pgfqpoint{1.219054in}{0.642814in}}%
\pgfpathcurveto{\pgfqpoint{1.221658in}{0.642814in}}{\pgfqpoint{1.224156in}{0.643848in}}{\pgfqpoint{1.225998in}{0.645690in}}%
\pgfpathcurveto{\pgfqpoint{1.227840in}{0.647532in}}{\pgfqpoint{1.228874in}{0.650030in}}{\pgfqpoint{1.228874in}{0.652635in}}%
\pgfpathcurveto{\pgfqpoint{1.228874in}{0.655239in}}{\pgfqpoint{1.227840in}{0.657737in}}{\pgfqpoint{1.225998in}{0.659579in}}%
\pgfpathcurveto{\pgfqpoint{1.224156in}{0.661421in}}{\pgfqpoint{1.221658in}{0.662456in}}{\pgfqpoint{1.219054in}{0.662456in}}%
\pgfpathcurveto{\pgfqpoint{1.216449in}{0.662456in}}{\pgfqpoint{1.213951in}{0.661421in}}{\pgfqpoint{1.212109in}{0.659579in}}%
\pgfpathcurveto{\pgfqpoint{1.210267in}{0.657737in}}{\pgfqpoint{1.209233in}{0.655239in}}{\pgfqpoint{1.209233in}{0.652635in}}%
\pgfpathcurveto{\pgfqpoint{1.209233in}{0.650030in}}{\pgfqpoint{1.210267in}{0.647532in}}{\pgfqpoint{1.212109in}{0.645690in}}%
\pgfpathcurveto{\pgfqpoint{1.213951in}{0.643848in}}{\pgfqpoint{1.216449in}{0.642814in}}{\pgfqpoint{1.219054in}{0.642814in}}%
\pgfpathclose%
\pgfusepath{stroke,fill}%
\end{pgfscope}%
\begin{pgfscope}%
\pgfpathrectangle{\pgfqpoint{0.750000in}{0.500000in}}{\pgfqpoint{4.650000in}{3.020000in}}%
\pgfusepath{clip}%
\pgfsetbuttcap%
\pgfsetroundjoin%
\definecolor{currentfill}{rgb}{0.121569,0.466667,0.705882}%
\pgfsetfillcolor{currentfill}%
\pgfsetlinewidth{1.003750pt}%
\definecolor{currentstroke}{rgb}{0.121569,0.466667,0.705882}%
\pgfsetstrokecolor{currentstroke}%
\pgfsetdash{}{0pt}%
\pgfpathmoveto{\pgfqpoint{1.650467in}{0.676418in}}%
\pgfpathcurveto{\pgfqpoint{1.653072in}{0.676418in}}{\pgfqpoint{1.655570in}{0.677453in}}{\pgfqpoint{1.657411in}{0.679294in}}%
\pgfpathcurveto{\pgfqpoint{1.659253in}{0.681136in}}{\pgfqpoint{1.660288in}{0.683634in}}{\pgfqpoint{1.660288in}{0.686239in}}%
\pgfpathcurveto{\pgfqpoint{1.660288in}{0.688843in}}{\pgfqpoint{1.659253in}{0.691341in}}{\pgfqpoint{1.657411in}{0.693183in}}%
\pgfpathcurveto{\pgfqpoint{1.655570in}{0.695025in}}{\pgfqpoint{1.653072in}{0.696060in}}{\pgfqpoint{1.650467in}{0.696060in}}%
\pgfpathcurveto{\pgfqpoint{1.647862in}{0.696060in}}{\pgfqpoint{1.645364in}{0.695025in}}{\pgfqpoint{1.643523in}{0.693183in}}%
\pgfpathcurveto{\pgfqpoint{1.641681in}{0.691341in}}{\pgfqpoint{1.640646in}{0.688843in}}{\pgfqpoint{1.640646in}{0.686239in}}%
\pgfpathcurveto{\pgfqpoint{1.640646in}{0.683634in}}{\pgfqpoint{1.641681in}{0.681136in}}{\pgfqpoint{1.643523in}{0.679294in}}%
\pgfpathcurveto{\pgfqpoint{1.645364in}{0.677453in}}{\pgfqpoint{1.647862in}{0.676418in}}{\pgfqpoint{1.650467in}{0.676418in}}%
\pgfpathclose%
\pgfusepath{stroke,fill}%
\end{pgfscope}%
\begin{pgfscope}%
\pgfpathrectangle{\pgfqpoint{0.750000in}{0.500000in}}{\pgfqpoint{4.650000in}{3.020000in}}%
\pgfusepath{clip}%
\pgfsetbuttcap%
\pgfsetroundjoin%
\definecolor{currentfill}{rgb}{0.121569,0.466667,0.705882}%
\pgfsetfillcolor{currentfill}%
\pgfsetlinewidth{1.003750pt}%
\definecolor{currentstroke}{rgb}{0.121569,0.466667,0.705882}%
\pgfsetstrokecolor{currentstroke}%
\pgfsetdash{}{0pt}%
\pgfpathmoveto{\pgfqpoint{1.608484in}{0.689092in}}%
\pgfpathcurveto{\pgfqpoint{1.611674in}{0.689092in}}{\pgfqpoint{1.614733in}{0.690360in}}{\pgfqpoint{1.616989in}{0.692615in}}%
\pgfpathcurveto{\pgfqpoint{1.619245in}{0.694871in}}{\pgfqpoint{1.620512in}{0.697931in}}{\pgfqpoint{1.620512in}{0.701121in}}%
\pgfpathcurveto{\pgfqpoint{1.620512in}{0.704310in}}{\pgfqpoint{1.619245in}{0.707370in}}{\pgfqpoint{1.616989in}{0.709626in}}%
\pgfpathcurveto{\pgfqpoint{1.614733in}{0.711881in}}{\pgfqpoint{1.611674in}{0.713149in}}{\pgfqpoint{1.608484in}{0.713149in}}%
\pgfpathcurveto{\pgfqpoint{1.605294in}{0.713149in}}{\pgfqpoint{1.602234in}{0.711881in}}{\pgfqpoint{1.599979in}{0.709626in}}%
\pgfpathcurveto{\pgfqpoint{1.597723in}{0.707370in}}{\pgfqpoint{1.596456in}{0.704310in}}{\pgfqpoint{1.596456in}{0.701121in}}%
\pgfpathcurveto{\pgfqpoint{1.596456in}{0.697931in}}{\pgfqpoint{1.597723in}{0.694871in}}{\pgfqpoint{1.599979in}{0.692615in}}%
\pgfpathcurveto{\pgfqpoint{1.602234in}{0.690360in}}{\pgfqpoint{1.605294in}{0.689092in}}{\pgfqpoint{1.608484in}{0.689092in}}%
\pgfpathclose%
\pgfusepath{stroke,fill}%
\end{pgfscope}%
\begin{pgfscope}%
\pgfpathrectangle{\pgfqpoint{0.750000in}{0.500000in}}{\pgfqpoint{4.650000in}{3.020000in}}%
\pgfusepath{clip}%
\pgfsetbuttcap%
\pgfsetroundjoin%
\definecolor{currentfill}{rgb}{0.121569,0.466667,0.705882}%
\pgfsetfillcolor{currentfill}%
\pgfsetlinewidth{1.003750pt}%
\definecolor{currentstroke}{rgb}{0.121569,0.466667,0.705882}%
\pgfsetstrokecolor{currentstroke}%
\pgfsetdash{}{0pt}%
\pgfpathmoveto{\pgfqpoint{1.100342in}{0.660348in}}%
\pgfpathcurveto{\pgfqpoint{1.104026in}{0.660348in}}{\pgfqpoint{1.107559in}{0.661812in}}{\pgfqpoint{1.110163in}{0.664416in}}%
\pgfpathcurveto{\pgfqpoint{1.112768in}{0.667021in}}{\pgfqpoint{1.114231in}{0.670554in}}{\pgfqpoint{1.114231in}{0.674237in}}%
\pgfpathcurveto{\pgfqpoint{1.114231in}{0.677921in}}{\pgfqpoint{1.112768in}{0.681454in}}{\pgfqpoint{1.110163in}{0.684058in}}%
\pgfpathcurveto{\pgfqpoint{1.107559in}{0.686663in}}{\pgfqpoint{1.104026in}{0.688126in}}{\pgfqpoint{1.100342in}{0.688126in}}%
\pgfpathcurveto{\pgfqpoint{1.096659in}{0.688126in}}{\pgfqpoint{1.093126in}{0.686663in}}{\pgfqpoint{1.090522in}{0.684058in}}%
\pgfpathcurveto{\pgfqpoint{1.087917in}{0.681454in}}{\pgfqpoint{1.086454in}{0.677921in}}{\pgfqpoint{1.086454in}{0.674237in}}%
\pgfpathcurveto{\pgfqpoint{1.086454in}{0.670554in}}{\pgfqpoint{1.087917in}{0.667021in}}{\pgfqpoint{1.090522in}{0.664416in}}%
\pgfpathcurveto{\pgfqpoint{1.093126in}{0.661812in}}{\pgfqpoint{1.096659in}{0.660348in}}{\pgfqpoint{1.100342in}{0.660348in}}%
\pgfpathclose%
\pgfusepath{stroke,fill}%
\end{pgfscope}%
\begin{pgfscope}%
\pgfpathrectangle{\pgfqpoint{0.750000in}{0.500000in}}{\pgfqpoint{4.650000in}{3.020000in}}%
\pgfusepath{clip}%
\pgfsetbuttcap%
\pgfsetroundjoin%
\definecolor{currentfill}{rgb}{0.121569,0.466667,0.705882}%
\pgfsetfillcolor{currentfill}%
\pgfsetlinewidth{1.003750pt}%
\definecolor{currentstroke}{rgb}{0.121569,0.466667,0.705882}%
\pgfsetstrokecolor{currentstroke}%
\pgfsetdash{}{0pt}%
\pgfpathmoveto{\pgfqpoint{4.380822in}{1.540585in}}%
\pgfpathcurveto{\pgfqpoint{4.390030in}{1.540585in}}{\pgfqpoint{4.398863in}{1.544243in}}{\pgfqpoint{4.405374in}{1.550755in}}%
\pgfpathcurveto{\pgfqpoint{4.411886in}{1.557266in}}{\pgfqpoint{4.415544in}{1.566099in}}{\pgfqpoint{4.415544in}{1.575307in}}%
\pgfpathcurveto{\pgfqpoint{4.415544in}{1.584515in}}{\pgfqpoint{4.411886in}{1.593348in}}{\pgfqpoint{4.405374in}{1.599859in}}%
\pgfpathcurveto{\pgfqpoint{4.398863in}{1.606371in}}{\pgfqpoint{4.390030in}{1.610029in}}{\pgfqpoint{4.380822in}{1.610029in}}%
\pgfpathcurveto{\pgfqpoint{4.371613in}{1.610029in}}{\pgfqpoint{4.362781in}{1.606371in}}{\pgfqpoint{4.356270in}{1.599859in}}%
\pgfpathcurveto{\pgfqpoint{4.349758in}{1.593348in}}{\pgfqpoint{4.346100in}{1.584515in}}{\pgfqpoint{4.346100in}{1.575307in}}%
\pgfpathcurveto{\pgfqpoint{4.346100in}{1.566099in}}{\pgfqpoint{4.349758in}{1.557266in}}{\pgfqpoint{4.356270in}{1.550755in}}%
\pgfpathcurveto{\pgfqpoint{4.362781in}{1.544243in}}{\pgfqpoint{4.371613in}{1.540585in}}{\pgfqpoint{4.380822in}{1.540585in}}%
\pgfpathclose%
\pgfusepath{stroke,fill}%
\end{pgfscope}%
\begin{pgfscope}%
\pgfpathrectangle{\pgfqpoint{0.750000in}{0.500000in}}{\pgfqpoint{4.650000in}{3.020000in}}%
\pgfusepath{clip}%
\pgfsetbuttcap%
\pgfsetroundjoin%
\definecolor{currentfill}{rgb}{0.121569,0.466667,0.705882}%
\pgfsetfillcolor{currentfill}%
\pgfsetlinewidth{1.003750pt}%
\definecolor{currentstroke}{rgb}{0.121569,0.466667,0.705882}%
\pgfsetstrokecolor{currentstroke}%
\pgfsetdash{}{0pt}%
\pgfpathmoveto{\pgfqpoint{1.242217in}{0.743626in}}%
\pgfpathcurveto{\pgfqpoint{1.244821in}{0.743626in}}{\pgfqpoint{1.247319in}{0.744661in}}{\pgfqpoint{1.249161in}{0.746502in}}%
\pgfpathcurveto{\pgfqpoint{1.251003in}{0.748344in}}{\pgfqpoint{1.252038in}{0.750842in}}{\pgfqpoint{1.252038in}{0.753447in}}%
\pgfpathcurveto{\pgfqpoint{1.252038in}{0.756051in}}{\pgfqpoint{1.251003in}{0.758550in}}{\pgfqpoint{1.249161in}{0.760391in}}%
\pgfpathcurveto{\pgfqpoint{1.247319in}{0.762233in}}{\pgfqpoint{1.244821in}{0.763268in}}{\pgfqpoint{1.242217in}{0.763268in}}%
\pgfpathcurveto{\pgfqpoint{1.239612in}{0.763268in}}{\pgfqpoint{1.237114in}{0.762233in}}{\pgfqpoint{1.235272in}{0.760391in}}%
\pgfpathcurveto{\pgfqpoint{1.233431in}{0.758550in}}{\pgfqpoint{1.232396in}{0.756051in}}{\pgfqpoint{1.232396in}{0.753447in}}%
\pgfpathcurveto{\pgfqpoint{1.232396in}{0.750842in}}{\pgfqpoint{1.233431in}{0.748344in}}{\pgfqpoint{1.235272in}{0.746502in}}%
\pgfpathcurveto{\pgfqpoint{1.237114in}{0.744661in}}{\pgfqpoint{1.239612in}{0.743626in}}{\pgfqpoint{1.242217in}{0.743626in}}%
\pgfpathclose%
\pgfusepath{stroke,fill}%
\end{pgfscope}%
\begin{pgfscope}%
\pgfpathrectangle{\pgfqpoint{0.750000in}{0.500000in}}{\pgfqpoint{4.650000in}{3.020000in}}%
\pgfusepath{clip}%
\pgfsetbuttcap%
\pgfsetroundjoin%
\definecolor{currentfill}{rgb}{0.121569,0.466667,0.705882}%
\pgfsetfillcolor{currentfill}%
\pgfsetlinewidth{1.003750pt}%
\definecolor{currentstroke}{rgb}{0.121569,0.466667,0.705882}%
\pgfsetstrokecolor{currentstroke}%
\pgfsetdash{}{0pt}%
\pgfpathmoveto{\pgfqpoint{3.444163in}{0.843532in}}%
\pgfpathcurveto{\pgfqpoint{3.448281in}{0.843532in}}{\pgfqpoint{3.452231in}{0.845168in}}{\pgfqpoint{3.455143in}{0.848080in}}%
\pgfpathcurveto{\pgfqpoint{3.458055in}{0.850992in}}{\pgfqpoint{3.459691in}{0.854942in}}{\pgfqpoint{3.459691in}{0.859060in}}%
\pgfpathcurveto{\pgfqpoint{3.459691in}{0.863178in}}{\pgfqpoint{3.458055in}{0.867128in}}{\pgfqpoint{3.455143in}{0.870040in}}%
\pgfpathcurveto{\pgfqpoint{3.452231in}{0.872952in}}{\pgfqpoint{3.448281in}{0.874588in}}{\pgfqpoint{3.444163in}{0.874588in}}%
\pgfpathcurveto{\pgfqpoint{3.440044in}{0.874588in}}{\pgfqpoint{3.436094in}{0.872952in}}{\pgfqpoint{3.433182in}{0.870040in}}%
\pgfpathcurveto{\pgfqpoint{3.430270in}{0.867128in}}{\pgfqpoint{3.428634in}{0.863178in}}{\pgfqpoint{3.428634in}{0.859060in}}%
\pgfpathcurveto{\pgfqpoint{3.428634in}{0.854942in}}{\pgfqpoint{3.430270in}{0.850992in}}{\pgfqpoint{3.433182in}{0.848080in}}%
\pgfpathcurveto{\pgfqpoint{3.436094in}{0.845168in}}{\pgfqpoint{3.440044in}{0.843532in}}{\pgfqpoint{3.444163in}{0.843532in}}%
\pgfpathclose%
\pgfusepath{stroke,fill}%
\end{pgfscope}%
\begin{pgfscope}%
\pgfpathrectangle{\pgfqpoint{0.750000in}{0.500000in}}{\pgfqpoint{4.650000in}{3.020000in}}%
\pgfusepath{clip}%
\pgfsetbuttcap%
\pgfsetroundjoin%
\definecolor{currentfill}{rgb}{0.121569,0.466667,0.705882}%
\pgfsetfillcolor{currentfill}%
\pgfsetlinewidth{1.003750pt}%
\definecolor{currentstroke}{rgb}{0.121569,0.466667,0.705882}%
\pgfsetstrokecolor{currentstroke}%
\pgfsetdash{}{0pt}%
\pgfpathmoveto{\pgfqpoint{1.172727in}{0.638493in}}%
\pgfpathcurveto{\pgfqpoint{1.175332in}{0.638493in}}{\pgfqpoint{1.177830in}{0.639528in}}{\pgfqpoint{1.179672in}{0.641370in}}%
\pgfpathcurveto{\pgfqpoint{1.181513in}{0.643211in}}{\pgfqpoint{1.182548in}{0.645710in}}{\pgfqpoint{1.182548in}{0.648314in}}%
\pgfpathcurveto{\pgfqpoint{1.182548in}{0.650919in}}{\pgfqpoint{1.181513in}{0.653417in}}{\pgfqpoint{1.179672in}{0.655259in}}%
\pgfpathcurveto{\pgfqpoint{1.177830in}{0.657100in}}{\pgfqpoint{1.175332in}{0.658135in}}{\pgfqpoint{1.172727in}{0.658135in}}%
\pgfpathcurveto{\pgfqpoint{1.170123in}{0.658135in}}{\pgfqpoint{1.167625in}{0.657100in}}{\pgfqpoint{1.165783in}{0.655259in}}%
\pgfpathcurveto{\pgfqpoint{1.163941in}{0.653417in}}{\pgfqpoint{1.162906in}{0.650919in}}{\pgfqpoint{1.162906in}{0.648314in}}%
\pgfpathcurveto{\pgfqpoint{1.162906in}{0.645710in}}{\pgfqpoint{1.163941in}{0.643211in}}{\pgfqpoint{1.165783in}{0.641370in}}%
\pgfpathcurveto{\pgfqpoint{1.167625in}{0.639528in}}{\pgfqpoint{1.170123in}{0.638493in}}{\pgfqpoint{1.172727in}{0.638493in}}%
\pgfpathclose%
\pgfusepath{stroke,fill}%
\end{pgfscope}%
\begin{pgfscope}%
\pgfpathrectangle{\pgfqpoint{0.750000in}{0.500000in}}{\pgfqpoint{4.650000in}{3.020000in}}%
\pgfusepath{clip}%
\pgfsetbuttcap%
\pgfsetroundjoin%
\definecolor{currentfill}{rgb}{0.121569,0.466667,0.705882}%
\pgfsetfillcolor{currentfill}%
\pgfsetlinewidth{1.003750pt}%
\definecolor{currentstroke}{rgb}{0.121569,0.466667,0.705882}%
\pgfsetstrokecolor{currentstroke}%
\pgfsetdash{}{0pt}%
\pgfpathmoveto{\pgfqpoint{4.157877in}{3.273314in}}%
\pgfpathcurveto{\pgfqpoint{4.163086in}{3.273314in}}{\pgfqpoint{4.168082in}{3.275384in}}{\pgfqpoint{4.171766in}{3.279067in}}%
\pgfpathcurveto{\pgfqpoint{4.175449in}{3.282751in}}{\pgfqpoint{4.177519in}{3.287747in}}{\pgfqpoint{4.177519in}{3.292956in}}%
\pgfpathcurveto{\pgfqpoint{4.177519in}{3.298165in}}{\pgfqpoint{4.175449in}{3.303162in}}{\pgfqpoint{4.171766in}{3.306845in}}%
\pgfpathcurveto{\pgfqpoint{4.168082in}{3.310529in}}{\pgfqpoint{4.163086in}{3.312598in}}{\pgfqpoint{4.157877in}{3.312598in}}%
\pgfpathcurveto{\pgfqpoint{4.152668in}{3.312598in}}{\pgfqpoint{4.147671in}{3.310529in}}{\pgfqpoint{4.143988in}{3.306845in}}%
\pgfpathcurveto{\pgfqpoint{4.140304in}{3.303162in}}{\pgfqpoint{4.138235in}{3.298165in}}{\pgfqpoint{4.138235in}{3.292956in}}%
\pgfpathcurveto{\pgfqpoint{4.138235in}{3.287747in}}{\pgfqpoint{4.140304in}{3.282751in}}{\pgfqpoint{4.143988in}{3.279067in}}%
\pgfpathcurveto{\pgfqpoint{4.147671in}{3.275384in}}{\pgfqpoint{4.152668in}{3.273314in}}{\pgfqpoint{4.157877in}{3.273314in}}%
\pgfpathclose%
\pgfusepath{stroke,fill}%
\end{pgfscope}%
\begin{pgfscope}%
\pgfpathrectangle{\pgfqpoint{0.750000in}{0.500000in}}{\pgfqpoint{4.650000in}{3.020000in}}%
\pgfusepath{clip}%
\pgfsetbuttcap%
\pgfsetroundjoin%
\definecolor{currentfill}{rgb}{0.121569,0.466667,0.705882}%
\pgfsetfillcolor{currentfill}%
\pgfsetlinewidth{1.003750pt}%
\definecolor{currentstroke}{rgb}{0.121569,0.466667,0.705882}%
\pgfsetstrokecolor{currentstroke}%
\pgfsetdash{}{0pt}%
\pgfpathmoveto{\pgfqpoint{1.792341in}{0.741899in}}%
\pgfpathcurveto{\pgfqpoint{1.795531in}{0.741899in}}{\pgfqpoint{1.798591in}{0.743166in}}{\pgfqpoint{1.800846in}{0.745422in}}%
\pgfpathcurveto{\pgfqpoint{1.803102in}{0.747677in}}{\pgfqpoint{1.804369in}{0.750737in}}{\pgfqpoint{1.804369in}{0.753927in}}%
\pgfpathcurveto{\pgfqpoint{1.804369in}{0.757117in}}{\pgfqpoint{1.803102in}{0.760177in}}{\pgfqpoint{1.800846in}{0.762432in}}%
\pgfpathcurveto{\pgfqpoint{1.798591in}{0.764688in}}{\pgfqpoint{1.795531in}{0.765955in}}{\pgfqpoint{1.792341in}{0.765955in}}%
\pgfpathcurveto{\pgfqpoint{1.789151in}{0.765955in}}{\pgfqpoint{1.786092in}{0.764688in}}{\pgfqpoint{1.783836in}{0.762432in}}%
\pgfpathcurveto{\pgfqpoint{1.781580in}{0.760177in}}{\pgfqpoint{1.780313in}{0.757117in}}{\pgfqpoint{1.780313in}{0.753927in}}%
\pgfpathcurveto{\pgfqpoint{1.780313in}{0.750737in}}{\pgfqpoint{1.781580in}{0.747677in}}{\pgfqpoint{1.783836in}{0.745422in}}%
\pgfpathcurveto{\pgfqpoint{1.786092in}{0.743166in}}{\pgfqpoint{1.789151in}{0.741899in}}{\pgfqpoint{1.792341in}{0.741899in}}%
\pgfpathclose%
\pgfusepath{stroke,fill}%
\end{pgfscope}%
\begin{pgfscope}%
\pgfpathrectangle{\pgfqpoint{0.750000in}{0.500000in}}{\pgfqpoint{4.650000in}{3.020000in}}%
\pgfusepath{clip}%
\pgfsetbuttcap%
\pgfsetroundjoin%
\definecolor{currentfill}{rgb}{0.121569,0.466667,0.705882}%
\pgfsetfillcolor{currentfill}%
\pgfsetlinewidth{1.003750pt}%
\definecolor{currentstroke}{rgb}{0.121569,0.466667,0.705882}%
\pgfsetstrokecolor{currentstroke}%
\pgfsetdash{}{0pt}%
\pgfpathmoveto{\pgfqpoint{2.272976in}{0.734277in}}%
\pgfpathcurveto{\pgfqpoint{2.276660in}{0.734277in}}{\pgfqpoint{2.280193in}{0.735741in}}{\pgfqpoint{2.282797in}{0.738345in}}%
\pgfpathcurveto{\pgfqpoint{2.285402in}{0.740950in}}{\pgfqpoint{2.286865in}{0.744483in}}{\pgfqpoint{2.286865in}{0.748166in}}%
\pgfpathcurveto{\pgfqpoint{2.286865in}{0.751850in}}{\pgfqpoint{2.285402in}{0.755383in}}{\pgfqpoint{2.282797in}{0.757987in}}%
\pgfpathcurveto{\pgfqpoint{2.280193in}{0.760592in}}{\pgfqpoint{2.276660in}{0.762055in}}{\pgfqpoint{2.272976in}{0.762055in}}%
\pgfpathcurveto{\pgfqpoint{2.269293in}{0.762055in}}{\pgfqpoint{2.265760in}{0.760592in}}{\pgfqpoint{2.263155in}{0.757987in}}%
\pgfpathcurveto{\pgfqpoint{2.260551in}{0.755383in}}{\pgfqpoint{2.259087in}{0.751850in}}{\pgfqpoint{2.259087in}{0.748166in}}%
\pgfpathcurveto{\pgfqpoint{2.259087in}{0.744483in}}{\pgfqpoint{2.260551in}{0.740950in}}{\pgfqpoint{2.263155in}{0.738345in}}%
\pgfpathcurveto{\pgfqpoint{2.265760in}{0.735741in}}{\pgfqpoint{2.269293in}{0.734277in}}{\pgfqpoint{2.272976in}{0.734277in}}%
\pgfpathclose%
\pgfusepath{stroke,fill}%
\end{pgfscope}%
\begin{pgfscope}%
\pgfpathrectangle{\pgfqpoint{0.750000in}{0.500000in}}{\pgfqpoint{4.650000in}{3.020000in}}%
\pgfusepath{clip}%
\pgfsetbuttcap%
\pgfsetroundjoin%
\definecolor{currentfill}{rgb}{0.121569,0.466667,0.705882}%
\pgfsetfillcolor{currentfill}%
\pgfsetlinewidth{1.003750pt}%
\definecolor{currentstroke}{rgb}{0.121569,0.466667,0.705882}%
\pgfsetstrokecolor{currentstroke}%
\pgfsetdash{}{0pt}%
\pgfpathmoveto{\pgfqpoint{1.081522in}{0.635133in}}%
\pgfpathcurveto{\pgfqpoint{1.084127in}{0.635133in}}{\pgfqpoint{1.086625in}{0.636168in}}{\pgfqpoint{1.088467in}{0.638009in}}%
\pgfpathcurveto{\pgfqpoint{1.090309in}{0.639851in}}{\pgfqpoint{1.091343in}{0.642349in}}{\pgfqpoint{1.091343in}{0.644954in}}%
\pgfpathcurveto{\pgfqpoint{1.091343in}{0.647558in}}{\pgfqpoint{1.090309in}{0.650056in}}{\pgfqpoint{1.088467in}{0.651898in}}%
\pgfpathcurveto{\pgfqpoint{1.086625in}{0.653740in}}{\pgfqpoint{1.084127in}{0.654775in}}{\pgfqpoint{1.081522in}{0.654775in}}%
\pgfpathcurveto{\pgfqpoint{1.078918in}{0.654775in}}{\pgfqpoint{1.076420in}{0.653740in}}{\pgfqpoint{1.074578in}{0.651898in}}%
\pgfpathcurveto{\pgfqpoint{1.072736in}{0.650056in}}{\pgfqpoint{1.071701in}{0.647558in}}{\pgfqpoint{1.071701in}{0.644954in}}%
\pgfpathcurveto{\pgfqpoint{1.071701in}{0.642349in}}{\pgfqpoint{1.072736in}{0.639851in}}{\pgfqpoint{1.074578in}{0.638009in}}%
\pgfpathcurveto{\pgfqpoint{1.076420in}{0.636168in}}{\pgfqpoint{1.078918in}{0.635133in}}{\pgfqpoint{1.081522in}{0.635133in}}%
\pgfpathclose%
\pgfusepath{stroke,fill}%
\end{pgfscope}%
\begin{pgfscope}%
\pgfpathrectangle{\pgfqpoint{0.750000in}{0.500000in}}{\pgfqpoint{4.650000in}{3.020000in}}%
\pgfusepath{clip}%
\pgfsetbuttcap%
\pgfsetroundjoin%
\definecolor{currentfill}{rgb}{0.121569,0.466667,0.705882}%
\pgfsetfillcolor{currentfill}%
\pgfsetlinewidth{1.003750pt}%
\definecolor{currentstroke}{rgb}{0.121569,0.466667,0.705882}%
\pgfsetstrokecolor{currentstroke}%
\pgfsetdash{}{0pt}%
\pgfpathmoveto{\pgfqpoint{1.256694in}{0.770041in}}%
\pgfpathcurveto{\pgfqpoint{1.261205in}{0.770041in}}{\pgfqpoint{1.265532in}{0.771833in}}{\pgfqpoint{1.268722in}{0.775023in}}%
\pgfpathcurveto{\pgfqpoint{1.271912in}{0.778213in}}{\pgfqpoint{1.273704in}{0.782540in}}{\pgfqpoint{1.273704in}{0.787051in}}%
\pgfpathcurveto{\pgfqpoint{1.273704in}{0.791562in}}{\pgfqpoint{1.271912in}{0.795889in}}{\pgfqpoint{1.268722in}{0.799079in}}%
\pgfpathcurveto{\pgfqpoint{1.265532in}{0.802269in}}{\pgfqpoint{1.261205in}{0.804061in}}{\pgfqpoint{1.256694in}{0.804061in}}%
\pgfpathcurveto{\pgfqpoint{1.252182in}{0.804061in}}{\pgfqpoint{1.247855in}{0.802269in}}{\pgfqpoint{1.244666in}{0.799079in}}%
\pgfpathcurveto{\pgfqpoint{1.241476in}{0.795889in}}{\pgfqpoint{1.239683in}{0.791562in}}{\pgfqpoint{1.239683in}{0.787051in}}%
\pgfpathcurveto{\pgfqpoint{1.239683in}{0.782540in}}{\pgfqpoint{1.241476in}{0.778213in}}{\pgfqpoint{1.244666in}{0.775023in}}%
\pgfpathcurveto{\pgfqpoint{1.247855in}{0.771833in}}{\pgfqpoint{1.252182in}{0.770041in}}{\pgfqpoint{1.256694in}{0.770041in}}%
\pgfpathclose%
\pgfusepath{stroke,fill}%
\end{pgfscope}%
\begin{pgfscope}%
\pgfpathrectangle{\pgfqpoint{0.750000in}{0.500000in}}{\pgfqpoint{4.650000in}{3.020000in}}%
\pgfusepath{clip}%
\pgfsetbuttcap%
\pgfsetroundjoin%
\definecolor{currentfill}{rgb}{0.121569,0.466667,0.705882}%
\pgfsetfillcolor{currentfill}%
\pgfsetlinewidth{1.003750pt}%
\definecolor{currentstroke}{rgb}{0.121569,0.466667,0.705882}%
\pgfsetstrokecolor{currentstroke}%
\pgfsetdash{}{0pt}%
\pgfpathmoveto{\pgfqpoint{1.398568in}{0.996527in}}%
\pgfpathcurveto{\pgfqpoint{1.406161in}{0.996527in}}{\pgfqpoint{1.413445in}{0.999544in}}{\pgfqpoint{1.418814in}{1.004914in}}%
\pgfpathcurveto{\pgfqpoint{1.424184in}{1.010283in}}{\pgfqpoint{1.427201in}{1.017567in}}{\pgfqpoint{1.427201in}{1.025160in}}%
\pgfpathcurveto{\pgfqpoint{1.427201in}{1.032753in}}{\pgfqpoint{1.424184in}{1.040037in}}{\pgfqpoint{1.418814in}{1.045406in}}%
\pgfpathcurveto{\pgfqpoint{1.413445in}{1.050776in}}{\pgfqpoint{1.406161in}{1.053793in}}{\pgfqpoint{1.398568in}{1.053793in}}%
\pgfpathcurveto{\pgfqpoint{1.390974in}{1.053793in}}{\pgfqpoint{1.383691in}{1.050776in}}{\pgfqpoint{1.378322in}{1.045406in}}%
\pgfpathcurveto{\pgfqpoint{1.372952in}{1.040037in}}{\pgfqpoint{1.369935in}{1.032753in}}{\pgfqpoint{1.369935in}{1.025160in}}%
\pgfpathcurveto{\pgfqpoint{1.369935in}{1.017567in}}{\pgfqpoint{1.372952in}{1.010283in}}{\pgfqpoint{1.378322in}{1.004914in}}%
\pgfpathcurveto{\pgfqpoint{1.383691in}{0.999544in}}{\pgfqpoint{1.390974in}{0.996527in}}{\pgfqpoint{1.398568in}{0.996527in}}%
\pgfpathclose%
\pgfusepath{stroke,fill}%
\end{pgfscope}%
\begin{pgfscope}%
\pgfpathrectangle{\pgfqpoint{0.750000in}{0.500000in}}{\pgfqpoint{4.650000in}{3.020000in}}%
\pgfusepath{clip}%
\pgfsetbuttcap%
\pgfsetroundjoin%
\definecolor{currentfill}{rgb}{0.121569,0.466667,0.705882}%
\pgfsetfillcolor{currentfill}%
\pgfsetlinewidth{1.003750pt}%
\definecolor{currentstroke}{rgb}{0.121569,0.466667,0.705882}%
\pgfsetstrokecolor{currentstroke}%
\pgfsetdash{}{0pt}%
\pgfpathmoveto{\pgfqpoint{1.200233in}{0.842127in}}%
\pgfpathcurveto{\pgfqpoint{1.205106in}{0.842127in}}{\pgfqpoint{1.209780in}{0.844063in}}{\pgfqpoint{1.213225in}{0.847508in}}%
\pgfpathcurveto{\pgfqpoint{1.216671in}{0.850954in}}{\pgfqpoint{1.218607in}{0.855627in}}{\pgfqpoint{1.218607in}{0.860500in}}%
\pgfpathcurveto{\pgfqpoint{1.218607in}{0.865373in}}{\pgfqpoint{1.216671in}{0.870046in}}{\pgfqpoint{1.213225in}{0.873492in}}%
\pgfpathcurveto{\pgfqpoint{1.209780in}{0.876937in}}{\pgfqpoint{1.205106in}{0.878873in}}{\pgfqpoint{1.200233in}{0.878873in}}%
\pgfpathcurveto{\pgfqpoint{1.195361in}{0.878873in}}{\pgfqpoint{1.190687in}{0.876937in}}{\pgfqpoint{1.187242in}{0.873492in}}%
\pgfpathcurveto{\pgfqpoint{1.183796in}{0.870046in}}{\pgfqpoint{1.181860in}{0.865373in}}{\pgfqpoint{1.181860in}{0.860500in}}%
\pgfpathcurveto{\pgfqpoint{1.181860in}{0.855627in}}{\pgfqpoint{1.183796in}{0.850954in}}{\pgfqpoint{1.187242in}{0.847508in}}%
\pgfpathcurveto{\pgfqpoint{1.190687in}{0.844063in}}{\pgfqpoint{1.195361in}{0.842127in}}{\pgfqpoint{1.200233in}{0.842127in}}%
\pgfpathclose%
\pgfusepath{stroke,fill}%
\end{pgfscope}%
\begin{pgfscope}%
\pgfpathrectangle{\pgfqpoint{0.750000in}{0.500000in}}{\pgfqpoint{4.650000in}{3.020000in}}%
\pgfusepath{clip}%
\pgfsetbuttcap%
\pgfsetroundjoin%
\definecolor{currentfill}{rgb}{0.121569,0.466667,0.705882}%
\pgfsetfillcolor{currentfill}%
\pgfsetlinewidth{1.003750pt}%
\definecolor{currentstroke}{rgb}{0.121569,0.466667,0.705882}%
\pgfsetstrokecolor{currentstroke}%
\pgfsetdash{}{0pt}%
\pgfpathmoveto{\pgfqpoint{1.006242in}{1.139387in}}%
\pgfpathcurveto{\pgfqpoint{1.009432in}{1.139387in}}{\pgfqpoint{1.012492in}{1.140655in}}{\pgfqpoint{1.014747in}{1.142910in}}%
\pgfpathcurveto{\pgfqpoint{1.017003in}{1.145166in}}{\pgfqpoint{1.018270in}{1.148225in}}{\pgfqpoint{1.018270in}{1.151415in}}%
\pgfpathcurveto{\pgfqpoint{1.018270in}{1.154605in}}{\pgfqpoint{1.017003in}{1.157665in}}{\pgfqpoint{1.014747in}{1.159921in}}%
\pgfpathcurveto{\pgfqpoint{1.012492in}{1.162176in}}{\pgfqpoint{1.009432in}{1.163444in}}{\pgfqpoint{1.006242in}{1.163444in}}%
\pgfpathcurveto{\pgfqpoint{1.003052in}{1.163444in}}{\pgfqpoint{0.999993in}{1.162176in}}{\pgfqpoint{0.997737in}{1.159921in}}%
\pgfpathcurveto{\pgfqpoint{0.995481in}{1.157665in}}{\pgfqpoint{0.994214in}{1.154605in}}{\pgfqpoint{0.994214in}{1.151415in}}%
\pgfpathcurveto{\pgfqpoint{0.994214in}{1.148225in}}{\pgfqpoint{0.995481in}{1.145166in}}{\pgfqpoint{0.997737in}{1.142910in}}%
\pgfpathcurveto{\pgfqpoint{0.999993in}{1.140655in}}{\pgfqpoint{1.003052in}{1.139387in}}{\pgfqpoint{1.006242in}{1.139387in}}%
\pgfpathclose%
\pgfusepath{stroke,fill}%
\end{pgfscope}%
\begin{pgfscope}%
\pgfpathrectangle{\pgfqpoint{0.750000in}{0.500000in}}{\pgfqpoint{4.650000in}{3.020000in}}%
\pgfusepath{clip}%
\pgfsetbuttcap%
\pgfsetroundjoin%
\definecolor{currentfill}{rgb}{0.121569,0.466667,0.705882}%
\pgfsetfillcolor{currentfill}%
\pgfsetlinewidth{1.003750pt}%
\definecolor{currentstroke}{rgb}{0.121569,0.466667,0.705882}%
\pgfsetstrokecolor{currentstroke}%
\pgfsetdash{}{0pt}%
\pgfpathmoveto{\pgfqpoint{1.178518in}{0.863413in}}%
\pgfpathcurveto{\pgfqpoint{1.182201in}{0.863413in}}{\pgfqpoint{1.185734in}{0.864877in}}{\pgfqpoint{1.188339in}{0.867481in}}%
\pgfpathcurveto{\pgfqpoint{1.190944in}{0.870086in}}{\pgfqpoint{1.192407in}{0.873619in}}{\pgfqpoint{1.192407in}{0.877302in}}%
\pgfpathcurveto{\pgfqpoint{1.192407in}{0.880985in}}{\pgfqpoint{1.190944in}{0.884518in}}{\pgfqpoint{1.188339in}{0.887123in}}%
\pgfpathcurveto{\pgfqpoint{1.185734in}{0.889727in}}{\pgfqpoint{1.182201in}{0.891191in}}{\pgfqpoint{1.178518in}{0.891191in}}%
\pgfpathcurveto{\pgfqpoint{1.174835in}{0.891191in}}{\pgfqpoint{1.171302in}{0.889727in}}{\pgfqpoint{1.168697in}{0.887123in}}%
\pgfpathcurveto{\pgfqpoint{1.166093in}{0.884518in}}{\pgfqpoint{1.164629in}{0.880985in}}{\pgfqpoint{1.164629in}{0.877302in}}%
\pgfpathcurveto{\pgfqpoint{1.164629in}{0.873619in}}{\pgfqpoint{1.166093in}{0.870086in}}{\pgfqpoint{1.168697in}{0.867481in}}%
\pgfpathcurveto{\pgfqpoint{1.171302in}{0.864877in}}{\pgfqpoint{1.174835in}{0.863413in}}{\pgfqpoint{1.178518in}{0.863413in}}%
\pgfpathclose%
\pgfusepath{stroke,fill}%
\end{pgfscope}%
\begin{pgfscope}%
\pgfpathrectangle{\pgfqpoint{0.750000in}{0.500000in}}{\pgfqpoint{4.650000in}{3.020000in}}%
\pgfusepath{clip}%
\pgfsetbuttcap%
\pgfsetroundjoin%
\definecolor{currentfill}{rgb}{0.121569,0.466667,0.705882}%
\pgfsetfillcolor{currentfill}%
\pgfsetlinewidth{1.003750pt}%
\definecolor{currentstroke}{rgb}{0.121569,0.466667,0.705882}%
\pgfsetstrokecolor{currentstroke}%
\pgfsetdash{}{0pt}%
\pgfpathmoveto{\pgfqpoint{1.158250in}{0.631292in}}%
\pgfpathcurveto{\pgfqpoint{1.160855in}{0.631292in}}{\pgfqpoint{1.163353in}{0.632327in}}{\pgfqpoint{1.165195in}{0.634169in}}%
\pgfpathcurveto{\pgfqpoint{1.167036in}{0.636010in}}{\pgfqpoint{1.168071in}{0.638509in}}{\pgfqpoint{1.168071in}{0.641113in}}%
\pgfpathcurveto{\pgfqpoint{1.168071in}{0.643718in}}{\pgfqpoint{1.167036in}{0.646216in}}{\pgfqpoint{1.165195in}{0.648058in}}%
\pgfpathcurveto{\pgfqpoint{1.163353in}{0.649899in}}{\pgfqpoint{1.160855in}{0.650934in}}{\pgfqpoint{1.158250in}{0.650934in}}%
\pgfpathcurveto{\pgfqpoint{1.155646in}{0.650934in}}{\pgfqpoint{1.153148in}{0.649899in}}{\pgfqpoint{1.151306in}{0.648058in}}%
\pgfpathcurveto{\pgfqpoint{1.149464in}{0.646216in}}{\pgfqpoint{1.148429in}{0.643718in}}{\pgfqpoint{1.148429in}{0.641113in}}%
\pgfpathcurveto{\pgfqpoint{1.148429in}{0.638509in}}{\pgfqpoint{1.149464in}{0.636010in}}{\pgfqpoint{1.151306in}{0.634169in}}%
\pgfpathcurveto{\pgfqpoint{1.153148in}{0.632327in}}{\pgfqpoint{1.155646in}{0.631292in}}{\pgfqpoint{1.158250in}{0.631292in}}%
\pgfpathclose%
\pgfusepath{stroke,fill}%
\end{pgfscope}%
\begin{pgfscope}%
\pgfpathrectangle{\pgfqpoint{0.750000in}{0.500000in}}{\pgfqpoint{4.650000in}{3.020000in}}%
\pgfusepath{clip}%
\pgfsetbuttcap%
\pgfsetroundjoin%
\definecolor{currentfill}{rgb}{0.121569,0.466667,0.705882}%
\pgfsetfillcolor{currentfill}%
\pgfsetlinewidth{1.003750pt}%
\definecolor{currentstroke}{rgb}{0.121569,0.466667,0.705882}%
\pgfsetstrokecolor{currentstroke}%
\pgfsetdash{}{0pt}%
\pgfpathmoveto{\pgfqpoint{2.168742in}{0.680750in}}%
\pgfpathcurveto{\pgfqpoint{2.173253in}{0.680750in}}{\pgfqpoint{2.177580in}{0.682542in}}{\pgfqpoint{2.180770in}{0.685732in}}%
\pgfpathcurveto{\pgfqpoint{2.183960in}{0.688922in}}{\pgfqpoint{2.185753in}{0.693249in}}{\pgfqpoint{2.185753in}{0.697760in}}%
\pgfpathcurveto{\pgfqpoint{2.185753in}{0.702271in}}{\pgfqpoint{2.183960in}{0.706598in}}{\pgfqpoint{2.180770in}{0.709788in}}%
\pgfpathcurveto{\pgfqpoint{2.177580in}{0.712978in}}{\pgfqpoint{2.173253in}{0.714770in}}{\pgfqpoint{2.168742in}{0.714770in}}%
\pgfpathcurveto{\pgfqpoint{2.164231in}{0.714770in}}{\pgfqpoint{2.159904in}{0.712978in}}{\pgfqpoint{2.156714in}{0.709788in}}%
\pgfpathcurveto{\pgfqpoint{2.153524in}{0.706598in}}{\pgfqpoint{2.151732in}{0.702271in}}{\pgfqpoint{2.151732in}{0.697760in}}%
\pgfpathcurveto{\pgfqpoint{2.151732in}{0.693249in}}{\pgfqpoint{2.153524in}{0.688922in}}{\pgfqpoint{2.156714in}{0.685732in}}%
\pgfpathcurveto{\pgfqpoint{2.159904in}{0.682542in}}{\pgfqpoint{2.164231in}{0.680750in}}{\pgfqpoint{2.168742in}{0.680750in}}%
\pgfpathclose%
\pgfusepath{stroke,fill}%
\end{pgfscope}%
\begin{pgfscope}%
\pgfpathrectangle{\pgfqpoint{0.750000in}{0.500000in}}{\pgfqpoint{4.650000in}{3.020000in}}%
\pgfusepath{clip}%
\pgfsetbuttcap%
\pgfsetroundjoin%
\definecolor{currentfill}{rgb}{0.121569,0.466667,0.705882}%
\pgfsetfillcolor{currentfill}%
\pgfsetlinewidth{1.003750pt}%
\definecolor{currentstroke}{rgb}{0.121569,0.466667,0.705882}%
\pgfsetstrokecolor{currentstroke}%
\pgfsetdash{}{0pt}%
\pgfpathmoveto{\pgfqpoint{0.993213in}{0.631772in}}%
\pgfpathcurveto{\pgfqpoint{0.995817in}{0.631772in}}{\pgfqpoint{0.998316in}{0.632807in}}{\pgfqpoint{1.000157in}{0.634649in}}%
\pgfpathcurveto{\pgfqpoint{1.001999in}{0.636490in}}{\pgfqpoint{1.003034in}{0.638989in}}{\pgfqpoint{1.003034in}{0.641593in}}%
\pgfpathcurveto{\pgfqpoint{1.003034in}{0.644198in}}{\pgfqpoint{1.001999in}{0.646696in}}{\pgfqpoint{1.000157in}{0.648538in}}%
\pgfpathcurveto{\pgfqpoint{0.998316in}{0.650379in}}{\pgfqpoint{0.995817in}{0.651414in}}{\pgfqpoint{0.993213in}{0.651414in}}%
\pgfpathcurveto{\pgfqpoint{0.990608in}{0.651414in}}{\pgfqpoint{0.988110in}{0.650379in}}{\pgfqpoint{0.986269in}{0.648538in}}%
\pgfpathcurveto{\pgfqpoint{0.984427in}{0.646696in}}{\pgfqpoint{0.983392in}{0.644198in}}{\pgfqpoint{0.983392in}{0.641593in}}%
\pgfpathcurveto{\pgfqpoint{0.983392in}{0.638989in}}{\pgfqpoint{0.984427in}{0.636490in}}{\pgfqpoint{0.986269in}{0.634649in}}%
\pgfpathcurveto{\pgfqpoint{0.988110in}{0.632807in}}{\pgfqpoint{0.990608in}{0.631772in}}{\pgfqpoint{0.993213in}{0.631772in}}%
\pgfpathclose%
\pgfusepath{stroke,fill}%
\end{pgfscope}%
\begin{pgfscope}%
\pgfpathrectangle{\pgfqpoint{0.750000in}{0.500000in}}{\pgfqpoint{4.650000in}{3.020000in}}%
\pgfusepath{clip}%
\pgfsetbuttcap%
\pgfsetroundjoin%
\definecolor{currentfill}{rgb}{0.121569,0.466667,0.705882}%
\pgfsetfillcolor{currentfill}%
\pgfsetlinewidth{1.003750pt}%
\definecolor{currentstroke}{rgb}{0.121569,0.466667,0.705882}%
\pgfsetstrokecolor{currentstroke}%
\pgfsetdash{}{0pt}%
\pgfpathmoveto{\pgfqpoint{1.960274in}{0.743339in}}%
\pgfpathcurveto{\pgfqpoint{1.963464in}{0.743339in}}{\pgfqpoint{1.966524in}{0.744606in}}{\pgfqpoint{1.968779in}{0.746862in}}%
\pgfpathcurveto{\pgfqpoint{1.971035in}{0.749118in}}{\pgfqpoint{1.972302in}{0.752177in}}{\pgfqpoint{1.972302in}{0.755367in}}%
\pgfpathcurveto{\pgfqpoint{1.972302in}{0.758557in}}{\pgfqpoint{1.971035in}{0.761617in}}{\pgfqpoint{1.968779in}{0.763872in}}%
\pgfpathcurveto{\pgfqpoint{1.966524in}{0.766128in}}{\pgfqpoint{1.963464in}{0.767395in}}{\pgfqpoint{1.960274in}{0.767395in}}%
\pgfpathcurveto{\pgfqpoint{1.957084in}{0.767395in}}{\pgfqpoint{1.954024in}{0.766128in}}{\pgfqpoint{1.951769in}{0.763872in}}%
\pgfpathcurveto{\pgfqpoint{1.949513in}{0.761617in}}{\pgfqpoint{1.948246in}{0.758557in}}{\pgfqpoint{1.948246in}{0.755367in}}%
\pgfpathcurveto{\pgfqpoint{1.948246in}{0.752177in}}{\pgfqpoint{1.949513in}{0.749118in}}{\pgfqpoint{1.951769in}{0.746862in}}%
\pgfpathcurveto{\pgfqpoint{1.954024in}{0.744606in}}{\pgfqpoint{1.957084in}{0.743339in}}{\pgfqpoint{1.960274in}{0.743339in}}%
\pgfpathclose%
\pgfusepath{stroke,fill}%
\end{pgfscope}%
\begin{pgfscope}%
\pgfpathrectangle{\pgfqpoint{0.750000in}{0.500000in}}{\pgfqpoint{4.650000in}{3.020000in}}%
\pgfusepath{clip}%
\pgfsetbuttcap%
\pgfsetroundjoin%
\definecolor{currentfill}{rgb}{0.121569,0.466667,0.705882}%
\pgfsetfillcolor{currentfill}%
\pgfsetlinewidth{1.003750pt}%
\definecolor{currentstroke}{rgb}{0.121569,0.466667,0.705882}%
\pgfsetstrokecolor{currentstroke}%
\pgfsetdash{}{0pt}%
\pgfpathmoveto{\pgfqpoint{3.403627in}{0.862911in}}%
\pgfpathcurveto{\pgfqpoint{3.409735in}{0.862911in}}{\pgfqpoint{3.415594in}{0.865338in}}{\pgfqpoint{3.419913in}{0.869657in}}%
\pgfpathcurveto{\pgfqpoint{3.424232in}{0.873976in}}{\pgfqpoint{3.426659in}{0.879835in}}{\pgfqpoint{3.426659in}{0.885943in}}%
\pgfpathcurveto{\pgfqpoint{3.426659in}{0.892051in}}{\pgfqpoint{3.424232in}{0.897910in}}{\pgfqpoint{3.419913in}{0.902229in}}%
\pgfpathcurveto{\pgfqpoint{3.415594in}{0.906548in}}{\pgfqpoint{3.409735in}{0.908975in}}{\pgfqpoint{3.403627in}{0.908975in}}%
\pgfpathcurveto{\pgfqpoint{3.397519in}{0.908975in}}{\pgfqpoint{3.391660in}{0.906548in}}{\pgfqpoint{3.387341in}{0.902229in}}%
\pgfpathcurveto{\pgfqpoint{3.383022in}{0.897910in}}{\pgfqpoint{3.380595in}{0.892051in}}{\pgfqpoint{3.380595in}{0.885943in}}%
\pgfpathcurveto{\pgfqpoint{3.380595in}{0.879835in}}{\pgfqpoint{3.383022in}{0.873976in}}{\pgfqpoint{3.387341in}{0.869657in}}%
\pgfpathcurveto{\pgfqpoint{3.391660in}{0.865338in}}{\pgfqpoint{3.397519in}{0.862911in}}{\pgfqpoint{3.403627in}{0.862911in}}%
\pgfpathclose%
\pgfusepath{stroke,fill}%
\end{pgfscope}%
\begin{pgfscope}%
\pgfpathrectangle{\pgfqpoint{0.750000in}{0.500000in}}{\pgfqpoint{4.650000in}{3.020000in}}%
\pgfusepath{clip}%
\pgfsetbuttcap%
\pgfsetroundjoin%
\definecolor{currentfill}{rgb}{0.121569,0.466667,0.705882}%
\pgfsetfillcolor{currentfill}%
\pgfsetlinewidth{1.003750pt}%
\definecolor{currentstroke}{rgb}{0.121569,0.466667,0.705882}%
\pgfsetstrokecolor{currentstroke}%
\pgfsetdash{}{0pt}%
\pgfpathmoveto{\pgfqpoint{1.229187in}{0.645407in}}%
\pgfpathcurveto{\pgfqpoint{1.232377in}{0.645407in}}{\pgfqpoint{1.235437in}{0.646674in}}{\pgfqpoint{1.237693in}{0.648930in}}%
\pgfpathcurveto{\pgfqpoint{1.239948in}{0.651186in}}{\pgfqpoint{1.241216in}{0.654245in}}{\pgfqpoint{1.241216in}{0.657435in}}%
\pgfpathcurveto{\pgfqpoint{1.241216in}{0.660625in}}{\pgfqpoint{1.239948in}{0.663685in}}{\pgfqpoint{1.237693in}{0.665940in}}%
\pgfpathcurveto{\pgfqpoint{1.235437in}{0.668196in}}{\pgfqpoint{1.232377in}{0.669463in}}{\pgfqpoint{1.229187in}{0.669463in}}%
\pgfpathcurveto{\pgfqpoint{1.225998in}{0.669463in}}{\pgfqpoint{1.222938in}{0.668196in}}{\pgfqpoint{1.220682in}{0.665940in}}%
\pgfpathcurveto{\pgfqpoint{1.218427in}{0.663685in}}{\pgfqpoint{1.217159in}{0.660625in}}{\pgfqpoint{1.217159in}{0.657435in}}%
\pgfpathcurveto{\pgfqpoint{1.217159in}{0.654245in}}{\pgfqpoint{1.218427in}{0.651186in}}{\pgfqpoint{1.220682in}{0.648930in}}%
\pgfpathcurveto{\pgfqpoint{1.222938in}{0.646674in}}{\pgfqpoint{1.225998in}{0.645407in}}{\pgfqpoint{1.229187in}{0.645407in}}%
\pgfpathclose%
\pgfusepath{stroke,fill}%
\end{pgfscope}%
\begin{pgfscope}%
\pgfpathrectangle{\pgfqpoint{0.750000in}{0.500000in}}{\pgfqpoint{4.650000in}{3.020000in}}%
\pgfusepath{clip}%
\pgfsetbuttcap%
\pgfsetroundjoin%
\definecolor{currentfill}{rgb}{0.121569,0.466667,0.705882}%
\pgfsetfillcolor{currentfill}%
\pgfsetlinewidth{1.003750pt}%
\definecolor{currentstroke}{rgb}{0.121569,0.466667,0.705882}%
\pgfsetstrokecolor{currentstroke}%
\pgfsetdash{}{0pt}%
\pgfpathmoveto{\pgfqpoint{1.359480in}{0.644734in}}%
\pgfpathcurveto{\pgfqpoint{1.362085in}{0.644734in}}{\pgfqpoint{1.364583in}{0.645769in}}{\pgfqpoint{1.366425in}{0.647610in}}%
\pgfpathcurveto{\pgfqpoint{1.368266in}{0.649452in}}{\pgfqpoint{1.369301in}{0.651950in}}{\pgfqpoint{1.369301in}{0.654555in}}%
\pgfpathcurveto{\pgfqpoint{1.369301in}{0.657159in}}{\pgfqpoint{1.368266in}{0.659658in}}{\pgfqpoint{1.366425in}{0.661499in}}%
\pgfpathcurveto{\pgfqpoint{1.364583in}{0.663341in}}{\pgfqpoint{1.362085in}{0.664376in}}{\pgfqpoint{1.359480in}{0.664376in}}%
\pgfpathcurveto{\pgfqpoint{1.356876in}{0.664376in}}{\pgfqpoint{1.354377in}{0.663341in}}{\pgfqpoint{1.352536in}{0.661499in}}%
\pgfpathcurveto{\pgfqpoint{1.350694in}{0.659658in}}{\pgfqpoint{1.349659in}{0.657159in}}{\pgfqpoint{1.349659in}{0.654555in}}%
\pgfpathcurveto{\pgfqpoint{1.349659in}{0.651950in}}{\pgfqpoint{1.350694in}{0.649452in}}{\pgfqpoint{1.352536in}{0.647610in}}%
\pgfpathcurveto{\pgfqpoint{1.354377in}{0.645769in}}{\pgfqpoint{1.356876in}{0.644734in}}{\pgfqpoint{1.359480in}{0.644734in}}%
\pgfpathclose%
\pgfusepath{stroke,fill}%
\end{pgfscope}%
\begin{pgfscope}%
\pgfpathrectangle{\pgfqpoint{0.750000in}{0.500000in}}{\pgfqpoint{4.650000in}{3.020000in}}%
\pgfusepath{clip}%
\pgfsetbuttcap%
\pgfsetroundjoin%
\definecolor{currentfill}{rgb}{0.121569,0.466667,0.705882}%
\pgfsetfillcolor{currentfill}%
\pgfsetlinewidth{1.003750pt}%
\definecolor{currentstroke}{rgb}{0.121569,0.466667,0.705882}%
\pgfsetstrokecolor{currentstroke}%
\pgfsetdash{}{0pt}%
\pgfpathmoveto{\pgfqpoint{1.385539in}{0.643007in}}%
\pgfpathcurveto{\pgfqpoint{1.388729in}{0.643007in}}{\pgfqpoint{1.391788in}{0.644274in}}{\pgfqpoint{1.394044in}{0.646530in}}%
\pgfpathcurveto{\pgfqpoint{1.396299in}{0.648785in}}{\pgfqpoint{1.397567in}{0.651845in}}{\pgfqpoint{1.397567in}{0.655035in}}%
\pgfpathcurveto{\pgfqpoint{1.397567in}{0.658225in}}{\pgfqpoint{1.396299in}{0.661284in}}{\pgfqpoint{1.394044in}{0.663540in}}%
\pgfpathcurveto{\pgfqpoint{1.391788in}{0.665796in}}{\pgfqpoint{1.388729in}{0.667063in}}{\pgfqpoint{1.385539in}{0.667063in}}%
\pgfpathcurveto{\pgfqpoint{1.382349in}{0.667063in}}{\pgfqpoint{1.379289in}{0.665796in}}{\pgfqpoint{1.377033in}{0.663540in}}%
\pgfpathcurveto{\pgfqpoint{1.374778in}{0.661284in}}{\pgfqpoint{1.373510in}{0.658225in}}{\pgfqpoint{1.373510in}{0.655035in}}%
\pgfpathcurveto{\pgfqpoint{1.373510in}{0.651845in}}{\pgfqpoint{1.374778in}{0.648785in}}{\pgfqpoint{1.377033in}{0.646530in}}%
\pgfpathcurveto{\pgfqpoint{1.379289in}{0.644274in}}{\pgfqpoint{1.382349in}{0.643007in}}{\pgfqpoint{1.385539in}{0.643007in}}%
\pgfpathclose%
\pgfusepath{stroke,fill}%
\end{pgfscope}%
\begin{pgfscope}%
\pgfpathrectangle{\pgfqpoint{0.750000in}{0.500000in}}{\pgfqpoint{4.650000in}{3.020000in}}%
\pgfusepath{clip}%
\pgfsetbuttcap%
\pgfsetroundjoin%
\definecolor{currentfill}{rgb}{0.121569,0.466667,0.705882}%
\pgfsetfillcolor{currentfill}%
\pgfsetlinewidth{1.003750pt}%
\definecolor{currentstroke}{rgb}{0.121569,0.466667,0.705882}%
\pgfsetstrokecolor{currentstroke}%
\pgfsetdash{}{0pt}%
\pgfpathmoveto{\pgfqpoint{1.373957in}{0.641087in}}%
\pgfpathcurveto{\pgfqpoint{1.377147in}{0.641087in}}{\pgfqpoint{1.380207in}{0.642354in}}{\pgfqpoint{1.382462in}{0.644609in}}%
\pgfpathcurveto{\pgfqpoint{1.384718in}{0.646865in}}{\pgfqpoint{1.385985in}{0.649925in}}{\pgfqpoint{1.385985in}{0.653115in}}%
\pgfpathcurveto{\pgfqpoint{1.385985in}{0.656305in}}{\pgfqpoint{1.384718in}{0.659364in}}{\pgfqpoint{1.382462in}{0.661620in}}%
\pgfpathcurveto{\pgfqpoint{1.380207in}{0.663875in}}{\pgfqpoint{1.377147in}{0.665143in}}{\pgfqpoint{1.373957in}{0.665143in}}%
\pgfpathcurveto{\pgfqpoint{1.370767in}{0.665143in}}{\pgfqpoint{1.367707in}{0.663875in}}{\pgfqpoint{1.365452in}{0.661620in}}%
\pgfpathcurveto{\pgfqpoint{1.363196in}{0.659364in}}{\pgfqpoint{1.361929in}{0.656305in}}{\pgfqpoint{1.361929in}{0.653115in}}%
\pgfpathcurveto{\pgfqpoint{1.361929in}{0.649925in}}{\pgfqpoint{1.363196in}{0.646865in}}{\pgfqpoint{1.365452in}{0.644609in}}%
\pgfpathcurveto{\pgfqpoint{1.367707in}{0.642354in}}{\pgfqpoint{1.370767in}{0.641087in}}{\pgfqpoint{1.373957in}{0.641087in}}%
\pgfpathclose%
\pgfusepath{stroke,fill}%
\end{pgfscope}%
\begin{pgfscope}%
\pgfpathrectangle{\pgfqpoint{0.750000in}{0.500000in}}{\pgfqpoint{4.650000in}{3.020000in}}%
\pgfusepath{clip}%
\pgfsetbuttcap%
\pgfsetroundjoin%
\definecolor{currentfill}{rgb}{0.121569,0.466667,0.705882}%
\pgfsetfillcolor{currentfill}%
\pgfsetlinewidth{1.003750pt}%
\definecolor{currentstroke}{rgb}{0.121569,0.466667,0.705882}%
\pgfsetstrokecolor{currentstroke}%
\pgfsetdash{}{0pt}%
\pgfpathmoveto{\pgfqpoint{1.249455in}{0.640413in}}%
\pgfpathcurveto{\pgfqpoint{1.252060in}{0.640413in}}{\pgfqpoint{1.254558in}{0.641448in}}{\pgfqpoint{1.256400in}{0.643290in}}%
\pgfpathcurveto{\pgfqpoint{1.258241in}{0.645132in}}{\pgfqpoint{1.259276in}{0.647630in}}{\pgfqpoint{1.259276in}{0.650234in}}%
\pgfpathcurveto{\pgfqpoint{1.259276in}{0.652839in}}{\pgfqpoint{1.258241in}{0.655337in}}{\pgfqpoint{1.256400in}{0.657179in}}%
\pgfpathcurveto{\pgfqpoint{1.254558in}{0.659020in}}{\pgfqpoint{1.252060in}{0.660055in}}{\pgfqpoint{1.249455in}{0.660055in}}%
\pgfpathcurveto{\pgfqpoint{1.246851in}{0.660055in}}{\pgfqpoint{1.244352in}{0.659020in}}{\pgfqpoint{1.242511in}{0.657179in}}%
\pgfpathcurveto{\pgfqpoint{1.240669in}{0.655337in}}{\pgfqpoint{1.239634in}{0.652839in}}{\pgfqpoint{1.239634in}{0.650234in}}%
\pgfpathcurveto{\pgfqpoint{1.239634in}{0.647630in}}{\pgfqpoint{1.240669in}{0.645132in}}{\pgfqpoint{1.242511in}{0.643290in}}%
\pgfpathcurveto{\pgfqpoint{1.244352in}{0.641448in}}{\pgfqpoint{1.246851in}{0.640413in}}{\pgfqpoint{1.249455in}{0.640413in}}%
\pgfpathclose%
\pgfusepath{stroke,fill}%
\end{pgfscope}%
\begin{pgfscope}%
\pgfpathrectangle{\pgfqpoint{0.750000in}{0.500000in}}{\pgfqpoint{4.650000in}{3.020000in}}%
\pgfusepath{clip}%
\pgfsetbuttcap%
\pgfsetroundjoin%
\definecolor{currentfill}{rgb}{0.121569,0.466667,0.705882}%
\pgfsetfillcolor{currentfill}%
\pgfsetlinewidth{1.003750pt}%
\definecolor{currentstroke}{rgb}{0.121569,0.466667,0.705882}%
\pgfsetstrokecolor{currentstroke}%
\pgfsetdash{}{0pt}%
\pgfpathmoveto{\pgfqpoint{3.361644in}{0.962212in}}%
\pgfpathcurveto{\pgfqpoint{3.368535in}{0.962212in}}{\pgfqpoint{3.375144in}{0.964950in}}{\pgfqpoint{3.380017in}{0.969822in}}%
\pgfpathcurveto{\pgfqpoint{3.384890in}{0.974695in}}{\pgfqpoint{3.387628in}{0.981305in}}{\pgfqpoint{3.387628in}{0.988195in}}%
\pgfpathcurveto{\pgfqpoint{3.387628in}{0.995086in}}{\pgfqpoint{3.384890in}{1.001696in}}{\pgfqpoint{3.380017in}{1.006569in}}%
\pgfpathcurveto{\pgfqpoint{3.375144in}{1.011441in}}{\pgfqpoint{3.368535in}{1.014179in}}{\pgfqpoint{3.361644in}{1.014179in}}%
\pgfpathcurveto{\pgfqpoint{3.354753in}{1.014179in}}{\pgfqpoint{3.348143in}{1.011441in}}{\pgfqpoint{3.343271in}{1.006569in}}%
\pgfpathcurveto{\pgfqpoint{3.338398in}{1.001696in}}{\pgfqpoint{3.335660in}{0.995086in}}{\pgfqpoint{3.335660in}{0.988195in}}%
\pgfpathcurveto{\pgfqpoint{3.335660in}{0.981305in}}{\pgfqpoint{3.338398in}{0.974695in}}{\pgfqpoint{3.343271in}{0.969822in}}%
\pgfpathcurveto{\pgfqpoint{3.348143in}{0.964950in}}{\pgfqpoint{3.354753in}{0.962212in}}{\pgfqpoint{3.361644in}{0.962212in}}%
\pgfpathclose%
\pgfusepath{stroke,fill}%
\end{pgfscope}%
\begin{pgfscope}%
\pgfpathrectangle{\pgfqpoint{0.750000in}{0.500000in}}{\pgfqpoint{4.650000in}{3.020000in}}%
\pgfusepath{clip}%
\pgfsetbuttcap%
\pgfsetroundjoin%
\definecolor{currentfill}{rgb}{0.121569,0.466667,0.705882}%
\pgfsetfillcolor{currentfill}%
\pgfsetlinewidth{1.003750pt}%
\definecolor{currentstroke}{rgb}{0.121569,0.466667,0.705882}%
\pgfsetstrokecolor{currentstroke}%
\pgfsetdash{}{0pt}%
\pgfpathmoveto{\pgfqpoint{2.647930in}{0.788502in}}%
\pgfpathcurveto{\pgfqpoint{2.654038in}{0.788502in}}{\pgfqpoint{2.659897in}{0.790929in}}{\pgfqpoint{2.664216in}{0.795248in}}%
\pgfpathcurveto{\pgfqpoint{2.668535in}{0.799567in}}{\pgfqpoint{2.670962in}{0.805426in}}{\pgfqpoint{2.670962in}{0.811534in}}%
\pgfpathcurveto{\pgfqpoint{2.670962in}{0.817642in}}{\pgfqpoint{2.668535in}{0.823501in}}{\pgfqpoint{2.664216in}{0.827820in}}%
\pgfpathcurveto{\pgfqpoint{2.659897in}{0.832139in}}{\pgfqpoint{2.654038in}{0.834566in}}{\pgfqpoint{2.647930in}{0.834566in}}%
\pgfpathcurveto{\pgfqpoint{2.641821in}{0.834566in}}{\pgfqpoint{2.635963in}{0.832139in}}{\pgfqpoint{2.631643in}{0.827820in}}%
\pgfpathcurveto{\pgfqpoint{2.627324in}{0.823501in}}{\pgfqpoint{2.624898in}{0.817642in}}{\pgfqpoint{2.624898in}{0.811534in}}%
\pgfpathcurveto{\pgfqpoint{2.624898in}{0.805426in}}{\pgfqpoint{2.627324in}{0.799567in}}{\pgfqpoint{2.631643in}{0.795248in}}%
\pgfpathcurveto{\pgfqpoint{2.635963in}{0.790929in}}{\pgfqpoint{2.641821in}{0.788502in}}{\pgfqpoint{2.647930in}{0.788502in}}%
\pgfpathclose%
\pgfusepath{stroke,fill}%
\end{pgfscope}%
\begin{pgfscope}%
\pgfpathrectangle{\pgfqpoint{0.750000in}{0.500000in}}{\pgfqpoint{4.650000in}{3.020000in}}%
\pgfusepath{clip}%
\pgfsetbuttcap%
\pgfsetroundjoin%
\definecolor{currentfill}{rgb}{0.121569,0.466667,0.705882}%
\pgfsetfillcolor{currentfill}%
\pgfsetlinewidth{1.003750pt}%
\definecolor{currentstroke}{rgb}{0.121569,0.466667,0.705882}%
\pgfsetstrokecolor{currentstroke}%
\pgfsetdash{}{0pt}%
\pgfpathmoveto{\pgfqpoint{1.802475in}{0.679387in}}%
\pgfpathcurveto{\pgfqpoint{1.807348in}{0.679387in}}{\pgfqpoint{1.812021in}{0.681323in}}{\pgfqpoint{1.815467in}{0.684768in}}%
\pgfpathcurveto{\pgfqpoint{1.818912in}{0.688214in}}{\pgfqpoint{1.820848in}{0.692887in}}{\pgfqpoint{1.820848in}{0.697760in}}%
\pgfpathcurveto{\pgfqpoint{1.820848in}{0.702633in}}{\pgfqpoint{1.818912in}{0.707306in}}{\pgfqpoint{1.815467in}{0.710752in}}%
\pgfpathcurveto{\pgfqpoint{1.812021in}{0.714197in}}{\pgfqpoint{1.807348in}{0.716133in}}{\pgfqpoint{1.802475in}{0.716133in}}%
\pgfpathcurveto{\pgfqpoint{1.797602in}{0.716133in}}{\pgfqpoint{1.792929in}{0.714197in}}{\pgfqpoint{1.789483in}{0.710752in}}%
\pgfpathcurveto{\pgfqpoint{1.786038in}{0.707306in}}{\pgfqpoint{1.784102in}{0.702633in}}{\pgfqpoint{1.784102in}{0.697760in}}%
\pgfpathcurveto{\pgfqpoint{1.784102in}{0.692887in}}{\pgfqpoint{1.786038in}{0.688214in}}{\pgfqpoint{1.789483in}{0.684768in}}%
\pgfpathcurveto{\pgfqpoint{1.792929in}{0.681323in}}{\pgfqpoint{1.797602in}{0.679387in}}{\pgfqpoint{1.802475in}{0.679387in}}%
\pgfpathclose%
\pgfusepath{stroke,fill}%
\end{pgfscope}%
\begin{pgfscope}%
\pgfpathrectangle{\pgfqpoint{0.750000in}{0.500000in}}{\pgfqpoint{4.650000in}{3.020000in}}%
\pgfusepath{clip}%
\pgfsetbuttcap%
\pgfsetroundjoin%
\definecolor{currentfill}{rgb}{0.121569,0.466667,0.705882}%
\pgfsetfillcolor{currentfill}%
\pgfsetlinewidth{1.003750pt}%
\definecolor{currentstroke}{rgb}{0.121569,0.466667,0.705882}%
\pgfsetstrokecolor{currentstroke}%
\pgfsetdash{}{0pt}%
\pgfpathmoveto{\pgfqpoint{1.120610in}{0.632252in}}%
\pgfpathcurveto{\pgfqpoint{1.123215in}{0.632252in}}{\pgfqpoint{1.125713in}{0.633287in}}{\pgfqpoint{1.127555in}{0.635129in}}%
\pgfpathcurveto{\pgfqpoint{1.129396in}{0.636971in}}{\pgfqpoint{1.130431in}{0.639469in}}{\pgfqpoint{1.130431in}{0.642073in}}%
\pgfpathcurveto{\pgfqpoint{1.130431in}{0.644678in}}{\pgfqpoint{1.129396in}{0.647176in}}{\pgfqpoint{1.127555in}{0.649018in}}%
\pgfpathcurveto{\pgfqpoint{1.125713in}{0.650859in}}{\pgfqpoint{1.123215in}{0.651894in}}{\pgfqpoint{1.120610in}{0.651894in}}%
\pgfpathcurveto{\pgfqpoint{1.118006in}{0.651894in}}{\pgfqpoint{1.115507in}{0.650859in}}{\pgfqpoint{1.113666in}{0.649018in}}%
\pgfpathcurveto{\pgfqpoint{1.111824in}{0.647176in}}{\pgfqpoint{1.110789in}{0.644678in}}{\pgfqpoint{1.110789in}{0.642073in}}%
\pgfpathcurveto{\pgfqpoint{1.110789in}{0.639469in}}{\pgfqpoint{1.111824in}{0.636971in}}{\pgfqpoint{1.113666in}{0.635129in}}%
\pgfpathcurveto{\pgfqpoint{1.115507in}{0.633287in}}{\pgfqpoint{1.118006in}{0.632252in}}{\pgfqpoint{1.120610in}{0.632252in}}%
\pgfpathclose%
\pgfusepath{stroke,fill}%
\end{pgfscope}%
\begin{pgfscope}%
\pgfpathrectangle{\pgfqpoint{0.750000in}{0.500000in}}{\pgfqpoint{4.650000in}{3.020000in}}%
\pgfusepath{clip}%
\pgfsetbuttcap%
\pgfsetroundjoin%
\definecolor{currentfill}{rgb}{0.121569,0.466667,0.705882}%
\pgfsetfillcolor{currentfill}%
\pgfsetlinewidth{1.003750pt}%
\definecolor{currentstroke}{rgb}{0.121569,0.466667,0.705882}%
\pgfsetstrokecolor{currentstroke}%
\pgfsetdash{}{0pt}%
\pgfpathmoveto{\pgfqpoint{1.291438in}{0.649534in}}%
\pgfpathcurveto{\pgfqpoint{1.294043in}{0.649534in}}{\pgfqpoint{1.296541in}{0.650569in}}{\pgfqpoint{1.298383in}{0.652411in}}%
\pgfpathcurveto{\pgfqpoint{1.300224in}{0.654253in}}{\pgfqpoint{1.301259in}{0.656751in}}{\pgfqpoint{1.301259in}{0.659355in}}%
\pgfpathcurveto{\pgfqpoint{1.301259in}{0.661960in}}{\pgfqpoint{1.300224in}{0.664458in}}{\pgfqpoint{1.298383in}{0.666300in}}%
\pgfpathcurveto{\pgfqpoint{1.296541in}{0.668142in}}{\pgfqpoint{1.294043in}{0.669176in}}{\pgfqpoint{1.291438in}{0.669176in}}%
\pgfpathcurveto{\pgfqpoint{1.288834in}{0.669176in}}{\pgfqpoint{1.286336in}{0.668142in}}{\pgfqpoint{1.284494in}{0.666300in}}%
\pgfpathcurveto{\pgfqpoint{1.282652in}{0.664458in}}{\pgfqpoint{1.281617in}{0.661960in}}{\pgfqpoint{1.281617in}{0.659355in}}%
\pgfpathcurveto{\pgfqpoint{1.281617in}{0.656751in}}{\pgfqpoint{1.282652in}{0.654253in}}{\pgfqpoint{1.284494in}{0.652411in}}%
\pgfpathcurveto{\pgfqpoint{1.286336in}{0.650569in}}{\pgfqpoint{1.288834in}{0.649534in}}{\pgfqpoint{1.291438in}{0.649534in}}%
\pgfpathclose%
\pgfusepath{stroke,fill}%
\end{pgfscope}%
\begin{pgfscope}%
\pgfpathrectangle{\pgfqpoint{0.750000in}{0.500000in}}{\pgfqpoint{4.650000in}{3.020000in}}%
\pgfusepath{clip}%
\pgfsetbuttcap%
\pgfsetroundjoin%
\definecolor{currentfill}{rgb}{0.121569,0.466667,0.705882}%
\pgfsetfillcolor{currentfill}%
\pgfsetlinewidth{1.003750pt}%
\definecolor{currentstroke}{rgb}{0.121569,0.466667,0.705882}%
\pgfsetstrokecolor{currentstroke}%
\pgfsetdash{}{0pt}%
\pgfpathmoveto{\pgfqpoint{2.875218in}{0.727357in}}%
\pgfpathcurveto{\pgfqpoint{2.879336in}{0.727357in}}{\pgfqpoint{2.883286in}{0.728994in}}{\pgfqpoint{2.886198in}{0.731905in}}%
\pgfpathcurveto{\pgfqpoint{2.889110in}{0.734817in}}{\pgfqpoint{2.890746in}{0.738767in}}{\pgfqpoint{2.890746in}{0.742886in}}%
\pgfpathcurveto{\pgfqpoint{2.890746in}{0.747004in}}{\pgfqpoint{2.889110in}{0.750954in}}{\pgfqpoint{2.886198in}{0.753866in}}%
\pgfpathcurveto{\pgfqpoint{2.883286in}{0.756778in}}{\pgfqpoint{2.879336in}{0.758414in}}{\pgfqpoint{2.875218in}{0.758414in}}%
\pgfpathcurveto{\pgfqpoint{2.871100in}{0.758414in}}{\pgfqpoint{2.867150in}{0.756778in}}{\pgfqpoint{2.864238in}{0.753866in}}%
\pgfpathcurveto{\pgfqpoint{2.861326in}{0.750954in}}{\pgfqpoint{2.859690in}{0.747004in}}{\pgfqpoint{2.859690in}{0.742886in}}%
\pgfpathcurveto{\pgfqpoint{2.859690in}{0.738767in}}{\pgfqpoint{2.861326in}{0.734817in}}{\pgfqpoint{2.864238in}{0.731905in}}%
\pgfpathcurveto{\pgfqpoint{2.867150in}{0.728994in}}{\pgfqpoint{2.871100in}{0.727357in}}{\pgfqpoint{2.875218in}{0.727357in}}%
\pgfpathclose%
\pgfusepath{stroke,fill}%
\end{pgfscope}%
\begin{pgfscope}%
\pgfpathrectangle{\pgfqpoint{0.750000in}{0.500000in}}{\pgfqpoint{4.650000in}{3.020000in}}%
\pgfusepath{clip}%
\pgfsetbuttcap%
\pgfsetroundjoin%
\definecolor{currentfill}{rgb}{0.121569,0.466667,0.705882}%
\pgfsetfillcolor{currentfill}%
\pgfsetlinewidth{1.003750pt}%
\definecolor{currentstroke}{rgb}{0.121569,0.466667,0.705882}%
\pgfsetstrokecolor{currentstroke}%
\pgfsetdash{}{0pt}%
\pgfpathmoveto{\pgfqpoint{1.177070in}{0.632252in}}%
\pgfpathcurveto{\pgfqpoint{1.179675in}{0.632252in}}{\pgfqpoint{1.182173in}{0.633287in}}{\pgfqpoint{1.184015in}{0.635129in}}%
\pgfpathcurveto{\pgfqpoint{1.185856in}{0.636971in}}{\pgfqpoint{1.186891in}{0.639469in}}{\pgfqpoint{1.186891in}{0.642073in}}%
\pgfpathcurveto{\pgfqpoint{1.186891in}{0.644678in}}{\pgfqpoint{1.185856in}{0.647176in}}{\pgfqpoint{1.184015in}{0.649018in}}%
\pgfpathcurveto{\pgfqpoint{1.182173in}{0.650859in}}{\pgfqpoint{1.179675in}{0.651894in}}{\pgfqpoint{1.177070in}{0.651894in}}%
\pgfpathcurveto{\pgfqpoint{1.174466in}{0.651894in}}{\pgfqpoint{1.171968in}{0.650859in}}{\pgfqpoint{1.170126in}{0.649018in}}%
\pgfpathcurveto{\pgfqpoint{1.168284in}{0.647176in}}{\pgfqpoint{1.167249in}{0.644678in}}{\pgfqpoint{1.167249in}{0.642073in}}%
\pgfpathcurveto{\pgfqpoint{1.167249in}{0.639469in}}{\pgfqpoint{1.168284in}{0.636971in}}{\pgfqpoint{1.170126in}{0.635129in}}%
\pgfpathcurveto{\pgfqpoint{1.171968in}{0.633287in}}{\pgfqpoint{1.174466in}{0.632252in}}{\pgfqpoint{1.177070in}{0.632252in}}%
\pgfpathclose%
\pgfusepath{stroke,fill}%
\end{pgfscope}%
\begin{pgfscope}%
\pgfpathrectangle{\pgfqpoint{0.750000in}{0.500000in}}{\pgfqpoint{4.650000in}{3.020000in}}%
\pgfusepath{clip}%
\pgfsetbuttcap%
\pgfsetroundjoin%
\definecolor{currentfill}{rgb}{0.121569,0.466667,0.705882}%
\pgfsetfillcolor{currentfill}%
\pgfsetlinewidth{1.003750pt}%
\definecolor{currentstroke}{rgb}{0.121569,0.466667,0.705882}%
\pgfsetstrokecolor{currentstroke}%
\pgfsetdash{}{0pt}%
\pgfpathmoveto{\pgfqpoint{2.720314in}{0.772713in}}%
\pgfpathcurveto{\pgfqpoint{2.729337in}{0.772713in}}{\pgfqpoint{2.737991in}{0.776297in}}{\pgfqpoint{2.744371in}{0.782677in}}%
\pgfpathcurveto{\pgfqpoint{2.750751in}{0.789057in}}{\pgfqpoint{2.754335in}{0.797711in}}{\pgfqpoint{2.754335in}{0.806733in}}%
\pgfpathcurveto{\pgfqpoint{2.754335in}{0.815756in}}{\pgfqpoint{2.750751in}{0.824410in}}{\pgfqpoint{2.744371in}{0.830790in}}%
\pgfpathcurveto{\pgfqpoint{2.737991in}{0.837169in}}{\pgfqpoint{2.729337in}{0.840754in}}{\pgfqpoint{2.720314in}{0.840754in}}%
\pgfpathcurveto{\pgfqpoint{2.711292in}{0.840754in}}{\pgfqpoint{2.702638in}{0.837169in}}{\pgfqpoint{2.696258in}{0.830790in}}%
\pgfpathcurveto{\pgfqpoint{2.689878in}{0.824410in}}{\pgfqpoint{2.686294in}{0.815756in}}{\pgfqpoint{2.686294in}{0.806733in}}%
\pgfpathcurveto{\pgfqpoint{2.686294in}{0.797711in}}{\pgfqpoint{2.689878in}{0.789057in}}{\pgfqpoint{2.696258in}{0.782677in}}%
\pgfpathcurveto{\pgfqpoint{2.702638in}{0.776297in}}{\pgfqpoint{2.711292in}{0.772713in}}{\pgfqpoint{2.720314in}{0.772713in}}%
\pgfpathclose%
\pgfusepath{stroke,fill}%
\end{pgfscope}%
\begin{pgfscope}%
\pgfpathrectangle{\pgfqpoint{0.750000in}{0.500000in}}{\pgfqpoint{4.650000in}{3.020000in}}%
\pgfusepath{clip}%
\pgfsetbuttcap%
\pgfsetroundjoin%
\definecolor{currentfill}{rgb}{0.121569,0.466667,0.705882}%
\pgfsetfillcolor{currentfill}%
\pgfsetlinewidth{1.003750pt}%
\definecolor{currentstroke}{rgb}{0.121569,0.466667,0.705882}%
\pgfsetstrokecolor{currentstroke}%
\pgfsetdash{}{0pt}%
\pgfpathmoveto{\pgfqpoint{2.928783in}{0.803207in}}%
\pgfpathcurveto{\pgfqpoint{2.932901in}{0.803207in}}{\pgfqpoint{2.936851in}{0.804843in}}{\pgfqpoint{2.939763in}{0.807755in}}%
\pgfpathcurveto{\pgfqpoint{2.942675in}{0.810667in}}{\pgfqpoint{2.944311in}{0.814617in}}{\pgfqpoint{2.944311in}{0.818735in}}%
\pgfpathcurveto{\pgfqpoint{2.944311in}{0.822853in}}{\pgfqpoint{2.942675in}{0.826803in}}{\pgfqpoint{2.939763in}{0.829715in}}%
\pgfpathcurveto{\pgfqpoint{2.936851in}{0.832627in}}{\pgfqpoint{2.932901in}{0.834263in}}{\pgfqpoint{2.928783in}{0.834263in}}%
\pgfpathcurveto{\pgfqpoint{2.924665in}{0.834263in}}{\pgfqpoint{2.920715in}{0.832627in}}{\pgfqpoint{2.917803in}{0.829715in}}%
\pgfpathcurveto{\pgfqpoint{2.914891in}{0.826803in}}{\pgfqpoint{2.913254in}{0.822853in}}{\pgfqpoint{2.913254in}{0.818735in}}%
\pgfpathcurveto{\pgfqpoint{2.913254in}{0.814617in}}{\pgfqpoint{2.914891in}{0.810667in}}{\pgfqpoint{2.917803in}{0.807755in}}%
\pgfpathcurveto{\pgfqpoint{2.920715in}{0.804843in}}{\pgfqpoint{2.924665in}{0.803207in}}{\pgfqpoint{2.928783in}{0.803207in}}%
\pgfpathclose%
\pgfusepath{stroke,fill}%
\end{pgfscope}%
\begin{pgfscope}%
\pgfpathrectangle{\pgfqpoint{0.750000in}{0.500000in}}{\pgfqpoint{4.650000in}{3.020000in}}%
\pgfusepath{clip}%
\pgfsetbuttcap%
\pgfsetroundjoin%
\definecolor{currentfill}{rgb}{0.121569,0.466667,0.705882}%
\pgfsetfillcolor{currentfill}%
\pgfsetlinewidth{1.003750pt}%
\definecolor{currentstroke}{rgb}{0.121569,0.466667,0.705882}%
\pgfsetstrokecolor{currentstroke}%
\pgfsetdash{}{0pt}%
\pgfpathmoveto{\pgfqpoint{1.505697in}{0.675710in}}%
\pgfpathcurveto{\pgfqpoint{1.509381in}{0.675710in}}{\pgfqpoint{1.512914in}{0.677174in}}{\pgfqpoint{1.515518in}{0.679778in}}%
\pgfpathcurveto{\pgfqpoint{1.518123in}{0.682383in}}{\pgfqpoint{1.519586in}{0.685916in}}{\pgfqpoint{1.519586in}{0.689599in}}%
\pgfpathcurveto{\pgfqpoint{1.519586in}{0.693282in}}{\pgfqpoint{1.518123in}{0.696815in}}{\pgfqpoint{1.515518in}{0.699420in}}%
\pgfpathcurveto{\pgfqpoint{1.512914in}{0.702025in}}{\pgfqpoint{1.509381in}{0.703488in}}{\pgfqpoint{1.505697in}{0.703488in}}%
\pgfpathcurveto{\pgfqpoint{1.502014in}{0.703488in}}{\pgfqpoint{1.498481in}{0.702025in}}{\pgfqpoint{1.495876in}{0.699420in}}%
\pgfpathcurveto{\pgfqpoint{1.493272in}{0.696815in}}{\pgfqpoint{1.491808in}{0.693282in}}{\pgfqpoint{1.491808in}{0.689599in}}%
\pgfpathcurveto{\pgfqpoint{1.491808in}{0.685916in}}{\pgfqpoint{1.493272in}{0.682383in}}{\pgfqpoint{1.495876in}{0.679778in}}%
\pgfpathcurveto{\pgfqpoint{1.498481in}{0.677174in}}{\pgfqpoint{1.502014in}{0.675710in}}{\pgfqpoint{1.505697in}{0.675710in}}%
\pgfpathclose%
\pgfusepath{stroke,fill}%
\end{pgfscope}%
\begin{pgfscope}%
\pgfpathrectangle{\pgfqpoint{0.750000in}{0.500000in}}{\pgfqpoint{4.650000in}{3.020000in}}%
\pgfusepath{clip}%
\pgfsetbuttcap%
\pgfsetroundjoin%
\definecolor{currentfill}{rgb}{0.121569,0.466667,0.705882}%
\pgfsetfillcolor{currentfill}%
\pgfsetlinewidth{1.003750pt}%
\definecolor{currentstroke}{rgb}{0.121569,0.466667,0.705882}%
\pgfsetstrokecolor{currentstroke}%
\pgfsetdash{}{0pt}%
\pgfpathmoveto{\pgfqpoint{2.854950in}{0.789723in}}%
\pgfpathcurveto{\pgfqpoint{2.859461in}{0.789723in}}{\pgfqpoint{2.863788in}{0.791515in}}{\pgfqpoint{2.866978in}{0.794705in}}%
\pgfpathcurveto{\pgfqpoint{2.870168in}{0.797895in}}{\pgfqpoint{2.871961in}{0.802222in}}{\pgfqpoint{2.871961in}{0.806733in}}%
\pgfpathcurveto{\pgfqpoint{2.871961in}{0.811245in}}{\pgfqpoint{2.870168in}{0.815572in}}{\pgfqpoint{2.866978in}{0.818762in}}%
\pgfpathcurveto{\pgfqpoint{2.863788in}{0.821951in}}{\pgfqpoint{2.859461in}{0.823744in}}{\pgfqpoint{2.854950in}{0.823744in}}%
\pgfpathcurveto{\pgfqpoint{2.850439in}{0.823744in}}{\pgfqpoint{2.846112in}{0.821951in}}{\pgfqpoint{2.842922in}{0.818762in}}%
\pgfpathcurveto{\pgfqpoint{2.839732in}{0.815572in}}{\pgfqpoint{2.837940in}{0.811245in}}{\pgfqpoint{2.837940in}{0.806733in}}%
\pgfpathcurveto{\pgfqpoint{2.837940in}{0.802222in}}{\pgfqpoint{2.839732in}{0.797895in}}{\pgfqpoint{2.842922in}{0.794705in}}%
\pgfpathcurveto{\pgfqpoint{2.846112in}{0.791515in}}{\pgfqpoint{2.850439in}{0.789723in}}{\pgfqpoint{2.854950in}{0.789723in}}%
\pgfpathclose%
\pgfusepath{stroke,fill}%
\end{pgfscope}%
\begin{pgfscope}%
\pgfpathrectangle{\pgfqpoint{0.750000in}{0.500000in}}{\pgfqpoint{4.650000in}{3.020000in}}%
\pgfusepath{clip}%
\pgfsetbuttcap%
\pgfsetroundjoin%
\definecolor{currentfill}{rgb}{0.121569,0.466667,0.705882}%
\pgfsetfillcolor{currentfill}%
\pgfsetlinewidth{1.003750pt}%
\definecolor{currentstroke}{rgb}{0.121569,0.466667,0.705882}%
\pgfsetstrokecolor{currentstroke}%
\pgfsetdash{}{0pt}%
\pgfpathmoveto{\pgfqpoint{2.029763in}{0.688508in}}%
\pgfpathcurveto{\pgfqpoint{2.034636in}{0.688508in}}{\pgfqpoint{2.039310in}{0.690444in}}{\pgfqpoint{2.042755in}{0.693889in}}%
\pgfpathcurveto{\pgfqpoint{2.046201in}{0.697335in}}{\pgfqpoint{2.048137in}{0.702009in}}{\pgfqpoint{2.048137in}{0.706881in}}%
\pgfpathcurveto{\pgfqpoint{2.048137in}{0.711754in}}{\pgfqpoint{2.046201in}{0.716428in}}{\pgfqpoint{2.042755in}{0.719873in}}%
\pgfpathcurveto{\pgfqpoint{2.039310in}{0.723319in}}{\pgfqpoint{2.034636in}{0.725254in}}{\pgfqpoint{2.029763in}{0.725254in}}%
\pgfpathcurveto{\pgfqpoint{2.024891in}{0.725254in}}{\pgfqpoint{2.020217in}{0.723319in}}{\pgfqpoint{2.016772in}{0.719873in}}%
\pgfpathcurveto{\pgfqpoint{2.013326in}{0.716428in}}{\pgfqpoint{2.011390in}{0.711754in}}{\pgfqpoint{2.011390in}{0.706881in}}%
\pgfpathcurveto{\pgfqpoint{2.011390in}{0.702009in}}{\pgfqpoint{2.013326in}{0.697335in}}{\pgfqpoint{2.016772in}{0.693889in}}%
\pgfpathcurveto{\pgfqpoint{2.020217in}{0.690444in}}{\pgfqpoint{2.024891in}{0.688508in}}{\pgfqpoint{2.029763in}{0.688508in}}%
\pgfpathclose%
\pgfusepath{stroke,fill}%
\end{pgfscope}%
\begin{pgfscope}%
\pgfpathrectangle{\pgfqpoint{0.750000in}{0.500000in}}{\pgfqpoint{4.650000in}{3.020000in}}%
\pgfusepath{clip}%
\pgfsetbuttcap%
\pgfsetroundjoin%
\definecolor{currentfill}{rgb}{0.121569,0.466667,0.705882}%
\pgfsetfillcolor{currentfill}%
\pgfsetlinewidth{1.003750pt}%
\definecolor{currentstroke}{rgb}{0.121569,0.466667,0.705882}%
\pgfsetstrokecolor{currentstroke}%
\pgfsetdash{}{0pt}%
\pgfpathmoveto{\pgfqpoint{1.025062in}{0.645214in}}%
\pgfpathcurveto{\pgfqpoint{1.027667in}{0.645214in}}{\pgfqpoint{1.030165in}{0.646249in}}{\pgfqpoint{1.032007in}{0.648090in}}%
\pgfpathcurveto{\pgfqpoint{1.033848in}{0.649932in}}{\pgfqpoint{1.034883in}{0.652430in}}{\pgfqpoint{1.034883in}{0.655035in}}%
\pgfpathcurveto{\pgfqpoint{1.034883in}{0.657639in}}{\pgfqpoint{1.033848in}{0.660138in}}{\pgfqpoint{1.032007in}{0.661979in}}%
\pgfpathcurveto{\pgfqpoint{1.030165in}{0.663821in}}{\pgfqpoint{1.027667in}{0.664856in}}{\pgfqpoint{1.025062in}{0.664856in}}%
\pgfpathcurveto{\pgfqpoint{1.022458in}{0.664856in}}{\pgfqpoint{1.019960in}{0.663821in}}{\pgfqpoint{1.018118in}{0.661979in}}%
\pgfpathcurveto{\pgfqpoint{1.016276in}{0.660138in}}{\pgfqpoint{1.015241in}{0.657639in}}{\pgfqpoint{1.015241in}{0.655035in}}%
\pgfpathcurveto{\pgfqpoint{1.015241in}{0.652430in}}{\pgfqpoint{1.016276in}{0.649932in}}{\pgfqpoint{1.018118in}{0.648090in}}%
\pgfpathcurveto{\pgfqpoint{1.019960in}{0.646249in}}{\pgfqpoint{1.022458in}{0.645214in}}{\pgfqpoint{1.025062in}{0.645214in}}%
\pgfpathclose%
\pgfusepath{stroke,fill}%
\end{pgfscope}%
\begin{pgfscope}%
\pgfpathrectangle{\pgfqpoint{0.750000in}{0.500000in}}{\pgfqpoint{4.650000in}{3.020000in}}%
\pgfusepath{clip}%
\pgfsetbuttcap%
\pgfsetroundjoin%
\definecolor{currentfill}{rgb}{0.121569,0.466667,0.705882}%
\pgfsetfillcolor{currentfill}%
\pgfsetlinewidth{1.003750pt}%
\definecolor{currentstroke}{rgb}{0.121569,0.466667,0.705882}%
\pgfsetstrokecolor{currentstroke}%
\pgfsetdash{}{0pt}%
\pgfpathmoveto{\pgfqpoint{4.382270in}{0.849912in}}%
\pgfpathcurveto{\pgfqpoint{4.385460in}{0.849912in}}{\pgfqpoint{4.388519in}{0.851179in}}{\pgfqpoint{4.390775in}{0.853435in}}%
\pgfpathcurveto{\pgfqpoint{4.393030in}{0.855691in}}{\pgfqpoint{4.394298in}{0.858750in}}{\pgfqpoint{4.394298in}{0.861940in}}%
\pgfpathcurveto{\pgfqpoint{4.394298in}{0.865130in}}{\pgfqpoint{4.393030in}{0.868190in}}{\pgfqpoint{4.390775in}{0.870445in}}%
\pgfpathcurveto{\pgfqpoint{4.388519in}{0.872701in}}{\pgfqpoint{4.385460in}{0.873968in}}{\pgfqpoint{4.382270in}{0.873968in}}%
\pgfpathcurveto{\pgfqpoint{4.379080in}{0.873968in}}{\pgfqpoint{4.376020in}{0.872701in}}{\pgfqpoint{4.373764in}{0.870445in}}%
\pgfpathcurveto{\pgfqpoint{4.371509in}{0.868190in}}{\pgfqpoint{4.370241in}{0.865130in}}{\pgfqpoint{4.370241in}{0.861940in}}%
\pgfpathcurveto{\pgfqpoint{4.370241in}{0.858750in}}{\pgfqpoint{4.371509in}{0.855691in}}{\pgfqpoint{4.373764in}{0.853435in}}%
\pgfpathcurveto{\pgfqpoint{4.376020in}{0.851179in}}{\pgfqpoint{4.379080in}{0.849912in}}{\pgfqpoint{4.382270in}{0.849912in}}%
\pgfpathclose%
\pgfusepath{stroke,fill}%
\end{pgfscope}%
\begin{pgfscope}%
\pgfpathrectangle{\pgfqpoint{0.750000in}{0.500000in}}{\pgfqpoint{4.650000in}{3.020000in}}%
\pgfusepath{clip}%
\pgfsetbuttcap%
\pgfsetroundjoin%
\definecolor{currentfill}{rgb}{0.121569,0.466667,0.705882}%
\pgfsetfillcolor{currentfill}%
\pgfsetlinewidth{1.003750pt}%
\definecolor{currentstroke}{rgb}{0.121569,0.466667,0.705882}%
\pgfsetstrokecolor{currentstroke}%
\pgfsetdash{}{0pt}%
\pgfpathmoveto{\pgfqpoint{2.921544in}{0.702874in}}%
\pgfpathcurveto{\pgfqpoint{2.925662in}{0.702874in}}{\pgfqpoint{2.929612in}{0.704511in}}{\pgfqpoint{2.932524in}{0.707422in}}%
\pgfpathcurveto{\pgfqpoint{2.935436in}{0.710334in}}{\pgfqpoint{2.937072in}{0.714284in}}{\pgfqpoint{2.937072in}{0.718403in}}%
\pgfpathcurveto{\pgfqpoint{2.937072in}{0.722521in}}{\pgfqpoint{2.935436in}{0.726471in}}{\pgfqpoint{2.932524in}{0.729383in}}%
\pgfpathcurveto{\pgfqpoint{2.929612in}{0.732295in}}{\pgfqpoint{2.925662in}{0.733931in}}{\pgfqpoint{2.921544in}{0.733931in}}%
\pgfpathcurveto{\pgfqpoint{2.917426in}{0.733931in}}{\pgfqpoint{2.913476in}{0.732295in}}{\pgfqpoint{2.910564in}{0.729383in}}%
\pgfpathcurveto{\pgfqpoint{2.907652in}{0.726471in}}{\pgfqpoint{2.906016in}{0.722521in}}{\pgfqpoint{2.906016in}{0.718403in}}%
\pgfpathcurveto{\pgfqpoint{2.906016in}{0.714284in}}{\pgfqpoint{2.907652in}{0.710334in}}{\pgfqpoint{2.910564in}{0.707422in}}%
\pgfpathcurveto{\pgfqpoint{2.913476in}{0.704511in}}{\pgfqpoint{2.917426in}{0.702874in}}{\pgfqpoint{2.921544in}{0.702874in}}%
\pgfpathclose%
\pgfusepath{stroke,fill}%
\end{pgfscope}%
\begin{pgfscope}%
\pgfpathrectangle{\pgfqpoint{0.750000in}{0.500000in}}{\pgfqpoint{4.650000in}{3.020000in}}%
\pgfusepath{clip}%
\pgfsetbuttcap%
\pgfsetroundjoin%
\definecolor{currentfill}{rgb}{0.121569,0.466667,0.705882}%
\pgfsetfillcolor{currentfill}%
\pgfsetlinewidth{1.003750pt}%
\definecolor{currentstroke}{rgb}{0.121569,0.466667,0.705882}%
\pgfsetstrokecolor{currentstroke}%
\pgfsetdash{}{0pt}%
\pgfpathmoveto{\pgfqpoint{2.633453in}{0.676471in}}%
\pgfpathcurveto{\pgfqpoint{2.637571in}{0.676471in}}{\pgfqpoint{2.641521in}{0.678107in}}{\pgfqpoint{2.644433in}{0.681019in}}%
\pgfpathcurveto{\pgfqpoint{2.647345in}{0.683931in}}{\pgfqpoint{2.648981in}{0.687881in}}{\pgfqpoint{2.648981in}{0.691999in}}%
\pgfpathcurveto{\pgfqpoint{2.648981in}{0.696118in}}{\pgfqpoint{2.647345in}{0.700068in}}{\pgfqpoint{2.644433in}{0.702980in}}%
\pgfpathcurveto{\pgfqpoint{2.641521in}{0.705891in}}{\pgfqpoint{2.637571in}{0.707528in}}{\pgfqpoint{2.633453in}{0.707528in}}%
\pgfpathcurveto{\pgfqpoint{2.629335in}{0.707528in}}{\pgfqpoint{2.625385in}{0.705891in}}{\pgfqpoint{2.622473in}{0.702980in}}%
\pgfpathcurveto{\pgfqpoint{2.619561in}{0.700068in}}{\pgfqpoint{2.617924in}{0.696118in}}{\pgfqpoint{2.617924in}{0.691999in}}%
\pgfpathcurveto{\pgfqpoint{2.617924in}{0.687881in}}{\pgfqpoint{2.619561in}{0.683931in}}{\pgfqpoint{2.622473in}{0.681019in}}%
\pgfpathcurveto{\pgfqpoint{2.625385in}{0.678107in}}{\pgfqpoint{2.629335in}{0.676471in}}{\pgfqpoint{2.633453in}{0.676471in}}%
\pgfpathclose%
\pgfusepath{stroke,fill}%
\end{pgfscope}%
\begin{pgfscope}%
\pgfpathrectangle{\pgfqpoint{0.750000in}{0.500000in}}{\pgfqpoint{4.650000in}{3.020000in}}%
\pgfusepath{clip}%
\pgfsetbuttcap%
\pgfsetroundjoin%
\definecolor{currentfill}{rgb}{0.121569,0.466667,0.705882}%
\pgfsetfillcolor{currentfill}%
\pgfsetlinewidth{1.003750pt}%
\definecolor{currentstroke}{rgb}{0.121569,0.466667,0.705882}%
\pgfsetstrokecolor{currentstroke}%
\pgfsetdash{}{0pt}%
\pgfpathmoveto{\pgfqpoint{1.815504in}{0.665856in}}%
\pgfpathcurveto{\pgfqpoint{1.818109in}{0.665856in}}{\pgfqpoint{1.820607in}{0.666891in}}{\pgfqpoint{1.822449in}{0.668733in}}%
\pgfpathcurveto{\pgfqpoint{1.824290in}{0.670575in}}{\pgfqpoint{1.825325in}{0.673073in}}{\pgfqpoint{1.825325in}{0.675677in}}%
\pgfpathcurveto{\pgfqpoint{1.825325in}{0.678282in}}{\pgfqpoint{1.824290in}{0.680780in}}{\pgfqpoint{1.822449in}{0.682622in}}%
\pgfpathcurveto{\pgfqpoint{1.820607in}{0.684464in}}{\pgfqpoint{1.818109in}{0.685498in}}{\pgfqpoint{1.815504in}{0.685498in}}%
\pgfpathcurveto{\pgfqpoint{1.812900in}{0.685498in}}{\pgfqpoint{1.810402in}{0.684464in}}{\pgfqpoint{1.808560in}{0.682622in}}%
\pgfpathcurveto{\pgfqpoint{1.806718in}{0.680780in}}{\pgfqpoint{1.805683in}{0.678282in}}{\pgfqpoint{1.805683in}{0.675677in}}%
\pgfpathcurveto{\pgfqpoint{1.805683in}{0.673073in}}{\pgfqpoint{1.806718in}{0.670575in}}{\pgfqpoint{1.808560in}{0.668733in}}%
\pgfpathcurveto{\pgfqpoint{1.810402in}{0.666891in}}{\pgfqpoint{1.812900in}{0.665856in}}{\pgfqpoint{1.815504in}{0.665856in}}%
\pgfpathclose%
\pgfusepath{stroke,fill}%
\end{pgfscope}%
\begin{pgfscope}%
\pgfpathrectangle{\pgfqpoint{0.750000in}{0.500000in}}{\pgfqpoint{4.650000in}{3.020000in}}%
\pgfusepath{clip}%
\pgfsetbuttcap%
\pgfsetroundjoin%
\definecolor{currentfill}{rgb}{0.121569,0.466667,0.705882}%
\pgfsetfillcolor{currentfill}%
\pgfsetlinewidth{1.003750pt}%
\definecolor{currentstroke}{rgb}{0.121569,0.466667,0.705882}%
\pgfsetstrokecolor{currentstroke}%
\pgfsetdash{}{0pt}%
\pgfpathmoveto{\pgfqpoint{3.974019in}{0.763501in}}%
\pgfpathcurveto{\pgfqpoint{3.977209in}{0.763501in}}{\pgfqpoint{3.980269in}{0.764769in}}{\pgfqpoint{3.982524in}{0.767024in}}%
\pgfpathcurveto{\pgfqpoint{3.984780in}{0.769280in}}{\pgfqpoint{3.986047in}{0.772340in}}{\pgfqpoint{3.986047in}{0.775530in}}%
\pgfpathcurveto{\pgfqpoint{3.986047in}{0.778719in}}{\pgfqpoint{3.984780in}{0.781779in}}{\pgfqpoint{3.982524in}{0.784035in}}%
\pgfpathcurveto{\pgfqpoint{3.980269in}{0.786290in}}{\pgfqpoint{3.977209in}{0.787558in}}{\pgfqpoint{3.974019in}{0.787558in}}%
\pgfpathcurveto{\pgfqpoint{3.970829in}{0.787558in}}{\pgfqpoint{3.967770in}{0.786290in}}{\pgfqpoint{3.965514in}{0.784035in}}%
\pgfpathcurveto{\pgfqpoint{3.963259in}{0.781779in}}{\pgfqpoint{3.961991in}{0.778719in}}{\pgfqpoint{3.961991in}{0.775530in}}%
\pgfpathcurveto{\pgfqpoint{3.961991in}{0.772340in}}{\pgfqpoint{3.963259in}{0.769280in}}{\pgfqpoint{3.965514in}{0.767024in}}%
\pgfpathcurveto{\pgfqpoint{3.967770in}{0.764769in}}{\pgfqpoint{3.970829in}{0.763501in}}{\pgfqpoint{3.974019in}{0.763501in}}%
\pgfpathclose%
\pgfusepath{stroke,fill}%
\end{pgfscope}%
\begin{pgfscope}%
\pgfpathrectangle{\pgfqpoint{0.750000in}{0.500000in}}{\pgfqpoint{4.650000in}{3.020000in}}%
\pgfusepath{clip}%
\pgfsetbuttcap%
\pgfsetroundjoin%
\definecolor{currentfill}{rgb}{0.121569,0.466667,0.705882}%
\pgfsetfillcolor{currentfill}%
\pgfsetlinewidth{1.003750pt}%
\definecolor{currentstroke}{rgb}{0.121569,0.466667,0.705882}%
\pgfsetstrokecolor{currentstroke}%
\pgfsetdash{}{0pt}%
\pgfpathmoveto{\pgfqpoint{1.549128in}{0.633213in}}%
\pgfpathcurveto{\pgfqpoint{1.551733in}{0.633213in}}{\pgfqpoint{1.554231in}{0.634247in}}{\pgfqpoint{1.556073in}{0.636089in}}%
\pgfpathcurveto{\pgfqpoint{1.557914in}{0.637931in}}{\pgfqpoint{1.558949in}{0.640429in}}{\pgfqpoint{1.558949in}{0.643033in}}%
\pgfpathcurveto{\pgfqpoint{1.558949in}{0.645638in}}{\pgfqpoint{1.557914in}{0.648136in}}{\pgfqpoint{1.556073in}{0.649978in}}%
\pgfpathcurveto{\pgfqpoint{1.554231in}{0.651820in}}{\pgfqpoint{1.551733in}{0.652854in}}{\pgfqpoint{1.549128in}{0.652854in}}%
\pgfpathcurveto{\pgfqpoint{1.546524in}{0.652854in}}{\pgfqpoint{1.544026in}{0.651820in}}{\pgfqpoint{1.542184in}{0.649978in}}%
\pgfpathcurveto{\pgfqpoint{1.540342in}{0.648136in}}{\pgfqpoint{1.539307in}{0.645638in}}{\pgfqpoint{1.539307in}{0.643033in}}%
\pgfpathcurveto{\pgfqpoint{1.539307in}{0.640429in}}{\pgfqpoint{1.540342in}{0.637931in}}{\pgfqpoint{1.542184in}{0.636089in}}%
\pgfpathcurveto{\pgfqpoint{1.544026in}{0.634247in}}{\pgfqpoint{1.546524in}{0.633213in}}{\pgfqpoint{1.549128in}{0.633213in}}%
\pgfpathclose%
\pgfusepath{stroke,fill}%
\end{pgfscope}%
\begin{pgfscope}%
\pgfpathrectangle{\pgfqpoint{0.750000in}{0.500000in}}{\pgfqpoint{4.650000in}{3.020000in}}%
\pgfusepath{clip}%
\pgfsetbuttcap%
\pgfsetroundjoin%
\definecolor{currentfill}{rgb}{0.121569,0.466667,0.705882}%
\pgfsetfillcolor{currentfill}%
\pgfsetlinewidth{1.003750pt}%
\definecolor{currentstroke}{rgb}{0.121569,0.466667,0.705882}%
\pgfsetstrokecolor{currentstroke}%
\pgfsetdash{}{0pt}%
\pgfpathmoveto{\pgfqpoint{1.607036in}{0.651168in}}%
\pgfpathcurveto{\pgfqpoint{1.610226in}{0.651168in}}{\pgfqpoint{1.613286in}{0.652435in}}{\pgfqpoint{1.615541in}{0.654691in}}%
\pgfpathcurveto{\pgfqpoint{1.617797in}{0.656946in}}{\pgfqpoint{1.619064in}{0.660006in}}{\pgfqpoint{1.619064in}{0.663196in}}%
\pgfpathcurveto{\pgfqpoint{1.619064in}{0.666386in}}{\pgfqpoint{1.617797in}{0.669445in}}{\pgfqpoint{1.615541in}{0.671701in}}%
\pgfpathcurveto{\pgfqpoint{1.613286in}{0.673957in}}{\pgfqpoint{1.610226in}{0.675224in}}{\pgfqpoint{1.607036in}{0.675224in}}%
\pgfpathcurveto{\pgfqpoint{1.603846in}{0.675224in}}{\pgfqpoint{1.600787in}{0.673957in}}{\pgfqpoint{1.598531in}{0.671701in}}%
\pgfpathcurveto{\pgfqpoint{1.596275in}{0.669445in}}{\pgfqpoint{1.595008in}{0.666386in}}{\pgfqpoint{1.595008in}{0.663196in}}%
\pgfpathcurveto{\pgfqpoint{1.595008in}{0.660006in}}{\pgfqpoint{1.596275in}{0.656946in}}{\pgfqpoint{1.598531in}{0.654691in}}%
\pgfpathcurveto{\pgfqpoint{1.600787in}{0.652435in}}{\pgfqpoint{1.603846in}{0.651168in}}{\pgfqpoint{1.607036in}{0.651168in}}%
\pgfpathclose%
\pgfusepath{stroke,fill}%
\end{pgfscope}%
\begin{pgfscope}%
\pgfpathrectangle{\pgfqpoint{0.750000in}{0.500000in}}{\pgfqpoint{4.650000in}{3.020000in}}%
\pgfusepath{clip}%
\pgfsetbuttcap%
\pgfsetroundjoin%
\definecolor{currentfill}{rgb}{0.121569,0.466667,0.705882}%
\pgfsetfillcolor{currentfill}%
\pgfsetlinewidth{1.003750pt}%
\definecolor{currentstroke}{rgb}{0.121569,0.466667,0.705882}%
\pgfsetstrokecolor{currentstroke}%
\pgfsetdash{}{0pt}%
\pgfpathmoveto{\pgfqpoint{2.905620in}{0.742379in}}%
\pgfpathcurveto{\pgfqpoint{2.908809in}{0.742379in}}{\pgfqpoint{2.911869in}{0.743646in}}{\pgfqpoint{2.914125in}{0.745902in}}%
\pgfpathcurveto{\pgfqpoint{2.916380in}{0.748157in}}{\pgfqpoint{2.917648in}{0.751217in}}{\pgfqpoint{2.917648in}{0.754407in}}%
\pgfpathcurveto{\pgfqpoint{2.917648in}{0.757597in}}{\pgfqpoint{2.916380in}{0.760657in}}{\pgfqpoint{2.914125in}{0.762912in}}%
\pgfpathcurveto{\pgfqpoint{2.911869in}{0.765168in}}{\pgfqpoint{2.908809in}{0.766435in}}{\pgfqpoint{2.905620in}{0.766435in}}%
\pgfpathcurveto{\pgfqpoint{2.902430in}{0.766435in}}{\pgfqpoint{2.899370in}{0.765168in}}{\pgfqpoint{2.897114in}{0.762912in}}%
\pgfpathcurveto{\pgfqpoint{2.894859in}{0.760657in}}{\pgfqpoint{2.893591in}{0.757597in}}{\pgfqpoint{2.893591in}{0.754407in}}%
\pgfpathcurveto{\pgfqpoint{2.893591in}{0.751217in}}{\pgfqpoint{2.894859in}{0.748157in}}{\pgfqpoint{2.897114in}{0.745902in}}%
\pgfpathcurveto{\pgfqpoint{2.899370in}{0.743646in}}{\pgfqpoint{2.902430in}{0.742379in}}{\pgfqpoint{2.905620in}{0.742379in}}%
\pgfpathclose%
\pgfusepath{stroke,fill}%
\end{pgfscope}%
\begin{pgfscope}%
\pgfpathrectangle{\pgfqpoint{0.750000in}{0.500000in}}{\pgfqpoint{4.650000in}{3.020000in}}%
\pgfusepath{clip}%
\pgfsetbuttcap%
\pgfsetroundjoin%
\definecolor{currentfill}{rgb}{0.121569,0.466667,0.705882}%
\pgfsetfillcolor{currentfill}%
\pgfsetlinewidth{1.003750pt}%
\definecolor{currentstroke}{rgb}{0.121569,0.466667,0.705882}%
\pgfsetstrokecolor{currentstroke}%
\pgfsetdash{}{0pt}%
\pgfpathmoveto{\pgfqpoint{3.843727in}{0.874462in}}%
\pgfpathcurveto{\pgfqpoint{3.848936in}{0.874462in}}{\pgfqpoint{3.853932in}{0.876532in}}{\pgfqpoint{3.857616in}{0.880215in}}%
\pgfpathcurveto{\pgfqpoint{3.861299in}{0.883899in}}{\pgfqpoint{3.863369in}{0.888895in}}{\pgfqpoint{3.863369in}{0.894104in}}%
\pgfpathcurveto{\pgfqpoint{3.863369in}{0.899313in}}{\pgfqpoint{3.861299in}{0.904310in}}{\pgfqpoint{3.857616in}{0.907993in}}%
\pgfpathcurveto{\pgfqpoint{3.853932in}{0.911676in}}{\pgfqpoint{3.848936in}{0.913746in}}{\pgfqpoint{3.843727in}{0.913746in}}%
\pgfpathcurveto{\pgfqpoint{3.838518in}{0.913746in}}{\pgfqpoint{3.833521in}{0.911676in}}{\pgfqpoint{3.829838in}{0.907993in}}%
\pgfpathcurveto{\pgfqpoint{3.826154in}{0.904310in}}{\pgfqpoint{3.824085in}{0.899313in}}{\pgfqpoint{3.824085in}{0.894104in}}%
\pgfpathcurveto{\pgfqpoint{3.824085in}{0.888895in}}{\pgfqpoint{3.826154in}{0.883899in}}{\pgfqpoint{3.829838in}{0.880215in}}%
\pgfpathcurveto{\pgfqpoint{3.833521in}{0.876532in}}{\pgfqpoint{3.838518in}{0.874462in}}{\pgfqpoint{3.843727in}{0.874462in}}%
\pgfpathclose%
\pgfusepath{stroke,fill}%
\end{pgfscope}%
\begin{pgfscope}%
\pgfpathrectangle{\pgfqpoint{0.750000in}{0.500000in}}{\pgfqpoint{4.650000in}{3.020000in}}%
\pgfusepath{clip}%
\pgfsetbuttcap%
\pgfsetroundjoin%
\definecolor{currentfill}{rgb}{0.121569,0.466667,0.705882}%
\pgfsetfillcolor{currentfill}%
\pgfsetlinewidth{1.003750pt}%
\definecolor{currentstroke}{rgb}{0.121569,0.466667,0.705882}%
\pgfsetstrokecolor{currentstroke}%
\pgfsetdash{}{0pt}%
\pgfpathmoveto{\pgfqpoint{4.350420in}{0.916020in}}%
\pgfpathcurveto{\pgfqpoint{4.354538in}{0.916020in}}{\pgfqpoint{4.358488in}{0.917656in}}{\pgfqpoint{4.361400in}{0.920568in}}%
\pgfpathcurveto{\pgfqpoint{4.364312in}{0.923480in}}{\pgfqpoint{4.365949in}{0.927430in}}{\pgfqpoint{4.365949in}{0.931549in}}%
\pgfpathcurveto{\pgfqpoint{4.365949in}{0.935667in}}{\pgfqpoint{4.364312in}{0.939617in}}{\pgfqpoint{4.361400in}{0.942529in}}%
\pgfpathcurveto{\pgfqpoint{4.358488in}{0.945441in}}{\pgfqpoint{4.354538in}{0.947077in}}{\pgfqpoint{4.350420in}{0.947077in}}%
\pgfpathcurveto{\pgfqpoint{4.346302in}{0.947077in}}{\pgfqpoint{4.342352in}{0.945441in}}{\pgfqpoint{4.339440in}{0.942529in}}%
\pgfpathcurveto{\pgfqpoint{4.336528in}{0.939617in}}{\pgfqpoint{4.334892in}{0.935667in}}{\pgfqpoint{4.334892in}{0.931549in}}%
\pgfpathcurveto{\pgfqpoint{4.334892in}{0.927430in}}{\pgfqpoint{4.336528in}{0.923480in}}{\pgfqpoint{4.339440in}{0.920568in}}%
\pgfpathcurveto{\pgfqpoint{4.342352in}{0.917656in}}{\pgfqpoint{4.346302in}{0.916020in}}{\pgfqpoint{4.350420in}{0.916020in}}%
\pgfpathclose%
\pgfusepath{stroke,fill}%
\end{pgfscope}%
\begin{pgfscope}%
\pgfpathrectangle{\pgfqpoint{0.750000in}{0.500000in}}{\pgfqpoint{4.650000in}{3.020000in}}%
\pgfusepath{clip}%
\pgfsetbuttcap%
\pgfsetroundjoin%
\definecolor{currentfill}{rgb}{0.121569,0.466667,0.705882}%
\pgfsetfillcolor{currentfill}%
\pgfsetlinewidth{1.003750pt}%
\definecolor{currentstroke}{rgb}{0.121569,0.466667,0.705882}%
\pgfsetstrokecolor{currentstroke}%
\pgfsetdash{}{0pt}%
\pgfpathmoveto{\pgfqpoint{2.511846in}{0.734996in}}%
\pgfpathcurveto{\pgfqpoint{2.516357in}{0.734996in}}{\pgfqpoint{2.520684in}{0.736789in}}{\pgfqpoint{2.523874in}{0.739979in}}%
\pgfpathcurveto{\pgfqpoint{2.527064in}{0.743168in}}{\pgfqpoint{2.528857in}{0.747496in}}{\pgfqpoint{2.528857in}{0.752007in}}%
\pgfpathcurveto{\pgfqpoint{2.528857in}{0.756518in}}{\pgfqpoint{2.527064in}{0.760845in}}{\pgfqpoint{2.523874in}{0.764035in}}%
\pgfpathcurveto{\pgfqpoint{2.520684in}{0.767225in}}{\pgfqpoint{2.516357in}{0.769017in}}{\pgfqpoint{2.511846in}{0.769017in}}%
\pgfpathcurveto{\pgfqpoint{2.507335in}{0.769017in}}{\pgfqpoint{2.503008in}{0.767225in}}{\pgfqpoint{2.499818in}{0.764035in}}%
\pgfpathcurveto{\pgfqpoint{2.496628in}{0.760845in}}{\pgfqpoint{2.494836in}{0.756518in}}{\pgfqpoint{2.494836in}{0.752007in}}%
\pgfpathcurveto{\pgfqpoint{2.494836in}{0.747496in}}{\pgfqpoint{2.496628in}{0.743168in}}{\pgfqpoint{2.499818in}{0.739979in}}%
\pgfpathcurveto{\pgfqpoint{2.503008in}{0.736789in}}{\pgfqpoint{2.507335in}{0.734996in}}{\pgfqpoint{2.511846in}{0.734996in}}%
\pgfpathclose%
\pgfusepath{stroke,fill}%
\end{pgfscope}%
\begin{pgfscope}%
\pgfpathrectangle{\pgfqpoint{0.750000in}{0.500000in}}{\pgfqpoint{4.650000in}{3.020000in}}%
\pgfusepath{clip}%
\pgfsetbuttcap%
\pgfsetroundjoin%
\definecolor{currentfill}{rgb}{0.121569,0.466667,0.705882}%
\pgfsetfillcolor{currentfill}%
\pgfsetlinewidth{1.003750pt}%
\definecolor{currentstroke}{rgb}{0.121569,0.466667,0.705882}%
\pgfsetstrokecolor{currentstroke}%
\pgfsetdash{}{0pt}%
\pgfpathmoveto{\pgfqpoint{1.402911in}{0.638505in}}%
\pgfpathcurveto{\pgfqpoint{1.407422in}{0.638505in}}{\pgfqpoint{1.411749in}{0.640297in}}{\pgfqpoint{1.414939in}{0.643487in}}%
\pgfpathcurveto{\pgfqpoint{1.418129in}{0.646677in}}{\pgfqpoint{1.419921in}{0.651004in}}{\pgfqpoint{1.419921in}{0.655515in}}%
\pgfpathcurveto{\pgfqpoint{1.419921in}{0.660026in}}{\pgfqpoint{1.418129in}{0.664353in}}{\pgfqpoint{1.414939in}{0.667543in}}%
\pgfpathcurveto{\pgfqpoint{1.411749in}{0.670733in}}{\pgfqpoint{1.407422in}{0.672525in}}{\pgfqpoint{1.402911in}{0.672525in}}%
\pgfpathcurveto{\pgfqpoint{1.398400in}{0.672525in}}{\pgfqpoint{1.394073in}{0.670733in}}{\pgfqpoint{1.390883in}{0.667543in}}%
\pgfpathcurveto{\pgfqpoint{1.387693in}{0.664353in}}{\pgfqpoint{1.385901in}{0.660026in}}{\pgfqpoint{1.385901in}{0.655515in}}%
\pgfpathcurveto{\pgfqpoint{1.385901in}{0.651004in}}{\pgfqpoint{1.387693in}{0.646677in}}{\pgfqpoint{1.390883in}{0.643487in}}%
\pgfpathcurveto{\pgfqpoint{1.394073in}{0.640297in}}{\pgfqpoint{1.398400in}{0.638505in}}{\pgfqpoint{1.402911in}{0.638505in}}%
\pgfpathclose%
\pgfusepath{stroke,fill}%
\end{pgfscope}%
\begin{pgfscope}%
\pgfpathrectangle{\pgfqpoint{0.750000in}{0.500000in}}{\pgfqpoint{4.650000in}{3.020000in}}%
\pgfusepath{clip}%
\pgfsetbuttcap%
\pgfsetroundjoin%
\definecolor{currentfill}{rgb}{0.121569,0.466667,0.705882}%
\pgfsetfillcolor{currentfill}%
\pgfsetlinewidth{1.003750pt}%
\definecolor{currentstroke}{rgb}{0.121569,0.466667,0.705882}%
\pgfsetstrokecolor{currentstroke}%
\pgfsetdash{}{0pt}%
\pgfpathmoveto{\pgfqpoint{0.985974in}{0.627932in}}%
\pgfpathcurveto{\pgfqpoint{0.988579in}{0.627932in}}{\pgfqpoint{0.991077in}{0.628967in}}{\pgfqpoint{0.992919in}{0.630808in}}%
\pgfpathcurveto{\pgfqpoint{0.994761in}{0.632650in}}{\pgfqpoint{0.995795in}{0.635148in}}{\pgfqpoint{0.995795in}{0.637753in}}%
\pgfpathcurveto{\pgfqpoint{0.995795in}{0.640357in}}{\pgfqpoint{0.994761in}{0.642856in}}{\pgfqpoint{0.992919in}{0.644697in}}%
\pgfpathcurveto{\pgfqpoint{0.991077in}{0.646539in}}{\pgfqpoint{0.988579in}{0.647574in}}{\pgfqpoint{0.985974in}{0.647574in}}%
\pgfpathcurveto{\pgfqpoint{0.983370in}{0.647574in}}{\pgfqpoint{0.980872in}{0.646539in}}{\pgfqpoint{0.979030in}{0.644697in}}%
\pgfpathcurveto{\pgfqpoint{0.977188in}{0.642856in}}{\pgfqpoint{0.976154in}{0.640357in}}{\pgfqpoint{0.976154in}{0.637753in}}%
\pgfpathcurveto{\pgfqpoint{0.976154in}{0.635148in}}{\pgfqpoint{0.977188in}{0.632650in}}{\pgfqpoint{0.979030in}{0.630808in}}%
\pgfpathcurveto{\pgfqpoint{0.980872in}{0.628967in}}{\pgfqpoint{0.983370in}{0.627932in}}{\pgfqpoint{0.985974in}{0.627932in}}%
\pgfpathclose%
\pgfusepath{stroke,fill}%
\end{pgfscope}%
\begin{pgfscope}%
\pgfpathrectangle{\pgfqpoint{0.750000in}{0.500000in}}{\pgfqpoint{4.650000in}{3.020000in}}%
\pgfusepath{clip}%
\pgfsetbuttcap%
\pgfsetroundjoin%
\definecolor{currentfill}{rgb}{0.121569,0.466667,0.705882}%
\pgfsetfillcolor{currentfill}%
\pgfsetlinewidth{1.003750pt}%
\definecolor{currentstroke}{rgb}{0.121569,0.466667,0.705882}%
\pgfsetstrokecolor{currentstroke}%
\pgfsetdash{}{0pt}%
\pgfpathmoveto{\pgfqpoint{5.188636in}{0.890304in}}%
\pgfpathcurveto{\pgfqpoint{5.193845in}{0.890304in}}{\pgfqpoint{5.198842in}{0.892374in}}{\pgfqpoint{5.202525in}{0.896057in}}%
\pgfpathcurveto{\pgfqpoint{5.206209in}{0.899740in}}{\pgfqpoint{5.208278in}{0.904737in}}{\pgfqpoint{5.208278in}{0.909946in}}%
\pgfpathcurveto{\pgfqpoint{5.208278in}{0.915155in}}{\pgfqpoint{5.206209in}{0.920151in}}{\pgfqpoint{5.202525in}{0.923835in}}%
\pgfpathcurveto{\pgfqpoint{5.198842in}{0.927518in}}{\pgfqpoint{5.193845in}{0.929588in}}{\pgfqpoint{5.188636in}{0.929588in}}%
\pgfpathcurveto{\pgfqpoint{5.183427in}{0.929588in}}{\pgfqpoint{5.178431in}{0.927518in}}{\pgfqpoint{5.174747in}{0.923835in}}%
\pgfpathcurveto{\pgfqpoint{5.171064in}{0.920151in}}{\pgfqpoint{5.168995in}{0.915155in}}{\pgfqpoint{5.168995in}{0.909946in}}%
\pgfpathcurveto{\pgfqpoint{5.168995in}{0.904737in}}{\pgfqpoint{5.171064in}{0.899740in}}{\pgfqpoint{5.174747in}{0.896057in}}%
\pgfpathcurveto{\pgfqpoint{5.178431in}{0.892374in}}{\pgfqpoint{5.183427in}{0.890304in}}{\pgfqpoint{5.188636in}{0.890304in}}%
\pgfpathclose%
\pgfusepath{stroke,fill}%
\end{pgfscope}%
\begin{pgfscope}%
\pgfpathrectangle{\pgfqpoint{0.750000in}{0.500000in}}{\pgfqpoint{4.650000in}{3.020000in}}%
\pgfusepath{clip}%
\pgfsetbuttcap%
\pgfsetroundjoin%
\definecolor{currentfill}{rgb}{0.121569,0.466667,0.705882}%
\pgfsetfillcolor{currentfill}%
\pgfsetlinewidth{1.003750pt}%
\definecolor{currentstroke}{rgb}{0.121569,0.466667,0.705882}%
\pgfsetstrokecolor{currentstroke}%
\pgfsetdash{}{0pt}%
\pgfpathmoveto{\pgfqpoint{1.093104in}{0.634653in}}%
\pgfpathcurveto{\pgfqpoint{1.095709in}{0.634653in}}{\pgfqpoint{1.098207in}{0.635687in}}{\pgfqpoint{1.100048in}{0.637529in}}%
\pgfpathcurveto{\pgfqpoint{1.101890in}{0.639371in}}{\pgfqpoint{1.102925in}{0.641869in}}{\pgfqpoint{1.102925in}{0.644474in}}%
\pgfpathcurveto{\pgfqpoint{1.102925in}{0.647078in}}{\pgfqpoint{1.101890in}{0.649576in}}{\pgfqpoint{1.100048in}{0.651418in}}%
\pgfpathcurveto{\pgfqpoint{1.098207in}{0.653260in}}{\pgfqpoint{1.095709in}{0.654295in}}{\pgfqpoint{1.093104in}{0.654295in}}%
\pgfpathcurveto{\pgfqpoint{1.090499in}{0.654295in}}{\pgfqpoint{1.088001in}{0.653260in}}{\pgfqpoint{1.086160in}{0.651418in}}%
\pgfpathcurveto{\pgfqpoint{1.084318in}{0.649576in}}{\pgfqpoint{1.083283in}{0.647078in}}{\pgfqpoint{1.083283in}{0.644474in}}%
\pgfpathcurveto{\pgfqpoint{1.083283in}{0.641869in}}{\pgfqpoint{1.084318in}{0.639371in}}{\pgfqpoint{1.086160in}{0.637529in}}%
\pgfpathcurveto{\pgfqpoint{1.088001in}{0.635687in}}{\pgfqpoint{1.090499in}{0.634653in}}{\pgfqpoint{1.093104in}{0.634653in}}%
\pgfpathclose%
\pgfusepath{stroke,fill}%
\end{pgfscope}%
\begin{pgfscope}%
\pgfpathrectangle{\pgfqpoint{0.750000in}{0.500000in}}{\pgfqpoint{4.650000in}{3.020000in}}%
\pgfusepath{clip}%
\pgfsetbuttcap%
\pgfsetroundjoin%
\definecolor{currentfill}{rgb}{0.121569,0.466667,0.705882}%
\pgfsetfillcolor{currentfill}%
\pgfsetlinewidth{1.003750pt}%
\definecolor{currentstroke}{rgb}{0.121569,0.466667,0.705882}%
\pgfsetstrokecolor{currentstroke}%
\pgfsetdash{}{0pt}%
\pgfpathmoveto{\pgfqpoint{3.118431in}{0.953892in}}%
\pgfpathcurveto{\pgfqpoint{3.121035in}{0.953892in}}{\pgfqpoint{3.123534in}{0.954926in}}{\pgfqpoint{3.125375in}{0.956768in}}%
\pgfpathcurveto{\pgfqpoint{3.127217in}{0.958610in}}{\pgfqpoint{3.128252in}{0.961108in}}{\pgfqpoint{3.128252in}{0.963713in}}%
\pgfpathcurveto{\pgfqpoint{3.128252in}{0.966317in}}{\pgfqpoint{3.127217in}{0.968815in}}{\pgfqpoint{3.125375in}{0.970657in}}%
\pgfpathcurveto{\pgfqpoint{3.123534in}{0.972499in}}{\pgfqpoint{3.121035in}{0.973533in}}{\pgfqpoint{3.118431in}{0.973533in}}%
\pgfpathcurveto{\pgfqpoint{3.115826in}{0.973533in}}{\pgfqpoint{3.113328in}{0.972499in}}{\pgfqpoint{3.111486in}{0.970657in}}%
\pgfpathcurveto{\pgfqpoint{3.109645in}{0.968815in}}{\pgfqpoint{3.108610in}{0.966317in}}{\pgfqpoint{3.108610in}{0.963713in}}%
\pgfpathcurveto{\pgfqpoint{3.108610in}{0.961108in}}{\pgfqpoint{3.109645in}{0.958610in}}{\pgfqpoint{3.111486in}{0.956768in}}%
\pgfpathcurveto{\pgfqpoint{3.113328in}{0.954926in}}{\pgfqpoint{3.115826in}{0.953892in}}{\pgfqpoint{3.118431in}{0.953892in}}%
\pgfpathclose%
\pgfusepath{stroke,fill}%
\end{pgfscope}%
\begin{pgfscope}%
\pgfpathrectangle{\pgfqpoint{0.750000in}{0.500000in}}{\pgfqpoint{4.650000in}{3.020000in}}%
\pgfusepath{clip}%
\pgfsetbuttcap%
\pgfsetroundjoin%
\definecolor{currentfill}{rgb}{0.121569,0.466667,0.705882}%
\pgfsetfillcolor{currentfill}%
\pgfsetlinewidth{1.003750pt}%
\definecolor{currentstroke}{rgb}{0.121569,0.466667,0.705882}%
\pgfsetstrokecolor{currentstroke}%
\pgfsetdash{}{0pt}%
\pgfpathmoveto{\pgfqpoint{3.348615in}{0.785164in}}%
\pgfpathcurveto{\pgfqpoint{3.352298in}{0.785164in}}{\pgfqpoint{3.355831in}{0.786627in}}{\pgfqpoint{3.358435in}{0.789232in}}%
\pgfpathcurveto{\pgfqpoint{3.361040in}{0.791836in}}{\pgfqpoint{3.362503in}{0.795369in}}{\pgfqpoint{3.362503in}{0.799052in}}%
\pgfpathcurveto{\pgfqpoint{3.362503in}{0.802736in}}{\pgfqpoint{3.361040in}{0.806269in}}{\pgfqpoint{3.358435in}{0.808873in}}%
\pgfpathcurveto{\pgfqpoint{3.355831in}{0.811478in}}{\pgfqpoint{3.352298in}{0.812941in}}{\pgfqpoint{3.348615in}{0.812941in}}%
\pgfpathcurveto{\pgfqpoint{3.344931in}{0.812941in}}{\pgfqpoint{3.341398in}{0.811478in}}{\pgfqpoint{3.338794in}{0.808873in}}%
\pgfpathcurveto{\pgfqpoint{3.336189in}{0.806269in}}{\pgfqpoint{3.334726in}{0.802736in}}{\pgfqpoint{3.334726in}{0.799052in}}%
\pgfpathcurveto{\pgfqpoint{3.334726in}{0.795369in}}{\pgfqpoint{3.336189in}{0.791836in}}{\pgfqpoint{3.338794in}{0.789232in}}%
\pgfpathcurveto{\pgfqpoint{3.341398in}{0.786627in}}{\pgfqpoint{3.344931in}{0.785164in}}{\pgfqpoint{3.348615in}{0.785164in}}%
\pgfpathclose%
\pgfusepath{stroke,fill}%
\end{pgfscope}%
\begin{pgfscope}%
\pgfpathrectangle{\pgfqpoint{0.750000in}{0.500000in}}{\pgfqpoint{4.650000in}{3.020000in}}%
\pgfusepath{clip}%
\pgfsetbuttcap%
\pgfsetroundjoin%
\definecolor{currentfill}{rgb}{0.121569,0.466667,0.705882}%
\pgfsetfillcolor{currentfill}%
\pgfsetlinewidth{1.003750pt}%
\definecolor{currentstroke}{rgb}{0.121569,0.466667,0.705882}%
\pgfsetstrokecolor{currentstroke}%
\pgfsetdash{}{0pt}%
\pgfpathmoveto{\pgfqpoint{1.796684in}{0.656928in}}%
\pgfpathcurveto{\pgfqpoint{1.799874in}{0.656928in}}{\pgfqpoint{1.802934in}{0.658196in}}{\pgfqpoint{1.805189in}{0.660451in}}%
\pgfpathcurveto{\pgfqpoint{1.807445in}{0.662707in}}{\pgfqpoint{1.808712in}{0.665767in}}{\pgfqpoint{1.808712in}{0.668957in}}%
\pgfpathcurveto{\pgfqpoint{1.808712in}{0.672146in}}{\pgfqpoint{1.807445in}{0.675206in}}{\pgfqpoint{1.805189in}{0.677462in}}%
\pgfpathcurveto{\pgfqpoint{1.802934in}{0.679717in}}{\pgfqpoint{1.799874in}{0.680985in}}{\pgfqpoint{1.796684in}{0.680985in}}%
\pgfpathcurveto{\pgfqpoint{1.793494in}{0.680985in}}{\pgfqpoint{1.790435in}{0.679717in}}{\pgfqpoint{1.788179in}{0.677462in}}%
\pgfpathcurveto{\pgfqpoint{1.785924in}{0.675206in}}{\pgfqpoint{1.784656in}{0.672146in}}{\pgfqpoint{1.784656in}{0.668957in}}%
\pgfpathcurveto{\pgfqpoint{1.784656in}{0.665767in}}{\pgfqpoint{1.785924in}{0.662707in}}{\pgfqpoint{1.788179in}{0.660451in}}%
\pgfpathcurveto{\pgfqpoint{1.790435in}{0.658196in}}{\pgfqpoint{1.793494in}{0.656928in}}{\pgfqpoint{1.796684in}{0.656928in}}%
\pgfpathclose%
\pgfusepath{stroke,fill}%
\end{pgfscope}%
\begin{pgfscope}%
\pgfpathrectangle{\pgfqpoint{0.750000in}{0.500000in}}{\pgfqpoint{4.650000in}{3.020000in}}%
\pgfusepath{clip}%
\pgfsetbuttcap%
\pgfsetroundjoin%
\definecolor{currentfill}{rgb}{0.121569,0.466667,0.705882}%
\pgfsetfillcolor{currentfill}%
\pgfsetlinewidth{1.003750pt}%
\definecolor{currentstroke}{rgb}{0.121569,0.466667,0.705882}%
\pgfsetstrokecolor{currentstroke}%
\pgfsetdash{}{0pt}%
\pgfpathmoveto{\pgfqpoint{4.102864in}{3.117124in}}%
\pgfpathcurveto{\pgfqpoint{4.107737in}{3.117124in}}{\pgfqpoint{4.112411in}{3.119060in}}{\pgfqpoint{4.115856in}{3.122505in}}%
\pgfpathcurveto{\pgfqpoint{4.119302in}{3.125951in}}{\pgfqpoint{4.121238in}{3.130624in}}{\pgfqpoint{4.121238in}{3.135497in}}%
\pgfpathcurveto{\pgfqpoint{4.121238in}{3.140370in}}{\pgfqpoint{4.119302in}{3.145044in}}{\pgfqpoint{4.115856in}{3.148489in}}%
\pgfpathcurveto{\pgfqpoint{4.112411in}{3.151934in}}{\pgfqpoint{4.107737in}{3.153870in}}{\pgfqpoint{4.102864in}{3.153870in}}%
\pgfpathcurveto{\pgfqpoint{4.097992in}{3.153870in}}{\pgfqpoint{4.093318in}{3.151934in}}{\pgfqpoint{4.089872in}{3.148489in}}%
\pgfpathcurveto{\pgfqpoint{4.086427in}{3.145044in}}{\pgfqpoint{4.084491in}{3.140370in}}{\pgfqpoint{4.084491in}{3.135497in}}%
\pgfpathcurveto{\pgfqpoint{4.084491in}{3.130624in}}{\pgfqpoint{4.086427in}{3.125951in}}{\pgfqpoint{4.089872in}{3.122505in}}%
\pgfpathcurveto{\pgfqpoint{4.093318in}{3.119060in}}{\pgfqpoint{4.097992in}{3.117124in}}{\pgfqpoint{4.102864in}{3.117124in}}%
\pgfpathclose%
\pgfusepath{stroke,fill}%
\end{pgfscope}%
\begin{pgfscope}%
\pgfpathrectangle{\pgfqpoint{0.750000in}{0.500000in}}{\pgfqpoint{4.650000in}{3.020000in}}%
\pgfusepath{clip}%
\pgfsetbuttcap%
\pgfsetroundjoin%
\definecolor{currentfill}{rgb}{0.121569,0.466667,0.705882}%
\pgfsetfillcolor{currentfill}%
\pgfsetlinewidth{1.003750pt}%
\definecolor{currentstroke}{rgb}{0.121569,0.466667,0.705882}%
\pgfsetstrokecolor{currentstroke}%
\pgfsetdash{}{0pt}%
\pgfpathmoveto{\pgfqpoint{2.723210in}{0.773881in}}%
\pgfpathcurveto{\pgfqpoint{2.727721in}{0.773881in}}{\pgfqpoint{2.732048in}{0.775673in}}{\pgfqpoint{2.735238in}{0.778863in}}%
\pgfpathcurveto{\pgfqpoint{2.738428in}{0.782053in}}{\pgfqpoint{2.740220in}{0.786380in}}{\pgfqpoint{2.740220in}{0.790891in}}%
\pgfpathcurveto{\pgfqpoint{2.740220in}{0.795403in}}{\pgfqpoint{2.738428in}{0.799730in}}{\pgfqpoint{2.735238in}{0.802920in}}%
\pgfpathcurveto{\pgfqpoint{2.732048in}{0.806109in}}{\pgfqpoint{2.727721in}{0.807902in}}{\pgfqpoint{2.723210in}{0.807902in}}%
\pgfpathcurveto{\pgfqpoint{2.718699in}{0.807902in}}{\pgfqpoint{2.714372in}{0.806109in}}{\pgfqpoint{2.711182in}{0.802920in}}%
\pgfpathcurveto{\pgfqpoint{2.707992in}{0.799730in}}{\pgfqpoint{2.706199in}{0.795403in}}{\pgfqpoint{2.706199in}{0.790891in}}%
\pgfpathcurveto{\pgfqpoint{2.706199in}{0.786380in}}{\pgfqpoint{2.707992in}{0.782053in}}{\pgfqpoint{2.711182in}{0.778863in}}%
\pgfpathcurveto{\pgfqpoint{2.714372in}{0.775673in}}{\pgfqpoint{2.718699in}{0.773881in}}{\pgfqpoint{2.723210in}{0.773881in}}%
\pgfpathclose%
\pgfusepath{stroke,fill}%
\end{pgfscope}%
\begin{pgfscope}%
\pgfpathrectangle{\pgfqpoint{0.750000in}{0.500000in}}{\pgfqpoint{4.650000in}{3.020000in}}%
\pgfusepath{clip}%
\pgfsetbuttcap%
\pgfsetroundjoin%
\definecolor{currentfill}{rgb}{0.121569,0.466667,0.705882}%
\pgfsetfillcolor{currentfill}%
\pgfsetlinewidth{1.003750pt}%
\definecolor{currentstroke}{rgb}{0.121569,0.466667,0.705882}%
\pgfsetstrokecolor{currentstroke}%
\pgfsetdash{}{0pt}%
\pgfpathmoveto{\pgfqpoint{1.075732in}{0.631292in}}%
\pgfpathcurveto{\pgfqpoint{1.078336in}{0.631292in}}{\pgfqpoint{1.080834in}{0.632327in}}{\pgfqpoint{1.082676in}{0.634169in}}%
\pgfpathcurveto{\pgfqpoint{1.084518in}{0.636010in}}{\pgfqpoint{1.085553in}{0.638509in}}{\pgfqpoint{1.085553in}{0.641113in}}%
\pgfpathcurveto{\pgfqpoint{1.085553in}{0.643718in}}{\pgfqpoint{1.084518in}{0.646216in}}{\pgfqpoint{1.082676in}{0.648058in}}%
\pgfpathcurveto{\pgfqpoint{1.080834in}{0.649899in}}{\pgfqpoint{1.078336in}{0.650934in}}{\pgfqpoint{1.075732in}{0.650934in}}%
\pgfpathcurveto{\pgfqpoint{1.073127in}{0.650934in}}{\pgfqpoint{1.070629in}{0.649899in}}{\pgfqpoint{1.068787in}{0.648058in}}%
\pgfpathcurveto{\pgfqpoint{1.066945in}{0.646216in}}{\pgfqpoint{1.065911in}{0.643718in}}{\pgfqpoint{1.065911in}{0.641113in}}%
\pgfpathcurveto{\pgfqpoint{1.065911in}{0.638509in}}{\pgfqpoint{1.066945in}{0.636010in}}{\pgfqpoint{1.068787in}{0.634169in}}%
\pgfpathcurveto{\pgfqpoint{1.070629in}{0.632327in}}{\pgfqpoint{1.073127in}{0.631292in}}{\pgfqpoint{1.075732in}{0.631292in}}%
\pgfpathclose%
\pgfusepath{stroke,fill}%
\end{pgfscope}%
\begin{pgfscope}%
\pgfpathrectangle{\pgfqpoint{0.750000in}{0.500000in}}{\pgfqpoint{4.650000in}{3.020000in}}%
\pgfusepath{clip}%
\pgfsetbuttcap%
\pgfsetroundjoin%
\definecolor{currentfill}{rgb}{0.121569,0.466667,0.705882}%
\pgfsetfillcolor{currentfill}%
\pgfsetlinewidth{1.003750pt}%
\definecolor{currentstroke}{rgb}{0.121569,0.466667,0.705882}%
\pgfsetstrokecolor{currentstroke}%
\pgfsetdash{}{0pt}%
\pgfpathmoveto{\pgfqpoint{1.737329in}{0.723657in}}%
\pgfpathcurveto{\pgfqpoint{1.740519in}{0.723657in}}{\pgfqpoint{1.743578in}{0.724924in}}{\pgfqpoint{1.745834in}{0.727180in}}%
\pgfpathcurveto{\pgfqpoint{1.748090in}{0.729435in}}{\pgfqpoint{1.749357in}{0.732495in}}{\pgfqpoint{1.749357in}{0.735685in}}%
\pgfpathcurveto{\pgfqpoint{1.749357in}{0.738875in}}{\pgfqpoint{1.748090in}{0.741934in}}{\pgfqpoint{1.745834in}{0.744190in}}%
\pgfpathcurveto{\pgfqpoint{1.743578in}{0.746445in}}{\pgfqpoint{1.740519in}{0.747713in}}{\pgfqpoint{1.737329in}{0.747713in}}%
\pgfpathcurveto{\pgfqpoint{1.734139in}{0.747713in}}{\pgfqpoint{1.731079in}{0.746445in}}{\pgfqpoint{1.728824in}{0.744190in}}%
\pgfpathcurveto{\pgfqpoint{1.726568in}{0.741934in}}{\pgfqpoint{1.725301in}{0.738875in}}{\pgfqpoint{1.725301in}{0.735685in}}%
\pgfpathcurveto{\pgfqpoint{1.725301in}{0.732495in}}{\pgfqpoint{1.726568in}{0.729435in}}{\pgfqpoint{1.728824in}{0.727180in}}%
\pgfpathcurveto{\pgfqpoint{1.731079in}{0.724924in}}{\pgfqpoint{1.734139in}{0.723657in}}{\pgfqpoint{1.737329in}{0.723657in}}%
\pgfpathclose%
\pgfusepath{stroke,fill}%
\end{pgfscope}%
\begin{pgfscope}%
\pgfpathrectangle{\pgfqpoint{0.750000in}{0.500000in}}{\pgfqpoint{4.650000in}{3.020000in}}%
\pgfusepath{clip}%
\pgfsetbuttcap%
\pgfsetroundjoin%
\definecolor{currentfill}{rgb}{0.121569,0.466667,0.705882}%
\pgfsetfillcolor{currentfill}%
\pgfsetlinewidth{1.003750pt}%
\definecolor{currentstroke}{rgb}{0.121569,0.466667,0.705882}%
\pgfsetstrokecolor{currentstroke}%
\pgfsetdash{}{0pt}%
\pgfpathmoveto{\pgfqpoint{1.007690in}{0.638013in}}%
\pgfpathcurveto{\pgfqpoint{1.010294in}{0.638013in}}{\pgfqpoint{1.012793in}{0.639048in}}{\pgfqpoint{1.014634in}{0.640890in}}%
\pgfpathcurveto{\pgfqpoint{1.016476in}{0.642731in}}{\pgfqpoint{1.017511in}{0.645229in}}{\pgfqpoint{1.017511in}{0.647834in}}%
\pgfpathcurveto{\pgfqpoint{1.017511in}{0.650439in}}{\pgfqpoint{1.016476in}{0.652937in}}{\pgfqpoint{1.014634in}{0.654778in}}%
\pgfpathcurveto{\pgfqpoint{1.012793in}{0.656620in}}{\pgfqpoint{1.010294in}{0.657655in}}{\pgfqpoint{1.007690in}{0.657655in}}%
\pgfpathcurveto{\pgfqpoint{1.005085in}{0.657655in}}{\pgfqpoint{1.002587in}{0.656620in}}{\pgfqpoint{1.000745in}{0.654778in}}%
\pgfpathcurveto{\pgfqpoint{0.998904in}{0.652937in}}{\pgfqpoint{0.997869in}{0.650439in}}{\pgfqpoint{0.997869in}{0.647834in}}%
\pgfpathcurveto{\pgfqpoint{0.997869in}{0.645229in}}{\pgfqpoint{0.998904in}{0.642731in}}{\pgfqpoint{1.000745in}{0.640890in}}%
\pgfpathcurveto{\pgfqpoint{1.002587in}{0.639048in}}{\pgfqpoint{1.005085in}{0.638013in}}{\pgfqpoint{1.007690in}{0.638013in}}%
\pgfpathclose%
\pgfusepath{stroke,fill}%
\end{pgfscope}%
\begin{pgfscope}%
\pgfpathrectangle{\pgfqpoint{0.750000in}{0.500000in}}{\pgfqpoint{4.650000in}{3.020000in}}%
\pgfusepath{clip}%
\pgfsetbuttcap%
\pgfsetroundjoin%
\definecolor{currentfill}{rgb}{0.121569,0.466667,0.705882}%
\pgfsetfillcolor{currentfill}%
\pgfsetlinewidth{1.003750pt}%
\definecolor{currentstroke}{rgb}{0.121569,0.466667,0.705882}%
\pgfsetstrokecolor{currentstroke}%
\pgfsetdash{}{0pt}%
\pgfpathmoveto{\pgfqpoint{4.214337in}{0.777963in}}%
\pgfpathcurveto{\pgfqpoint{4.218020in}{0.777963in}}{\pgfqpoint{4.221553in}{0.779426in}}{\pgfqpoint{4.224158in}{0.782031in}}%
\pgfpathcurveto{\pgfqpoint{4.226762in}{0.784635in}}{\pgfqpoint{4.228226in}{0.788168in}}{\pgfqpoint{4.228226in}{0.791852in}}%
\pgfpathcurveto{\pgfqpoint{4.228226in}{0.795535in}}{\pgfqpoint{4.226762in}{0.799068in}}{\pgfqpoint{4.224158in}{0.801672in}}%
\pgfpathcurveto{\pgfqpoint{4.221553in}{0.804277in}}{\pgfqpoint{4.218020in}{0.805740in}}{\pgfqpoint{4.214337in}{0.805740in}}%
\pgfpathcurveto{\pgfqpoint{4.210653in}{0.805740in}}{\pgfqpoint{4.207120in}{0.804277in}}{\pgfqpoint{4.204516in}{0.801672in}}%
\pgfpathcurveto{\pgfqpoint{4.201911in}{0.799068in}}{\pgfqpoint{4.200448in}{0.795535in}}{\pgfqpoint{4.200448in}{0.791852in}}%
\pgfpathcurveto{\pgfqpoint{4.200448in}{0.788168in}}{\pgfqpoint{4.201911in}{0.784635in}}{\pgfqpoint{4.204516in}{0.782031in}}%
\pgfpathcurveto{\pgfqpoint{4.207120in}{0.779426in}}{\pgfqpoint{4.210653in}{0.777963in}}{\pgfqpoint{4.214337in}{0.777963in}}%
\pgfpathclose%
\pgfusepath{stroke,fill}%
\end{pgfscope}%
\begin{pgfscope}%
\pgfpathrectangle{\pgfqpoint{0.750000in}{0.500000in}}{\pgfqpoint{4.650000in}{3.020000in}}%
\pgfusepath{clip}%
\pgfsetbuttcap%
\pgfsetroundjoin%
\definecolor{currentfill}{rgb}{0.121569,0.466667,0.705882}%
\pgfsetfillcolor{currentfill}%
\pgfsetlinewidth{1.003750pt}%
\definecolor{currentstroke}{rgb}{0.121569,0.466667,0.705882}%
\pgfsetstrokecolor{currentstroke}%
\pgfsetdash{}{0pt}%
\pgfpathmoveto{\pgfqpoint{2.381554in}{0.829750in}}%
\pgfpathcurveto{\pgfqpoint{2.384743in}{0.829750in}}{\pgfqpoint{2.387803in}{0.831017in}}{\pgfqpoint{2.390059in}{0.833272in}}%
\pgfpathcurveto{\pgfqpoint{2.392314in}{0.835528in}}{\pgfqpoint{2.393582in}{0.838588in}}{\pgfqpoint{2.393582in}{0.841778in}}%
\pgfpathcurveto{\pgfqpoint{2.393582in}{0.844968in}}{\pgfqpoint{2.392314in}{0.848027in}}{\pgfqpoint{2.390059in}{0.850283in}}%
\pgfpathcurveto{\pgfqpoint{2.387803in}{0.852538in}}{\pgfqpoint{2.384743in}{0.853806in}}{\pgfqpoint{2.381554in}{0.853806in}}%
\pgfpathcurveto{\pgfqpoint{2.378364in}{0.853806in}}{\pgfqpoint{2.375304in}{0.852538in}}{\pgfqpoint{2.373048in}{0.850283in}}%
\pgfpathcurveto{\pgfqpoint{2.370793in}{0.848027in}}{\pgfqpoint{2.369525in}{0.844968in}}{\pgfqpoint{2.369525in}{0.841778in}}%
\pgfpathcurveto{\pgfqpoint{2.369525in}{0.838588in}}{\pgfqpoint{2.370793in}{0.835528in}}{\pgfqpoint{2.373048in}{0.833272in}}%
\pgfpathcurveto{\pgfqpoint{2.375304in}{0.831017in}}{\pgfqpoint{2.378364in}{0.829750in}}{\pgfqpoint{2.381554in}{0.829750in}}%
\pgfpathclose%
\pgfusepath{stroke,fill}%
\end{pgfscope}%
\begin{pgfscope}%
\pgfpathrectangle{\pgfqpoint{0.750000in}{0.500000in}}{\pgfqpoint{4.650000in}{3.020000in}}%
\pgfusepath{clip}%
\pgfsetbuttcap%
\pgfsetroundjoin%
\definecolor{currentfill}{rgb}{0.121569,0.466667,0.705882}%
\pgfsetfillcolor{currentfill}%
\pgfsetlinewidth{1.003750pt}%
\definecolor{currentstroke}{rgb}{0.121569,0.466667,0.705882}%
\pgfsetstrokecolor{currentstroke}%
\pgfsetdash{}{0pt}%
\pgfpathmoveto{\pgfqpoint{2.488683in}{0.704741in}}%
\pgfpathcurveto{\pgfqpoint{2.491288in}{0.704741in}}{\pgfqpoint{2.493786in}{0.705776in}}{\pgfqpoint{2.495628in}{0.707618in}}%
\pgfpathcurveto{\pgfqpoint{2.497469in}{0.709459in}}{\pgfqpoint{2.498504in}{0.711958in}}{\pgfqpoint{2.498504in}{0.714562in}}%
\pgfpathcurveto{\pgfqpoint{2.498504in}{0.717167in}}{\pgfqpoint{2.497469in}{0.719665in}}{\pgfqpoint{2.495628in}{0.721507in}}%
\pgfpathcurveto{\pgfqpoint{2.493786in}{0.723348in}}{\pgfqpoint{2.491288in}{0.724383in}}{\pgfqpoint{2.488683in}{0.724383in}}%
\pgfpathcurveto{\pgfqpoint{2.486079in}{0.724383in}}{\pgfqpoint{2.483580in}{0.723348in}}{\pgfqpoint{2.481739in}{0.721507in}}%
\pgfpathcurveto{\pgfqpoint{2.479897in}{0.719665in}}{\pgfqpoint{2.478862in}{0.717167in}}{\pgfqpoint{2.478862in}{0.714562in}}%
\pgfpathcurveto{\pgfqpoint{2.478862in}{0.711958in}}{\pgfqpoint{2.479897in}{0.709459in}}{\pgfqpoint{2.481739in}{0.707618in}}%
\pgfpathcurveto{\pgfqpoint{2.483580in}{0.705776in}}{\pgfqpoint{2.486079in}{0.704741in}}{\pgfqpoint{2.488683in}{0.704741in}}%
\pgfpathclose%
\pgfusepath{stroke,fill}%
\end{pgfscope}%
\begin{pgfscope}%
\pgfpathrectangle{\pgfqpoint{0.750000in}{0.500000in}}{\pgfqpoint{4.650000in}{3.020000in}}%
\pgfusepath{clip}%
\pgfsetbuttcap%
\pgfsetroundjoin%
\definecolor{currentfill}{rgb}{0.121569,0.466667,0.705882}%
\pgfsetfillcolor{currentfill}%
\pgfsetlinewidth{1.003750pt}%
\definecolor{currentstroke}{rgb}{0.121569,0.466667,0.705882}%
\pgfsetstrokecolor{currentstroke}%
\pgfsetdash{}{0pt}%
\pgfpathmoveto{\pgfqpoint{1.437656in}{0.645694in}}%
\pgfpathcurveto{\pgfqpoint{1.440260in}{0.645694in}}{\pgfqpoint{1.442758in}{0.646729in}}{\pgfqpoint{1.444600in}{0.648571in}}%
\pgfpathcurveto{\pgfqpoint{1.446442in}{0.650412in}}{\pgfqpoint{1.447477in}{0.652910in}}{\pgfqpoint{1.447477in}{0.655515in}}%
\pgfpathcurveto{\pgfqpoint{1.447477in}{0.658119in}}{\pgfqpoint{1.446442in}{0.660618in}}{\pgfqpoint{1.444600in}{0.662459in}}%
\pgfpathcurveto{\pgfqpoint{1.442758in}{0.664301in}}{\pgfqpoint{1.440260in}{0.665336in}}{\pgfqpoint{1.437656in}{0.665336in}}%
\pgfpathcurveto{\pgfqpoint{1.435051in}{0.665336in}}{\pgfqpoint{1.432553in}{0.664301in}}{\pgfqpoint{1.430711in}{0.662459in}}%
\pgfpathcurveto{\pgfqpoint{1.428870in}{0.660618in}}{\pgfqpoint{1.427835in}{0.658119in}}{\pgfqpoint{1.427835in}{0.655515in}}%
\pgfpathcurveto{\pgfqpoint{1.427835in}{0.652910in}}{\pgfqpoint{1.428870in}{0.650412in}}{\pgfqpoint{1.430711in}{0.648571in}}%
\pgfpathcurveto{\pgfqpoint{1.432553in}{0.646729in}}{\pgfqpoint{1.435051in}{0.645694in}}{\pgfqpoint{1.437656in}{0.645694in}}%
\pgfpathclose%
\pgfusepath{stroke,fill}%
\end{pgfscope}%
\begin{pgfscope}%
\pgfpathrectangle{\pgfqpoint{0.750000in}{0.500000in}}{\pgfqpoint{4.650000in}{3.020000in}}%
\pgfusepath{clip}%
\pgfsetbuttcap%
\pgfsetroundjoin%
\definecolor{currentfill}{rgb}{0.121569,0.466667,0.705882}%
\pgfsetfillcolor{currentfill}%
\pgfsetlinewidth{1.003750pt}%
\definecolor{currentstroke}{rgb}{0.121569,0.466667,0.705882}%
\pgfsetstrokecolor{currentstroke}%
\pgfsetdash{}{0pt}%
\pgfpathmoveto{\pgfqpoint{1.793789in}{0.651935in}}%
\pgfpathcurveto{\pgfqpoint{1.796393in}{0.651935in}}{\pgfqpoint{1.798892in}{0.652970in}}{\pgfqpoint{1.800733in}{0.654811in}}%
\pgfpathcurveto{\pgfqpoint{1.802575in}{0.656653in}}{\pgfqpoint{1.803610in}{0.659151in}}{\pgfqpoint{1.803610in}{0.661756in}}%
\pgfpathcurveto{\pgfqpoint{1.803610in}{0.664360in}}{\pgfqpoint{1.802575in}{0.666858in}}{\pgfqpoint{1.800733in}{0.668700in}}%
\pgfpathcurveto{\pgfqpoint{1.798892in}{0.670542in}}{\pgfqpoint{1.796393in}{0.671577in}}{\pgfqpoint{1.793789in}{0.671577in}}%
\pgfpathcurveto{\pgfqpoint{1.791184in}{0.671577in}}{\pgfqpoint{1.788686in}{0.670542in}}{\pgfqpoint{1.786844in}{0.668700in}}%
\pgfpathcurveto{\pgfqpoint{1.785003in}{0.666858in}}{\pgfqpoint{1.783968in}{0.664360in}}{\pgfqpoint{1.783968in}{0.661756in}}%
\pgfpathcurveto{\pgfqpoint{1.783968in}{0.659151in}}{\pgfqpoint{1.785003in}{0.656653in}}{\pgfqpoint{1.786844in}{0.654811in}}%
\pgfpathcurveto{\pgfqpoint{1.788686in}{0.652970in}}{\pgfqpoint{1.791184in}{0.651935in}}{\pgfqpoint{1.793789in}{0.651935in}}%
\pgfpathclose%
\pgfusepath{stroke,fill}%
\end{pgfscope}%
\begin{pgfscope}%
\pgfpathrectangle{\pgfqpoint{0.750000in}{0.500000in}}{\pgfqpoint{4.650000in}{3.020000in}}%
\pgfusepath{clip}%
\pgfsetbuttcap%
\pgfsetroundjoin%
\definecolor{currentfill}{rgb}{0.121569,0.466667,0.705882}%
\pgfsetfillcolor{currentfill}%
\pgfsetlinewidth{1.003750pt}%
\definecolor{currentstroke}{rgb}{0.121569,0.466667,0.705882}%
\pgfsetstrokecolor{currentstroke}%
\pgfsetdash{}{0pt}%
\pgfpathmoveto{\pgfqpoint{1.434760in}{0.654108in}}%
\pgfpathcurveto{\pgfqpoint{1.438444in}{0.654108in}}{\pgfqpoint{1.441977in}{0.655571in}}{\pgfqpoint{1.444581in}{0.658176in}}%
\pgfpathcurveto{\pgfqpoint{1.447186in}{0.660780in}}{\pgfqpoint{1.448649in}{0.664313in}}{\pgfqpoint{1.448649in}{0.667996in}}%
\pgfpathcurveto{\pgfqpoint{1.448649in}{0.671680in}}{\pgfqpoint{1.447186in}{0.675213in}}{\pgfqpoint{1.444581in}{0.677817in}}%
\pgfpathcurveto{\pgfqpoint{1.441977in}{0.680422in}}{\pgfqpoint{1.438444in}{0.681885in}}{\pgfqpoint{1.434760in}{0.681885in}}%
\pgfpathcurveto{\pgfqpoint{1.431077in}{0.681885in}}{\pgfqpoint{1.427544in}{0.680422in}}{\pgfqpoint{1.424939in}{0.677817in}}%
\pgfpathcurveto{\pgfqpoint{1.422335in}{0.675213in}}{\pgfqpoint{1.420871in}{0.671680in}}{\pgfqpoint{1.420871in}{0.667996in}}%
\pgfpathcurveto{\pgfqpoint{1.420871in}{0.664313in}}{\pgfqpoint{1.422335in}{0.660780in}}{\pgfqpoint{1.424939in}{0.658176in}}%
\pgfpathcurveto{\pgfqpoint{1.427544in}{0.655571in}}{\pgfqpoint{1.431077in}{0.654108in}}{\pgfqpoint{1.434760in}{0.654108in}}%
\pgfpathclose%
\pgfusepath{stroke,fill}%
\end{pgfscope}%
\begin{pgfscope}%
\pgfpathrectangle{\pgfqpoint{0.750000in}{0.500000in}}{\pgfqpoint{4.650000in}{3.020000in}}%
\pgfusepath{clip}%
\pgfsetbuttcap%
\pgfsetroundjoin%
\definecolor{currentfill}{rgb}{0.121569,0.466667,0.705882}%
\pgfsetfillcolor{currentfill}%
\pgfsetlinewidth{1.003750pt}%
\definecolor{currentstroke}{rgb}{0.121569,0.466667,0.705882}%
\pgfsetstrokecolor{currentstroke}%
\pgfsetdash{}{0pt}%
\pgfpathmoveto{\pgfqpoint{1.266828in}{0.628892in}}%
\pgfpathcurveto{\pgfqpoint{1.269432in}{0.628892in}}{\pgfqpoint{1.271930in}{0.629927in}}{\pgfqpoint{1.273772in}{0.631768in}}%
\pgfpathcurveto{\pgfqpoint{1.275614in}{0.633610in}}{\pgfqpoint{1.276648in}{0.636108in}}{\pgfqpoint{1.276648in}{0.638713in}}%
\pgfpathcurveto{\pgfqpoint{1.276648in}{0.641317in}}{\pgfqpoint{1.275614in}{0.643816in}}{\pgfqpoint{1.273772in}{0.645657in}}%
\pgfpathcurveto{\pgfqpoint{1.271930in}{0.647499in}}{\pgfqpoint{1.269432in}{0.648534in}}{\pgfqpoint{1.266828in}{0.648534in}}%
\pgfpathcurveto{\pgfqpoint{1.264223in}{0.648534in}}{\pgfqpoint{1.261725in}{0.647499in}}{\pgfqpoint{1.259883in}{0.645657in}}%
\pgfpathcurveto{\pgfqpoint{1.258041in}{0.643816in}}{\pgfqpoint{1.257007in}{0.641317in}}{\pgfqpoint{1.257007in}{0.638713in}}%
\pgfpathcurveto{\pgfqpoint{1.257007in}{0.636108in}}{\pgfqpoint{1.258041in}{0.633610in}}{\pgfqpoint{1.259883in}{0.631768in}}%
\pgfpathcurveto{\pgfqpoint{1.261725in}{0.629927in}}{\pgfqpoint{1.264223in}{0.628892in}}{\pgfqpoint{1.266828in}{0.628892in}}%
\pgfpathclose%
\pgfusepath{stroke,fill}%
\end{pgfscope}%
\begin{pgfscope}%
\pgfpathrectangle{\pgfqpoint{0.750000in}{0.500000in}}{\pgfqpoint{4.650000in}{3.020000in}}%
\pgfusepath{clip}%
\pgfsetbuttcap%
\pgfsetroundjoin%
\definecolor{currentfill}{rgb}{0.121569,0.466667,0.705882}%
\pgfsetfillcolor{currentfill}%
\pgfsetlinewidth{1.003750pt}%
\definecolor{currentstroke}{rgb}{0.121569,0.466667,0.705882}%
\pgfsetstrokecolor{currentstroke}%
\pgfsetdash{}{0pt}%
\pgfpathmoveto{\pgfqpoint{1.879203in}{0.659388in}}%
\pgfpathcurveto{\pgfqpoint{1.882886in}{0.659388in}}{\pgfqpoint{1.886419in}{0.660852in}}{\pgfqpoint{1.889024in}{0.663456in}}%
\pgfpathcurveto{\pgfqpoint{1.891628in}{0.666061in}}{\pgfqpoint{1.893092in}{0.669594in}}{\pgfqpoint{1.893092in}{0.673277in}}%
\pgfpathcurveto{\pgfqpoint{1.893092in}{0.676960in}}{\pgfqpoint{1.891628in}{0.680494in}}{\pgfqpoint{1.889024in}{0.683098in}}%
\pgfpathcurveto{\pgfqpoint{1.886419in}{0.685703in}}{\pgfqpoint{1.882886in}{0.687166in}}{\pgfqpoint{1.879203in}{0.687166in}}%
\pgfpathcurveto{\pgfqpoint{1.875520in}{0.687166in}}{\pgfqpoint{1.871987in}{0.685703in}}{\pgfqpoint{1.869382in}{0.683098in}}%
\pgfpathcurveto{\pgfqpoint{1.866778in}{0.680494in}}{\pgfqpoint{1.865314in}{0.676960in}}{\pgfqpoint{1.865314in}{0.673277in}}%
\pgfpathcurveto{\pgfqpoint{1.865314in}{0.669594in}}{\pgfqpoint{1.866778in}{0.666061in}}{\pgfqpoint{1.869382in}{0.663456in}}%
\pgfpathcurveto{\pgfqpoint{1.871987in}{0.660852in}}{\pgfqpoint{1.875520in}{0.659388in}}{\pgfqpoint{1.879203in}{0.659388in}}%
\pgfpathclose%
\pgfusepath{stroke,fill}%
\end{pgfscope}%
\begin{pgfscope}%
\pgfpathrectangle{\pgfqpoint{0.750000in}{0.500000in}}{\pgfqpoint{4.650000in}{3.020000in}}%
\pgfusepath{clip}%
\pgfsetbuttcap%
\pgfsetroundjoin%
\definecolor{currentfill}{rgb}{0.121569,0.466667,0.705882}%
\pgfsetfillcolor{currentfill}%
\pgfsetlinewidth{1.003750pt}%
\definecolor{currentstroke}{rgb}{0.121569,0.466667,0.705882}%
\pgfsetstrokecolor{currentstroke}%
\pgfsetdash{}{0pt}%
\pgfpathmoveto{\pgfqpoint{2.202039in}{0.666050in}}%
\pgfpathcurveto{\pgfqpoint{2.205229in}{0.666050in}}{\pgfqpoint{2.208289in}{0.667317in}}{\pgfqpoint{2.210544in}{0.669573in}}%
\pgfpathcurveto{\pgfqpoint{2.212800in}{0.671828in}}{\pgfqpoint{2.214067in}{0.674888in}}{\pgfqpoint{2.214067in}{0.678078in}}%
\pgfpathcurveto{\pgfqpoint{2.214067in}{0.681268in}}{\pgfqpoint{2.212800in}{0.684327in}}{\pgfqpoint{2.210544in}{0.686583in}}%
\pgfpathcurveto{\pgfqpoint{2.208289in}{0.688838in}}{\pgfqpoint{2.205229in}{0.690106in}}{\pgfqpoint{2.202039in}{0.690106in}}%
\pgfpathcurveto{\pgfqpoint{2.198849in}{0.690106in}}{\pgfqpoint{2.195790in}{0.688838in}}{\pgfqpoint{2.193534in}{0.686583in}}%
\pgfpathcurveto{\pgfqpoint{2.191278in}{0.684327in}}{\pgfqpoint{2.190011in}{0.681268in}}{\pgfqpoint{2.190011in}{0.678078in}}%
\pgfpathcurveto{\pgfqpoint{2.190011in}{0.674888in}}{\pgfqpoint{2.191278in}{0.671828in}}{\pgfqpoint{2.193534in}{0.669573in}}%
\pgfpathcurveto{\pgfqpoint{2.195790in}{0.667317in}}{\pgfqpoint{2.198849in}{0.666050in}}{\pgfqpoint{2.202039in}{0.666050in}}%
\pgfpathclose%
\pgfusepath{stroke,fill}%
\end{pgfscope}%
\begin{pgfscope}%
\pgfpathrectangle{\pgfqpoint{0.750000in}{0.500000in}}{\pgfqpoint{4.650000in}{3.020000in}}%
\pgfusepath{clip}%
\pgfsetbuttcap%
\pgfsetroundjoin%
\definecolor{currentfill}{rgb}{0.121569,0.466667,0.705882}%
\pgfsetfillcolor{currentfill}%
\pgfsetlinewidth{1.003750pt}%
\definecolor{currentstroke}{rgb}{0.121569,0.466667,0.705882}%
\pgfsetstrokecolor{currentstroke}%
\pgfsetdash{}{0pt}%
\pgfpathmoveto{\pgfqpoint{1.256694in}{0.640126in}}%
\pgfpathcurveto{\pgfqpoint{1.259884in}{0.640126in}}{\pgfqpoint{1.262943in}{0.641394in}}{\pgfqpoint{1.265199in}{0.643649in}}%
\pgfpathcurveto{\pgfqpoint{1.267454in}{0.645905in}}{\pgfqpoint{1.268722in}{0.648965in}}{\pgfqpoint{1.268722in}{0.652155in}}%
\pgfpathcurveto{\pgfqpoint{1.268722in}{0.655344in}}{\pgfqpoint{1.267454in}{0.658404in}}{\pgfqpoint{1.265199in}{0.660660in}}%
\pgfpathcurveto{\pgfqpoint{1.262943in}{0.662915in}}{\pgfqpoint{1.259884in}{0.664183in}}{\pgfqpoint{1.256694in}{0.664183in}}%
\pgfpathcurveto{\pgfqpoint{1.253504in}{0.664183in}}{\pgfqpoint{1.250444in}{0.662915in}}{\pgfqpoint{1.248188in}{0.660660in}}%
\pgfpathcurveto{\pgfqpoint{1.245933in}{0.658404in}}{\pgfqpoint{1.244666in}{0.655344in}}{\pgfqpoint{1.244666in}{0.652155in}}%
\pgfpathcurveto{\pgfqpoint{1.244666in}{0.648965in}}{\pgfqpoint{1.245933in}{0.645905in}}{\pgfqpoint{1.248188in}{0.643649in}}%
\pgfpathcurveto{\pgfqpoint{1.250444in}{0.641394in}}{\pgfqpoint{1.253504in}{0.640126in}}{\pgfqpoint{1.256694in}{0.640126in}}%
\pgfpathclose%
\pgfusepath{stroke,fill}%
\end{pgfscope}%
\begin{pgfscope}%
\pgfpathrectangle{\pgfqpoint{0.750000in}{0.500000in}}{\pgfqpoint{4.650000in}{3.020000in}}%
\pgfusepath{clip}%
\pgfsetbuttcap%
\pgfsetroundjoin%
\definecolor{currentfill}{rgb}{0.121569,0.466667,0.705882}%
\pgfsetfillcolor{currentfill}%
\pgfsetlinewidth{1.003750pt}%
\definecolor{currentstroke}{rgb}{0.121569,0.466667,0.705882}%
\pgfsetstrokecolor{currentstroke}%
\pgfsetdash{}{0pt}%
\pgfpathmoveto{\pgfqpoint{1.029405in}{0.630812in}}%
\pgfpathcurveto{\pgfqpoint{1.032010in}{0.630812in}}{\pgfqpoint{1.034508in}{0.631847in}}{\pgfqpoint{1.036350in}{0.633689in}}%
\pgfpathcurveto{\pgfqpoint{1.038191in}{0.635530in}}{\pgfqpoint{1.039226in}{0.638029in}}{\pgfqpoint{1.039226in}{0.640633in}}%
\pgfpathcurveto{\pgfqpoint{1.039226in}{0.643238in}}{\pgfqpoint{1.038191in}{0.645736in}}{\pgfqpoint{1.036350in}{0.647578in}}%
\pgfpathcurveto{\pgfqpoint{1.034508in}{0.649419in}}{\pgfqpoint{1.032010in}{0.650454in}}{\pgfqpoint{1.029405in}{0.650454in}}%
\pgfpathcurveto{\pgfqpoint{1.026801in}{0.650454in}}{\pgfqpoint{1.024303in}{0.649419in}}{\pgfqpoint{1.022461in}{0.647578in}}%
\pgfpathcurveto{\pgfqpoint{1.020619in}{0.645736in}}{\pgfqpoint{1.019584in}{0.643238in}}{\pgfqpoint{1.019584in}{0.640633in}}%
\pgfpathcurveto{\pgfqpoint{1.019584in}{0.638029in}}{\pgfqpoint{1.020619in}{0.635530in}}{\pgfqpoint{1.022461in}{0.633689in}}%
\pgfpathcurveto{\pgfqpoint{1.024303in}{0.631847in}}{\pgfqpoint{1.026801in}{0.630812in}}{\pgfqpoint{1.029405in}{0.630812in}}%
\pgfpathclose%
\pgfusepath{stroke,fill}%
\end{pgfscope}%
\begin{pgfscope}%
\pgfpathrectangle{\pgfqpoint{0.750000in}{0.500000in}}{\pgfqpoint{4.650000in}{3.020000in}}%
\pgfusepath{clip}%
\pgfsetbuttcap%
\pgfsetroundjoin%
\definecolor{currentfill}{rgb}{0.121569,0.466667,0.705882}%
\pgfsetfillcolor{currentfill}%
\pgfsetlinewidth{1.003750pt}%
\definecolor{currentstroke}{rgb}{0.121569,0.466667,0.705882}%
\pgfsetstrokecolor{currentstroke}%
\pgfsetdash{}{0pt}%
\pgfpathmoveto{\pgfqpoint{2.342466in}{0.692032in}}%
\pgfpathcurveto{\pgfqpoint{2.346149in}{0.692032in}}{\pgfqpoint{2.349682in}{0.693496in}}{\pgfqpoint{2.352287in}{0.696100in}}%
\pgfpathcurveto{\pgfqpoint{2.354891in}{0.698705in}}{\pgfqpoint{2.356355in}{0.702238in}}{\pgfqpoint{2.356355in}{0.705921in}}%
\pgfpathcurveto{\pgfqpoint{2.356355in}{0.709604in}}{\pgfqpoint{2.354891in}{0.713137in}}{\pgfqpoint{2.352287in}{0.715742in}}%
\pgfpathcurveto{\pgfqpoint{2.349682in}{0.718347in}}{\pgfqpoint{2.346149in}{0.719810in}}{\pgfqpoint{2.342466in}{0.719810in}}%
\pgfpathcurveto{\pgfqpoint{2.338782in}{0.719810in}}{\pgfqpoint{2.335249in}{0.718347in}}{\pgfqpoint{2.332645in}{0.715742in}}%
\pgfpathcurveto{\pgfqpoint{2.330040in}{0.713137in}}{\pgfqpoint{2.328577in}{0.709604in}}{\pgfqpoint{2.328577in}{0.705921in}}%
\pgfpathcurveto{\pgfqpoint{2.328577in}{0.702238in}}{\pgfqpoint{2.330040in}{0.698705in}}{\pgfqpoint{2.332645in}{0.696100in}}%
\pgfpathcurveto{\pgfqpoint{2.335249in}{0.693496in}}{\pgfqpoint{2.338782in}{0.692032in}}{\pgfqpoint{2.342466in}{0.692032in}}%
\pgfpathclose%
\pgfusepath{stroke,fill}%
\end{pgfscope}%
\begin{pgfscope}%
\pgfpathrectangle{\pgfqpoint{0.750000in}{0.500000in}}{\pgfqpoint{4.650000in}{3.020000in}}%
\pgfusepath{clip}%
\pgfsetbuttcap%
\pgfsetroundjoin%
\definecolor{currentfill}{rgb}{0.121569,0.466667,0.705882}%
\pgfsetfillcolor{currentfill}%
\pgfsetlinewidth{1.003750pt}%
\definecolor{currentstroke}{rgb}{0.121569,0.466667,0.705882}%
\pgfsetstrokecolor{currentstroke}%
\pgfsetdash{}{0pt}%
\pgfpathmoveto{\pgfqpoint{4.642855in}{0.875415in}}%
\pgfpathcurveto{\pgfqpoint{4.646538in}{0.875415in}}{\pgfqpoint{4.650071in}{0.876878in}}{\pgfqpoint{4.652676in}{0.879483in}}%
\pgfpathcurveto{\pgfqpoint{4.655280in}{0.882087in}}{\pgfqpoint{4.656744in}{0.885620in}}{\pgfqpoint{4.656744in}{0.889303in}}%
\pgfpathcurveto{\pgfqpoint{4.656744in}{0.892987in}}{\pgfqpoint{4.655280in}{0.896520in}}{\pgfqpoint{4.652676in}{0.899124in}}%
\pgfpathcurveto{\pgfqpoint{4.650071in}{0.901729in}}{\pgfqpoint{4.646538in}{0.903192in}}{\pgfqpoint{4.642855in}{0.903192in}}%
\pgfpathcurveto{\pgfqpoint{4.639172in}{0.903192in}}{\pgfqpoint{4.635639in}{0.901729in}}{\pgfqpoint{4.633034in}{0.899124in}}%
\pgfpathcurveto{\pgfqpoint{4.630429in}{0.896520in}}{\pgfqpoint{4.628966in}{0.892987in}}{\pgfqpoint{4.628966in}{0.889303in}}%
\pgfpathcurveto{\pgfqpoint{4.628966in}{0.885620in}}{\pgfqpoint{4.630429in}{0.882087in}}{\pgfqpoint{4.633034in}{0.879483in}}%
\pgfpathcurveto{\pgfqpoint{4.635639in}{0.876878in}}{\pgfqpoint{4.639172in}{0.875415in}}{\pgfqpoint{4.642855in}{0.875415in}}%
\pgfpathclose%
\pgfusepath{stroke,fill}%
\end{pgfscope}%
\begin{pgfscope}%
\pgfpathrectangle{\pgfqpoint{0.750000in}{0.500000in}}{\pgfqpoint{4.650000in}{3.020000in}}%
\pgfusepath{clip}%
\pgfsetbuttcap%
\pgfsetroundjoin%
\definecolor{currentfill}{rgb}{0.121569,0.466667,0.705882}%
\pgfsetfillcolor{currentfill}%
\pgfsetlinewidth{1.003750pt}%
\definecolor{currentstroke}{rgb}{0.121569,0.466667,0.705882}%
\pgfsetstrokecolor{currentstroke}%
\pgfsetdash{}{0pt}%
\pgfpathmoveto{\pgfqpoint{1.307363in}{0.634846in}}%
\pgfpathcurveto{\pgfqpoint{1.310553in}{0.634846in}}{\pgfqpoint{1.313613in}{0.636113in}}{\pgfqpoint{1.315868in}{0.638369in}}%
\pgfpathcurveto{\pgfqpoint{1.318124in}{0.640624in}}{\pgfqpoint{1.319391in}{0.643684in}}{\pgfqpoint{1.319391in}{0.646874in}}%
\pgfpathcurveto{\pgfqpoint{1.319391in}{0.650064in}}{\pgfqpoint{1.318124in}{0.653123in}}{\pgfqpoint{1.315868in}{0.655379in}}%
\pgfpathcurveto{\pgfqpoint{1.313613in}{0.657635in}}{\pgfqpoint{1.310553in}{0.658902in}}{\pgfqpoint{1.307363in}{0.658902in}}%
\pgfpathcurveto{\pgfqpoint{1.304173in}{0.658902in}}{\pgfqpoint{1.301113in}{0.657635in}}{\pgfqpoint{1.298858in}{0.655379in}}%
\pgfpathcurveto{\pgfqpoint{1.296602in}{0.653123in}}{\pgfqpoint{1.295335in}{0.650064in}}{\pgfqpoint{1.295335in}{0.646874in}}%
\pgfpathcurveto{\pgfqpoint{1.295335in}{0.643684in}}{\pgfqpoint{1.296602in}{0.640624in}}{\pgfqpoint{1.298858in}{0.638369in}}%
\pgfpathcurveto{\pgfqpoint{1.301113in}{0.636113in}}{\pgfqpoint{1.304173in}{0.634846in}}{\pgfqpoint{1.307363in}{0.634846in}}%
\pgfpathclose%
\pgfusepath{stroke,fill}%
\end{pgfscope}%
\begin{pgfscope}%
\pgfpathrectangle{\pgfqpoint{0.750000in}{0.500000in}}{\pgfqpoint{4.650000in}{3.020000in}}%
\pgfusepath{clip}%
\pgfsetbuttcap%
\pgfsetroundjoin%
\definecolor{currentfill}{rgb}{0.121569,0.466667,0.705882}%
\pgfsetfillcolor{currentfill}%
\pgfsetlinewidth{1.003750pt}%
\definecolor{currentstroke}{rgb}{0.121569,0.466667,0.705882}%
\pgfsetstrokecolor{currentstroke}%
\pgfsetdash{}{0pt}%
\pgfpathmoveto{\pgfqpoint{2.147027in}{0.695632in}}%
\pgfpathcurveto{\pgfqpoint{2.151538in}{0.695632in}}{\pgfqpoint{2.155865in}{0.697424in}}{\pgfqpoint{2.159055in}{0.700614in}}%
\pgfpathcurveto{\pgfqpoint{2.162245in}{0.703804in}}{\pgfqpoint{2.164037in}{0.708131in}}{\pgfqpoint{2.164037in}{0.712642in}}%
\pgfpathcurveto{\pgfqpoint{2.164037in}{0.717153in}}{\pgfqpoint{2.162245in}{0.721480in}}{\pgfqpoint{2.159055in}{0.724670in}}%
\pgfpathcurveto{\pgfqpoint{2.155865in}{0.727860in}}{\pgfqpoint{2.151538in}{0.729652in}}{\pgfqpoint{2.147027in}{0.729652in}}%
\pgfpathcurveto{\pgfqpoint{2.142516in}{0.729652in}}{\pgfqpoint{2.138189in}{0.727860in}}{\pgfqpoint{2.134999in}{0.724670in}}%
\pgfpathcurveto{\pgfqpoint{2.131809in}{0.721480in}}{\pgfqpoint{2.130016in}{0.717153in}}{\pgfqpoint{2.130016in}{0.712642in}}%
\pgfpathcurveto{\pgfqpoint{2.130016in}{0.708131in}}{\pgfqpoint{2.131809in}{0.703804in}}{\pgfqpoint{2.134999in}{0.700614in}}%
\pgfpathcurveto{\pgfqpoint{2.138189in}{0.697424in}}{\pgfqpoint{2.142516in}{0.695632in}}{\pgfqpoint{2.147027in}{0.695632in}}%
\pgfpathclose%
\pgfusepath{stroke,fill}%
\end{pgfscope}%
\begin{pgfscope}%
\pgfpathrectangle{\pgfqpoint{0.750000in}{0.500000in}}{\pgfqpoint{4.650000in}{3.020000in}}%
\pgfusepath{clip}%
\pgfsetbuttcap%
\pgfsetroundjoin%
\definecolor{currentfill}{rgb}{0.121569,0.466667,0.705882}%
\pgfsetfillcolor{currentfill}%
\pgfsetlinewidth{1.003750pt}%
\definecolor{currentstroke}{rgb}{0.121569,0.466667,0.705882}%
\pgfsetstrokecolor{currentstroke}%
\pgfsetdash{}{0pt}%
\pgfpathmoveto{\pgfqpoint{3.115535in}{0.698500in}}%
\pgfpathcurveto{\pgfqpoint{3.118140in}{0.698500in}}{\pgfqpoint{3.120638in}{0.699535in}}{\pgfqpoint{3.122480in}{0.701377in}}%
\pgfpathcurveto{\pgfqpoint{3.124322in}{0.703219in}}{\pgfqpoint{3.125356in}{0.705717in}}{\pgfqpoint{3.125356in}{0.708321in}}%
\pgfpathcurveto{\pgfqpoint{3.125356in}{0.710926in}}{\pgfqpoint{3.124322in}{0.713424in}}{\pgfqpoint{3.122480in}{0.715266in}}%
\pgfpathcurveto{\pgfqpoint{3.120638in}{0.717108in}}{\pgfqpoint{3.118140in}{0.718142in}}{\pgfqpoint{3.115535in}{0.718142in}}%
\pgfpathcurveto{\pgfqpoint{3.112931in}{0.718142in}}{\pgfqpoint{3.110433in}{0.717108in}}{\pgfqpoint{3.108591in}{0.715266in}}%
\pgfpathcurveto{\pgfqpoint{3.106749in}{0.713424in}}{\pgfqpoint{3.105715in}{0.710926in}}{\pgfqpoint{3.105715in}{0.708321in}}%
\pgfpathcurveto{\pgfqpoint{3.105715in}{0.705717in}}{\pgfqpoint{3.106749in}{0.703219in}}{\pgfqpoint{3.108591in}{0.701377in}}%
\pgfpathcurveto{\pgfqpoint{3.110433in}{0.699535in}}{\pgfqpoint{3.112931in}{0.698500in}}{\pgfqpoint{3.115535in}{0.698500in}}%
\pgfpathclose%
\pgfusepath{stroke,fill}%
\end{pgfscope}%
\begin{pgfscope}%
\pgfpathrectangle{\pgfqpoint{0.750000in}{0.500000in}}{\pgfqpoint{4.650000in}{3.020000in}}%
\pgfusepath{clip}%
\pgfsetbuttcap%
\pgfsetroundjoin%
\definecolor{currentfill}{rgb}{0.121569,0.466667,0.705882}%
\pgfsetfillcolor{currentfill}%
\pgfsetlinewidth{1.003750pt}%
\definecolor{currentstroke}{rgb}{0.121569,0.466667,0.705882}%
\pgfsetstrokecolor{currentstroke}%
\pgfsetdash{}{0pt}%
\pgfpathmoveto{\pgfqpoint{1.405806in}{0.633693in}}%
\pgfpathcurveto{\pgfqpoint{1.408411in}{0.633693in}}{\pgfqpoint{1.410909in}{0.634727in}}{\pgfqpoint{1.412751in}{0.636569in}}%
\pgfpathcurveto{\pgfqpoint{1.414592in}{0.638411in}}{\pgfqpoint{1.415627in}{0.640909in}}{\pgfqpoint{1.415627in}{0.643513in}}%
\pgfpathcurveto{\pgfqpoint{1.415627in}{0.646118in}}{\pgfqpoint{1.414592in}{0.648616in}}{\pgfqpoint{1.412751in}{0.650458in}}%
\pgfpathcurveto{\pgfqpoint{1.410909in}{0.652300in}}{\pgfqpoint{1.408411in}{0.653334in}}{\pgfqpoint{1.405806in}{0.653334in}}%
\pgfpathcurveto{\pgfqpoint{1.403202in}{0.653334in}}{\pgfqpoint{1.400704in}{0.652300in}}{\pgfqpoint{1.398862in}{0.650458in}}%
\pgfpathcurveto{\pgfqpoint{1.397020in}{0.648616in}}{\pgfqpoint{1.395985in}{0.646118in}}{\pgfqpoint{1.395985in}{0.643513in}}%
\pgfpathcurveto{\pgfqpoint{1.395985in}{0.640909in}}{\pgfqpoint{1.397020in}{0.638411in}}{\pgfqpoint{1.398862in}{0.636569in}}%
\pgfpathcurveto{\pgfqpoint{1.400704in}{0.634727in}}{\pgfqpoint{1.403202in}{0.633693in}}{\pgfqpoint{1.405806in}{0.633693in}}%
\pgfpathclose%
\pgfusepath{stroke,fill}%
\end{pgfscope}%
\begin{pgfscope}%
\pgfpathrectangle{\pgfqpoint{0.750000in}{0.500000in}}{\pgfqpoint{4.650000in}{3.020000in}}%
\pgfusepath{clip}%
\pgfsetbuttcap%
\pgfsetroundjoin%
\definecolor{currentfill}{rgb}{0.121569,0.466667,0.705882}%
\pgfsetfillcolor{currentfill}%
\pgfsetlinewidth{1.003750pt}%
\definecolor{currentstroke}{rgb}{0.121569,0.466667,0.705882}%
\pgfsetstrokecolor{currentstroke}%
\pgfsetdash{}{0pt}%
\pgfpathmoveto{\pgfqpoint{1.019271in}{0.627452in}}%
\pgfpathcurveto{\pgfqpoint{1.021876in}{0.627452in}}{\pgfqpoint{1.024374in}{0.628487in}}{\pgfqpoint{1.026216in}{0.630328in}}%
\pgfpathcurveto{\pgfqpoint{1.028058in}{0.632170in}}{\pgfqpoint{1.029092in}{0.634668in}}{\pgfqpoint{1.029092in}{0.637273in}}%
\pgfpathcurveto{\pgfqpoint{1.029092in}{0.639877in}}{\pgfqpoint{1.028058in}{0.642375in}}{\pgfqpoint{1.026216in}{0.644217in}}%
\pgfpathcurveto{\pgfqpoint{1.024374in}{0.646059in}}{\pgfqpoint{1.021876in}{0.647094in}}{\pgfqpoint{1.019271in}{0.647094in}}%
\pgfpathcurveto{\pgfqpoint{1.016667in}{0.647094in}}{\pgfqpoint{1.014169in}{0.646059in}}{\pgfqpoint{1.012327in}{0.644217in}}%
\pgfpathcurveto{\pgfqpoint{1.010485in}{0.642375in}}{\pgfqpoint{1.009451in}{0.639877in}}{\pgfqpoint{1.009451in}{0.637273in}}%
\pgfpathcurveto{\pgfqpoint{1.009451in}{0.634668in}}{\pgfqpoint{1.010485in}{0.632170in}}{\pgfqpoint{1.012327in}{0.630328in}}%
\pgfpathcurveto{\pgfqpoint{1.014169in}{0.628487in}}{\pgfqpoint{1.016667in}{0.627452in}}{\pgfqpoint{1.019271in}{0.627452in}}%
\pgfpathclose%
\pgfusepath{stroke,fill}%
\end{pgfscope}%
\begin{pgfscope}%
\pgfpathrectangle{\pgfqpoint{0.750000in}{0.500000in}}{\pgfqpoint{4.650000in}{3.020000in}}%
\pgfusepath{clip}%
\pgfsetbuttcap%
\pgfsetroundjoin%
\definecolor{currentfill}{rgb}{0.121569,0.466667,0.705882}%
\pgfsetfillcolor{currentfill}%
\pgfsetlinewidth{1.003750pt}%
\definecolor{currentstroke}{rgb}{0.121569,0.466667,0.705882}%
\pgfsetstrokecolor{currentstroke}%
\pgfsetdash{}{0pt}%
\pgfpathmoveto{\pgfqpoint{1.511488in}{0.648574in}}%
\pgfpathcurveto{\pgfqpoint{1.514093in}{0.648574in}}{\pgfqpoint{1.516591in}{0.649609in}}{\pgfqpoint{1.518433in}{0.651451in}}%
\pgfpathcurveto{\pgfqpoint{1.520274in}{0.653293in}}{\pgfqpoint{1.521309in}{0.655791in}}{\pgfqpoint{1.521309in}{0.658395in}}%
\pgfpathcurveto{\pgfqpoint{1.521309in}{0.661000in}}{\pgfqpoint{1.520274in}{0.663498in}}{\pgfqpoint{1.518433in}{0.665340in}}%
\pgfpathcurveto{\pgfqpoint{1.516591in}{0.667181in}}{\pgfqpoint{1.514093in}{0.668216in}}{\pgfqpoint{1.511488in}{0.668216in}}%
\pgfpathcurveto{\pgfqpoint{1.508884in}{0.668216in}}{\pgfqpoint{1.506385in}{0.667181in}}{\pgfqpoint{1.504544in}{0.665340in}}%
\pgfpathcurveto{\pgfqpoint{1.502702in}{0.663498in}}{\pgfqpoint{1.501667in}{0.661000in}}{\pgfqpoint{1.501667in}{0.658395in}}%
\pgfpathcurveto{\pgfqpoint{1.501667in}{0.655791in}}{\pgfqpoint{1.502702in}{0.653293in}}{\pgfqpoint{1.504544in}{0.651451in}}%
\pgfpathcurveto{\pgfqpoint{1.506385in}{0.649609in}}{\pgfqpoint{1.508884in}{0.648574in}}{\pgfqpoint{1.511488in}{0.648574in}}%
\pgfpathclose%
\pgfusepath{stroke,fill}%
\end{pgfscope}%
\begin{pgfscope}%
\pgfpathrectangle{\pgfqpoint{0.750000in}{0.500000in}}{\pgfqpoint{4.650000in}{3.020000in}}%
\pgfusepath{clip}%
\pgfsetbuttcap%
\pgfsetroundjoin%
\definecolor{currentfill}{rgb}{0.121569,0.466667,0.705882}%
\pgfsetfillcolor{currentfill}%
\pgfsetlinewidth{1.003750pt}%
\definecolor{currentstroke}{rgb}{0.121569,0.466667,0.705882}%
\pgfsetstrokecolor{currentstroke}%
\pgfsetdash{}{0pt}%
\pgfpathmoveto{\pgfqpoint{1.801027in}{0.695229in}}%
\pgfpathcurveto{\pgfqpoint{1.805900in}{0.695229in}}{\pgfqpoint{1.810574in}{0.697165in}}{\pgfqpoint{1.814019in}{0.700610in}}%
\pgfpathcurveto{\pgfqpoint{1.817465in}{0.704056in}}{\pgfqpoint{1.819401in}{0.708729in}}{\pgfqpoint{1.819401in}{0.713602in}}%
\pgfpathcurveto{\pgfqpoint{1.819401in}{0.718475in}}{\pgfqpoint{1.817465in}{0.723148in}}{\pgfqpoint{1.814019in}{0.726594in}}%
\pgfpathcurveto{\pgfqpoint{1.810574in}{0.730039in}}{\pgfqpoint{1.805900in}{0.731975in}}{\pgfqpoint{1.801027in}{0.731975in}}%
\pgfpathcurveto{\pgfqpoint{1.796155in}{0.731975in}}{\pgfqpoint{1.791481in}{0.730039in}}{\pgfqpoint{1.788036in}{0.726594in}}%
\pgfpathcurveto{\pgfqpoint{1.784590in}{0.723148in}}{\pgfqpoint{1.782654in}{0.718475in}}{\pgfqpoint{1.782654in}{0.713602in}}%
\pgfpathcurveto{\pgfqpoint{1.782654in}{0.708729in}}{\pgfqpoint{1.784590in}{0.704056in}}{\pgfqpoint{1.788036in}{0.700610in}}%
\pgfpathcurveto{\pgfqpoint{1.791481in}{0.697165in}}{\pgfqpoint{1.796155in}{0.695229in}}{\pgfqpoint{1.801027in}{0.695229in}}%
\pgfpathclose%
\pgfusepath{stroke,fill}%
\end{pgfscope}%
\begin{pgfscope}%
\pgfpathrectangle{\pgfqpoint{0.750000in}{0.500000in}}{\pgfqpoint{4.650000in}{3.020000in}}%
\pgfusepath{clip}%
\pgfsetbuttcap%
\pgfsetroundjoin%
\definecolor{currentfill}{rgb}{0.121569,0.466667,0.705882}%
\pgfsetfillcolor{currentfill}%
\pgfsetlinewidth{1.003750pt}%
\definecolor{currentstroke}{rgb}{0.121569,0.466667,0.705882}%
\pgfsetstrokecolor{currentstroke}%
\pgfsetdash{}{0pt}%
\pgfpathmoveto{\pgfqpoint{3.480355in}{0.963013in}}%
\pgfpathcurveto{\pgfqpoint{3.482959in}{0.963013in}}{\pgfqpoint{3.485458in}{0.964047in}}{\pgfqpoint{3.487299in}{0.965889in}}%
\pgfpathcurveto{\pgfqpoint{3.489141in}{0.967731in}}{\pgfqpoint{3.490176in}{0.970229in}}{\pgfqpoint{3.490176in}{0.972834in}}%
\pgfpathcurveto{\pgfqpoint{3.490176in}{0.975438in}}{\pgfqpoint{3.489141in}{0.977936in}}{\pgfqpoint{3.487299in}{0.979778in}}%
\pgfpathcurveto{\pgfqpoint{3.485458in}{0.981620in}}{\pgfqpoint{3.482959in}{0.982655in}}{\pgfqpoint{3.480355in}{0.982655in}}%
\pgfpathcurveto{\pgfqpoint{3.477750in}{0.982655in}}{\pgfqpoint{3.475252in}{0.981620in}}{\pgfqpoint{3.473410in}{0.979778in}}%
\pgfpathcurveto{\pgfqpoint{3.471569in}{0.977936in}}{\pgfqpoint{3.470534in}{0.975438in}}{\pgfqpoint{3.470534in}{0.972834in}}%
\pgfpathcurveto{\pgfqpoint{3.470534in}{0.970229in}}{\pgfqpoint{3.471569in}{0.967731in}}{\pgfqpoint{3.473410in}{0.965889in}}%
\pgfpathcurveto{\pgfqpoint{3.475252in}{0.964047in}}{\pgfqpoint{3.477750in}{0.963013in}}{\pgfqpoint{3.480355in}{0.963013in}}%
\pgfpathclose%
\pgfusepath{stroke,fill}%
\end{pgfscope}%
\begin{pgfscope}%
\pgfpathrectangle{\pgfqpoint{0.750000in}{0.500000in}}{\pgfqpoint{4.650000in}{3.020000in}}%
\pgfusepath{clip}%
\pgfsetbuttcap%
\pgfsetroundjoin%
\definecolor{currentfill}{rgb}{0.121569,0.466667,0.705882}%
\pgfsetfillcolor{currentfill}%
\pgfsetlinewidth{1.003750pt}%
\definecolor{currentstroke}{rgb}{0.121569,0.466667,0.705882}%
\pgfsetstrokecolor{currentstroke}%
\pgfsetdash{}{0pt}%
\pgfpathmoveto{\pgfqpoint{3.843727in}{0.944457in}}%
\pgfpathcurveto{\pgfqpoint{3.850106in}{0.944457in}}{\pgfqpoint{3.856226in}{0.946992in}}{\pgfqpoint{3.860737in}{0.951503in}}%
\pgfpathcurveto{\pgfqpoint{3.865248in}{0.956014in}}{\pgfqpoint{3.867783in}{0.962133in}}{\pgfqpoint{3.867783in}{0.968513in}}%
\pgfpathcurveto{\pgfqpoint{3.867783in}{0.974893in}}{\pgfqpoint{3.865248in}{0.981012in}}{\pgfqpoint{3.860737in}{0.985523in}}%
\pgfpathcurveto{\pgfqpoint{3.856226in}{0.990035in}}{\pgfqpoint{3.850106in}{0.992569in}}{\pgfqpoint{3.843727in}{0.992569in}}%
\pgfpathcurveto{\pgfqpoint{3.837347in}{0.992569in}}{\pgfqpoint{3.831228in}{0.990035in}}{\pgfqpoint{3.826716in}{0.985523in}}%
\pgfpathcurveto{\pgfqpoint{3.822205in}{0.981012in}}{\pgfqpoint{3.819670in}{0.974893in}}{\pgfqpoint{3.819670in}{0.968513in}}%
\pgfpathcurveto{\pgfqpoint{3.819670in}{0.962133in}}{\pgfqpoint{3.822205in}{0.956014in}}{\pgfqpoint{3.826716in}{0.951503in}}%
\pgfpathcurveto{\pgfqpoint{3.831228in}{0.946992in}}{\pgfqpoint{3.837347in}{0.944457in}}{\pgfqpoint{3.843727in}{0.944457in}}%
\pgfpathclose%
\pgfusepath{stroke,fill}%
\end{pgfscope}%
\begin{pgfscope}%
\pgfpathrectangle{\pgfqpoint{0.750000in}{0.500000in}}{\pgfqpoint{4.650000in}{3.020000in}}%
\pgfusepath{clip}%
\pgfsetbuttcap%
\pgfsetroundjoin%
\definecolor{currentfill}{rgb}{0.121569,0.466667,0.705882}%
\pgfsetfillcolor{currentfill}%
\pgfsetlinewidth{1.003750pt}%
\definecolor{currentstroke}{rgb}{0.121569,0.466667,0.705882}%
\pgfsetstrokecolor{currentstroke}%
\pgfsetdash{}{0pt}%
\pgfpathmoveto{\pgfqpoint{1.618618in}{0.648574in}}%
\pgfpathcurveto{\pgfqpoint{1.621222in}{0.648574in}}{\pgfqpoint{1.623720in}{0.649609in}}{\pgfqpoint{1.625562in}{0.651451in}}%
\pgfpathcurveto{\pgfqpoint{1.627404in}{0.653293in}}{\pgfqpoint{1.628439in}{0.655791in}}{\pgfqpoint{1.628439in}{0.658395in}}%
\pgfpathcurveto{\pgfqpoint{1.628439in}{0.661000in}}{\pgfqpoint{1.627404in}{0.663498in}}{\pgfqpoint{1.625562in}{0.665340in}}%
\pgfpathcurveto{\pgfqpoint{1.623720in}{0.667181in}}{\pgfqpoint{1.621222in}{0.668216in}}{\pgfqpoint{1.618618in}{0.668216in}}%
\pgfpathcurveto{\pgfqpoint{1.616013in}{0.668216in}}{\pgfqpoint{1.613515in}{0.667181in}}{\pgfqpoint{1.611673in}{0.665340in}}%
\pgfpathcurveto{\pgfqpoint{1.609832in}{0.663498in}}{\pgfqpoint{1.608797in}{0.661000in}}{\pgfqpoint{1.608797in}{0.658395in}}%
\pgfpathcurveto{\pgfqpoint{1.608797in}{0.655791in}}{\pgfqpoint{1.609832in}{0.653293in}}{\pgfqpoint{1.611673in}{0.651451in}}%
\pgfpathcurveto{\pgfqpoint{1.613515in}{0.649609in}}{\pgfqpoint{1.616013in}{0.648574in}}{\pgfqpoint{1.618618in}{0.648574in}}%
\pgfpathclose%
\pgfusepath{stroke,fill}%
\end{pgfscope}%
\begin{pgfscope}%
\pgfpathrectangle{\pgfqpoint{0.750000in}{0.500000in}}{\pgfqpoint{4.650000in}{3.020000in}}%
\pgfusepath{clip}%
\pgfsetbuttcap%
\pgfsetroundjoin%
\definecolor{currentfill}{rgb}{0.121569,0.466667,0.705882}%
\pgfsetfillcolor{currentfill}%
\pgfsetlinewidth{1.003750pt}%
\definecolor{currentstroke}{rgb}{0.121569,0.466667,0.705882}%
\pgfsetstrokecolor{currentstroke}%
\pgfsetdash{}{0pt}%
\pgfpathmoveto{\pgfqpoint{2.430775in}{0.740526in}}%
\pgfpathcurveto{\pgfqpoint{2.435984in}{0.740526in}}{\pgfqpoint{2.440981in}{0.742595in}}{\pgfqpoint{2.444664in}{0.746279in}}%
\pgfpathcurveto{\pgfqpoint{2.448347in}{0.749962in}}{\pgfqpoint{2.450417in}{0.754959in}}{\pgfqpoint{2.450417in}{0.760168in}}%
\pgfpathcurveto{\pgfqpoint{2.450417in}{0.765377in}}{\pgfqpoint{2.448347in}{0.770373in}}{\pgfqpoint{2.444664in}{0.774057in}}%
\pgfpathcurveto{\pgfqpoint{2.440981in}{0.777740in}}{\pgfqpoint{2.435984in}{0.779810in}}{\pgfqpoint{2.430775in}{0.779810in}}%
\pgfpathcurveto{\pgfqpoint{2.425566in}{0.779810in}}{\pgfqpoint{2.420570in}{0.777740in}}{\pgfqpoint{2.416886in}{0.774057in}}%
\pgfpathcurveto{\pgfqpoint{2.413203in}{0.770373in}}{\pgfqpoint{2.411133in}{0.765377in}}{\pgfqpoint{2.411133in}{0.760168in}}%
\pgfpathcurveto{\pgfqpoint{2.411133in}{0.754959in}}{\pgfqpoint{2.413203in}{0.749962in}}{\pgfqpoint{2.416886in}{0.746279in}}%
\pgfpathcurveto{\pgfqpoint{2.420570in}{0.742595in}}{\pgfqpoint{2.425566in}{0.740526in}}{\pgfqpoint{2.430775in}{0.740526in}}%
\pgfpathclose%
\pgfusepath{stroke,fill}%
\end{pgfscope}%
\begin{pgfscope}%
\pgfpathrectangle{\pgfqpoint{0.750000in}{0.500000in}}{\pgfqpoint{4.650000in}{3.020000in}}%
\pgfusepath{clip}%
\pgfsetbuttcap%
\pgfsetroundjoin%
\definecolor{currentfill}{rgb}{0.121569,0.466667,0.705882}%
\pgfsetfillcolor{currentfill}%
\pgfsetlinewidth{1.003750pt}%
\definecolor{currentstroke}{rgb}{0.121569,0.466667,0.705882}%
\pgfsetstrokecolor{currentstroke}%
\pgfsetdash{}{0pt}%
\pgfpathmoveto{\pgfqpoint{2.023973in}{0.661056in}}%
\pgfpathcurveto{\pgfqpoint{2.026577in}{0.661056in}}{\pgfqpoint{2.029075in}{0.662091in}}{\pgfqpoint{2.030917in}{0.663932in}}%
\pgfpathcurveto{\pgfqpoint{2.032759in}{0.665774in}}{\pgfqpoint{2.033794in}{0.668272in}}{\pgfqpoint{2.033794in}{0.670877in}}%
\pgfpathcurveto{\pgfqpoint{2.033794in}{0.673481in}}{\pgfqpoint{2.032759in}{0.675980in}}{\pgfqpoint{2.030917in}{0.677821in}}%
\pgfpathcurveto{\pgfqpoint{2.029075in}{0.679663in}}{\pgfqpoint{2.026577in}{0.680698in}}{\pgfqpoint{2.023973in}{0.680698in}}%
\pgfpathcurveto{\pgfqpoint{2.021368in}{0.680698in}}{\pgfqpoint{2.018870in}{0.679663in}}{\pgfqpoint{2.017028in}{0.677821in}}%
\pgfpathcurveto{\pgfqpoint{2.015186in}{0.675980in}}{\pgfqpoint{2.014152in}{0.673481in}}{\pgfqpoint{2.014152in}{0.670877in}}%
\pgfpathcurveto{\pgfqpoint{2.014152in}{0.668272in}}{\pgfqpoint{2.015186in}{0.665774in}}{\pgfqpoint{2.017028in}{0.663932in}}%
\pgfpathcurveto{\pgfqpoint{2.018870in}{0.662091in}}{\pgfqpoint{2.021368in}{0.661056in}}{\pgfqpoint{2.023973in}{0.661056in}}%
\pgfpathclose%
\pgfusepath{stroke,fill}%
\end{pgfscope}%
\begin{pgfscope}%
\pgfpathrectangle{\pgfqpoint{0.750000in}{0.500000in}}{\pgfqpoint{4.650000in}{3.020000in}}%
\pgfusepath{clip}%
\pgfsetbuttcap%
\pgfsetroundjoin%
\definecolor{currentfill}{rgb}{0.121569,0.466667,0.705882}%
\pgfsetfillcolor{currentfill}%
\pgfsetlinewidth{1.003750pt}%
\definecolor{currentstroke}{rgb}{0.121569,0.466667,0.705882}%
\pgfsetstrokecolor{currentstroke}%
\pgfsetdash{}{0pt}%
\pgfpathmoveto{\pgfqpoint{1.110476in}{0.628892in}}%
\pgfpathcurveto{\pgfqpoint{1.113081in}{0.628892in}}{\pgfqpoint{1.115579in}{0.629927in}}{\pgfqpoint{1.117421in}{0.631768in}}%
\pgfpathcurveto{\pgfqpoint{1.119262in}{0.633610in}}{\pgfqpoint{1.120297in}{0.636108in}}{\pgfqpoint{1.120297in}{0.638713in}}%
\pgfpathcurveto{\pgfqpoint{1.120297in}{0.641317in}}{\pgfqpoint{1.119262in}{0.643816in}}{\pgfqpoint{1.117421in}{0.645657in}}%
\pgfpathcurveto{\pgfqpoint{1.115579in}{0.647499in}}{\pgfqpoint{1.113081in}{0.648534in}}{\pgfqpoint{1.110476in}{0.648534in}}%
\pgfpathcurveto{\pgfqpoint{1.107872in}{0.648534in}}{\pgfqpoint{1.105374in}{0.647499in}}{\pgfqpoint{1.103532in}{0.645657in}}%
\pgfpathcurveto{\pgfqpoint{1.101690in}{0.643816in}}{\pgfqpoint{1.100655in}{0.641317in}}{\pgfqpoint{1.100655in}{0.638713in}}%
\pgfpathcurveto{\pgfqpoint{1.100655in}{0.636108in}}{\pgfqpoint{1.101690in}{0.633610in}}{\pgfqpoint{1.103532in}{0.631768in}}%
\pgfpathcurveto{\pgfqpoint{1.105374in}{0.629927in}}{\pgfqpoint{1.107872in}{0.628892in}}{\pgfqpoint{1.110476in}{0.628892in}}%
\pgfpathclose%
\pgfusepath{stroke,fill}%
\end{pgfscope}%
\begin{pgfscope}%
\pgfpathrectangle{\pgfqpoint{0.750000in}{0.500000in}}{\pgfqpoint{4.650000in}{3.020000in}}%
\pgfusepath{clip}%
\pgfsetbuttcap%
\pgfsetroundjoin%
\definecolor{currentfill}{rgb}{0.121569,0.466667,0.705882}%
\pgfsetfillcolor{currentfill}%
\pgfsetlinewidth{1.003750pt}%
\definecolor{currentstroke}{rgb}{0.121569,0.466667,0.705882}%
\pgfsetstrokecolor{currentstroke}%
\pgfsetdash{}{0pt}%
\pgfpathmoveto{\pgfqpoint{1.692450in}{0.676198in}}%
\pgfpathcurveto{\pgfqpoint{1.697659in}{0.676198in}}{\pgfqpoint{1.702656in}{0.678268in}}{\pgfqpoint{1.706339in}{0.681951in}}%
\pgfpathcurveto{\pgfqpoint{1.710022in}{0.685634in}}{\pgfqpoint{1.712092in}{0.690631in}}{\pgfqpoint{1.712092in}{0.695840in}}%
\pgfpathcurveto{\pgfqpoint{1.712092in}{0.701049in}}{\pgfqpoint{1.710022in}{0.706045in}}{\pgfqpoint{1.706339in}{0.709729in}}%
\pgfpathcurveto{\pgfqpoint{1.702656in}{0.713412in}}{\pgfqpoint{1.697659in}{0.715482in}}{\pgfqpoint{1.692450in}{0.715482in}}%
\pgfpathcurveto{\pgfqpoint{1.687241in}{0.715482in}}{\pgfqpoint{1.682245in}{0.713412in}}{\pgfqpoint{1.678561in}{0.709729in}}%
\pgfpathcurveto{\pgfqpoint{1.674878in}{0.706045in}}{\pgfqpoint{1.672808in}{0.701049in}}{\pgfqpoint{1.672808in}{0.695840in}}%
\pgfpathcurveto{\pgfqpoint{1.672808in}{0.690631in}}{\pgfqpoint{1.674878in}{0.685634in}}{\pgfqpoint{1.678561in}{0.681951in}}%
\pgfpathcurveto{\pgfqpoint{1.682245in}{0.678268in}}{\pgfqpoint{1.687241in}{0.676198in}}{\pgfqpoint{1.692450in}{0.676198in}}%
\pgfpathclose%
\pgfusepath{stroke,fill}%
\end{pgfscope}%
\begin{pgfscope}%
\pgfpathrectangle{\pgfqpoint{0.750000in}{0.500000in}}{\pgfqpoint{4.650000in}{3.020000in}}%
\pgfusepath{clip}%
\pgfsetbuttcap%
\pgfsetroundjoin%
\definecolor{currentfill}{rgb}{0.121569,0.466667,0.705882}%
\pgfsetfillcolor{currentfill}%
\pgfsetlinewidth{1.003750pt}%
\definecolor{currentstroke}{rgb}{0.121569,0.466667,0.705882}%
\pgfsetstrokecolor{currentstroke}%
\pgfsetdash{}{0pt}%
\pgfpathmoveto{\pgfqpoint{2.912858in}{0.722696in}}%
\pgfpathcurveto{\pgfqpoint{2.916048in}{0.722696in}}{\pgfqpoint{2.919108in}{0.723964in}}{\pgfqpoint{2.921363in}{0.726219in}}%
\pgfpathcurveto{\pgfqpoint{2.923619in}{0.728475in}}{\pgfqpoint{2.924886in}{0.731535in}}{\pgfqpoint{2.924886in}{0.734725in}}%
\pgfpathcurveto{\pgfqpoint{2.924886in}{0.737914in}}{\pgfqpoint{2.923619in}{0.740974in}}{\pgfqpoint{2.921363in}{0.743230in}}%
\pgfpathcurveto{\pgfqpoint{2.919108in}{0.745485in}}{\pgfqpoint{2.916048in}{0.746753in}}{\pgfqpoint{2.912858in}{0.746753in}}%
\pgfpathcurveto{\pgfqpoint{2.909668in}{0.746753in}}{\pgfqpoint{2.906608in}{0.745485in}}{\pgfqpoint{2.904353in}{0.743230in}}%
\pgfpathcurveto{\pgfqpoint{2.902097in}{0.740974in}}{\pgfqpoint{2.900830in}{0.737914in}}{\pgfqpoint{2.900830in}{0.734725in}}%
\pgfpathcurveto{\pgfqpoint{2.900830in}{0.731535in}}{\pgfqpoint{2.902097in}{0.728475in}}{\pgfqpoint{2.904353in}{0.726219in}}%
\pgfpathcurveto{\pgfqpoint{2.906608in}{0.723964in}}{\pgfqpoint{2.909668in}{0.722696in}}{\pgfqpoint{2.912858in}{0.722696in}}%
\pgfpathclose%
\pgfusepath{stroke,fill}%
\end{pgfscope}%
\begin{pgfscope}%
\pgfpathrectangle{\pgfqpoint{0.750000in}{0.500000in}}{\pgfqpoint{4.650000in}{3.020000in}}%
\pgfusepath{clip}%
\pgfsetbuttcap%
\pgfsetroundjoin%
\definecolor{currentfill}{rgb}{0.121569,0.466667,0.705882}%
\pgfsetfillcolor{currentfill}%
\pgfsetlinewidth{1.003750pt}%
\definecolor{currentstroke}{rgb}{0.121569,0.466667,0.705882}%
\pgfsetstrokecolor{currentstroke}%
\pgfsetdash{}{0pt}%
\pgfpathmoveto{\pgfqpoint{2.080433in}{0.646654in}}%
\pgfpathcurveto{\pgfqpoint{2.083037in}{0.646654in}}{\pgfqpoint{2.085536in}{0.647689in}}{\pgfqpoint{2.087377in}{0.649531in}}%
\pgfpathcurveto{\pgfqpoint{2.089219in}{0.651372in}}{\pgfqpoint{2.090254in}{0.653871in}}{\pgfqpoint{2.090254in}{0.656475in}}%
\pgfpathcurveto{\pgfqpoint{2.090254in}{0.659080in}}{\pgfqpoint{2.089219in}{0.661578in}}{\pgfqpoint{2.087377in}{0.663420in}}%
\pgfpathcurveto{\pgfqpoint{2.085536in}{0.665261in}}{\pgfqpoint{2.083037in}{0.666296in}}{\pgfqpoint{2.080433in}{0.666296in}}%
\pgfpathcurveto{\pgfqpoint{2.077828in}{0.666296in}}{\pgfqpoint{2.075330in}{0.665261in}}{\pgfqpoint{2.073488in}{0.663420in}}%
\pgfpathcurveto{\pgfqpoint{2.071647in}{0.661578in}}{\pgfqpoint{2.070612in}{0.659080in}}{\pgfqpoint{2.070612in}{0.656475in}}%
\pgfpathcurveto{\pgfqpoint{2.070612in}{0.653871in}}{\pgfqpoint{2.071647in}{0.651372in}}{\pgfqpoint{2.073488in}{0.649531in}}%
\pgfpathcurveto{\pgfqpoint{2.075330in}{0.647689in}}{\pgfqpoint{2.077828in}{0.646654in}}{\pgfqpoint{2.080433in}{0.646654in}}%
\pgfpathclose%
\pgfusepath{stroke,fill}%
\end{pgfscope}%
\begin{pgfscope}%
\pgfpathrectangle{\pgfqpoint{0.750000in}{0.500000in}}{\pgfqpoint{4.650000in}{3.020000in}}%
\pgfusepath{clip}%
\pgfsetbuttcap%
\pgfsetroundjoin%
\definecolor{currentfill}{rgb}{0.121569,0.466667,0.705882}%
\pgfsetfillcolor{currentfill}%
\pgfsetlinewidth{1.003750pt}%
\definecolor{currentstroke}{rgb}{0.121569,0.466667,0.705882}%
\pgfsetstrokecolor{currentstroke}%
\pgfsetdash{}{0pt}%
\pgfpathmoveto{\pgfqpoint{1.434760in}{0.634653in}}%
\pgfpathcurveto{\pgfqpoint{1.437365in}{0.634653in}}{\pgfqpoint{1.439863in}{0.635687in}}{\pgfqpoint{1.441705in}{0.637529in}}%
\pgfpathcurveto{\pgfqpoint{1.443546in}{0.639371in}}{\pgfqpoint{1.444581in}{0.641869in}}{\pgfqpoint{1.444581in}{0.644474in}}%
\pgfpathcurveto{\pgfqpoint{1.444581in}{0.647078in}}{\pgfqpoint{1.443546in}{0.649576in}}{\pgfqpoint{1.441705in}{0.651418in}}%
\pgfpathcurveto{\pgfqpoint{1.439863in}{0.653260in}}{\pgfqpoint{1.437365in}{0.654295in}}{\pgfqpoint{1.434760in}{0.654295in}}%
\pgfpathcurveto{\pgfqpoint{1.432156in}{0.654295in}}{\pgfqpoint{1.429658in}{0.653260in}}{\pgfqpoint{1.427816in}{0.651418in}}%
\pgfpathcurveto{\pgfqpoint{1.425974in}{0.649576in}}{\pgfqpoint{1.424939in}{0.647078in}}{\pgfqpoint{1.424939in}{0.644474in}}%
\pgfpathcurveto{\pgfqpoint{1.424939in}{0.641869in}}{\pgfqpoint{1.425974in}{0.639371in}}{\pgfqpoint{1.427816in}{0.637529in}}%
\pgfpathcurveto{\pgfqpoint{1.429658in}{0.635687in}}{\pgfqpoint{1.432156in}{0.634653in}}{\pgfqpoint{1.434760in}{0.634653in}}%
\pgfpathclose%
\pgfusepath{stroke,fill}%
\end{pgfscope}%
\begin{pgfscope}%
\pgfpathrectangle{\pgfqpoint{0.750000in}{0.500000in}}{\pgfqpoint{4.650000in}{3.020000in}}%
\pgfusepath{clip}%
\pgfsetbuttcap%
\pgfsetroundjoin%
\definecolor{currentfill}{rgb}{0.121569,0.466667,0.705882}%
\pgfsetfillcolor{currentfill}%
\pgfsetlinewidth{1.003750pt}%
\definecolor{currentstroke}{rgb}{0.121569,0.466667,0.705882}%
\pgfsetstrokecolor{currentstroke}%
\pgfsetdash{}{0pt}%
\pgfpathmoveto{\pgfqpoint{1.359480in}{0.777710in}}%
\pgfpathcurveto{\pgfqpoint{1.362085in}{0.777710in}}{\pgfqpoint{1.364583in}{0.778745in}}{\pgfqpoint{1.366425in}{0.780587in}}%
\pgfpathcurveto{\pgfqpoint{1.368266in}{0.782428in}}{\pgfqpoint{1.369301in}{0.784926in}}{\pgfqpoint{1.369301in}{0.787531in}}%
\pgfpathcurveto{\pgfqpoint{1.369301in}{0.790136in}}{\pgfqpoint{1.368266in}{0.792634in}}{\pgfqpoint{1.366425in}{0.794475in}}%
\pgfpathcurveto{\pgfqpoint{1.364583in}{0.796317in}}{\pgfqpoint{1.362085in}{0.797352in}}{\pgfqpoint{1.359480in}{0.797352in}}%
\pgfpathcurveto{\pgfqpoint{1.356876in}{0.797352in}}{\pgfqpoint{1.354377in}{0.796317in}}{\pgfqpoint{1.352536in}{0.794475in}}%
\pgfpathcurveto{\pgfqpoint{1.350694in}{0.792634in}}{\pgfqpoint{1.349659in}{0.790136in}}{\pgfqpoint{1.349659in}{0.787531in}}%
\pgfpathcurveto{\pgfqpoint{1.349659in}{0.784926in}}{\pgfqpoint{1.350694in}{0.782428in}}{\pgfqpoint{1.352536in}{0.780587in}}%
\pgfpathcurveto{\pgfqpoint{1.354377in}{0.778745in}}{\pgfqpoint{1.356876in}{0.777710in}}{\pgfqpoint{1.359480in}{0.777710in}}%
\pgfpathclose%
\pgfusepath{stroke,fill}%
\end{pgfscope}%
\begin{pgfscope}%
\pgfpathrectangle{\pgfqpoint{0.750000in}{0.500000in}}{\pgfqpoint{4.650000in}{3.020000in}}%
\pgfusepath{clip}%
\pgfsetbuttcap%
\pgfsetroundjoin%
\definecolor{currentfill}{rgb}{0.121569,0.466667,0.705882}%
\pgfsetfillcolor{currentfill}%
\pgfsetlinewidth{1.003750pt}%
\definecolor{currentstroke}{rgb}{0.121569,0.466667,0.705882}%
\pgfsetstrokecolor{currentstroke}%
\pgfsetdash{}{0pt}%
\pgfpathmoveto{\pgfqpoint{2.232441in}{0.684590in}}%
\pgfpathcurveto{\pgfqpoint{2.236952in}{0.684590in}}{\pgfqpoint{2.241279in}{0.686383in}}{\pgfqpoint{2.244469in}{0.689572in}}%
\pgfpathcurveto{\pgfqpoint{2.247659in}{0.692762in}}{\pgfqpoint{2.249451in}{0.697089in}}{\pgfqpoint{2.249451in}{0.701601in}}%
\pgfpathcurveto{\pgfqpoint{2.249451in}{0.706112in}}{\pgfqpoint{2.247659in}{0.710439in}}{\pgfqpoint{2.244469in}{0.713629in}}%
\pgfpathcurveto{\pgfqpoint{2.241279in}{0.716819in}}{\pgfqpoint{2.236952in}{0.718611in}}{\pgfqpoint{2.232441in}{0.718611in}}%
\pgfpathcurveto{\pgfqpoint{2.227930in}{0.718611in}}{\pgfqpoint{2.223603in}{0.716819in}}{\pgfqpoint{2.220413in}{0.713629in}}%
\pgfpathcurveto{\pgfqpoint{2.217223in}{0.710439in}}{\pgfqpoint{2.215431in}{0.706112in}}{\pgfqpoint{2.215431in}{0.701601in}}%
\pgfpathcurveto{\pgfqpoint{2.215431in}{0.697089in}}{\pgfqpoint{2.217223in}{0.692762in}}{\pgfqpoint{2.220413in}{0.689572in}}%
\pgfpathcurveto{\pgfqpoint{2.223603in}{0.686383in}}{\pgfqpoint{2.227930in}{0.684590in}}{\pgfqpoint{2.232441in}{0.684590in}}%
\pgfpathclose%
\pgfusepath{stroke,fill}%
\end{pgfscope}%
\begin{pgfscope}%
\pgfpathrectangle{\pgfqpoint{0.750000in}{0.500000in}}{\pgfqpoint{4.650000in}{3.020000in}}%
\pgfusepath{clip}%
\pgfsetbuttcap%
\pgfsetroundjoin%
\definecolor{currentfill}{rgb}{0.121569,0.466667,0.705882}%
\pgfsetfillcolor{currentfill}%
\pgfsetlinewidth{1.003750pt}%
\definecolor{currentstroke}{rgb}{0.121569,0.466667,0.705882}%
\pgfsetstrokecolor{currentstroke}%
\pgfsetdash{}{0pt}%
\pgfpathmoveto{\pgfqpoint{1.226292in}{0.632252in}}%
\pgfpathcurveto{\pgfqpoint{1.228897in}{0.632252in}}{\pgfqpoint{1.231395in}{0.633287in}}{\pgfqpoint{1.233236in}{0.635129in}}%
\pgfpathcurveto{\pgfqpoint{1.235078in}{0.636971in}}{\pgfqpoint{1.236113in}{0.639469in}}{\pgfqpoint{1.236113in}{0.642073in}}%
\pgfpathcurveto{\pgfqpoint{1.236113in}{0.644678in}}{\pgfqpoint{1.235078in}{0.647176in}}{\pgfqpoint{1.233236in}{0.649018in}}%
\pgfpathcurveto{\pgfqpoint{1.231395in}{0.650859in}}{\pgfqpoint{1.228897in}{0.651894in}}{\pgfqpoint{1.226292in}{0.651894in}}%
\pgfpathcurveto{\pgfqpoint{1.223687in}{0.651894in}}{\pgfqpoint{1.221189in}{0.650859in}}{\pgfqpoint{1.219348in}{0.649018in}}%
\pgfpathcurveto{\pgfqpoint{1.217506in}{0.647176in}}{\pgfqpoint{1.216471in}{0.644678in}}{\pgfqpoint{1.216471in}{0.642073in}}%
\pgfpathcurveto{\pgfqpoint{1.216471in}{0.639469in}}{\pgfqpoint{1.217506in}{0.636971in}}{\pgfqpoint{1.219348in}{0.635129in}}%
\pgfpathcurveto{\pgfqpoint{1.221189in}{0.633287in}}{\pgfqpoint{1.223687in}{0.632252in}}{\pgfqpoint{1.226292in}{0.632252in}}%
\pgfpathclose%
\pgfusepath{stroke,fill}%
\end{pgfscope}%
\begin{pgfscope}%
\pgfpathrectangle{\pgfqpoint{0.750000in}{0.500000in}}{\pgfqpoint{4.650000in}{3.020000in}}%
\pgfusepath{clip}%
\pgfsetbuttcap%
\pgfsetroundjoin%
\definecolor{currentfill}{rgb}{0.121569,0.466667,0.705882}%
\pgfsetfillcolor{currentfill}%
\pgfsetlinewidth{1.003750pt}%
\definecolor{currentstroke}{rgb}{0.121569,0.466667,0.705882}%
\pgfsetstrokecolor{currentstroke}%
\pgfsetdash{}{0pt}%
\pgfpathmoveto{\pgfqpoint{1.064150in}{0.627452in}}%
\pgfpathcurveto{\pgfqpoint{1.066755in}{0.627452in}}{\pgfqpoint{1.069253in}{0.628487in}}{\pgfqpoint{1.071095in}{0.630328in}}%
\pgfpathcurveto{\pgfqpoint{1.072936in}{0.632170in}}{\pgfqpoint{1.073971in}{0.634668in}}{\pgfqpoint{1.073971in}{0.637273in}}%
\pgfpathcurveto{\pgfqpoint{1.073971in}{0.639877in}}{\pgfqpoint{1.072936in}{0.642375in}}{\pgfqpoint{1.071095in}{0.644217in}}%
\pgfpathcurveto{\pgfqpoint{1.069253in}{0.646059in}}{\pgfqpoint{1.066755in}{0.647094in}}{\pgfqpoint{1.064150in}{0.647094in}}%
\pgfpathcurveto{\pgfqpoint{1.061546in}{0.647094in}}{\pgfqpoint{1.059047in}{0.646059in}}{\pgfqpoint{1.057206in}{0.644217in}}%
\pgfpathcurveto{\pgfqpoint{1.055364in}{0.642375in}}{\pgfqpoint{1.054329in}{0.639877in}}{\pgfqpoint{1.054329in}{0.637273in}}%
\pgfpathcurveto{\pgfqpoint{1.054329in}{0.634668in}}{\pgfqpoint{1.055364in}{0.632170in}}{\pgfqpoint{1.057206in}{0.630328in}}%
\pgfpathcurveto{\pgfqpoint{1.059047in}{0.628487in}}{\pgfqpoint{1.061546in}{0.627452in}}{\pgfqpoint{1.064150in}{0.627452in}}%
\pgfpathclose%
\pgfusepath{stroke,fill}%
\end{pgfscope}%
\begin{pgfscope}%
\pgfpathrectangle{\pgfqpoint{0.750000in}{0.500000in}}{\pgfqpoint{4.650000in}{3.020000in}}%
\pgfusepath{clip}%
\pgfsetbuttcap%
\pgfsetroundjoin%
\definecolor{currentfill}{rgb}{0.121569,0.466667,0.705882}%
\pgfsetfillcolor{currentfill}%
\pgfsetlinewidth{1.003750pt}%
\definecolor{currentstroke}{rgb}{0.121569,0.466667,0.705882}%
\pgfsetstrokecolor{currentstroke}%
\pgfsetdash{}{0pt}%
\pgfpathmoveto{\pgfqpoint{1.181413in}{0.627932in}}%
\pgfpathcurveto{\pgfqpoint{1.184018in}{0.627932in}}{\pgfqpoint{1.186516in}{0.628967in}}{\pgfqpoint{1.188358in}{0.630808in}}%
\pgfpathcurveto{\pgfqpoint{1.190200in}{0.632650in}}{\pgfqpoint{1.191234in}{0.635148in}}{\pgfqpoint{1.191234in}{0.637753in}}%
\pgfpathcurveto{\pgfqpoint{1.191234in}{0.640357in}}{\pgfqpoint{1.190200in}{0.642856in}}{\pgfqpoint{1.188358in}{0.644697in}}%
\pgfpathcurveto{\pgfqpoint{1.186516in}{0.646539in}}{\pgfqpoint{1.184018in}{0.647574in}}{\pgfqpoint{1.181413in}{0.647574in}}%
\pgfpathcurveto{\pgfqpoint{1.178809in}{0.647574in}}{\pgfqpoint{1.176311in}{0.646539in}}{\pgfqpoint{1.174469in}{0.644697in}}%
\pgfpathcurveto{\pgfqpoint{1.172627in}{0.642856in}}{\pgfqpoint{1.171593in}{0.640357in}}{\pgfqpoint{1.171593in}{0.637753in}}%
\pgfpathcurveto{\pgfqpoint{1.171593in}{0.635148in}}{\pgfqpoint{1.172627in}{0.632650in}}{\pgfqpoint{1.174469in}{0.630808in}}%
\pgfpathcurveto{\pgfqpoint{1.176311in}{0.628967in}}{\pgfqpoint{1.178809in}{0.627932in}}{\pgfqpoint{1.181413in}{0.627932in}}%
\pgfpathclose%
\pgfusepath{stroke,fill}%
\end{pgfscope}%
\begin{pgfscope}%
\pgfpathrectangle{\pgfqpoint{0.750000in}{0.500000in}}{\pgfqpoint{4.650000in}{3.020000in}}%
\pgfusepath{clip}%
\pgfsetbuttcap%
\pgfsetroundjoin%
\definecolor{currentfill}{rgb}{0.121569,0.466667,0.705882}%
\pgfsetfillcolor{currentfill}%
\pgfsetlinewidth{1.003750pt}%
\definecolor{currentstroke}{rgb}{0.121569,0.466667,0.705882}%
\pgfsetstrokecolor{currentstroke}%
\pgfsetdash{}{0pt}%
\pgfpathmoveto{\pgfqpoint{2.643587in}{0.686072in}}%
\pgfpathcurveto{\pgfqpoint{2.647705in}{0.686072in}}{\pgfqpoint{2.651655in}{0.687708in}}{\pgfqpoint{2.654567in}{0.690620in}}%
\pgfpathcurveto{\pgfqpoint{2.657479in}{0.693532in}}{\pgfqpoint{2.659115in}{0.697482in}}{\pgfqpoint{2.659115in}{0.701601in}}%
\pgfpathcurveto{\pgfqpoint{2.659115in}{0.705719in}}{\pgfqpoint{2.657479in}{0.709669in}}{\pgfqpoint{2.654567in}{0.712581in}}%
\pgfpathcurveto{\pgfqpoint{2.651655in}{0.715493in}}{\pgfqpoint{2.647705in}{0.717129in}}{\pgfqpoint{2.643587in}{0.717129in}}%
\pgfpathcurveto{\pgfqpoint{2.639468in}{0.717129in}}{\pgfqpoint{2.635518in}{0.715493in}}{\pgfqpoint{2.632606in}{0.712581in}}%
\pgfpathcurveto{\pgfqpoint{2.629694in}{0.709669in}}{\pgfqpoint{2.628058in}{0.705719in}}{\pgfqpoint{2.628058in}{0.701601in}}%
\pgfpathcurveto{\pgfqpoint{2.628058in}{0.697482in}}{\pgfqpoint{2.629694in}{0.693532in}}{\pgfqpoint{2.632606in}{0.690620in}}%
\pgfpathcurveto{\pgfqpoint{2.635518in}{0.687708in}}{\pgfqpoint{2.639468in}{0.686072in}}{\pgfqpoint{2.643587in}{0.686072in}}%
\pgfpathclose%
\pgfusepath{stroke,fill}%
\end{pgfscope}%
\begin{pgfscope}%
\pgfpathrectangle{\pgfqpoint{0.750000in}{0.500000in}}{\pgfqpoint{4.650000in}{3.020000in}}%
\pgfusepath{clip}%
\pgfsetbuttcap%
\pgfsetroundjoin%
\definecolor{currentfill}{rgb}{0.121569,0.466667,0.705882}%
\pgfsetfillcolor{currentfill}%
\pgfsetlinewidth{1.003750pt}%
\definecolor{currentstroke}{rgb}{0.121569,0.466667,0.705882}%
\pgfsetstrokecolor{currentstroke}%
\pgfsetdash{}{0pt}%
\pgfpathmoveto{\pgfqpoint{3.519443in}{0.780199in}}%
\pgfpathcurveto{\pgfqpoint{3.524315in}{0.780199in}}{\pgfqpoint{3.528989in}{0.782135in}}{\pgfqpoint{3.532435in}{0.785581in}}%
\pgfpathcurveto{\pgfqpoint{3.535880in}{0.789026in}}{\pgfqpoint{3.537816in}{0.793700in}}{\pgfqpoint{3.537816in}{0.798572in}}%
\pgfpathcurveto{\pgfqpoint{3.537816in}{0.803445in}}{\pgfqpoint{3.535880in}{0.808119in}}{\pgfqpoint{3.532435in}{0.811564in}}%
\pgfpathcurveto{\pgfqpoint{3.528989in}{0.815010in}}{\pgfqpoint{3.524315in}{0.816946in}}{\pgfqpoint{3.519443in}{0.816946in}}%
\pgfpathcurveto{\pgfqpoint{3.514570in}{0.816946in}}{\pgfqpoint{3.509896in}{0.815010in}}{\pgfqpoint{3.506451in}{0.811564in}}%
\pgfpathcurveto{\pgfqpoint{3.503005in}{0.808119in}}{\pgfqpoint{3.501069in}{0.803445in}}{\pgfqpoint{3.501069in}{0.798572in}}%
\pgfpathcurveto{\pgfqpoint{3.501069in}{0.793700in}}{\pgfqpoint{3.503005in}{0.789026in}}{\pgfqpoint{3.506451in}{0.785581in}}%
\pgfpathcurveto{\pgfqpoint{3.509896in}{0.782135in}}{\pgfqpoint{3.514570in}{0.780199in}}{\pgfqpoint{3.519443in}{0.780199in}}%
\pgfpathclose%
\pgfusepath{stroke,fill}%
\end{pgfscope}%
\begin{pgfscope}%
\pgfpathrectangle{\pgfqpoint{0.750000in}{0.500000in}}{\pgfqpoint{4.650000in}{3.020000in}}%
\pgfusepath{clip}%
\pgfsetbuttcap%
\pgfsetroundjoin%
\definecolor{currentfill}{rgb}{0.121569,0.466667,0.705882}%
\pgfsetfillcolor{currentfill}%
\pgfsetlinewidth{1.003750pt}%
\definecolor{currentstroke}{rgb}{0.121569,0.466667,0.705882}%
\pgfsetstrokecolor{currentstroke}%
\pgfsetdash{}{0pt}%
\pgfpathmoveto{\pgfqpoint{4.427148in}{0.809405in}}%
\pgfpathcurveto{\pgfqpoint{4.431659in}{0.809405in}}{\pgfqpoint{4.435986in}{0.811198in}}{\pgfqpoint{4.439176in}{0.814388in}}%
\pgfpathcurveto{\pgfqpoint{4.442366in}{0.817578in}}{\pgfqpoint{4.444159in}{0.821905in}}{\pgfqpoint{4.444159in}{0.826416in}}%
\pgfpathcurveto{\pgfqpoint{4.444159in}{0.830927in}}{\pgfqpoint{4.442366in}{0.835254in}}{\pgfqpoint{4.439176in}{0.838444in}}%
\pgfpathcurveto{\pgfqpoint{4.435986in}{0.841634in}}{\pgfqpoint{4.431659in}{0.843426in}}{\pgfqpoint{4.427148in}{0.843426in}}%
\pgfpathcurveto{\pgfqpoint{4.422637in}{0.843426in}}{\pgfqpoint{4.418310in}{0.841634in}}{\pgfqpoint{4.415120in}{0.838444in}}%
\pgfpathcurveto{\pgfqpoint{4.411930in}{0.835254in}}{\pgfqpoint{4.410138in}{0.830927in}}{\pgfqpoint{4.410138in}{0.826416in}}%
\pgfpathcurveto{\pgfqpoint{4.410138in}{0.821905in}}{\pgfqpoint{4.411930in}{0.817578in}}{\pgfqpoint{4.415120in}{0.814388in}}%
\pgfpathcurveto{\pgfqpoint{4.418310in}{0.811198in}}{\pgfqpoint{4.422637in}{0.809405in}}{\pgfqpoint{4.427148in}{0.809405in}}%
\pgfpathclose%
\pgfusepath{stroke,fill}%
\end{pgfscope}%
\begin{pgfscope}%
\pgfpathrectangle{\pgfqpoint{0.750000in}{0.500000in}}{\pgfqpoint{4.650000in}{3.020000in}}%
\pgfusepath{clip}%
\pgfsetbuttcap%
\pgfsetroundjoin%
\definecolor{currentfill}{rgb}{0.121569,0.466667,0.705882}%
\pgfsetfillcolor{currentfill}%
\pgfsetlinewidth{1.003750pt}%
\definecolor{currentstroke}{rgb}{0.121569,0.466667,0.705882}%
\pgfsetstrokecolor{currentstroke}%
\pgfsetdash{}{0pt}%
\pgfpathmoveto{\pgfqpoint{1.624408in}{0.638266in}}%
\pgfpathcurveto{\pgfqpoint{1.628092in}{0.638266in}}{\pgfqpoint{1.631625in}{0.639729in}}{\pgfqpoint{1.634229in}{0.642334in}}%
\pgfpathcurveto{\pgfqpoint{1.636834in}{0.644938in}}{\pgfqpoint{1.638297in}{0.648471in}}{\pgfqpoint{1.638297in}{0.652155in}}%
\pgfpathcurveto{\pgfqpoint{1.638297in}{0.655838in}}{\pgfqpoint{1.636834in}{0.659371in}}{\pgfqpoint{1.634229in}{0.661975in}}%
\pgfpathcurveto{\pgfqpoint{1.631625in}{0.664580in}}{\pgfqpoint{1.628092in}{0.666043in}}{\pgfqpoint{1.624408in}{0.666043in}}%
\pgfpathcurveto{\pgfqpoint{1.620725in}{0.666043in}}{\pgfqpoint{1.617192in}{0.664580in}}{\pgfqpoint{1.614588in}{0.661975in}}%
\pgfpathcurveto{\pgfqpoint{1.611983in}{0.659371in}}{\pgfqpoint{1.610520in}{0.655838in}}{\pgfqpoint{1.610520in}{0.652155in}}%
\pgfpathcurveto{\pgfqpoint{1.610520in}{0.648471in}}{\pgfqpoint{1.611983in}{0.644938in}}{\pgfqpoint{1.614588in}{0.642334in}}%
\pgfpathcurveto{\pgfqpoint{1.617192in}{0.639729in}}{\pgfqpoint{1.620725in}{0.638266in}}{\pgfqpoint{1.624408in}{0.638266in}}%
\pgfpathclose%
\pgfusepath{stroke,fill}%
\end{pgfscope}%
\begin{pgfscope}%
\pgfpathrectangle{\pgfqpoint{0.750000in}{0.500000in}}{\pgfqpoint{4.650000in}{3.020000in}}%
\pgfusepath{clip}%
\pgfsetbuttcap%
\pgfsetroundjoin%
\definecolor{currentfill}{rgb}{0.121569,0.466667,0.705882}%
\pgfsetfillcolor{currentfill}%
\pgfsetlinewidth{1.003750pt}%
\definecolor{currentstroke}{rgb}{0.121569,0.466667,0.705882}%
\pgfsetstrokecolor{currentstroke}%
\pgfsetdash{}{0pt}%
\pgfpathmoveto{\pgfqpoint{2.763745in}{0.742859in}}%
\pgfpathcurveto{\pgfqpoint{2.766935in}{0.742859in}}{\pgfqpoint{2.769995in}{0.744126in}}{\pgfqpoint{2.772251in}{0.746382in}}%
\pgfpathcurveto{\pgfqpoint{2.774506in}{0.748637in}}{\pgfqpoint{2.775773in}{0.751697in}}{\pgfqpoint{2.775773in}{0.754887in}}%
\pgfpathcurveto{\pgfqpoint{2.775773in}{0.758077in}}{\pgfqpoint{2.774506in}{0.761137in}}{\pgfqpoint{2.772251in}{0.763392in}}%
\pgfpathcurveto{\pgfqpoint{2.769995in}{0.765648in}}{\pgfqpoint{2.766935in}{0.766915in}}{\pgfqpoint{2.763745in}{0.766915in}}%
\pgfpathcurveto{\pgfqpoint{2.760555in}{0.766915in}}{\pgfqpoint{2.757496in}{0.765648in}}{\pgfqpoint{2.755240in}{0.763392in}}%
\pgfpathcurveto{\pgfqpoint{2.752985in}{0.761137in}}{\pgfqpoint{2.751717in}{0.758077in}}{\pgfqpoint{2.751717in}{0.754887in}}%
\pgfpathcurveto{\pgfqpoint{2.751717in}{0.751697in}}{\pgfqpoint{2.752985in}{0.748637in}}{\pgfqpoint{2.755240in}{0.746382in}}%
\pgfpathcurveto{\pgfqpoint{2.757496in}{0.744126in}}{\pgfqpoint{2.760555in}{0.742859in}}{\pgfqpoint{2.763745in}{0.742859in}}%
\pgfpathclose%
\pgfusepath{stroke,fill}%
\end{pgfscope}%
\begin{pgfscope}%
\pgfpathrectangle{\pgfqpoint{0.750000in}{0.500000in}}{\pgfqpoint{4.650000in}{3.020000in}}%
\pgfusepath{clip}%
\pgfsetbuttcap%
\pgfsetroundjoin%
\definecolor{currentfill}{rgb}{0.121569,0.466667,0.705882}%
\pgfsetfillcolor{currentfill}%
\pgfsetlinewidth{1.003750pt}%
\definecolor{currentstroke}{rgb}{0.121569,0.466667,0.705882}%
\pgfsetstrokecolor{currentstroke}%
\pgfsetdash{}{0pt}%
\pgfpathmoveto{\pgfqpoint{4.851323in}{1.041796in}}%
\pgfpathcurveto{\pgfqpoint{4.855441in}{1.041796in}}{\pgfqpoint{4.859391in}{1.043432in}}{\pgfqpoint{4.862303in}{1.046344in}}%
\pgfpathcurveto{\pgfqpoint{4.865215in}{1.049256in}}{\pgfqpoint{4.866851in}{1.053206in}}{\pgfqpoint{4.866851in}{1.057324in}}%
\pgfpathcurveto{\pgfqpoint{4.866851in}{1.061442in}}{\pgfqpoint{4.865215in}{1.065392in}}{\pgfqpoint{4.862303in}{1.068304in}}%
\pgfpathcurveto{\pgfqpoint{4.859391in}{1.071216in}}{\pgfqpoint{4.855441in}{1.072852in}}{\pgfqpoint{4.851323in}{1.072852in}}%
\pgfpathcurveto{\pgfqpoint{4.847205in}{1.072852in}}{\pgfqpoint{4.843255in}{1.071216in}}{\pgfqpoint{4.840343in}{1.068304in}}%
\pgfpathcurveto{\pgfqpoint{4.837431in}{1.065392in}}{\pgfqpoint{4.835795in}{1.061442in}}{\pgfqpoint{4.835795in}{1.057324in}}%
\pgfpathcurveto{\pgfqpoint{4.835795in}{1.053206in}}{\pgfqpoint{4.837431in}{1.049256in}}{\pgfqpoint{4.840343in}{1.046344in}}%
\pgfpathcurveto{\pgfqpoint{4.843255in}{1.043432in}}{\pgfqpoint{4.847205in}{1.041796in}}{\pgfqpoint{4.851323in}{1.041796in}}%
\pgfpathclose%
\pgfusepath{stroke,fill}%
\end{pgfscope}%
\begin{pgfscope}%
\pgfpathrectangle{\pgfqpoint{0.750000in}{0.500000in}}{\pgfqpoint{4.650000in}{3.020000in}}%
\pgfusepath{clip}%
\pgfsetbuttcap%
\pgfsetroundjoin%
\definecolor{currentfill}{rgb}{0.121569,0.466667,0.705882}%
\pgfsetfillcolor{currentfill}%
\pgfsetlinewidth{1.003750pt}%
\definecolor{currentstroke}{rgb}{0.121569,0.466667,0.705882}%
\pgfsetstrokecolor{currentstroke}%
\pgfsetdash{}{0pt}%
\pgfpathmoveto{\pgfqpoint{0.990318in}{0.628412in}}%
\pgfpathcurveto{\pgfqpoint{0.992922in}{0.628412in}}{\pgfqpoint{0.995420in}{0.629447in}}{\pgfqpoint{0.997262in}{0.631288in}}%
\pgfpathcurveto{\pgfqpoint{0.999104in}{0.633130in}}{\pgfqpoint{1.000138in}{0.635628in}}{\pgfqpoint{1.000138in}{0.638233in}}%
\pgfpathcurveto{\pgfqpoint{1.000138in}{0.640837in}}{\pgfqpoint{0.999104in}{0.643336in}}{\pgfqpoint{0.997262in}{0.645177in}}%
\pgfpathcurveto{\pgfqpoint{0.995420in}{0.647019in}}{\pgfqpoint{0.992922in}{0.648054in}}{\pgfqpoint{0.990318in}{0.648054in}}%
\pgfpathcurveto{\pgfqpoint{0.987713in}{0.648054in}}{\pgfqpoint{0.985215in}{0.647019in}}{\pgfqpoint{0.983373in}{0.645177in}}%
\pgfpathcurveto{\pgfqpoint{0.981531in}{0.643336in}}{\pgfqpoint{0.980497in}{0.640837in}}{\pgfqpoint{0.980497in}{0.638233in}}%
\pgfpathcurveto{\pgfqpoint{0.980497in}{0.635628in}}{\pgfqpoint{0.981531in}{0.633130in}}{\pgfqpoint{0.983373in}{0.631288in}}%
\pgfpathcurveto{\pgfqpoint{0.985215in}{0.629447in}}{\pgfqpoint{0.987713in}{0.628412in}}{\pgfqpoint{0.990318in}{0.628412in}}%
\pgfpathclose%
\pgfusepath{stroke,fill}%
\end{pgfscope}%
\begin{pgfscope}%
\pgfpathrectangle{\pgfqpoint{0.750000in}{0.500000in}}{\pgfqpoint{4.650000in}{3.020000in}}%
\pgfusepath{clip}%
\pgfsetbuttcap%
\pgfsetroundjoin%
\definecolor{currentfill}{rgb}{0.121569,0.466667,0.705882}%
\pgfsetfillcolor{currentfill}%
\pgfsetlinewidth{1.003750pt}%
\definecolor{currentstroke}{rgb}{0.121569,0.466667,0.705882}%
\pgfsetstrokecolor{currentstroke}%
\pgfsetdash{}{0pt}%
\pgfpathmoveto{\pgfqpoint{2.249813in}{0.649054in}}%
\pgfpathcurveto{\pgfqpoint{2.252418in}{0.649054in}}{\pgfqpoint{2.254916in}{0.650089in}}{\pgfqpoint{2.256758in}{0.651931in}}%
\pgfpathcurveto{\pgfqpoint{2.258599in}{0.653773in}}{\pgfqpoint{2.259634in}{0.656271in}}{\pgfqpoint{2.259634in}{0.658875in}}%
\pgfpathcurveto{\pgfqpoint{2.259634in}{0.661480in}}{\pgfqpoint{2.258599in}{0.663978in}}{\pgfqpoint{2.256758in}{0.665820in}}%
\pgfpathcurveto{\pgfqpoint{2.254916in}{0.667661in}}{\pgfqpoint{2.252418in}{0.668696in}}{\pgfqpoint{2.249813in}{0.668696in}}%
\pgfpathcurveto{\pgfqpoint{2.247209in}{0.668696in}}{\pgfqpoint{2.244710in}{0.667661in}}{\pgfqpoint{2.242869in}{0.665820in}}%
\pgfpathcurveto{\pgfqpoint{2.241027in}{0.663978in}}{\pgfqpoint{2.239992in}{0.661480in}}{\pgfqpoint{2.239992in}{0.658875in}}%
\pgfpathcurveto{\pgfqpoint{2.239992in}{0.656271in}}{\pgfqpoint{2.241027in}{0.653773in}}{\pgfqpoint{2.242869in}{0.651931in}}%
\pgfpathcurveto{\pgfqpoint{2.244710in}{0.650089in}}{\pgfqpoint{2.247209in}{0.649054in}}{\pgfqpoint{2.249813in}{0.649054in}}%
\pgfpathclose%
\pgfusepath{stroke,fill}%
\end{pgfscope}%
\begin{pgfscope}%
\pgfpathrectangle{\pgfqpoint{0.750000in}{0.500000in}}{\pgfqpoint{4.650000in}{3.020000in}}%
\pgfusepath{clip}%
\pgfsetbuttcap%
\pgfsetroundjoin%
\definecolor{currentfill}{rgb}{0.121569,0.466667,0.705882}%
\pgfsetfillcolor{currentfill}%
\pgfsetlinewidth{1.003750pt}%
\definecolor{currentstroke}{rgb}{0.121569,0.466667,0.705882}%
\pgfsetstrokecolor{currentstroke}%
\pgfsetdash{}{0pt}%
\pgfpathmoveto{\pgfqpoint{1.236426in}{0.632732in}}%
\pgfpathcurveto{\pgfqpoint{1.239030in}{0.632732in}}{\pgfqpoint{1.241529in}{0.633767in}}{\pgfqpoint{1.243370in}{0.635609in}}%
\pgfpathcurveto{\pgfqpoint{1.245212in}{0.637451in}}{\pgfqpoint{1.246247in}{0.639949in}}{\pgfqpoint{1.246247in}{0.642553in}}%
\pgfpathcurveto{\pgfqpoint{1.246247in}{0.645158in}}{\pgfqpoint{1.245212in}{0.647656in}}{\pgfqpoint{1.243370in}{0.649498in}}%
\pgfpathcurveto{\pgfqpoint{1.241529in}{0.651340in}}{\pgfqpoint{1.239030in}{0.652374in}}{\pgfqpoint{1.236426in}{0.652374in}}%
\pgfpathcurveto{\pgfqpoint{1.233821in}{0.652374in}}{\pgfqpoint{1.231323in}{0.651340in}}{\pgfqpoint{1.229481in}{0.649498in}}%
\pgfpathcurveto{\pgfqpoint{1.227640in}{0.647656in}}{\pgfqpoint{1.226605in}{0.645158in}}{\pgfqpoint{1.226605in}{0.642553in}}%
\pgfpathcurveto{\pgfqpoint{1.226605in}{0.639949in}}{\pgfqpoint{1.227640in}{0.637451in}}{\pgfqpoint{1.229481in}{0.635609in}}%
\pgfpathcurveto{\pgfqpoint{1.231323in}{0.633767in}}{\pgfqpoint{1.233821in}{0.632732in}}{\pgfqpoint{1.236426in}{0.632732in}}%
\pgfpathclose%
\pgfusepath{stroke,fill}%
\end{pgfscope}%
\begin{pgfscope}%
\pgfpathrectangle{\pgfqpoint{0.750000in}{0.500000in}}{\pgfqpoint{4.650000in}{3.020000in}}%
\pgfusepath{clip}%
\pgfsetbuttcap%
\pgfsetroundjoin%
\definecolor{currentfill}{rgb}{0.121569,0.466667,0.705882}%
\pgfsetfillcolor{currentfill}%
\pgfsetlinewidth{1.003750pt}%
\definecolor{currentstroke}{rgb}{0.121569,0.466667,0.705882}%
\pgfsetstrokecolor{currentstroke}%
\pgfsetdash{}{0pt}%
\pgfpathmoveto{\pgfqpoint{2.883904in}{0.711175in}}%
\pgfpathcurveto{\pgfqpoint{2.887094in}{0.711175in}}{\pgfqpoint{2.890154in}{0.712442in}}{\pgfqpoint{2.892409in}{0.714698in}}%
\pgfpathcurveto{\pgfqpoint{2.894665in}{0.716954in}}{\pgfqpoint{2.895932in}{0.720013in}}{\pgfqpoint{2.895932in}{0.723203in}}%
\pgfpathcurveto{\pgfqpoint{2.895932in}{0.726393in}}{\pgfqpoint{2.894665in}{0.729453in}}{\pgfqpoint{2.892409in}{0.731708in}}%
\pgfpathcurveto{\pgfqpoint{2.890154in}{0.733964in}}{\pgfqpoint{2.887094in}{0.735231in}}{\pgfqpoint{2.883904in}{0.735231in}}%
\pgfpathcurveto{\pgfqpoint{2.880714in}{0.735231in}}{\pgfqpoint{2.877655in}{0.733964in}}{\pgfqpoint{2.875399in}{0.731708in}}%
\pgfpathcurveto{\pgfqpoint{2.873143in}{0.729453in}}{\pgfqpoint{2.871876in}{0.726393in}}{\pgfqpoint{2.871876in}{0.723203in}}%
\pgfpathcurveto{\pgfqpoint{2.871876in}{0.720013in}}{\pgfqpoint{2.873143in}{0.716954in}}{\pgfqpoint{2.875399in}{0.714698in}}%
\pgfpathcurveto{\pgfqpoint{2.877655in}{0.712442in}}{\pgfqpoint{2.880714in}{0.711175in}}{\pgfqpoint{2.883904in}{0.711175in}}%
\pgfpathclose%
\pgfusepath{stroke,fill}%
\end{pgfscope}%
\begin{pgfscope}%
\pgfpathrectangle{\pgfqpoint{0.750000in}{0.500000in}}{\pgfqpoint{4.650000in}{3.020000in}}%
\pgfusepath{clip}%
\pgfsetbuttcap%
\pgfsetroundjoin%
\definecolor{currentfill}{rgb}{0.121569,0.466667,0.705882}%
\pgfsetfillcolor{currentfill}%
\pgfsetlinewidth{1.003750pt}%
\definecolor{currentstroke}{rgb}{0.121569,0.466667,0.705882}%
\pgfsetstrokecolor{currentstroke}%
\pgfsetdash{}{0pt}%
\pgfpathmoveto{\pgfqpoint{3.898739in}{0.840598in}}%
\pgfpathcurveto{\pgfqpoint{3.901344in}{0.840598in}}{\pgfqpoint{3.903842in}{0.841633in}}{\pgfqpoint{3.905684in}{0.843474in}}%
\pgfpathcurveto{\pgfqpoint{3.907525in}{0.845316in}}{\pgfqpoint{3.908560in}{0.847814in}}{\pgfqpoint{3.908560in}{0.850419in}}%
\pgfpathcurveto{\pgfqpoint{3.908560in}{0.853023in}}{\pgfqpoint{3.907525in}{0.855521in}}{\pgfqpoint{3.905684in}{0.857363in}}%
\pgfpathcurveto{\pgfqpoint{3.903842in}{0.859205in}}{\pgfqpoint{3.901344in}{0.860240in}}{\pgfqpoint{3.898739in}{0.860240in}}%
\pgfpathcurveto{\pgfqpoint{3.896135in}{0.860240in}}{\pgfqpoint{3.893636in}{0.859205in}}{\pgfqpoint{3.891795in}{0.857363in}}%
\pgfpathcurveto{\pgfqpoint{3.889953in}{0.855521in}}{\pgfqpoint{3.888918in}{0.853023in}}{\pgfqpoint{3.888918in}{0.850419in}}%
\pgfpathcurveto{\pgfqpoint{3.888918in}{0.847814in}}{\pgfqpoint{3.889953in}{0.845316in}}{\pgfqpoint{3.891795in}{0.843474in}}%
\pgfpathcurveto{\pgfqpoint{3.893636in}{0.841633in}}{\pgfqpoint{3.896135in}{0.840598in}}{\pgfqpoint{3.898739in}{0.840598in}}%
\pgfpathclose%
\pgfusepath{stroke,fill}%
\end{pgfscope}%
\begin{pgfscope}%
\pgfpathrectangle{\pgfqpoint{0.750000in}{0.500000in}}{\pgfqpoint{4.650000in}{3.020000in}}%
\pgfusepath{clip}%
\pgfsetbuttcap%
\pgfsetroundjoin%
\definecolor{currentfill}{rgb}{0.121569,0.466667,0.705882}%
\pgfsetfillcolor{currentfill}%
\pgfsetlinewidth{1.003750pt}%
\definecolor{currentstroke}{rgb}{0.121569,0.466667,0.705882}%
\pgfsetstrokecolor{currentstroke}%
\pgfsetdash{}{0pt}%
\pgfpathmoveto{\pgfqpoint{4.605215in}{0.907716in}}%
\pgfpathcurveto{\pgfqpoint{4.612808in}{0.907716in}}{\pgfqpoint{4.620092in}{0.910733in}}{\pgfqpoint{4.625461in}{0.916103in}}%
\pgfpathcurveto{\pgfqpoint{4.630831in}{0.921472in}}{\pgfqpoint{4.633847in}{0.928756in}}{\pgfqpoint{4.633847in}{0.936349in}}%
\pgfpathcurveto{\pgfqpoint{4.633847in}{0.943943in}}{\pgfqpoint{4.630831in}{0.951226in}}{\pgfqpoint{4.625461in}{0.956596in}}%
\pgfpathcurveto{\pgfqpoint{4.620092in}{0.961965in}}{\pgfqpoint{4.612808in}{0.964982in}}{\pgfqpoint{4.605215in}{0.964982in}}%
\pgfpathcurveto{\pgfqpoint{4.597621in}{0.964982in}}{\pgfqpoint{4.590338in}{0.961965in}}{\pgfqpoint{4.584968in}{0.956596in}}%
\pgfpathcurveto{\pgfqpoint{4.579599in}{0.951226in}}{\pgfqpoint{4.576582in}{0.943943in}}{\pgfqpoint{4.576582in}{0.936349in}}%
\pgfpathcurveto{\pgfqpoint{4.576582in}{0.928756in}}{\pgfqpoint{4.579599in}{0.921472in}}{\pgfqpoint{4.584968in}{0.916103in}}%
\pgfpathcurveto{\pgfqpoint{4.590338in}{0.910733in}}{\pgfqpoint{4.597621in}{0.907716in}}{\pgfqpoint{4.605215in}{0.907716in}}%
\pgfpathclose%
\pgfusepath{stroke,fill}%
\end{pgfscope}%
\begin{pgfscope}%
\pgfpathrectangle{\pgfqpoint{0.750000in}{0.500000in}}{\pgfqpoint{4.650000in}{3.020000in}}%
\pgfusepath{clip}%
\pgfsetbuttcap%
\pgfsetroundjoin%
\definecolor{currentfill}{rgb}{0.121569,0.466667,0.705882}%
\pgfsetfillcolor{currentfill}%
\pgfsetlinewidth{1.003750pt}%
\definecolor{currentstroke}{rgb}{0.121569,0.466667,0.705882}%
\pgfsetstrokecolor{currentstroke}%
\pgfsetdash{}{0pt}%
\pgfpathmoveto{\pgfqpoint{1.006242in}{0.629852in}}%
\pgfpathcurveto{\pgfqpoint{1.008847in}{0.629852in}}{\pgfqpoint{1.011345in}{0.630887in}}{\pgfqpoint{1.013187in}{0.632729in}}%
\pgfpathcurveto{\pgfqpoint{1.015028in}{0.634570in}}{\pgfqpoint{1.016063in}{0.637068in}}{\pgfqpoint{1.016063in}{0.639673in}}%
\pgfpathcurveto{\pgfqpoint{1.016063in}{0.642278in}}{\pgfqpoint{1.015028in}{0.644776in}}{\pgfqpoint{1.013187in}{0.646617in}}%
\pgfpathcurveto{\pgfqpoint{1.011345in}{0.648459in}}{\pgfqpoint{1.008847in}{0.649494in}}{\pgfqpoint{1.006242in}{0.649494in}}%
\pgfpathcurveto{\pgfqpoint{1.003638in}{0.649494in}}{\pgfqpoint{1.001139in}{0.648459in}}{\pgfqpoint{0.999298in}{0.646617in}}%
\pgfpathcurveto{\pgfqpoint{0.997456in}{0.644776in}}{\pgfqpoint{0.996421in}{0.642278in}}{\pgfqpoint{0.996421in}{0.639673in}}%
\pgfpathcurveto{\pgfqpoint{0.996421in}{0.637068in}}{\pgfqpoint{0.997456in}{0.634570in}}{\pgfqpoint{0.999298in}{0.632729in}}%
\pgfpathcurveto{\pgfqpoint{1.001139in}{0.630887in}}{\pgfqpoint{1.003638in}{0.629852in}}{\pgfqpoint{1.006242in}{0.629852in}}%
\pgfpathclose%
\pgfusepath{stroke,fill}%
\end{pgfscope}%
\begin{pgfscope}%
\pgfpathrectangle{\pgfqpoint{0.750000in}{0.500000in}}{\pgfqpoint{4.650000in}{3.020000in}}%
\pgfusepath{clip}%
\pgfsetbuttcap%
\pgfsetroundjoin%
\definecolor{currentfill}{rgb}{0.121569,0.466667,0.705882}%
\pgfsetfillcolor{currentfill}%
\pgfsetlinewidth{1.003750pt}%
\definecolor{currentstroke}{rgb}{0.121569,0.466667,0.705882}%
\pgfsetstrokecolor{currentstroke}%
\pgfsetdash{}{0pt}%
\pgfpathmoveto{\pgfqpoint{2.044240in}{0.711235in}}%
\pgfpathcurveto{\pgfqpoint{2.047924in}{0.711235in}}{\pgfqpoint{2.051457in}{0.712698in}}{\pgfqpoint{2.054061in}{0.715303in}}%
\pgfpathcurveto{\pgfqpoint{2.056666in}{0.717907in}}{\pgfqpoint{2.058129in}{0.721440in}}{\pgfqpoint{2.058129in}{0.725123in}}%
\pgfpathcurveto{\pgfqpoint{2.058129in}{0.728807in}}{\pgfqpoint{2.056666in}{0.732340in}}{\pgfqpoint{2.054061in}{0.734944in}}%
\pgfpathcurveto{\pgfqpoint{2.051457in}{0.737549in}}{\pgfqpoint{2.047924in}{0.739012in}}{\pgfqpoint{2.044240in}{0.739012in}}%
\pgfpathcurveto{\pgfqpoint{2.040557in}{0.739012in}}{\pgfqpoint{2.037024in}{0.737549in}}{\pgfqpoint{2.034419in}{0.734944in}}%
\pgfpathcurveto{\pgfqpoint{2.031815in}{0.732340in}}{\pgfqpoint{2.030351in}{0.728807in}}{\pgfqpoint{2.030351in}{0.725123in}}%
\pgfpathcurveto{\pgfqpoint{2.030351in}{0.721440in}}{\pgfqpoint{2.031815in}{0.717907in}}{\pgfqpoint{2.034419in}{0.715303in}}%
\pgfpathcurveto{\pgfqpoint{2.037024in}{0.712698in}}{\pgfqpoint{2.040557in}{0.711235in}}{\pgfqpoint{2.044240in}{0.711235in}}%
\pgfpathclose%
\pgfusepath{stroke,fill}%
\end{pgfscope}%
\begin{pgfscope}%
\pgfpathrectangle{\pgfqpoint{0.750000in}{0.500000in}}{\pgfqpoint{4.650000in}{3.020000in}}%
\pgfusepath{clip}%
\pgfsetbuttcap%
\pgfsetroundjoin%
\definecolor{currentfill}{rgb}{0.121569,0.466667,0.705882}%
\pgfsetfillcolor{currentfill}%
\pgfsetlinewidth{1.003750pt}%
\definecolor{currentstroke}{rgb}{0.121569,0.466667,0.705882}%
\pgfsetstrokecolor{currentstroke}%
\pgfsetdash{}{0pt}%
\pgfpathmoveto{\pgfqpoint{1.311706in}{0.647614in}}%
\pgfpathcurveto{\pgfqpoint{1.314311in}{0.647614in}}{\pgfqpoint{1.316809in}{0.648649in}}{\pgfqpoint{1.318651in}{0.650491in}}%
\pgfpathcurveto{\pgfqpoint{1.320492in}{0.652332in}}{\pgfqpoint{1.321527in}{0.654831in}}{\pgfqpoint{1.321527in}{0.657435in}}%
\pgfpathcurveto{\pgfqpoint{1.321527in}{0.660040in}}{\pgfqpoint{1.320492in}{0.662538in}}{\pgfqpoint{1.318651in}{0.664380in}}%
\pgfpathcurveto{\pgfqpoint{1.316809in}{0.666221in}}{\pgfqpoint{1.314311in}{0.667256in}}{\pgfqpoint{1.311706in}{0.667256in}}%
\pgfpathcurveto{\pgfqpoint{1.309102in}{0.667256in}}{\pgfqpoint{1.306603in}{0.666221in}}{\pgfqpoint{1.304762in}{0.664380in}}%
\pgfpathcurveto{\pgfqpoint{1.302920in}{0.662538in}}{\pgfqpoint{1.301885in}{0.660040in}}{\pgfqpoint{1.301885in}{0.657435in}}%
\pgfpathcurveto{\pgfqpoint{1.301885in}{0.654831in}}{\pgfqpoint{1.302920in}{0.652332in}}{\pgfqpoint{1.304762in}{0.650491in}}%
\pgfpathcurveto{\pgfqpoint{1.306603in}{0.648649in}}{\pgfqpoint{1.309102in}{0.647614in}}{\pgfqpoint{1.311706in}{0.647614in}}%
\pgfpathclose%
\pgfusepath{stroke,fill}%
\end{pgfscope}%
\begin{pgfscope}%
\pgfpathrectangle{\pgfqpoint{0.750000in}{0.500000in}}{\pgfqpoint{4.650000in}{3.020000in}}%
\pgfusepath{clip}%
\pgfsetbuttcap%
\pgfsetroundjoin%
\definecolor{currentfill}{rgb}{0.121569,0.466667,0.705882}%
\pgfsetfillcolor{currentfill}%
\pgfsetlinewidth{1.003750pt}%
\definecolor{currentstroke}{rgb}{0.121569,0.466667,0.705882}%
\pgfsetstrokecolor{currentstroke}%
\pgfsetdash{}{0pt}%
\pgfpathmoveto{\pgfqpoint{1.955931in}{0.687068in}}%
\pgfpathcurveto{\pgfqpoint{1.960804in}{0.687068in}}{\pgfqpoint{1.965477in}{0.689004in}}{\pgfqpoint{1.968923in}{0.692449in}}%
\pgfpathcurveto{\pgfqpoint{1.972368in}{0.695895in}}{\pgfqpoint{1.974304in}{0.700568in}}{\pgfqpoint{1.974304in}{0.705441in}}%
\pgfpathcurveto{\pgfqpoint{1.974304in}{0.710314in}}{\pgfqpoint{1.972368in}{0.714987in}}{\pgfqpoint{1.968923in}{0.718433in}}%
\pgfpathcurveto{\pgfqpoint{1.965477in}{0.721878in}}{\pgfqpoint{1.960804in}{0.723814in}}{\pgfqpoint{1.955931in}{0.723814in}}%
\pgfpathcurveto{\pgfqpoint{1.951058in}{0.723814in}}{\pgfqpoint{1.946385in}{0.721878in}}{\pgfqpoint{1.942939in}{0.718433in}}%
\pgfpathcurveto{\pgfqpoint{1.939494in}{0.714987in}}{\pgfqpoint{1.937558in}{0.710314in}}{\pgfqpoint{1.937558in}{0.705441in}}%
\pgfpathcurveto{\pgfqpoint{1.937558in}{0.700568in}}{\pgfqpoint{1.939494in}{0.695895in}}{\pgfqpoint{1.942939in}{0.692449in}}%
\pgfpathcurveto{\pgfqpoint{1.946385in}{0.689004in}}{\pgfqpoint{1.951058in}{0.687068in}}{\pgfqpoint{1.955931in}{0.687068in}}%
\pgfpathclose%
\pgfusepath{stroke,fill}%
\end{pgfscope}%
\begin{pgfscope}%
\pgfpathrectangle{\pgfqpoint{0.750000in}{0.500000in}}{\pgfqpoint{4.650000in}{3.020000in}}%
\pgfusepath{clip}%
\pgfsetbuttcap%
\pgfsetroundjoin%
\definecolor{currentfill}{rgb}{0.121569,0.466667,0.705882}%
\pgfsetfillcolor{currentfill}%
\pgfsetlinewidth{1.003750pt}%
\definecolor{currentstroke}{rgb}{0.121569,0.466667,0.705882}%
\pgfsetstrokecolor{currentstroke}%
\pgfsetdash{}{0pt}%
\pgfpathmoveto{\pgfqpoint{2.917201in}{0.739499in}}%
\pgfpathcurveto{\pgfqpoint{2.920391in}{0.739499in}}{\pgfqpoint{2.923451in}{0.740766in}}{\pgfqpoint{2.925706in}{0.743021in}}%
\pgfpathcurveto{\pgfqpoint{2.927962in}{0.745277in}}{\pgfqpoint{2.929229in}{0.748337in}}{\pgfqpoint{2.929229in}{0.751527in}}%
\pgfpathcurveto{\pgfqpoint{2.929229in}{0.754717in}}{\pgfqpoint{2.927962in}{0.757776in}}{\pgfqpoint{2.925706in}{0.760032in}}%
\pgfpathcurveto{\pgfqpoint{2.923451in}{0.762287in}}{\pgfqpoint{2.920391in}{0.763555in}}{\pgfqpoint{2.917201in}{0.763555in}}%
\pgfpathcurveto{\pgfqpoint{2.914011in}{0.763555in}}{\pgfqpoint{2.910952in}{0.762287in}}{\pgfqpoint{2.908696in}{0.760032in}}%
\pgfpathcurveto{\pgfqpoint{2.906440in}{0.757776in}}{\pgfqpoint{2.905173in}{0.754717in}}{\pgfqpoint{2.905173in}{0.751527in}}%
\pgfpathcurveto{\pgfqpoint{2.905173in}{0.748337in}}{\pgfqpoint{2.906440in}{0.745277in}}{\pgfqpoint{2.908696in}{0.743021in}}%
\pgfpathcurveto{\pgfqpoint{2.910952in}{0.740766in}}{\pgfqpoint{2.914011in}{0.739499in}}{\pgfqpoint{2.917201in}{0.739499in}}%
\pgfpathclose%
\pgfusepath{stroke,fill}%
\end{pgfscope}%
\begin{pgfscope}%
\pgfpathrectangle{\pgfqpoint{0.750000in}{0.500000in}}{\pgfqpoint{4.650000in}{3.020000in}}%
\pgfusepath{clip}%
\pgfsetbuttcap%
\pgfsetroundjoin%
\definecolor{currentfill}{rgb}{0.121569,0.466667,0.705882}%
\pgfsetfillcolor{currentfill}%
\pgfsetlinewidth{1.003750pt}%
\definecolor{currentstroke}{rgb}{0.121569,0.466667,0.705882}%
\pgfsetstrokecolor{currentstroke}%
\pgfsetdash{}{0pt}%
\pgfpathmoveto{\pgfqpoint{1.117715in}{0.650975in}}%
\pgfpathcurveto{\pgfqpoint{1.120319in}{0.650975in}}{\pgfqpoint{1.122818in}{0.652009in}}{\pgfqpoint{1.124659in}{0.653851in}}%
\pgfpathcurveto{\pgfqpoint{1.126501in}{0.655693in}}{\pgfqpoint{1.127536in}{0.658191in}}{\pgfqpoint{1.127536in}{0.660796in}}%
\pgfpathcurveto{\pgfqpoint{1.127536in}{0.663400in}}{\pgfqpoint{1.126501in}{0.665898in}}{\pgfqpoint{1.124659in}{0.667740in}}%
\pgfpathcurveto{\pgfqpoint{1.122818in}{0.669582in}}{\pgfqpoint{1.120319in}{0.670617in}}{\pgfqpoint{1.117715in}{0.670617in}}%
\pgfpathcurveto{\pgfqpoint{1.115110in}{0.670617in}}{\pgfqpoint{1.112612in}{0.669582in}}{\pgfqpoint{1.110770in}{0.667740in}}%
\pgfpathcurveto{\pgfqpoint{1.108929in}{0.665898in}}{\pgfqpoint{1.107894in}{0.663400in}}{\pgfqpoint{1.107894in}{0.660796in}}%
\pgfpathcurveto{\pgfqpoint{1.107894in}{0.658191in}}{\pgfqpoint{1.108929in}{0.655693in}}{\pgfqpoint{1.110770in}{0.653851in}}%
\pgfpathcurveto{\pgfqpoint{1.112612in}{0.652009in}}{\pgfqpoint{1.115110in}{0.650975in}}{\pgfqpoint{1.117715in}{0.650975in}}%
\pgfpathclose%
\pgfusepath{stroke,fill}%
\end{pgfscope}%
\begin{pgfscope}%
\pgfpathrectangle{\pgfqpoint{0.750000in}{0.500000in}}{\pgfqpoint{4.650000in}{3.020000in}}%
\pgfusepath{clip}%
\pgfsetbuttcap%
\pgfsetroundjoin%
\definecolor{currentfill}{rgb}{0.121569,0.466667,0.705882}%
\pgfsetfillcolor{currentfill}%
\pgfsetlinewidth{1.003750pt}%
\definecolor{currentstroke}{rgb}{0.121569,0.466667,0.705882}%
\pgfsetstrokecolor{currentstroke}%
\pgfsetdash{}{0pt}%
\pgfpathmoveto{\pgfqpoint{1.094552in}{0.628892in}}%
\pgfpathcurveto{\pgfqpoint{1.097156in}{0.628892in}}{\pgfqpoint{1.099654in}{0.629927in}}{\pgfqpoint{1.101496in}{0.631768in}}%
\pgfpathcurveto{\pgfqpoint{1.103338in}{0.633610in}}{\pgfqpoint{1.104373in}{0.636108in}}{\pgfqpoint{1.104373in}{0.638713in}}%
\pgfpathcurveto{\pgfqpoint{1.104373in}{0.641317in}}{\pgfqpoint{1.103338in}{0.643816in}}{\pgfqpoint{1.101496in}{0.645657in}}%
\pgfpathcurveto{\pgfqpoint{1.099654in}{0.647499in}}{\pgfqpoint{1.097156in}{0.648534in}}{\pgfqpoint{1.094552in}{0.648534in}}%
\pgfpathcurveto{\pgfqpoint{1.091947in}{0.648534in}}{\pgfqpoint{1.089449in}{0.647499in}}{\pgfqpoint{1.087607in}{0.645657in}}%
\pgfpathcurveto{\pgfqpoint{1.085766in}{0.643816in}}{\pgfqpoint{1.084731in}{0.641317in}}{\pgfqpoint{1.084731in}{0.638713in}}%
\pgfpathcurveto{\pgfqpoint{1.084731in}{0.636108in}}{\pgfqpoint{1.085766in}{0.633610in}}{\pgfqpoint{1.087607in}{0.631768in}}%
\pgfpathcurveto{\pgfqpoint{1.089449in}{0.629927in}}{\pgfqpoint{1.091947in}{0.628892in}}{\pgfqpoint{1.094552in}{0.628892in}}%
\pgfpathclose%
\pgfusepath{stroke,fill}%
\end{pgfscope}%
\begin{pgfscope}%
\pgfpathrectangle{\pgfqpoint{0.750000in}{0.500000in}}{\pgfqpoint{4.650000in}{3.020000in}}%
\pgfusepath{clip}%
\pgfsetbuttcap%
\pgfsetroundjoin%
\definecolor{currentfill}{rgb}{0.121569,0.466667,0.705882}%
\pgfsetfillcolor{currentfill}%
\pgfsetlinewidth{1.003750pt}%
\definecolor{currentstroke}{rgb}{0.121569,0.466667,0.705882}%
\pgfsetstrokecolor{currentstroke}%
\pgfsetdash{}{0pt}%
\pgfpathmoveto{\pgfqpoint{3.360196in}{0.748907in}}%
\pgfpathcurveto{\pgfqpoint{3.362801in}{0.748907in}}{\pgfqpoint{3.365299in}{0.749941in}}{\pgfqpoint{3.367141in}{0.751783in}}%
\pgfpathcurveto{\pgfqpoint{3.368982in}{0.753625in}}{\pgfqpoint{3.370017in}{0.756123in}}{\pgfqpoint{3.370017in}{0.758728in}}%
\pgfpathcurveto{\pgfqpoint{3.370017in}{0.761332in}}{\pgfqpoint{3.368982in}{0.763830in}}{\pgfqpoint{3.367141in}{0.765672in}}%
\pgfpathcurveto{\pgfqpoint{3.365299in}{0.767514in}}{\pgfqpoint{3.362801in}{0.768548in}}{\pgfqpoint{3.360196in}{0.768548in}}%
\pgfpathcurveto{\pgfqpoint{3.357592in}{0.768548in}}{\pgfqpoint{3.355093in}{0.767514in}}{\pgfqpoint{3.353252in}{0.765672in}}%
\pgfpathcurveto{\pgfqpoint{3.351410in}{0.763830in}}{\pgfqpoint{3.350375in}{0.761332in}}{\pgfqpoint{3.350375in}{0.758728in}}%
\pgfpathcurveto{\pgfqpoint{3.350375in}{0.756123in}}{\pgfqpoint{3.351410in}{0.753625in}}{\pgfqpoint{3.353252in}{0.751783in}}%
\pgfpathcurveto{\pgfqpoint{3.355093in}{0.749941in}}{\pgfqpoint{3.357592in}{0.748907in}}{\pgfqpoint{3.360196in}{0.748907in}}%
\pgfpathclose%
\pgfusepath{stroke,fill}%
\end{pgfscope}%
\begin{pgfscope}%
\pgfpathrectangle{\pgfqpoint{0.750000in}{0.500000in}}{\pgfqpoint{4.650000in}{3.020000in}}%
\pgfusepath{clip}%
\pgfsetbuttcap%
\pgfsetroundjoin%
\definecolor{currentfill}{rgb}{0.121569,0.466667,0.705882}%
\pgfsetfillcolor{currentfill}%
\pgfsetlinewidth{1.003750pt}%
\definecolor{currentstroke}{rgb}{0.121569,0.466667,0.705882}%
\pgfsetstrokecolor{currentstroke}%
\pgfsetdash{}{0pt}%
\pgfpathmoveto{\pgfqpoint{2.356943in}{0.729278in}}%
\pgfpathcurveto{\pgfqpoint{2.361061in}{0.729278in}}{\pgfqpoint{2.365011in}{0.730914in}}{\pgfqpoint{2.367923in}{0.733826in}}%
\pgfpathcurveto{\pgfqpoint{2.370835in}{0.736738in}}{\pgfqpoint{2.372471in}{0.740688in}}{\pgfqpoint{2.372471in}{0.744806in}}%
\pgfpathcurveto{\pgfqpoint{2.372471in}{0.748924in}}{\pgfqpoint{2.370835in}{0.752874in}}{\pgfqpoint{2.367923in}{0.755786in}}%
\pgfpathcurveto{\pgfqpoint{2.365011in}{0.758698in}}{\pgfqpoint{2.361061in}{0.760334in}}{\pgfqpoint{2.356943in}{0.760334in}}%
\pgfpathcurveto{\pgfqpoint{2.352825in}{0.760334in}}{\pgfqpoint{2.348875in}{0.758698in}}{\pgfqpoint{2.345963in}{0.755786in}}%
\pgfpathcurveto{\pgfqpoint{2.343051in}{0.752874in}}{\pgfqpoint{2.341414in}{0.748924in}}{\pgfqpoint{2.341414in}{0.744806in}}%
\pgfpathcurveto{\pgfqpoint{2.341414in}{0.740688in}}{\pgfqpoint{2.343051in}{0.736738in}}{\pgfqpoint{2.345963in}{0.733826in}}%
\pgfpathcurveto{\pgfqpoint{2.348875in}{0.730914in}}{\pgfqpoint{2.352825in}{0.729278in}}{\pgfqpoint{2.356943in}{0.729278in}}%
\pgfpathclose%
\pgfusepath{stroke,fill}%
\end{pgfscope}%
\begin{pgfscope}%
\pgfpathrectangle{\pgfqpoint{0.750000in}{0.500000in}}{\pgfqpoint{4.650000in}{3.020000in}}%
\pgfusepath{clip}%
\pgfsetbuttcap%
\pgfsetroundjoin%
\definecolor{currentfill}{rgb}{0.121569,0.466667,0.705882}%
\pgfsetfillcolor{currentfill}%
\pgfsetlinewidth{1.003750pt}%
\definecolor{currentstroke}{rgb}{0.121569,0.466667,0.705882}%
\pgfsetstrokecolor{currentstroke}%
\pgfsetdash{}{0pt}%
\pgfpathmoveto{\pgfqpoint{1.116267in}{0.634653in}}%
\pgfpathcurveto{\pgfqpoint{1.118872in}{0.634653in}}{\pgfqpoint{1.121370in}{0.635687in}}{\pgfqpoint{1.123212in}{0.637529in}}%
\pgfpathcurveto{\pgfqpoint{1.125053in}{0.639371in}}{\pgfqpoint{1.126088in}{0.641869in}}{\pgfqpoint{1.126088in}{0.644474in}}%
\pgfpathcurveto{\pgfqpoint{1.126088in}{0.647078in}}{\pgfqpoint{1.125053in}{0.649576in}}{\pgfqpoint{1.123212in}{0.651418in}}%
\pgfpathcurveto{\pgfqpoint{1.121370in}{0.653260in}}{\pgfqpoint{1.118872in}{0.654295in}}{\pgfqpoint{1.116267in}{0.654295in}}%
\pgfpathcurveto{\pgfqpoint{1.113663in}{0.654295in}}{\pgfqpoint{1.111164in}{0.653260in}}{\pgfqpoint{1.109323in}{0.651418in}}%
\pgfpathcurveto{\pgfqpoint{1.107481in}{0.649576in}}{\pgfqpoint{1.106446in}{0.647078in}}{\pgfqpoint{1.106446in}{0.644474in}}%
\pgfpathcurveto{\pgfqpoint{1.106446in}{0.641869in}}{\pgfqpoint{1.107481in}{0.639371in}}{\pgfqpoint{1.109323in}{0.637529in}}%
\pgfpathcurveto{\pgfqpoint{1.111164in}{0.635687in}}{\pgfqpoint{1.113663in}{0.634653in}}{\pgfqpoint{1.116267in}{0.634653in}}%
\pgfpathclose%
\pgfusepath{stroke,fill}%
\end{pgfscope}%
\begin{pgfscope}%
\pgfpathrectangle{\pgfqpoint{0.750000in}{0.500000in}}{\pgfqpoint{4.650000in}{3.020000in}}%
\pgfusepath{clip}%
\pgfsetbuttcap%
\pgfsetroundjoin%
\definecolor{currentfill}{rgb}{0.121569,0.466667,0.705882}%
\pgfsetfillcolor{currentfill}%
\pgfsetlinewidth{1.003750pt}%
\definecolor{currentstroke}{rgb}{0.121569,0.466667,0.705882}%
\pgfsetstrokecolor{currentstroke}%
\pgfsetdash{}{0pt}%
\pgfpathmoveto{\pgfqpoint{2.271529in}{0.718716in}}%
\pgfpathcurveto{\pgfqpoint{2.275647in}{0.718716in}}{\pgfqpoint{2.279597in}{0.720352in}}{\pgfqpoint{2.282509in}{0.723264in}}%
\pgfpathcurveto{\pgfqpoint{2.285421in}{0.726176in}}{\pgfqpoint{2.287057in}{0.730126in}}{\pgfqpoint{2.287057in}{0.734245in}}%
\pgfpathcurveto{\pgfqpoint{2.287057in}{0.738363in}}{\pgfqpoint{2.285421in}{0.742313in}}{\pgfqpoint{2.282509in}{0.745225in}}%
\pgfpathcurveto{\pgfqpoint{2.279597in}{0.748137in}}{\pgfqpoint{2.275647in}{0.749773in}}{\pgfqpoint{2.271529in}{0.749773in}}%
\pgfpathcurveto{\pgfqpoint{2.267411in}{0.749773in}}{\pgfqpoint{2.263460in}{0.748137in}}{\pgfqpoint{2.260549in}{0.745225in}}%
\pgfpathcurveto{\pgfqpoint{2.257637in}{0.742313in}}{\pgfqpoint{2.256000in}{0.738363in}}{\pgfqpoint{2.256000in}{0.734245in}}%
\pgfpathcurveto{\pgfqpoint{2.256000in}{0.730126in}}{\pgfqpoint{2.257637in}{0.726176in}}{\pgfqpoint{2.260549in}{0.723264in}}%
\pgfpathcurveto{\pgfqpoint{2.263460in}{0.720352in}}{\pgfqpoint{2.267411in}{0.718716in}}{\pgfqpoint{2.271529in}{0.718716in}}%
\pgfpathclose%
\pgfusepath{stroke,fill}%
\end{pgfscope}%
\begin{pgfscope}%
\pgfpathrectangle{\pgfqpoint{0.750000in}{0.500000in}}{\pgfqpoint{4.650000in}{3.020000in}}%
\pgfusepath{clip}%
\pgfsetbuttcap%
\pgfsetroundjoin%
\definecolor{currentfill}{rgb}{0.121569,0.466667,0.705882}%
\pgfsetfillcolor{currentfill}%
\pgfsetlinewidth{1.003750pt}%
\definecolor{currentstroke}{rgb}{0.121569,0.466667,0.705882}%
\pgfsetstrokecolor{currentstroke}%
\pgfsetdash{}{0pt}%
\pgfpathmoveto{\pgfqpoint{2.261395in}{1.633654in}}%
\pgfpathcurveto{\pgfqpoint{2.263999in}{1.633654in}}{\pgfqpoint{2.266498in}{1.634689in}}{\pgfqpoint{2.268339in}{1.636531in}}%
\pgfpathcurveto{\pgfqpoint{2.270181in}{1.638373in}}{\pgfqpoint{2.271216in}{1.640871in}}{\pgfqpoint{2.271216in}{1.643475in}}%
\pgfpathcurveto{\pgfqpoint{2.271216in}{1.646080in}}{\pgfqpoint{2.270181in}{1.648578in}}{\pgfqpoint{2.268339in}{1.650420in}}%
\pgfpathcurveto{\pgfqpoint{2.266498in}{1.652261in}}{\pgfqpoint{2.263999in}{1.653296in}}{\pgfqpoint{2.261395in}{1.653296in}}%
\pgfpathcurveto{\pgfqpoint{2.258790in}{1.653296in}}{\pgfqpoint{2.256292in}{1.652261in}}{\pgfqpoint{2.254450in}{1.650420in}}%
\pgfpathcurveto{\pgfqpoint{2.252609in}{1.648578in}}{\pgfqpoint{2.251574in}{1.646080in}}{\pgfqpoint{2.251574in}{1.643475in}}%
\pgfpathcurveto{\pgfqpoint{2.251574in}{1.640871in}}{\pgfqpoint{2.252609in}{1.638373in}}{\pgfqpoint{2.254450in}{1.636531in}}%
\pgfpathcurveto{\pgfqpoint{2.256292in}{1.634689in}}{\pgfqpoint{2.258790in}{1.633654in}}{\pgfqpoint{2.261395in}{1.633654in}}%
\pgfpathclose%
\pgfusepath{stroke,fill}%
\end{pgfscope}%
\begin{pgfscope}%
\pgfpathrectangle{\pgfqpoint{0.750000in}{0.500000in}}{\pgfqpoint{4.650000in}{3.020000in}}%
\pgfusepath{clip}%
\pgfsetbuttcap%
\pgfsetroundjoin%
\definecolor{currentfill}{rgb}{0.121569,0.466667,0.705882}%
\pgfsetfillcolor{currentfill}%
\pgfsetlinewidth{1.003750pt}%
\definecolor{currentstroke}{rgb}{0.121569,0.466667,0.705882}%
\pgfsetstrokecolor{currentstroke}%
\pgfsetdash{}{0pt}%
\pgfpathmoveto{\pgfqpoint{1.025062in}{0.637053in}}%
\pgfpathcurveto{\pgfqpoint{1.027667in}{0.637053in}}{\pgfqpoint{1.030165in}{0.638088in}}{\pgfqpoint{1.032007in}{0.639929in}}%
\pgfpathcurveto{\pgfqpoint{1.033848in}{0.641771in}}{\pgfqpoint{1.034883in}{0.644269in}}{\pgfqpoint{1.034883in}{0.646874in}}%
\pgfpathcurveto{\pgfqpoint{1.034883in}{0.649478in}}{\pgfqpoint{1.033848in}{0.651977in}}{\pgfqpoint{1.032007in}{0.653818in}}%
\pgfpathcurveto{\pgfqpoint{1.030165in}{0.655660in}}{\pgfqpoint{1.027667in}{0.656695in}}{\pgfqpoint{1.025062in}{0.656695in}}%
\pgfpathcurveto{\pgfqpoint{1.022458in}{0.656695in}}{\pgfqpoint{1.019960in}{0.655660in}}{\pgfqpoint{1.018118in}{0.653818in}}%
\pgfpathcurveto{\pgfqpoint{1.016276in}{0.651977in}}{\pgfqpoint{1.015241in}{0.649478in}}{\pgfqpoint{1.015241in}{0.646874in}}%
\pgfpathcurveto{\pgfqpoint{1.015241in}{0.644269in}}{\pgfqpoint{1.016276in}{0.641771in}}{\pgfqpoint{1.018118in}{0.639929in}}%
\pgfpathcurveto{\pgfqpoint{1.019960in}{0.638088in}}{\pgfqpoint{1.022458in}{0.637053in}}{\pgfqpoint{1.025062in}{0.637053in}}%
\pgfpathclose%
\pgfusepath{stroke,fill}%
\end{pgfscope}%
\begin{pgfscope}%
\pgfpathrectangle{\pgfqpoint{0.750000in}{0.500000in}}{\pgfqpoint{4.650000in}{3.020000in}}%
\pgfusepath{clip}%
\pgfsetbuttcap%
\pgfsetroundjoin%
\definecolor{currentfill}{rgb}{0.121569,0.466667,0.705882}%
\pgfsetfillcolor{currentfill}%
\pgfsetlinewidth{1.003750pt}%
\definecolor{currentstroke}{rgb}{0.121569,0.466667,0.705882}%
\pgfsetstrokecolor{currentstroke}%
\pgfsetdash{}{0pt}%
\pgfpathmoveto{\pgfqpoint{2.535009in}{0.878522in}}%
\pgfpathcurveto{\pgfqpoint{2.537614in}{0.878522in}}{\pgfqpoint{2.540112in}{0.879557in}}{\pgfqpoint{2.541954in}{0.881399in}}%
\pgfpathcurveto{\pgfqpoint{2.543795in}{0.883241in}}{\pgfqpoint{2.544830in}{0.885739in}}{\pgfqpoint{2.544830in}{0.888343in}}%
\pgfpathcurveto{\pgfqpoint{2.544830in}{0.890948in}}{\pgfqpoint{2.543795in}{0.893446in}}{\pgfqpoint{2.541954in}{0.895288in}}%
\pgfpathcurveto{\pgfqpoint{2.540112in}{0.897129in}}{\pgfqpoint{2.537614in}{0.898164in}}{\pgfqpoint{2.535009in}{0.898164in}}%
\pgfpathcurveto{\pgfqpoint{2.532405in}{0.898164in}}{\pgfqpoint{2.529907in}{0.897129in}}{\pgfqpoint{2.528065in}{0.895288in}}%
\pgfpathcurveto{\pgfqpoint{2.526223in}{0.893446in}}{\pgfqpoint{2.525188in}{0.890948in}}{\pgfqpoint{2.525188in}{0.888343in}}%
\pgfpathcurveto{\pgfqpoint{2.525188in}{0.885739in}}{\pgfqpoint{2.526223in}{0.883241in}}{\pgfqpoint{2.528065in}{0.881399in}}%
\pgfpathcurveto{\pgfqpoint{2.529907in}{0.879557in}}{\pgfqpoint{2.532405in}{0.878522in}}{\pgfqpoint{2.535009in}{0.878522in}}%
\pgfpathclose%
\pgfusepath{stroke,fill}%
\end{pgfscope}%
\begin{pgfscope}%
\pgfpathrectangle{\pgfqpoint{0.750000in}{0.500000in}}{\pgfqpoint{4.650000in}{3.020000in}}%
\pgfusepath{clip}%
\pgfsetbuttcap%
\pgfsetroundjoin%
\definecolor{currentfill}{rgb}{0.121569,0.466667,0.705882}%
\pgfsetfillcolor{currentfill}%
\pgfsetlinewidth{1.003750pt}%
\definecolor{currentstroke}{rgb}{0.121569,0.466667,0.705882}%
\pgfsetstrokecolor{currentstroke}%
\pgfsetdash{}{0pt}%
\pgfpathmoveto{\pgfqpoint{1.766283in}{0.781551in}}%
\pgfpathcurveto{\pgfqpoint{1.768887in}{0.781551in}}{\pgfqpoint{1.771385in}{0.782585in}}{\pgfqpoint{1.773227in}{0.784427in}}%
\pgfpathcurveto{\pgfqpoint{1.775069in}{0.786269in}}{\pgfqpoint{1.776104in}{0.788767in}}{\pgfqpoint{1.776104in}{0.791372in}}%
\pgfpathcurveto{\pgfqpoint{1.776104in}{0.793976in}}{\pgfqpoint{1.775069in}{0.796474in}}{\pgfqpoint{1.773227in}{0.798316in}}%
\pgfpathcurveto{\pgfqpoint{1.771385in}{0.800158in}}{\pgfqpoint{1.768887in}{0.801192in}}{\pgfqpoint{1.766283in}{0.801192in}}%
\pgfpathcurveto{\pgfqpoint{1.763678in}{0.801192in}}{\pgfqpoint{1.761180in}{0.800158in}}{\pgfqpoint{1.759338in}{0.798316in}}%
\pgfpathcurveto{\pgfqpoint{1.757497in}{0.796474in}}{\pgfqpoint{1.756462in}{0.793976in}}{\pgfqpoint{1.756462in}{0.791372in}}%
\pgfpathcurveto{\pgfqpoint{1.756462in}{0.788767in}}{\pgfqpoint{1.757497in}{0.786269in}}{\pgfqpoint{1.759338in}{0.784427in}}%
\pgfpathcurveto{\pgfqpoint{1.761180in}{0.782585in}}{\pgfqpoint{1.763678in}{0.781551in}}{\pgfqpoint{1.766283in}{0.781551in}}%
\pgfpathclose%
\pgfusepath{stroke,fill}%
\end{pgfscope}%
\begin{pgfscope}%
\pgfpathrectangle{\pgfqpoint{0.750000in}{0.500000in}}{\pgfqpoint{4.650000in}{3.020000in}}%
\pgfusepath{clip}%
\pgfsetbuttcap%
\pgfsetroundjoin%
\definecolor{currentfill}{rgb}{0.121569,0.466667,0.705882}%
\pgfsetfillcolor{currentfill}%
\pgfsetlinewidth{1.003750pt}%
\definecolor{currentstroke}{rgb}{0.121569,0.466667,0.705882}%
\pgfsetstrokecolor{currentstroke}%
\pgfsetdash{}{0pt}%
\pgfpathmoveto{\pgfqpoint{0.990318in}{0.628892in}}%
\pgfpathcurveto{\pgfqpoint{0.992922in}{0.628892in}}{\pgfqpoint{0.995420in}{0.629927in}}{\pgfqpoint{0.997262in}{0.631768in}}%
\pgfpathcurveto{\pgfqpoint{0.999104in}{0.633610in}}{\pgfqpoint{1.000138in}{0.636108in}}{\pgfqpoint{1.000138in}{0.638713in}}%
\pgfpathcurveto{\pgfqpoint{1.000138in}{0.641317in}}{\pgfqpoint{0.999104in}{0.643816in}}{\pgfqpoint{0.997262in}{0.645657in}}%
\pgfpathcurveto{\pgfqpoint{0.995420in}{0.647499in}}{\pgfqpoint{0.992922in}{0.648534in}}{\pgfqpoint{0.990318in}{0.648534in}}%
\pgfpathcurveto{\pgfqpoint{0.987713in}{0.648534in}}{\pgfqpoint{0.985215in}{0.647499in}}{\pgfqpoint{0.983373in}{0.645657in}}%
\pgfpathcurveto{\pgfqpoint{0.981531in}{0.643816in}}{\pgfqpoint{0.980497in}{0.641317in}}{\pgfqpoint{0.980497in}{0.638713in}}%
\pgfpathcurveto{\pgfqpoint{0.980497in}{0.636108in}}{\pgfqpoint{0.981531in}{0.633610in}}{\pgfqpoint{0.983373in}{0.631768in}}%
\pgfpathcurveto{\pgfqpoint{0.985215in}{0.629927in}}{\pgfqpoint{0.987713in}{0.628892in}}{\pgfqpoint{0.990318in}{0.628892in}}%
\pgfpathclose%
\pgfusepath{stroke,fill}%
\end{pgfscope}%
\begin{pgfscope}%
\pgfpathrectangle{\pgfqpoint{0.750000in}{0.500000in}}{\pgfqpoint{4.650000in}{3.020000in}}%
\pgfusepath{clip}%
\pgfsetbuttcap%
\pgfsetroundjoin%
\definecolor{currentfill}{rgb}{0.121569,0.466667,0.705882}%
\pgfsetfillcolor{currentfill}%
\pgfsetlinewidth{1.003750pt}%
\definecolor{currentstroke}{rgb}{0.121569,0.466667,0.705882}%
\pgfsetstrokecolor{currentstroke}%
\pgfsetdash{}{0pt}%
\pgfpathmoveto{\pgfqpoint{2.675436in}{2.225827in}}%
\pgfpathcurveto{\pgfqpoint{2.680645in}{2.225827in}}{\pgfqpoint{2.685641in}{2.227896in}}{\pgfqpoint{2.689325in}{2.231580in}}%
\pgfpathcurveto{\pgfqpoint{2.693008in}{2.235263in}}{\pgfqpoint{2.695078in}{2.240260in}}{\pgfqpoint{2.695078in}{2.245469in}}%
\pgfpathcurveto{\pgfqpoint{2.695078in}{2.250678in}}{\pgfqpoint{2.693008in}{2.255674in}}{\pgfqpoint{2.689325in}{2.259358in}}%
\pgfpathcurveto{\pgfqpoint{2.685641in}{2.263041in}}{\pgfqpoint{2.680645in}{2.265111in}}{\pgfqpoint{2.675436in}{2.265111in}}%
\pgfpathcurveto{\pgfqpoint{2.670227in}{2.265111in}}{\pgfqpoint{2.665230in}{2.263041in}}{\pgfqpoint{2.661547in}{2.259358in}}%
\pgfpathcurveto{\pgfqpoint{2.657864in}{2.255674in}}{\pgfqpoint{2.655794in}{2.250678in}}{\pgfqpoint{2.655794in}{2.245469in}}%
\pgfpathcurveto{\pgfqpoint{2.655794in}{2.240260in}}{\pgfqpoint{2.657864in}{2.235263in}}{\pgfqpoint{2.661547in}{2.231580in}}%
\pgfpathcurveto{\pgfqpoint{2.665230in}{2.227896in}}{\pgfqpoint{2.670227in}{2.225827in}}{\pgfqpoint{2.675436in}{2.225827in}}%
\pgfpathclose%
\pgfusepath{stroke,fill}%
\end{pgfscope}%
\begin{pgfscope}%
\pgfpathrectangle{\pgfqpoint{0.750000in}{0.500000in}}{\pgfqpoint{4.650000in}{3.020000in}}%
\pgfusepath{clip}%
\pgfsetbuttcap%
\pgfsetroundjoin%
\definecolor{currentfill}{rgb}{0.121569,0.466667,0.705882}%
\pgfsetfillcolor{currentfill}%
\pgfsetlinewidth{1.003750pt}%
\definecolor{currentstroke}{rgb}{0.121569,0.466667,0.705882}%
\pgfsetstrokecolor{currentstroke}%
\pgfsetdash{}{0pt}%
\pgfpathmoveto{\pgfqpoint{2.477101in}{0.790683in}}%
\pgfpathcurveto{\pgfqpoint{2.481613in}{0.790683in}}{\pgfqpoint{2.485940in}{0.792475in}}{\pgfqpoint{2.489130in}{0.795665in}}%
\pgfpathcurveto{\pgfqpoint{2.492320in}{0.798855in}}{\pgfqpoint{2.494112in}{0.803182in}}{\pgfqpoint{2.494112in}{0.807693in}}%
\pgfpathcurveto{\pgfqpoint{2.494112in}{0.812205in}}{\pgfqpoint{2.492320in}{0.816532in}}{\pgfqpoint{2.489130in}{0.819722in}}%
\pgfpathcurveto{\pgfqpoint{2.485940in}{0.822912in}}{\pgfqpoint{2.481613in}{0.824704in}}{\pgfqpoint{2.477101in}{0.824704in}}%
\pgfpathcurveto{\pgfqpoint{2.472590in}{0.824704in}}{\pgfqpoint{2.468263in}{0.822912in}}{\pgfqpoint{2.465073in}{0.819722in}}%
\pgfpathcurveto{\pgfqpoint{2.461883in}{0.816532in}}{\pgfqpoint{2.460091in}{0.812205in}}{\pgfqpoint{2.460091in}{0.807693in}}%
\pgfpathcurveto{\pgfqpoint{2.460091in}{0.803182in}}{\pgfqpoint{2.461883in}{0.798855in}}{\pgfqpoint{2.465073in}{0.795665in}}%
\pgfpathcurveto{\pgfqpoint{2.468263in}{0.792475in}}{\pgfqpoint{2.472590in}{0.790683in}}{\pgfqpoint{2.477101in}{0.790683in}}%
\pgfpathclose%
\pgfusepath{stroke,fill}%
\end{pgfscope}%
\begin{pgfscope}%
\pgfpathrectangle{\pgfqpoint{0.750000in}{0.500000in}}{\pgfqpoint{4.650000in}{3.020000in}}%
\pgfusepath{clip}%
\pgfsetbuttcap%
\pgfsetroundjoin%
\definecolor{currentfill}{rgb}{0.121569,0.466667,0.705882}%
\pgfsetfillcolor{currentfill}%
\pgfsetlinewidth{1.003750pt}%
\definecolor{currentstroke}{rgb}{0.121569,0.466667,0.705882}%
\pgfsetstrokecolor{currentstroke}%
\pgfsetdash{}{0pt}%
\pgfpathmoveto{\pgfqpoint{2.419194in}{0.891004in}}%
\pgfpathcurveto{\pgfqpoint{2.421798in}{0.891004in}}{\pgfqpoint{2.424296in}{0.892039in}}{\pgfqpoint{2.426138in}{0.893880in}}%
\pgfpathcurveto{\pgfqpoint{2.427980in}{0.895722in}}{\pgfqpoint{2.429015in}{0.898220in}}{\pgfqpoint{2.429015in}{0.900825in}}%
\pgfpathcurveto{\pgfqpoint{2.429015in}{0.903429in}}{\pgfqpoint{2.427980in}{0.905928in}}{\pgfqpoint{2.426138in}{0.907769in}}%
\pgfpathcurveto{\pgfqpoint{2.424296in}{0.909611in}}{\pgfqpoint{2.421798in}{0.910646in}}{\pgfqpoint{2.419194in}{0.910646in}}%
\pgfpathcurveto{\pgfqpoint{2.416589in}{0.910646in}}{\pgfqpoint{2.414091in}{0.909611in}}{\pgfqpoint{2.412249in}{0.907769in}}%
\pgfpathcurveto{\pgfqpoint{2.410408in}{0.905928in}}{\pgfqpoint{2.409373in}{0.903429in}}{\pgfqpoint{2.409373in}{0.900825in}}%
\pgfpathcurveto{\pgfqpoint{2.409373in}{0.898220in}}{\pgfqpoint{2.410408in}{0.895722in}}{\pgfqpoint{2.412249in}{0.893880in}}%
\pgfpathcurveto{\pgfqpoint{2.414091in}{0.892039in}}{\pgfqpoint{2.416589in}{0.891004in}}{\pgfqpoint{2.419194in}{0.891004in}}%
\pgfpathclose%
\pgfusepath{stroke,fill}%
\end{pgfscope}%
\begin{pgfscope}%
\pgfpathrectangle{\pgfqpoint{0.750000in}{0.500000in}}{\pgfqpoint{4.650000in}{3.020000in}}%
\pgfusepath{clip}%
\pgfsetbuttcap%
\pgfsetroundjoin%
\definecolor{currentfill}{rgb}{0.121569,0.466667,0.705882}%
\pgfsetfillcolor{currentfill}%
\pgfsetlinewidth{1.003750pt}%
\definecolor{currentstroke}{rgb}{0.121569,0.466667,0.705882}%
\pgfsetstrokecolor{currentstroke}%
\pgfsetdash{}{0pt}%
\pgfpathmoveto{\pgfqpoint{1.995019in}{0.954852in}}%
\pgfpathcurveto{\pgfqpoint{1.997623in}{0.954852in}}{\pgfqpoint{2.000121in}{0.955886in}}{\pgfqpoint{2.001963in}{0.957728in}}%
\pgfpathcurveto{\pgfqpoint{2.003805in}{0.959570in}}{\pgfqpoint{2.004840in}{0.962068in}}{\pgfqpoint{2.004840in}{0.964673in}}%
\pgfpathcurveto{\pgfqpoint{2.004840in}{0.967277in}}{\pgfqpoint{2.003805in}{0.969775in}}{\pgfqpoint{2.001963in}{0.971617in}}%
\pgfpathcurveto{\pgfqpoint{2.000121in}{0.973459in}}{\pgfqpoint{1.997623in}{0.974494in}}{\pgfqpoint{1.995019in}{0.974494in}}%
\pgfpathcurveto{\pgfqpoint{1.992414in}{0.974494in}}{\pgfqpoint{1.989916in}{0.973459in}}{\pgfqpoint{1.988074in}{0.971617in}}%
\pgfpathcurveto{\pgfqpoint{1.986233in}{0.969775in}}{\pgfqpoint{1.985198in}{0.967277in}}{\pgfqpoint{1.985198in}{0.964673in}}%
\pgfpathcurveto{\pgfqpoint{1.985198in}{0.962068in}}{\pgfqpoint{1.986233in}{0.959570in}}{\pgfqpoint{1.988074in}{0.957728in}}%
\pgfpathcurveto{\pgfqpoint{1.989916in}{0.955886in}}{\pgfqpoint{1.992414in}{0.954852in}}{\pgfqpoint{1.995019in}{0.954852in}}%
\pgfpathclose%
\pgfusepath{stroke,fill}%
\end{pgfscope}%
\begin{pgfscope}%
\pgfpathrectangle{\pgfqpoint{0.750000in}{0.500000in}}{\pgfqpoint{4.650000in}{3.020000in}}%
\pgfusepath{clip}%
\pgfsetbuttcap%
\pgfsetroundjoin%
\definecolor{currentfill}{rgb}{0.121569,0.466667,0.705882}%
\pgfsetfillcolor{currentfill}%
\pgfsetlinewidth{1.003750pt}%
\definecolor{currentstroke}{rgb}{0.121569,0.466667,0.705882}%
\pgfsetstrokecolor{currentstroke}%
\pgfsetdash{}{0pt}%
\pgfpathmoveto{\pgfqpoint{1.356585in}{0.696580in}}%
\pgfpathcurveto{\pgfqpoint{1.359189in}{0.696580in}}{\pgfqpoint{1.361687in}{0.697615in}}{\pgfqpoint{1.363529in}{0.699457in}}%
\pgfpathcurveto{\pgfqpoint{1.365371in}{0.701298in}}{\pgfqpoint{1.366406in}{0.703797in}}{\pgfqpoint{1.366406in}{0.706401in}}%
\pgfpathcurveto{\pgfqpoint{1.366406in}{0.709006in}}{\pgfqpoint{1.365371in}{0.711504in}}{\pgfqpoint{1.363529in}{0.713346in}}%
\pgfpathcurveto{\pgfqpoint{1.361687in}{0.715187in}}{\pgfqpoint{1.359189in}{0.716222in}}{\pgfqpoint{1.356585in}{0.716222in}}%
\pgfpathcurveto{\pgfqpoint{1.353980in}{0.716222in}}{\pgfqpoint{1.351482in}{0.715187in}}{\pgfqpoint{1.349640in}{0.713346in}}%
\pgfpathcurveto{\pgfqpoint{1.347799in}{0.711504in}}{\pgfqpoint{1.346764in}{0.709006in}}{\pgfqpoint{1.346764in}{0.706401in}}%
\pgfpathcurveto{\pgfqpoint{1.346764in}{0.703797in}}{\pgfqpoint{1.347799in}{0.701298in}}{\pgfqpoint{1.349640in}{0.699457in}}%
\pgfpathcurveto{\pgfqpoint{1.351482in}{0.697615in}}{\pgfqpoint{1.353980in}{0.696580in}}{\pgfqpoint{1.356585in}{0.696580in}}%
\pgfpathclose%
\pgfusepath{stroke,fill}%
\end{pgfscope}%
\begin{pgfscope}%
\pgfpathrectangle{\pgfqpoint{0.750000in}{0.500000in}}{\pgfqpoint{4.650000in}{3.020000in}}%
\pgfusepath{clip}%
\pgfsetbuttcap%
\pgfsetroundjoin%
\definecolor{currentfill}{rgb}{0.121569,0.466667,0.705882}%
\pgfsetfillcolor{currentfill}%
\pgfsetlinewidth{1.003750pt}%
\definecolor{currentstroke}{rgb}{0.121569,0.466667,0.705882}%
\pgfsetstrokecolor{currentstroke}%
\pgfsetdash{}{0pt}%
\pgfpathmoveto{\pgfqpoint{1.067045in}{0.637533in}}%
\pgfpathcurveto{\pgfqpoint{1.069650in}{0.637533in}}{\pgfqpoint{1.072148in}{0.638568in}}{\pgfqpoint{1.073990in}{0.640410in}}%
\pgfpathcurveto{\pgfqpoint{1.075832in}{0.642251in}}{\pgfqpoint{1.076866in}{0.644749in}}{\pgfqpoint{1.076866in}{0.647354in}}%
\pgfpathcurveto{\pgfqpoint{1.076866in}{0.649958in}}{\pgfqpoint{1.075832in}{0.652457in}}{\pgfqpoint{1.073990in}{0.654298in}}%
\pgfpathcurveto{\pgfqpoint{1.072148in}{0.656140in}}{\pgfqpoint{1.069650in}{0.657175in}}{\pgfqpoint{1.067045in}{0.657175in}}%
\pgfpathcurveto{\pgfqpoint{1.064441in}{0.657175in}}{\pgfqpoint{1.061943in}{0.656140in}}{\pgfqpoint{1.060101in}{0.654298in}}%
\pgfpathcurveto{\pgfqpoint{1.058259in}{0.652457in}}{\pgfqpoint{1.057225in}{0.649958in}}{\pgfqpoint{1.057225in}{0.647354in}}%
\pgfpathcurveto{\pgfqpoint{1.057225in}{0.644749in}}{\pgfqpoint{1.058259in}{0.642251in}}{\pgfqpoint{1.060101in}{0.640410in}}%
\pgfpathcurveto{\pgfqpoint{1.061943in}{0.638568in}}{\pgfqpoint{1.064441in}{0.637533in}}{\pgfqpoint{1.067045in}{0.637533in}}%
\pgfpathclose%
\pgfusepath{stroke,fill}%
\end{pgfscope}%
\begin{pgfscope}%
\pgfpathrectangle{\pgfqpoint{0.750000in}{0.500000in}}{\pgfqpoint{4.650000in}{3.020000in}}%
\pgfusepath{clip}%
\pgfsetbuttcap%
\pgfsetroundjoin%
\definecolor{currentfill}{rgb}{0.121569,0.466667,0.705882}%
\pgfsetfillcolor{currentfill}%
\pgfsetlinewidth{1.003750pt}%
\definecolor{currentstroke}{rgb}{0.121569,0.466667,0.705882}%
\pgfsetstrokecolor{currentstroke}%
\pgfsetdash{}{0pt}%
\pgfpathmoveto{\pgfqpoint{2.265738in}{0.746080in}}%
\pgfpathcurveto{\pgfqpoint{2.269856in}{0.746080in}}{\pgfqpoint{2.273806in}{0.747716in}}{\pgfqpoint{2.276718in}{0.750628in}}%
\pgfpathcurveto{\pgfqpoint{2.279630in}{0.753540in}}{\pgfqpoint{2.281266in}{0.757490in}}{\pgfqpoint{2.281266in}{0.761608in}}%
\pgfpathcurveto{\pgfqpoint{2.281266in}{0.765726in}}{\pgfqpoint{2.279630in}{0.769676in}}{\pgfqpoint{2.276718in}{0.772588in}}%
\pgfpathcurveto{\pgfqpoint{2.273806in}{0.775500in}}{\pgfqpoint{2.269856in}{0.777136in}}{\pgfqpoint{2.265738in}{0.777136in}}%
\pgfpathcurveto{\pgfqpoint{2.261620in}{0.777136in}}{\pgfqpoint{2.257670in}{0.775500in}}{\pgfqpoint{2.254758in}{0.772588in}}%
\pgfpathcurveto{\pgfqpoint{2.251846in}{0.769676in}}{\pgfqpoint{2.250210in}{0.765726in}}{\pgfqpoint{2.250210in}{0.761608in}}%
\pgfpathcurveto{\pgfqpoint{2.250210in}{0.757490in}}{\pgfqpoint{2.251846in}{0.753540in}}{\pgfqpoint{2.254758in}{0.750628in}}%
\pgfpathcurveto{\pgfqpoint{2.257670in}{0.747716in}}{\pgfqpoint{2.261620in}{0.746080in}}{\pgfqpoint{2.265738in}{0.746080in}}%
\pgfpathclose%
\pgfusepath{stroke,fill}%
\end{pgfscope}%
\begin{pgfscope}%
\pgfpathrectangle{\pgfqpoint{0.750000in}{0.500000in}}{\pgfqpoint{4.650000in}{3.020000in}}%
\pgfusepath{clip}%
\pgfsetbuttcap%
\pgfsetroundjoin%
\definecolor{currentfill}{rgb}{0.121569,0.466667,0.705882}%
\pgfsetfillcolor{currentfill}%
\pgfsetlinewidth{1.003750pt}%
\definecolor{currentstroke}{rgb}{0.121569,0.466667,0.705882}%
\pgfsetstrokecolor{currentstroke}%
\pgfsetdash{}{0pt}%
\pgfpathmoveto{\pgfqpoint{1.181413in}{0.640413in}}%
\pgfpathcurveto{\pgfqpoint{1.184018in}{0.640413in}}{\pgfqpoint{1.186516in}{0.641448in}}{\pgfqpoint{1.188358in}{0.643290in}}%
\pgfpathcurveto{\pgfqpoint{1.190200in}{0.645132in}}{\pgfqpoint{1.191234in}{0.647630in}}{\pgfqpoint{1.191234in}{0.650234in}}%
\pgfpathcurveto{\pgfqpoint{1.191234in}{0.652839in}}{\pgfqpoint{1.190200in}{0.655337in}}{\pgfqpoint{1.188358in}{0.657179in}}%
\pgfpathcurveto{\pgfqpoint{1.186516in}{0.659020in}}{\pgfqpoint{1.184018in}{0.660055in}}{\pgfqpoint{1.181413in}{0.660055in}}%
\pgfpathcurveto{\pgfqpoint{1.178809in}{0.660055in}}{\pgfqpoint{1.176311in}{0.659020in}}{\pgfqpoint{1.174469in}{0.657179in}}%
\pgfpathcurveto{\pgfqpoint{1.172627in}{0.655337in}}{\pgfqpoint{1.171593in}{0.652839in}}{\pgfqpoint{1.171593in}{0.650234in}}%
\pgfpathcurveto{\pgfqpoint{1.171593in}{0.647630in}}{\pgfqpoint{1.172627in}{0.645132in}}{\pgfqpoint{1.174469in}{0.643290in}}%
\pgfpathcurveto{\pgfqpoint{1.176311in}{0.641448in}}{\pgfqpoint{1.178809in}{0.640413in}}{\pgfqpoint{1.181413in}{0.640413in}}%
\pgfpathclose%
\pgfusepath{stroke,fill}%
\end{pgfscope}%
\begin{pgfscope}%
\pgfpathrectangle{\pgfqpoint{0.750000in}{0.500000in}}{\pgfqpoint{4.650000in}{3.020000in}}%
\pgfusepath{clip}%
\pgfsetbuttcap%
\pgfsetroundjoin%
\definecolor{currentfill}{rgb}{0.121569,0.466667,0.705882}%
\pgfsetfillcolor{currentfill}%
\pgfsetlinewidth{1.003750pt}%
\definecolor{currentstroke}{rgb}{0.121569,0.466667,0.705882}%
\pgfsetstrokecolor{currentstroke}%
\pgfsetdash{}{0pt}%
\pgfpathmoveto{\pgfqpoint{1.190100in}{0.634653in}}%
\pgfpathcurveto{\pgfqpoint{1.192704in}{0.634653in}}{\pgfqpoint{1.195202in}{0.635687in}}{\pgfqpoint{1.197044in}{0.637529in}}%
\pgfpathcurveto{\pgfqpoint{1.198886in}{0.639371in}}{\pgfqpoint{1.199921in}{0.641869in}}{\pgfqpoint{1.199921in}{0.644474in}}%
\pgfpathcurveto{\pgfqpoint{1.199921in}{0.647078in}}{\pgfqpoint{1.198886in}{0.649576in}}{\pgfqpoint{1.197044in}{0.651418in}}%
\pgfpathcurveto{\pgfqpoint{1.195202in}{0.653260in}}{\pgfqpoint{1.192704in}{0.654295in}}{\pgfqpoint{1.190100in}{0.654295in}}%
\pgfpathcurveto{\pgfqpoint{1.187495in}{0.654295in}}{\pgfqpoint{1.184997in}{0.653260in}}{\pgfqpoint{1.183155in}{0.651418in}}%
\pgfpathcurveto{\pgfqpoint{1.181313in}{0.649576in}}{\pgfqpoint{1.180279in}{0.647078in}}{\pgfqpoint{1.180279in}{0.644474in}}%
\pgfpathcurveto{\pgfqpoint{1.180279in}{0.641869in}}{\pgfqpoint{1.181313in}{0.639371in}}{\pgfqpoint{1.183155in}{0.637529in}}%
\pgfpathcurveto{\pgfqpoint{1.184997in}{0.635687in}}{\pgfqpoint{1.187495in}{0.634653in}}{\pgfqpoint{1.190100in}{0.634653in}}%
\pgfpathclose%
\pgfusepath{stroke,fill}%
\end{pgfscope}%
\begin{pgfscope}%
\pgfpathrectangle{\pgfqpoint{0.750000in}{0.500000in}}{\pgfqpoint{4.650000in}{3.020000in}}%
\pgfusepath{clip}%
\pgfsetbuttcap%
\pgfsetroundjoin%
\definecolor{currentfill}{rgb}{0.121569,0.466667,0.705882}%
\pgfsetfillcolor{currentfill}%
\pgfsetlinewidth{1.003750pt}%
\definecolor{currentstroke}{rgb}{0.121569,0.466667,0.705882}%
\pgfsetstrokecolor{currentstroke}%
\pgfsetdash{}{0pt}%
\pgfpathmoveto{\pgfqpoint{1.310258in}{0.679778in}}%
\pgfpathcurveto{\pgfqpoint{1.312863in}{0.679778in}}{\pgfqpoint{1.315361in}{0.680813in}}{\pgfqpoint{1.317203in}{0.682655in}}%
\pgfpathcurveto{\pgfqpoint{1.319045in}{0.684496in}}{\pgfqpoint{1.320079in}{0.686995in}}{\pgfqpoint{1.320079in}{0.689599in}}%
\pgfpathcurveto{\pgfqpoint{1.320079in}{0.692204in}}{\pgfqpoint{1.319045in}{0.694702in}}{\pgfqpoint{1.317203in}{0.696544in}}%
\pgfpathcurveto{\pgfqpoint{1.315361in}{0.698385in}}{\pgfqpoint{1.312863in}{0.699420in}}{\pgfqpoint{1.310258in}{0.699420in}}%
\pgfpathcurveto{\pgfqpoint{1.307654in}{0.699420in}}{\pgfqpoint{1.305156in}{0.698385in}}{\pgfqpoint{1.303314in}{0.696544in}}%
\pgfpathcurveto{\pgfqpoint{1.301472in}{0.694702in}}{\pgfqpoint{1.300437in}{0.692204in}}{\pgfqpoint{1.300437in}{0.689599in}}%
\pgfpathcurveto{\pgfqpoint{1.300437in}{0.686995in}}{\pgfqpoint{1.301472in}{0.684496in}}{\pgfqpoint{1.303314in}{0.682655in}}%
\pgfpathcurveto{\pgfqpoint{1.305156in}{0.680813in}}{\pgfqpoint{1.307654in}{0.679778in}}{\pgfqpoint{1.310258in}{0.679778in}}%
\pgfpathclose%
\pgfusepath{stroke,fill}%
\end{pgfscope}%
\begin{pgfscope}%
\pgfpathrectangle{\pgfqpoint{0.750000in}{0.500000in}}{\pgfqpoint{4.650000in}{3.020000in}}%
\pgfusepath{clip}%
\pgfsetbuttcap%
\pgfsetroundjoin%
\definecolor{currentfill}{rgb}{0.121569,0.466667,0.705882}%
\pgfsetfillcolor{currentfill}%
\pgfsetlinewidth{1.003750pt}%
\definecolor{currentstroke}{rgb}{0.121569,0.466667,0.705882}%
\pgfsetstrokecolor{currentstroke}%
\pgfsetdash{}{0pt}%
\pgfpathmoveto{\pgfqpoint{1.463714in}{0.682178in}}%
\pgfpathcurveto{\pgfqpoint{1.466319in}{0.682178in}}{\pgfqpoint{1.468817in}{0.683213in}}{\pgfqpoint{1.470659in}{0.685055in}}%
\pgfpathcurveto{\pgfqpoint{1.472500in}{0.686897in}}{\pgfqpoint{1.473535in}{0.689395in}}{\pgfqpoint{1.473535in}{0.691999in}}%
\pgfpathcurveto{\pgfqpoint{1.473535in}{0.694604in}}{\pgfqpoint{1.472500in}{0.697102in}}{\pgfqpoint{1.470659in}{0.698944in}}%
\pgfpathcurveto{\pgfqpoint{1.468817in}{0.700786in}}{\pgfqpoint{1.466319in}{0.701820in}}{\pgfqpoint{1.463714in}{0.701820in}}%
\pgfpathcurveto{\pgfqpoint{1.461110in}{0.701820in}}{\pgfqpoint{1.458611in}{0.700786in}}{\pgfqpoint{1.456770in}{0.698944in}}%
\pgfpathcurveto{\pgfqpoint{1.454928in}{0.697102in}}{\pgfqpoint{1.453893in}{0.694604in}}{\pgfqpoint{1.453893in}{0.691999in}}%
\pgfpathcurveto{\pgfqpoint{1.453893in}{0.689395in}}{\pgfqpoint{1.454928in}{0.686897in}}{\pgfqpoint{1.456770in}{0.685055in}}%
\pgfpathcurveto{\pgfqpoint{1.458611in}{0.683213in}}{\pgfqpoint{1.461110in}{0.682178in}}{\pgfqpoint{1.463714in}{0.682178in}}%
\pgfpathclose%
\pgfusepath{stroke,fill}%
\end{pgfscope}%
\begin{pgfscope}%
\pgfpathrectangle{\pgfqpoint{0.750000in}{0.500000in}}{\pgfqpoint{4.650000in}{3.020000in}}%
\pgfusepath{clip}%
\pgfsetbuttcap%
\pgfsetroundjoin%
\definecolor{currentfill}{rgb}{0.121569,0.466667,0.705882}%
\pgfsetfillcolor{currentfill}%
\pgfsetlinewidth{1.003750pt}%
\definecolor{currentstroke}{rgb}{0.121569,0.466667,0.705882}%
\pgfsetstrokecolor{currentstroke}%
\pgfsetdash{}{0pt}%
\pgfpathmoveto{\pgfqpoint{1.049673in}{1.230405in}}%
\pgfpathcurveto{\pgfqpoint{1.052278in}{1.230405in}}{\pgfqpoint{1.054776in}{1.231440in}}{\pgfqpoint{1.056618in}{1.233282in}}%
\pgfpathcurveto{\pgfqpoint{1.058459in}{1.235123in}}{\pgfqpoint{1.059494in}{1.237622in}}{\pgfqpoint{1.059494in}{1.240226in}}%
\pgfpathcurveto{\pgfqpoint{1.059494in}{1.242831in}}{\pgfqpoint{1.058459in}{1.245329in}}{\pgfqpoint{1.056618in}{1.247171in}}%
\pgfpathcurveto{\pgfqpoint{1.054776in}{1.249012in}}{\pgfqpoint{1.052278in}{1.250047in}}{\pgfqpoint{1.049673in}{1.250047in}}%
\pgfpathcurveto{\pgfqpoint{1.047069in}{1.250047in}}{\pgfqpoint{1.044570in}{1.249012in}}{\pgfqpoint{1.042729in}{1.247171in}}%
\pgfpathcurveto{\pgfqpoint{1.040887in}{1.245329in}}{\pgfqpoint{1.039852in}{1.242831in}}{\pgfqpoint{1.039852in}{1.240226in}}%
\pgfpathcurveto{\pgfqpoint{1.039852in}{1.237622in}}{\pgfqpoint{1.040887in}{1.235123in}}{\pgfqpoint{1.042729in}{1.233282in}}%
\pgfpathcurveto{\pgfqpoint{1.044570in}{1.231440in}}{\pgfqpoint{1.047069in}{1.230405in}}{\pgfqpoint{1.049673in}{1.230405in}}%
\pgfpathclose%
\pgfusepath{stroke,fill}%
\end{pgfscope}%
\begin{pgfscope}%
\pgfpathrectangle{\pgfqpoint{0.750000in}{0.500000in}}{\pgfqpoint{4.650000in}{3.020000in}}%
\pgfusepath{clip}%
\pgfsetbuttcap%
\pgfsetroundjoin%
\definecolor{currentfill}{rgb}{0.121569,0.466667,0.705882}%
\pgfsetfillcolor{currentfill}%
\pgfsetlinewidth{1.003750pt}%
\definecolor{currentstroke}{rgb}{0.121569,0.466667,0.705882}%
\pgfsetstrokecolor{currentstroke}%
\pgfsetdash{}{0pt}%
\pgfpathmoveto{\pgfqpoint{1.006242in}{0.630812in}}%
\pgfpathcurveto{\pgfqpoint{1.008847in}{0.630812in}}{\pgfqpoint{1.011345in}{0.631847in}}{\pgfqpoint{1.013187in}{0.633689in}}%
\pgfpathcurveto{\pgfqpoint{1.015028in}{0.635530in}}{\pgfqpoint{1.016063in}{0.638029in}}{\pgfqpoint{1.016063in}{0.640633in}}%
\pgfpathcurveto{\pgfqpoint{1.016063in}{0.643238in}}{\pgfqpoint{1.015028in}{0.645736in}}{\pgfqpoint{1.013187in}{0.647578in}}%
\pgfpathcurveto{\pgfqpoint{1.011345in}{0.649419in}}{\pgfqpoint{1.008847in}{0.650454in}}{\pgfqpoint{1.006242in}{0.650454in}}%
\pgfpathcurveto{\pgfqpoint{1.003638in}{0.650454in}}{\pgfqpoint{1.001139in}{0.649419in}}{\pgfqpoint{0.999298in}{0.647578in}}%
\pgfpathcurveto{\pgfqpoint{0.997456in}{0.645736in}}{\pgfqpoint{0.996421in}{0.643238in}}{\pgfqpoint{0.996421in}{0.640633in}}%
\pgfpathcurveto{\pgfqpoint{0.996421in}{0.638029in}}{\pgfqpoint{0.997456in}{0.635530in}}{\pgfqpoint{0.999298in}{0.633689in}}%
\pgfpathcurveto{\pgfqpoint{1.001139in}{0.631847in}}{\pgfqpoint{1.003638in}{0.630812in}}{\pgfqpoint{1.006242in}{0.630812in}}%
\pgfpathclose%
\pgfusepath{stroke,fill}%
\end{pgfscope}%
\begin{pgfscope}%
\pgfpathrectangle{\pgfqpoint{0.750000in}{0.500000in}}{\pgfqpoint{4.650000in}{3.020000in}}%
\pgfusepath{clip}%
\pgfsetbuttcap%
\pgfsetroundjoin%
\definecolor{currentfill}{rgb}{0.121569,0.466667,0.705882}%
\pgfsetfillcolor{currentfill}%
\pgfsetlinewidth{1.003750pt}%
\definecolor{currentstroke}{rgb}{0.121569,0.466667,0.705882}%
\pgfsetstrokecolor{currentstroke}%
\pgfsetdash{}{0pt}%
\pgfpathmoveto{\pgfqpoint{1.761940in}{0.722023in}}%
\pgfpathcurveto{\pgfqpoint{1.764544in}{0.722023in}}{\pgfqpoint{1.767042in}{0.723058in}}{\pgfqpoint{1.768884in}{0.724900in}}%
\pgfpathcurveto{\pgfqpoint{1.770726in}{0.726741in}}{\pgfqpoint{1.771761in}{0.729240in}}{\pgfqpoint{1.771761in}{0.731844in}}%
\pgfpathcurveto{\pgfqpoint{1.771761in}{0.734449in}}{\pgfqpoint{1.770726in}{0.736947in}}{\pgfqpoint{1.768884in}{0.738789in}}%
\pgfpathcurveto{\pgfqpoint{1.767042in}{0.740630in}}{\pgfqpoint{1.764544in}{0.741665in}}{\pgfqpoint{1.761940in}{0.741665in}}%
\pgfpathcurveto{\pgfqpoint{1.759335in}{0.741665in}}{\pgfqpoint{1.756837in}{0.740630in}}{\pgfqpoint{1.754995in}{0.738789in}}%
\pgfpathcurveto{\pgfqpoint{1.753153in}{0.736947in}}{\pgfqpoint{1.752119in}{0.734449in}}{\pgfqpoint{1.752119in}{0.731844in}}%
\pgfpathcurveto{\pgfqpoint{1.752119in}{0.729240in}}{\pgfqpoint{1.753153in}{0.726741in}}{\pgfqpoint{1.754995in}{0.724900in}}%
\pgfpathcurveto{\pgfqpoint{1.756837in}{0.723058in}}{\pgfqpoint{1.759335in}{0.722023in}}{\pgfqpoint{1.761940in}{0.722023in}}%
\pgfpathclose%
\pgfusepath{stroke,fill}%
\end{pgfscope}%
\begin{pgfscope}%
\pgfpathrectangle{\pgfqpoint{0.750000in}{0.500000in}}{\pgfqpoint{4.650000in}{3.020000in}}%
\pgfusepath{clip}%
\pgfsetbuttcap%
\pgfsetroundjoin%
\definecolor{currentfill}{rgb}{0.121569,0.466667,0.705882}%
\pgfsetfillcolor{currentfill}%
\pgfsetlinewidth{1.003750pt}%
\definecolor{currentstroke}{rgb}{0.121569,0.466667,0.705882}%
\pgfsetstrokecolor{currentstroke}%
\pgfsetdash{}{0pt}%
\pgfpathmoveto{\pgfqpoint{1.747463in}{1.083760in}}%
\pgfpathcurveto{\pgfqpoint{1.751146in}{1.083760in}}{\pgfqpoint{1.754679in}{1.085223in}}{\pgfqpoint{1.757284in}{1.087828in}}%
\pgfpathcurveto{\pgfqpoint{1.759888in}{1.090432in}}{\pgfqpoint{1.761352in}{1.093965in}}{\pgfqpoint{1.761352in}{1.097649in}}%
\pgfpathcurveto{\pgfqpoint{1.761352in}{1.101332in}}{\pgfqpoint{1.759888in}{1.104865in}}{\pgfqpoint{1.757284in}{1.107470in}}%
\pgfpathcurveto{\pgfqpoint{1.754679in}{1.110074in}}{\pgfqpoint{1.751146in}{1.111538in}}{\pgfqpoint{1.747463in}{1.111538in}}%
\pgfpathcurveto{\pgfqpoint{1.743779in}{1.111538in}}{\pgfqpoint{1.740246in}{1.110074in}}{\pgfqpoint{1.737642in}{1.107470in}}%
\pgfpathcurveto{\pgfqpoint{1.735037in}{1.104865in}}{\pgfqpoint{1.733574in}{1.101332in}}{\pgfqpoint{1.733574in}{1.097649in}}%
\pgfpathcurveto{\pgfqpoint{1.733574in}{1.093965in}}{\pgfqpoint{1.735037in}{1.090432in}}{\pgfqpoint{1.737642in}{1.087828in}}%
\pgfpathcurveto{\pgfqpoint{1.740246in}{1.085223in}}{\pgfqpoint{1.743779in}{1.083760in}}{\pgfqpoint{1.747463in}{1.083760in}}%
\pgfpathclose%
\pgfusepath{stroke,fill}%
\end{pgfscope}%
\begin{pgfscope}%
\pgfpathrectangle{\pgfqpoint{0.750000in}{0.500000in}}{\pgfqpoint{4.650000in}{3.020000in}}%
\pgfusepath{clip}%
\pgfsetbuttcap%
\pgfsetroundjoin%
\definecolor{currentfill}{rgb}{0.121569,0.466667,0.705882}%
\pgfsetfillcolor{currentfill}%
\pgfsetlinewidth{1.003750pt}%
\definecolor{currentstroke}{rgb}{0.121569,0.466667,0.705882}%
\pgfsetstrokecolor{currentstroke}%
\pgfsetdash{}{0pt}%
\pgfpathmoveto{\pgfqpoint{1.835772in}{0.859513in}}%
\pgfpathcurveto{\pgfqpoint{1.838962in}{0.859513in}}{\pgfqpoint{1.842022in}{0.860781in}}{\pgfqpoint{1.844277in}{0.863036in}}%
\pgfpathcurveto{\pgfqpoint{1.846533in}{0.865292in}}{\pgfqpoint{1.847800in}{0.868351in}}{\pgfqpoint{1.847800in}{0.871541in}}%
\pgfpathcurveto{\pgfqpoint{1.847800in}{0.874731in}}{\pgfqpoint{1.846533in}{0.877791in}}{\pgfqpoint{1.844277in}{0.880046in}}%
\pgfpathcurveto{\pgfqpoint{1.842022in}{0.882302in}}{\pgfqpoint{1.838962in}{0.883569in}}{\pgfqpoint{1.835772in}{0.883569in}}%
\pgfpathcurveto{\pgfqpoint{1.832582in}{0.883569in}}{\pgfqpoint{1.829523in}{0.882302in}}{\pgfqpoint{1.827267in}{0.880046in}}%
\pgfpathcurveto{\pgfqpoint{1.825011in}{0.877791in}}{\pgfqpoint{1.823744in}{0.874731in}}{\pgfqpoint{1.823744in}{0.871541in}}%
\pgfpathcurveto{\pgfqpoint{1.823744in}{0.868351in}}{\pgfqpoint{1.825011in}{0.865292in}}{\pgfqpoint{1.827267in}{0.863036in}}%
\pgfpathcurveto{\pgfqpoint{1.829523in}{0.860781in}}{\pgfqpoint{1.832582in}{0.859513in}}{\pgfqpoint{1.835772in}{0.859513in}}%
\pgfpathclose%
\pgfusepath{stroke,fill}%
\end{pgfscope}%
\begin{pgfscope}%
\pgfpathrectangle{\pgfqpoint{0.750000in}{0.500000in}}{\pgfqpoint{4.650000in}{3.020000in}}%
\pgfusepath{clip}%
\pgfsetbuttcap%
\pgfsetroundjoin%
\definecolor{currentfill}{rgb}{0.121569,0.466667,0.705882}%
\pgfsetfillcolor{currentfill}%
\pgfsetlinewidth{1.003750pt}%
\definecolor{currentstroke}{rgb}{0.121569,0.466667,0.705882}%
\pgfsetstrokecolor{currentstroke}%
\pgfsetdash{}{0pt}%
\pgfpathmoveto{\pgfqpoint{2.840473in}{3.345330in}}%
\pgfpathcurveto{\pgfqpoint{2.850391in}{3.345330in}}{\pgfqpoint{2.859904in}{3.349271in}}{\pgfqpoint{2.866917in}{3.356284in}}%
\pgfpathcurveto{\pgfqpoint{2.873930in}{3.363297in}}{\pgfqpoint{2.877870in}{3.372809in}}{\pgfqpoint{2.877870in}{3.382727in}}%
\pgfpathcurveto{\pgfqpoint{2.877870in}{3.392645in}}{\pgfqpoint{2.873930in}{3.402158in}}{\pgfqpoint{2.866917in}{3.409171in}}%
\pgfpathcurveto{\pgfqpoint{2.859904in}{3.416184in}}{\pgfqpoint{2.850391in}{3.420124in}}{\pgfqpoint{2.840473in}{3.420124in}}%
\pgfpathcurveto{\pgfqpoint{2.830555in}{3.420124in}}{\pgfqpoint{2.821043in}{3.416184in}}{\pgfqpoint{2.814030in}{3.409171in}}%
\pgfpathcurveto{\pgfqpoint{2.807017in}{3.402158in}}{\pgfqpoint{2.803076in}{3.392645in}}{\pgfqpoint{2.803076in}{3.382727in}}%
\pgfpathcurveto{\pgfqpoint{2.803076in}{3.372809in}}{\pgfqpoint{2.807017in}{3.363297in}}{\pgfqpoint{2.814030in}{3.356284in}}%
\pgfpathcurveto{\pgfqpoint{2.821043in}{3.349271in}}{\pgfqpoint{2.830555in}{3.345330in}}{\pgfqpoint{2.840473in}{3.345330in}}%
\pgfpathclose%
\pgfusepath{stroke,fill}%
\end{pgfscope}%
\begin{pgfscope}%
\pgfpathrectangle{\pgfqpoint{0.750000in}{0.500000in}}{\pgfqpoint{4.650000in}{3.020000in}}%
\pgfusepath{clip}%
\pgfsetbuttcap%
\pgfsetroundjoin%
\definecolor{currentfill}{rgb}{0.121569,0.466667,0.705882}%
\pgfsetfillcolor{currentfill}%
\pgfsetlinewidth{1.003750pt}%
\definecolor{currentstroke}{rgb}{0.121569,0.466667,0.705882}%
\pgfsetstrokecolor{currentstroke}%
\pgfsetdash{}{0pt}%
\pgfpathmoveto{\pgfqpoint{4.204203in}{2.830553in}}%
\pgfpathcurveto{\pgfqpoint{4.216963in}{2.830553in}}{\pgfqpoint{4.229201in}{2.835623in}}{\pgfqpoint{4.238224in}{2.844645in}}%
\pgfpathcurveto{\pgfqpoint{4.247246in}{2.853668in}}{\pgfqpoint{4.252316in}{2.865906in}}{\pgfqpoint{4.252316in}{2.878666in}}%
\pgfpathcurveto{\pgfqpoint{4.252316in}{2.891425in}}{\pgfqpoint{4.247246in}{2.903664in}}{\pgfqpoint{4.238224in}{2.912687in}}%
\pgfpathcurveto{\pgfqpoint{4.229201in}{2.921709in}}{\pgfqpoint{4.216963in}{2.926778in}}{\pgfqpoint{4.204203in}{2.926778in}}%
\pgfpathcurveto{\pgfqpoint{4.191443in}{2.926778in}}{\pgfqpoint{4.179205in}{2.921709in}}{\pgfqpoint{4.170182in}{2.912687in}}%
\pgfpathcurveto{\pgfqpoint{4.161160in}{2.903664in}}{\pgfqpoint{4.156090in}{2.891425in}}{\pgfqpoint{4.156090in}{2.878666in}}%
\pgfpathcurveto{\pgfqpoint{4.156090in}{2.865906in}}{\pgfqpoint{4.161160in}{2.853668in}}{\pgfqpoint{4.170182in}{2.844645in}}%
\pgfpathcurveto{\pgfqpoint{4.179205in}{2.835623in}}{\pgfqpoint{4.191443in}{2.830553in}}{\pgfqpoint{4.204203in}{2.830553in}}%
\pgfpathclose%
\pgfusepath{stroke,fill}%
\end{pgfscope}%
\begin{pgfscope}%
\pgfpathrectangle{\pgfqpoint{0.750000in}{0.500000in}}{\pgfqpoint{4.650000in}{3.020000in}}%
\pgfusepath{clip}%
\pgfsetbuttcap%
\pgfsetroundjoin%
\definecolor{currentfill}{rgb}{0.121569,0.466667,0.705882}%
\pgfsetfillcolor{currentfill}%
\pgfsetlinewidth{1.003750pt}%
\definecolor{currentstroke}{rgb}{0.121569,0.466667,0.705882}%
\pgfsetstrokecolor{currentstroke}%
\pgfsetdash{}{0pt}%
\pgfpathmoveto{\pgfqpoint{1.715613in}{1.796874in}}%
\pgfpathcurveto{\pgfqpoint{1.718218in}{1.796874in}}{\pgfqpoint{1.720716in}{1.797909in}}{\pgfqpoint{1.722558in}{1.799751in}}%
\pgfpathcurveto{\pgfqpoint{1.724399in}{1.801592in}}{\pgfqpoint{1.725434in}{1.804091in}}{\pgfqpoint{1.725434in}{1.806695in}}%
\pgfpathcurveto{\pgfqpoint{1.725434in}{1.809300in}}{\pgfqpoint{1.724399in}{1.811798in}}{\pgfqpoint{1.722558in}{1.813640in}}%
\pgfpathcurveto{\pgfqpoint{1.720716in}{1.815481in}}{\pgfqpoint{1.718218in}{1.816516in}}{\pgfqpoint{1.715613in}{1.816516in}}%
\pgfpathcurveto{\pgfqpoint{1.713009in}{1.816516in}}{\pgfqpoint{1.710511in}{1.815481in}}{\pgfqpoint{1.708669in}{1.813640in}}%
\pgfpathcurveto{\pgfqpoint{1.706827in}{1.811798in}}{\pgfqpoint{1.705792in}{1.809300in}}{\pgfqpoint{1.705792in}{1.806695in}}%
\pgfpathcurveto{\pgfqpoint{1.705792in}{1.804091in}}{\pgfqpoint{1.706827in}{1.801592in}}{\pgfqpoint{1.708669in}{1.799751in}}%
\pgfpathcurveto{\pgfqpoint{1.710511in}{1.797909in}}{\pgfqpoint{1.713009in}{1.796874in}}{\pgfqpoint{1.715613in}{1.796874in}}%
\pgfpathclose%
\pgfusepath{stroke,fill}%
\end{pgfscope}%
\begin{pgfscope}%
\pgfpathrectangle{\pgfqpoint{0.750000in}{0.500000in}}{\pgfqpoint{4.650000in}{3.020000in}}%
\pgfusepath{clip}%
\pgfsetbuttcap%
\pgfsetroundjoin%
\definecolor{currentfill}{rgb}{0.121569,0.466667,0.705882}%
\pgfsetfillcolor{currentfill}%
\pgfsetlinewidth{1.003750pt}%
\definecolor{currentstroke}{rgb}{0.121569,0.466667,0.705882}%
\pgfsetstrokecolor{currentstroke}%
\pgfsetdash{}{0pt}%
\pgfpathmoveto{\pgfqpoint{1.340660in}{0.675938in}}%
\pgfpathcurveto{\pgfqpoint{1.343265in}{0.675938in}}{\pgfqpoint{1.345763in}{0.676973in}}{\pgfqpoint{1.347604in}{0.678814in}}%
\pgfpathcurveto{\pgfqpoint{1.349446in}{0.680656in}}{\pgfqpoint{1.350481in}{0.683154in}}{\pgfqpoint{1.350481in}{0.685759in}}%
\pgfpathcurveto{\pgfqpoint{1.350481in}{0.688363in}}{\pgfqpoint{1.349446in}{0.690861in}}{\pgfqpoint{1.347604in}{0.692703in}}%
\pgfpathcurveto{\pgfqpoint{1.345763in}{0.694545in}}{\pgfqpoint{1.343265in}{0.695580in}}{\pgfqpoint{1.340660in}{0.695580in}}%
\pgfpathcurveto{\pgfqpoint{1.338055in}{0.695580in}}{\pgfqpoint{1.335557in}{0.694545in}}{\pgfqpoint{1.333716in}{0.692703in}}%
\pgfpathcurveto{\pgfqpoint{1.331874in}{0.690861in}}{\pgfqpoint{1.330839in}{0.688363in}}{\pgfqpoint{1.330839in}{0.685759in}}%
\pgfpathcurveto{\pgfqpoint{1.330839in}{0.683154in}}{\pgfqpoint{1.331874in}{0.680656in}}{\pgfqpoint{1.333716in}{0.678814in}}%
\pgfpathcurveto{\pgfqpoint{1.335557in}{0.676973in}}{\pgfqpoint{1.338055in}{0.675938in}}{\pgfqpoint{1.340660in}{0.675938in}}%
\pgfpathclose%
\pgfusepath{stroke,fill}%
\end{pgfscope}%
\begin{pgfscope}%
\pgfpathrectangle{\pgfqpoint{0.750000in}{0.500000in}}{\pgfqpoint{4.650000in}{3.020000in}}%
\pgfusepath{clip}%
\pgfsetbuttcap%
\pgfsetroundjoin%
\definecolor{currentfill}{rgb}{0.121569,0.466667,0.705882}%
\pgfsetfillcolor{currentfill}%
\pgfsetlinewidth{1.003750pt}%
\definecolor{currentstroke}{rgb}{0.121569,0.466667,0.705882}%
\pgfsetstrokecolor{currentstroke}%
\pgfsetdash{}{0pt}%
\pgfpathmoveto{\pgfqpoint{0.991765in}{0.888124in}}%
\pgfpathcurveto{\pgfqpoint{0.994370in}{0.888124in}}{\pgfqpoint{0.996868in}{0.889158in}}{\pgfqpoint{0.998710in}{0.891000in}}%
\pgfpathcurveto{\pgfqpoint{1.000551in}{0.892842in}}{\pgfqpoint{1.001586in}{0.895340in}}{\pgfqpoint{1.001586in}{0.897944in}}%
\pgfpathcurveto{\pgfqpoint{1.001586in}{0.900549in}}{\pgfqpoint{1.000551in}{0.903047in}}{\pgfqpoint{0.998710in}{0.904889in}}%
\pgfpathcurveto{\pgfqpoint{0.996868in}{0.906731in}}{\pgfqpoint{0.994370in}{0.907765in}}{\pgfqpoint{0.991765in}{0.907765in}}%
\pgfpathcurveto{\pgfqpoint{0.989161in}{0.907765in}}{\pgfqpoint{0.986662in}{0.906731in}}{\pgfqpoint{0.984821in}{0.904889in}}%
\pgfpathcurveto{\pgfqpoint{0.982979in}{0.903047in}}{\pgfqpoint{0.981944in}{0.900549in}}{\pgfqpoint{0.981944in}{0.897944in}}%
\pgfpathcurveto{\pgfqpoint{0.981944in}{0.895340in}}{\pgfqpoint{0.982979in}{0.892842in}}{\pgfqpoint{0.984821in}{0.891000in}}%
\pgfpathcurveto{\pgfqpoint{0.986662in}{0.889158in}}{\pgfqpoint{0.989161in}{0.888124in}}{\pgfqpoint{0.991765in}{0.888124in}}%
\pgfpathclose%
\pgfusepath{stroke,fill}%
\end{pgfscope}%
\begin{pgfscope}%
\pgfpathrectangle{\pgfqpoint{0.750000in}{0.500000in}}{\pgfqpoint{4.650000in}{3.020000in}}%
\pgfusepath{clip}%
\pgfsetbuttcap%
\pgfsetroundjoin%
\definecolor{currentfill}{rgb}{0.121569,0.466667,0.705882}%
\pgfsetfillcolor{currentfill}%
\pgfsetlinewidth{1.003750pt}%
\definecolor{currentstroke}{rgb}{0.121569,0.466667,0.705882}%
\pgfsetstrokecolor{currentstroke}%
\pgfsetdash{}{0pt}%
\pgfpathmoveto{\pgfqpoint{2.168742in}{1.283464in}}%
\pgfpathcurveto{\pgfqpoint{2.172426in}{1.283464in}}{\pgfqpoint{2.175959in}{1.284928in}}{\pgfqpoint{2.178563in}{1.287532in}}%
\pgfpathcurveto{\pgfqpoint{2.181168in}{1.290137in}}{\pgfqpoint{2.182631in}{1.293670in}}{\pgfqpoint{2.182631in}{1.297353in}}%
\pgfpathcurveto{\pgfqpoint{2.182631in}{1.301037in}}{\pgfqpoint{2.181168in}{1.304570in}}{\pgfqpoint{2.178563in}{1.307174in}}%
\pgfpathcurveto{\pgfqpoint{2.175959in}{1.309779in}}{\pgfqpoint{2.172426in}{1.311242in}}{\pgfqpoint{2.168742in}{1.311242in}}%
\pgfpathcurveto{\pgfqpoint{2.165059in}{1.311242in}}{\pgfqpoint{2.161526in}{1.309779in}}{\pgfqpoint{2.158921in}{1.307174in}}%
\pgfpathcurveto{\pgfqpoint{2.156317in}{1.304570in}}{\pgfqpoint{2.154853in}{1.301037in}}{\pgfqpoint{2.154853in}{1.297353in}}%
\pgfpathcurveto{\pgfqpoint{2.154853in}{1.293670in}}{\pgfqpoint{2.156317in}{1.290137in}}{\pgfqpoint{2.158921in}{1.287532in}}%
\pgfpathcurveto{\pgfqpoint{2.161526in}{1.284928in}}{\pgfqpoint{2.165059in}{1.283464in}}{\pgfqpoint{2.168742in}{1.283464in}}%
\pgfpathclose%
\pgfusepath{stroke,fill}%
\end{pgfscope}%
\begin{pgfscope}%
\pgfpathrectangle{\pgfqpoint{0.750000in}{0.500000in}}{\pgfqpoint{4.650000in}{3.020000in}}%
\pgfusepath{clip}%
\pgfsetbuttcap%
\pgfsetroundjoin%
\definecolor{currentfill}{rgb}{0.121569,0.466667,0.705882}%
\pgfsetfillcolor{currentfill}%
\pgfsetlinewidth{1.003750pt}%
\definecolor{currentstroke}{rgb}{0.121569,0.466667,0.705882}%
\pgfsetstrokecolor{currentstroke}%
\pgfsetdash{}{0pt}%
\pgfpathmoveto{\pgfqpoint{1.478191in}{0.731818in}}%
\pgfpathcurveto{\pgfqpoint{1.481381in}{0.731818in}}{\pgfqpoint{1.484441in}{0.733085in}}{\pgfqpoint{1.486696in}{0.735341in}}%
\pgfpathcurveto{\pgfqpoint{1.488952in}{0.737596in}}{\pgfqpoint{1.490219in}{0.740656in}}{\pgfqpoint{1.490219in}{0.743846in}}%
\pgfpathcurveto{\pgfqpoint{1.490219in}{0.747036in}}{\pgfqpoint{1.488952in}{0.750095in}}{\pgfqpoint{1.486696in}{0.752351in}}%
\pgfpathcurveto{\pgfqpoint{1.484441in}{0.754606in}}{\pgfqpoint{1.481381in}{0.755874in}}{\pgfqpoint{1.478191in}{0.755874in}}%
\pgfpathcurveto{\pgfqpoint{1.475001in}{0.755874in}}{\pgfqpoint{1.471942in}{0.754606in}}{\pgfqpoint{1.469686in}{0.752351in}}%
\pgfpathcurveto{\pgfqpoint{1.467430in}{0.750095in}}{\pgfqpoint{1.466163in}{0.747036in}}{\pgfqpoint{1.466163in}{0.743846in}}%
\pgfpathcurveto{\pgfqpoint{1.466163in}{0.740656in}}{\pgfqpoint{1.467430in}{0.737596in}}{\pgfqpoint{1.469686in}{0.735341in}}%
\pgfpathcurveto{\pgfqpoint{1.471942in}{0.733085in}}{\pgfqpoint{1.475001in}{0.731818in}}{\pgfqpoint{1.478191in}{0.731818in}}%
\pgfpathclose%
\pgfusepath{stroke,fill}%
\end{pgfscope}%
\begin{pgfscope}%
\pgfpathrectangle{\pgfqpoint{0.750000in}{0.500000in}}{\pgfqpoint{4.650000in}{3.020000in}}%
\pgfusepath{clip}%
\pgfsetbuttcap%
\pgfsetroundjoin%
\definecolor{currentfill}{rgb}{0.121569,0.466667,0.705882}%
\pgfsetfillcolor{currentfill}%
\pgfsetlinewidth{1.003750pt}%
\definecolor{currentstroke}{rgb}{0.121569,0.466667,0.705882}%
\pgfsetstrokecolor{currentstroke}%
\pgfsetdash{}{0pt}%
\pgfpathmoveto{\pgfqpoint{1.178518in}{0.791152in}}%
\pgfpathcurveto{\pgfqpoint{1.181123in}{0.791152in}}{\pgfqpoint{1.183621in}{0.792187in}}{\pgfqpoint{1.185463in}{0.794028in}}%
\pgfpathcurveto{\pgfqpoint{1.187304in}{0.795870in}}{\pgfqpoint{1.188339in}{0.798368in}}{\pgfqpoint{1.188339in}{0.800973in}}%
\pgfpathcurveto{\pgfqpoint{1.188339in}{0.803577in}}{\pgfqpoint{1.187304in}{0.806075in}}{\pgfqpoint{1.185463in}{0.807917in}}%
\pgfpathcurveto{\pgfqpoint{1.183621in}{0.809759in}}{\pgfqpoint{1.181123in}{0.810794in}}{\pgfqpoint{1.178518in}{0.810794in}}%
\pgfpathcurveto{\pgfqpoint{1.175914in}{0.810794in}}{\pgfqpoint{1.173415in}{0.809759in}}{\pgfqpoint{1.171574in}{0.807917in}}%
\pgfpathcurveto{\pgfqpoint{1.169732in}{0.806075in}}{\pgfqpoint{1.168697in}{0.803577in}}{\pgfqpoint{1.168697in}{0.800973in}}%
\pgfpathcurveto{\pgfqpoint{1.168697in}{0.798368in}}{\pgfqpoint{1.169732in}{0.795870in}}{\pgfqpoint{1.171574in}{0.794028in}}%
\pgfpathcurveto{\pgfqpoint{1.173415in}{0.792187in}}{\pgfqpoint{1.175914in}{0.791152in}}{\pgfqpoint{1.178518in}{0.791152in}}%
\pgfpathclose%
\pgfusepath{stroke,fill}%
\end{pgfscope}%
\begin{pgfscope}%
\pgfpathrectangle{\pgfqpoint{0.750000in}{0.500000in}}{\pgfqpoint{4.650000in}{3.020000in}}%
\pgfusepath{clip}%
\pgfsetbuttcap%
\pgfsetroundjoin%
\definecolor{currentfill}{rgb}{0.121569,0.466667,0.705882}%
\pgfsetfillcolor{currentfill}%
\pgfsetlinewidth{1.003750pt}%
\definecolor{currentstroke}{rgb}{0.121569,0.466667,0.705882}%
\pgfsetstrokecolor{currentstroke}%
\pgfsetdash{}{0pt}%
\pgfpathmoveto{\pgfqpoint{1.217606in}{1.282732in}}%
\pgfpathcurveto{\pgfqpoint{1.220210in}{1.282732in}}{\pgfqpoint{1.222709in}{1.283766in}}{\pgfqpoint{1.224550in}{1.285608in}}%
\pgfpathcurveto{\pgfqpoint{1.226392in}{1.287450in}}{\pgfqpoint{1.227427in}{1.289948in}}{\pgfqpoint{1.227427in}{1.292553in}}%
\pgfpathcurveto{\pgfqpoint{1.227427in}{1.295157in}}{\pgfqpoint{1.226392in}{1.297655in}}{\pgfqpoint{1.224550in}{1.299497in}}%
\pgfpathcurveto{\pgfqpoint{1.222709in}{1.301339in}}{\pgfqpoint{1.220210in}{1.302374in}}{\pgfqpoint{1.217606in}{1.302374in}}%
\pgfpathcurveto{\pgfqpoint{1.215001in}{1.302374in}}{\pgfqpoint{1.212503in}{1.301339in}}{\pgfqpoint{1.210661in}{1.299497in}}%
\pgfpathcurveto{\pgfqpoint{1.208820in}{1.297655in}}{\pgfqpoint{1.207785in}{1.295157in}}{\pgfqpoint{1.207785in}{1.292553in}}%
\pgfpathcurveto{\pgfqpoint{1.207785in}{1.289948in}}{\pgfqpoint{1.208820in}{1.287450in}}{\pgfqpoint{1.210661in}{1.285608in}}%
\pgfpathcurveto{\pgfqpoint{1.212503in}{1.283766in}}{\pgfqpoint{1.215001in}{1.282732in}}{\pgfqpoint{1.217606in}{1.282732in}}%
\pgfpathclose%
\pgfusepath{stroke,fill}%
\end{pgfscope}%
\begin{pgfscope}%
\pgfpathrectangle{\pgfqpoint{0.750000in}{0.500000in}}{\pgfqpoint{4.650000in}{3.020000in}}%
\pgfusepath{clip}%
\pgfsetbuttcap%
\pgfsetroundjoin%
\definecolor{currentfill}{rgb}{0.121569,0.466667,0.705882}%
\pgfsetfillcolor{currentfill}%
\pgfsetlinewidth{1.003750pt}%
\definecolor{currentstroke}{rgb}{0.121569,0.466667,0.705882}%
\pgfsetstrokecolor{currentstroke}%
\pgfsetdash{}{0pt}%
\pgfpathmoveto{\pgfqpoint{1.198786in}{0.726344in}}%
\pgfpathcurveto{\pgfqpoint{1.201390in}{0.726344in}}{\pgfqpoint{1.203889in}{0.727379in}}{\pgfqpoint{1.205730in}{0.729220in}}%
\pgfpathcurveto{\pgfqpoint{1.207572in}{0.731062in}}{\pgfqpoint{1.208607in}{0.733560in}}{\pgfqpoint{1.208607in}{0.736165in}}%
\pgfpathcurveto{\pgfqpoint{1.208607in}{0.738769in}}{\pgfqpoint{1.207572in}{0.741268in}}{\pgfqpoint{1.205730in}{0.743109in}}%
\pgfpathcurveto{\pgfqpoint{1.203889in}{0.744951in}}{\pgfqpoint{1.201390in}{0.745986in}}{\pgfqpoint{1.198786in}{0.745986in}}%
\pgfpathcurveto{\pgfqpoint{1.196181in}{0.745986in}}{\pgfqpoint{1.193683in}{0.744951in}}{\pgfqpoint{1.191841in}{0.743109in}}%
\pgfpathcurveto{\pgfqpoint{1.190000in}{0.741268in}}{\pgfqpoint{1.188965in}{0.738769in}}{\pgfqpoint{1.188965in}{0.736165in}}%
\pgfpathcurveto{\pgfqpoint{1.188965in}{0.733560in}}{\pgfqpoint{1.190000in}{0.731062in}}{\pgfqpoint{1.191841in}{0.729220in}}%
\pgfpathcurveto{\pgfqpoint{1.193683in}{0.727379in}}{\pgfqpoint{1.196181in}{0.726344in}}{\pgfqpoint{1.198786in}{0.726344in}}%
\pgfpathclose%
\pgfusepath{stroke,fill}%
\end{pgfscope}%
\begin{pgfscope}%
\pgfpathrectangle{\pgfqpoint{0.750000in}{0.500000in}}{\pgfqpoint{4.650000in}{3.020000in}}%
\pgfusepath{clip}%
\pgfsetbuttcap%
\pgfsetroundjoin%
\definecolor{currentfill}{rgb}{0.121569,0.466667,0.705882}%
\pgfsetfillcolor{currentfill}%
\pgfsetlinewidth{1.003750pt}%
\definecolor{currentstroke}{rgb}{0.121569,0.466667,0.705882}%
\pgfsetstrokecolor{currentstroke}%
\pgfsetdash{}{0pt}%
\pgfpathmoveto{\pgfqpoint{1.443446in}{0.732105in}}%
\pgfpathcurveto{\pgfqpoint{1.446051in}{0.732105in}}{\pgfqpoint{1.448549in}{0.733139in}}{\pgfqpoint{1.450391in}{0.734981in}}%
\pgfpathcurveto{\pgfqpoint{1.452233in}{0.736823in}}{\pgfqpoint{1.453267in}{0.739321in}}{\pgfqpoint{1.453267in}{0.741925in}}%
\pgfpathcurveto{\pgfqpoint{1.453267in}{0.744530in}}{\pgfqpoint{1.452233in}{0.747028in}}{\pgfqpoint{1.450391in}{0.748870in}}%
\pgfpathcurveto{\pgfqpoint{1.448549in}{0.750712in}}{\pgfqpoint{1.446051in}{0.751746in}}{\pgfqpoint{1.443446in}{0.751746in}}%
\pgfpathcurveto{\pgfqpoint{1.440842in}{0.751746in}}{\pgfqpoint{1.438344in}{0.750712in}}{\pgfqpoint{1.436502in}{0.748870in}}%
\pgfpathcurveto{\pgfqpoint{1.434660in}{0.747028in}}{\pgfqpoint{1.433626in}{0.744530in}}{\pgfqpoint{1.433626in}{0.741925in}}%
\pgfpathcurveto{\pgfqpoint{1.433626in}{0.739321in}}{\pgfqpoint{1.434660in}{0.736823in}}{\pgfqpoint{1.436502in}{0.734981in}}%
\pgfpathcurveto{\pgfqpoint{1.438344in}{0.733139in}}{\pgfqpoint{1.440842in}{0.732105in}}{\pgfqpoint{1.443446in}{0.732105in}}%
\pgfpathclose%
\pgfusepath{stroke,fill}%
\end{pgfscope}%
\begin{pgfscope}%
\pgfpathrectangle{\pgfqpoint{0.750000in}{0.500000in}}{\pgfqpoint{4.650000in}{3.020000in}}%
\pgfusepath{clip}%
\pgfsetbuttcap%
\pgfsetroundjoin%
\definecolor{currentfill}{rgb}{0.121569,0.466667,0.705882}%
\pgfsetfillcolor{currentfill}%
\pgfsetlinewidth{1.003750pt}%
\definecolor{currentstroke}{rgb}{0.121569,0.466667,0.705882}%
\pgfsetstrokecolor{currentstroke}%
\pgfsetdash{}{0pt}%
\pgfpathmoveto{\pgfqpoint{1.237874in}{0.689859in}}%
\pgfpathcurveto{\pgfqpoint{1.240478in}{0.689859in}}{\pgfqpoint{1.242976in}{0.690894in}}{\pgfqpoint{1.244818in}{0.692736in}}%
\pgfpathcurveto{\pgfqpoint{1.246660in}{0.694578in}}{\pgfqpoint{1.247695in}{0.697076in}}{\pgfqpoint{1.247695in}{0.699680in}}%
\pgfpathcurveto{\pgfqpoint{1.247695in}{0.702285in}}{\pgfqpoint{1.246660in}{0.704783in}}{\pgfqpoint{1.244818in}{0.706625in}}%
\pgfpathcurveto{\pgfqpoint{1.242976in}{0.708466in}}{\pgfqpoint{1.240478in}{0.709501in}}{\pgfqpoint{1.237874in}{0.709501in}}%
\pgfpathcurveto{\pgfqpoint{1.235269in}{0.709501in}}{\pgfqpoint{1.232771in}{0.708466in}}{\pgfqpoint{1.230929in}{0.706625in}}%
\pgfpathcurveto{\pgfqpoint{1.229087in}{0.704783in}}{\pgfqpoint{1.228053in}{0.702285in}}{\pgfqpoint{1.228053in}{0.699680in}}%
\pgfpathcurveto{\pgfqpoint{1.228053in}{0.697076in}}{\pgfqpoint{1.229087in}{0.694578in}}{\pgfqpoint{1.230929in}{0.692736in}}%
\pgfpathcurveto{\pgfqpoint{1.232771in}{0.690894in}}{\pgfqpoint{1.235269in}{0.689859in}}{\pgfqpoint{1.237874in}{0.689859in}}%
\pgfpathclose%
\pgfusepath{stroke,fill}%
\end{pgfscope}%
\begin{pgfscope}%
\pgfpathrectangle{\pgfqpoint{0.750000in}{0.500000in}}{\pgfqpoint{4.650000in}{3.020000in}}%
\pgfusepath{clip}%
\pgfsetbuttcap%
\pgfsetroundjoin%
\definecolor{currentfill}{rgb}{0.121569,0.466667,0.705882}%
\pgfsetfillcolor{currentfill}%
\pgfsetlinewidth{1.003750pt}%
\definecolor{currentstroke}{rgb}{0.121569,0.466667,0.705882}%
\pgfsetstrokecolor{currentstroke}%
\pgfsetdash{}{0pt}%
\pgfpathmoveto{\pgfqpoint{1.084418in}{0.645214in}}%
\pgfpathcurveto{\pgfqpoint{1.087022in}{0.645214in}}{\pgfqpoint{1.089521in}{0.646249in}}{\pgfqpoint{1.091362in}{0.648090in}}%
\pgfpathcurveto{\pgfqpoint{1.093204in}{0.649932in}}{\pgfqpoint{1.094239in}{0.652430in}}{\pgfqpoint{1.094239in}{0.655035in}}%
\pgfpathcurveto{\pgfqpoint{1.094239in}{0.657639in}}{\pgfqpoint{1.093204in}{0.660138in}}{\pgfqpoint{1.091362in}{0.661979in}}%
\pgfpathcurveto{\pgfqpoint{1.089521in}{0.663821in}}{\pgfqpoint{1.087022in}{0.664856in}}{\pgfqpoint{1.084418in}{0.664856in}}%
\pgfpathcurveto{\pgfqpoint{1.081813in}{0.664856in}}{\pgfqpoint{1.079315in}{0.663821in}}{\pgfqpoint{1.077473in}{0.661979in}}%
\pgfpathcurveto{\pgfqpoint{1.075632in}{0.660138in}}{\pgfqpoint{1.074597in}{0.657639in}}{\pgfqpoint{1.074597in}{0.655035in}}%
\pgfpathcurveto{\pgfqpoint{1.074597in}{0.652430in}}{\pgfqpoint{1.075632in}{0.649932in}}{\pgfqpoint{1.077473in}{0.648090in}}%
\pgfpathcurveto{\pgfqpoint{1.079315in}{0.646249in}}{\pgfqpoint{1.081813in}{0.645214in}}{\pgfqpoint{1.084418in}{0.645214in}}%
\pgfpathclose%
\pgfusepath{stroke,fill}%
\end{pgfscope}%
\begin{pgfscope}%
\pgfpathrectangle{\pgfqpoint{0.750000in}{0.500000in}}{\pgfqpoint{4.650000in}{3.020000in}}%
\pgfusepath{clip}%
\pgfsetbuttcap%
\pgfsetroundjoin%
\definecolor{currentfill}{rgb}{0.121569,0.466667,0.705882}%
\pgfsetfillcolor{currentfill}%
\pgfsetlinewidth{1.003750pt}%
\definecolor{currentstroke}{rgb}{0.121569,0.466667,0.705882}%
\pgfsetstrokecolor{currentstroke}%
\pgfsetdash{}{0pt}%
\pgfpathmoveto{\pgfqpoint{1.113372in}{0.646174in}}%
\pgfpathcurveto{\pgfqpoint{1.115976in}{0.646174in}}{\pgfqpoint{1.118474in}{0.647209in}}{\pgfqpoint{1.120316in}{0.649051in}}%
\pgfpathcurveto{\pgfqpoint{1.122158in}{0.650892in}}{\pgfqpoint{1.123193in}{0.653390in}}{\pgfqpoint{1.123193in}{0.655995in}}%
\pgfpathcurveto{\pgfqpoint{1.123193in}{0.658600in}}{\pgfqpoint{1.122158in}{0.661098in}}{\pgfqpoint{1.120316in}{0.662939in}}%
\pgfpathcurveto{\pgfqpoint{1.118474in}{0.664781in}}{\pgfqpoint{1.115976in}{0.665816in}}{\pgfqpoint{1.113372in}{0.665816in}}%
\pgfpathcurveto{\pgfqpoint{1.110767in}{0.665816in}}{\pgfqpoint{1.108269in}{0.664781in}}{\pgfqpoint{1.106427in}{0.662939in}}%
\pgfpathcurveto{\pgfqpoint{1.104586in}{0.661098in}}{\pgfqpoint{1.103551in}{0.658600in}}{\pgfqpoint{1.103551in}{0.655995in}}%
\pgfpathcurveto{\pgfqpoint{1.103551in}{0.653390in}}{\pgfqpoint{1.104586in}{0.650892in}}{\pgfqpoint{1.106427in}{0.649051in}}%
\pgfpathcurveto{\pgfqpoint{1.108269in}{0.647209in}}{\pgfqpoint{1.110767in}{0.646174in}}{\pgfqpoint{1.113372in}{0.646174in}}%
\pgfpathclose%
\pgfusepath{stroke,fill}%
\end{pgfscope}%
\begin{pgfscope}%
\pgfpathrectangle{\pgfqpoint{0.750000in}{0.500000in}}{\pgfqpoint{4.650000in}{3.020000in}}%
\pgfusepath{clip}%
\pgfsetbuttcap%
\pgfsetroundjoin%
\definecolor{currentfill}{rgb}{0.121569,0.466667,0.705882}%
\pgfsetfillcolor{currentfill}%
\pgfsetlinewidth{1.003750pt}%
\definecolor{currentstroke}{rgb}{0.121569,0.466667,0.705882}%
\pgfsetstrokecolor{currentstroke}%
\pgfsetdash{}{0pt}%
\pgfpathmoveto{\pgfqpoint{1.107581in}{0.656735in}}%
\pgfpathcurveto{\pgfqpoint{1.110185in}{0.656735in}}{\pgfqpoint{1.112684in}{0.657770in}}{\pgfqpoint{1.114525in}{0.659612in}}%
\pgfpathcurveto{\pgfqpoint{1.116367in}{0.661454in}}{\pgfqpoint{1.117402in}{0.663952in}}{\pgfqpoint{1.117402in}{0.666556in}}%
\pgfpathcurveto{\pgfqpoint{1.117402in}{0.669161in}}{\pgfqpoint{1.116367in}{0.671659in}}{\pgfqpoint{1.114525in}{0.673501in}}%
\pgfpathcurveto{\pgfqpoint{1.112684in}{0.675342in}}{\pgfqpoint{1.110185in}{0.676377in}}{\pgfqpoint{1.107581in}{0.676377in}}%
\pgfpathcurveto{\pgfqpoint{1.104976in}{0.676377in}}{\pgfqpoint{1.102478in}{0.675342in}}{\pgfqpoint{1.100637in}{0.673501in}}%
\pgfpathcurveto{\pgfqpoint{1.098795in}{0.671659in}}{\pgfqpoint{1.097760in}{0.669161in}}{\pgfqpoint{1.097760in}{0.666556in}}%
\pgfpathcurveto{\pgfqpoint{1.097760in}{0.663952in}}{\pgfqpoint{1.098795in}{0.661454in}}{\pgfqpoint{1.100637in}{0.659612in}}%
\pgfpathcurveto{\pgfqpoint{1.102478in}{0.657770in}}{\pgfqpoint{1.104976in}{0.656735in}}{\pgfqpoint{1.107581in}{0.656735in}}%
\pgfpathclose%
\pgfusepath{stroke,fill}%
\end{pgfscope}%
\begin{pgfscope}%
\pgfpathrectangle{\pgfqpoint{0.750000in}{0.500000in}}{\pgfqpoint{4.650000in}{3.020000in}}%
\pgfusepath{clip}%
\pgfsetbuttcap%
\pgfsetroundjoin%
\definecolor{currentfill}{rgb}{0.121569,0.466667,0.705882}%
\pgfsetfillcolor{currentfill}%
\pgfsetlinewidth{1.003750pt}%
\definecolor{currentstroke}{rgb}{0.121569,0.466667,0.705882}%
\pgfsetstrokecolor{currentstroke}%
\pgfsetdash{}{0pt}%
\pgfpathmoveto{\pgfqpoint{1.195890in}{1.151676in}}%
\pgfpathcurveto{\pgfqpoint{1.198495in}{1.151676in}}{\pgfqpoint{1.200993in}{1.152710in}}{\pgfqpoint{1.202835in}{1.154552in}}%
\pgfpathcurveto{\pgfqpoint{1.204677in}{1.156394in}}{\pgfqpoint{1.205711in}{1.158892in}}{\pgfqpoint{1.205711in}{1.161497in}}%
\pgfpathcurveto{\pgfqpoint{1.205711in}{1.164101in}}{\pgfqpoint{1.204677in}{1.166599in}}{\pgfqpoint{1.202835in}{1.168441in}}%
\pgfpathcurveto{\pgfqpoint{1.200993in}{1.170283in}}{\pgfqpoint{1.198495in}{1.171318in}}{\pgfqpoint{1.195890in}{1.171318in}}%
\pgfpathcurveto{\pgfqpoint{1.193286in}{1.171318in}}{\pgfqpoint{1.190788in}{1.170283in}}{\pgfqpoint{1.188946in}{1.168441in}}%
\pgfpathcurveto{\pgfqpoint{1.187104in}{1.166599in}}{\pgfqpoint{1.186069in}{1.164101in}}{\pgfqpoint{1.186069in}{1.161497in}}%
\pgfpathcurveto{\pgfqpoint{1.186069in}{1.158892in}}{\pgfqpoint{1.187104in}{1.156394in}}{\pgfqpoint{1.188946in}{1.154552in}}%
\pgfpathcurveto{\pgfqpoint{1.190788in}{1.152710in}}{\pgfqpoint{1.193286in}{1.151676in}}{\pgfqpoint{1.195890in}{1.151676in}}%
\pgfpathclose%
\pgfusepath{stroke,fill}%
\end{pgfscope}%
\begin{pgfscope}%
\pgfpathrectangle{\pgfqpoint{0.750000in}{0.500000in}}{\pgfqpoint{4.650000in}{3.020000in}}%
\pgfusepath{clip}%
\pgfsetbuttcap%
\pgfsetroundjoin%
\definecolor{currentfill}{rgb}{0.121569,0.466667,0.705882}%
\pgfsetfillcolor{currentfill}%
\pgfsetlinewidth{1.003750pt}%
\definecolor{currentstroke}{rgb}{0.121569,0.466667,0.705882}%
\pgfsetstrokecolor{currentstroke}%
\pgfsetdash{}{0pt}%
\pgfpathmoveto{\pgfqpoint{1.811161in}{0.951491in}}%
\pgfpathcurveto{\pgfqpoint{1.813766in}{0.951491in}}{\pgfqpoint{1.816264in}{0.952526in}}{\pgfqpoint{1.818106in}{0.954368in}}%
\pgfpathcurveto{\pgfqpoint{1.819947in}{0.956209in}}{\pgfqpoint{1.820982in}{0.958708in}}{\pgfqpoint{1.820982in}{0.961312in}}%
\pgfpathcurveto{\pgfqpoint{1.820982in}{0.963917in}}{\pgfqpoint{1.819947in}{0.966415in}}{\pgfqpoint{1.818106in}{0.968257in}}%
\pgfpathcurveto{\pgfqpoint{1.816264in}{0.970098in}}{\pgfqpoint{1.813766in}{0.971133in}}{\pgfqpoint{1.811161in}{0.971133in}}%
\pgfpathcurveto{\pgfqpoint{1.808557in}{0.971133in}}{\pgfqpoint{1.806059in}{0.970098in}}{\pgfqpoint{1.804217in}{0.968257in}}%
\pgfpathcurveto{\pgfqpoint{1.802375in}{0.966415in}}{\pgfqpoint{1.801340in}{0.963917in}}{\pgfqpoint{1.801340in}{0.961312in}}%
\pgfpathcurveto{\pgfqpoint{1.801340in}{0.958708in}}{\pgfqpoint{1.802375in}{0.956209in}}{\pgfqpoint{1.804217in}{0.954368in}}%
\pgfpathcurveto{\pgfqpoint{1.806059in}{0.952526in}}{\pgfqpoint{1.808557in}{0.951491in}}{\pgfqpoint{1.811161in}{0.951491in}}%
\pgfpathclose%
\pgfusepath{stroke,fill}%
\end{pgfscope}%
\begin{pgfscope}%
\pgfpathrectangle{\pgfqpoint{0.750000in}{0.500000in}}{\pgfqpoint{4.650000in}{3.020000in}}%
\pgfusepath{clip}%
\pgfsetbuttcap%
\pgfsetroundjoin%
\definecolor{currentfill}{rgb}{0.121569,0.466667,0.705882}%
\pgfsetfillcolor{currentfill}%
\pgfsetlinewidth{1.003750pt}%
\definecolor{currentstroke}{rgb}{0.121569,0.466667,0.705882}%
\pgfsetstrokecolor{currentstroke}%
\pgfsetdash{}{0pt}%
\pgfpathmoveto{\pgfqpoint{1.845906in}{0.746026in}}%
\pgfpathcurveto{\pgfqpoint{1.848511in}{0.746026in}}{\pgfqpoint{1.851009in}{0.747061in}}{\pgfqpoint{1.852850in}{0.748903in}}%
\pgfpathcurveto{\pgfqpoint{1.854692in}{0.750744in}}{\pgfqpoint{1.855727in}{0.753243in}}{\pgfqpoint{1.855727in}{0.755847in}}%
\pgfpathcurveto{\pgfqpoint{1.855727in}{0.758452in}}{\pgfqpoint{1.854692in}{0.760950in}}{\pgfqpoint{1.852850in}{0.762792in}}%
\pgfpathcurveto{\pgfqpoint{1.851009in}{0.764633in}}{\pgfqpoint{1.848511in}{0.765668in}}{\pgfqpoint{1.845906in}{0.765668in}}%
\pgfpathcurveto{\pgfqpoint{1.843301in}{0.765668in}}{\pgfqpoint{1.840803in}{0.764633in}}{\pgfqpoint{1.838962in}{0.762792in}}%
\pgfpathcurveto{\pgfqpoint{1.837120in}{0.760950in}}{\pgfqpoint{1.836085in}{0.758452in}}{\pgfqpoint{1.836085in}{0.755847in}}%
\pgfpathcurveto{\pgfqpoint{1.836085in}{0.753243in}}{\pgfqpoint{1.837120in}{0.750744in}}{\pgfqpoint{1.838962in}{0.748903in}}%
\pgfpathcurveto{\pgfqpoint{1.840803in}{0.747061in}}{\pgfqpoint{1.843301in}{0.746026in}}{\pgfqpoint{1.845906in}{0.746026in}}%
\pgfpathclose%
\pgfusepath{stroke,fill}%
\end{pgfscope}%
\begin{pgfscope}%
\pgfpathrectangle{\pgfqpoint{0.750000in}{0.500000in}}{\pgfqpoint{4.650000in}{3.020000in}}%
\pgfusepath{clip}%
\pgfsetbuttcap%
\pgfsetroundjoin%
\definecolor{currentfill}{rgb}{0.121569,0.466667,0.705882}%
\pgfsetfillcolor{currentfill}%
\pgfsetlinewidth{1.003750pt}%
\definecolor{currentstroke}{rgb}{0.121569,0.466667,0.705882}%
\pgfsetstrokecolor{currentstroke}%
\pgfsetdash{}{0pt}%
\pgfpathmoveto{\pgfqpoint{1.838667in}{0.794992in}}%
\pgfpathcurveto{\pgfqpoint{1.841272in}{0.794992in}}{\pgfqpoint{1.843770in}{0.796027in}}{\pgfqpoint{1.845612in}{0.797869in}}%
\pgfpathcurveto{\pgfqpoint{1.847454in}{0.799710in}}{\pgfqpoint{1.848488in}{0.802209in}}{\pgfqpoint{1.848488in}{0.804813in}}%
\pgfpathcurveto{\pgfqpoint{1.848488in}{0.807418in}}{\pgfqpoint{1.847454in}{0.809916in}}{\pgfqpoint{1.845612in}{0.811758in}}%
\pgfpathcurveto{\pgfqpoint{1.843770in}{0.813599in}}{\pgfqpoint{1.841272in}{0.814634in}}{\pgfqpoint{1.838667in}{0.814634in}}%
\pgfpathcurveto{\pgfqpoint{1.836063in}{0.814634in}}{\pgfqpoint{1.833565in}{0.813599in}}{\pgfqpoint{1.831723in}{0.811758in}}%
\pgfpathcurveto{\pgfqpoint{1.829881in}{0.809916in}}{\pgfqpoint{1.828847in}{0.807418in}}{\pgfqpoint{1.828847in}{0.804813in}}%
\pgfpathcurveto{\pgfqpoint{1.828847in}{0.802209in}}{\pgfqpoint{1.829881in}{0.799710in}}{\pgfqpoint{1.831723in}{0.797869in}}%
\pgfpathcurveto{\pgfqpoint{1.833565in}{0.796027in}}{\pgfqpoint{1.836063in}{0.794992in}}{\pgfqpoint{1.838667in}{0.794992in}}%
\pgfpathclose%
\pgfusepath{stroke,fill}%
\end{pgfscope}%
\begin{pgfscope}%
\pgfpathrectangle{\pgfqpoint{0.750000in}{0.500000in}}{\pgfqpoint{4.650000in}{3.020000in}}%
\pgfusepath{clip}%
\pgfsetbuttcap%
\pgfsetroundjoin%
\definecolor{currentfill}{rgb}{0.121569,0.466667,0.705882}%
\pgfsetfillcolor{currentfill}%
\pgfsetlinewidth{1.003750pt}%
\definecolor{currentstroke}{rgb}{0.121569,0.466667,0.705882}%
\pgfsetstrokecolor{currentstroke}%
\pgfsetdash{}{0pt}%
\pgfpathmoveto{\pgfqpoint{1.287095in}{0.684099in}}%
\pgfpathcurveto{\pgfqpoint{1.289700in}{0.684099in}}{\pgfqpoint{1.292198in}{0.685133in}}{\pgfqpoint{1.294040in}{0.686975in}}%
\pgfpathcurveto{\pgfqpoint{1.295881in}{0.688817in}}{\pgfqpoint{1.296916in}{0.691315in}}{\pgfqpoint{1.296916in}{0.693920in}}%
\pgfpathcurveto{\pgfqpoint{1.296916in}{0.696524in}}{\pgfqpoint{1.295881in}{0.699022in}}{\pgfqpoint{1.294040in}{0.700864in}}%
\pgfpathcurveto{\pgfqpoint{1.292198in}{0.702706in}}{\pgfqpoint{1.289700in}{0.703741in}}{\pgfqpoint{1.287095in}{0.703741in}}%
\pgfpathcurveto{\pgfqpoint{1.284491in}{0.703741in}}{\pgfqpoint{1.281993in}{0.702706in}}{\pgfqpoint{1.280151in}{0.700864in}}%
\pgfpathcurveto{\pgfqpoint{1.278309in}{0.699022in}}{\pgfqpoint{1.277274in}{0.696524in}}{\pgfqpoint{1.277274in}{0.693920in}}%
\pgfpathcurveto{\pgfqpoint{1.277274in}{0.691315in}}{\pgfqpoint{1.278309in}{0.688817in}}{\pgfqpoint{1.280151in}{0.686975in}}%
\pgfpathcurveto{\pgfqpoint{1.281993in}{0.685133in}}{\pgfqpoint{1.284491in}{0.684099in}}{\pgfqpoint{1.287095in}{0.684099in}}%
\pgfpathclose%
\pgfusepath{stroke,fill}%
\end{pgfscope}%
\begin{pgfscope}%
\pgfpathrectangle{\pgfqpoint{0.750000in}{0.500000in}}{\pgfqpoint{4.650000in}{3.020000in}}%
\pgfusepath{clip}%
\pgfsetbuttcap%
\pgfsetroundjoin%
\definecolor{currentfill}{rgb}{0.121569,0.466667,0.705882}%
\pgfsetfillcolor{currentfill}%
\pgfsetlinewidth{1.003750pt}%
\definecolor{currentstroke}{rgb}{0.121569,0.466667,0.705882}%
\pgfsetstrokecolor{currentstroke}%
\pgfsetdash{}{0pt}%
\pgfpathmoveto{\pgfqpoint{1.333422in}{0.695620in}}%
\pgfpathcurveto{\pgfqpoint{1.336026in}{0.695620in}}{\pgfqpoint{1.338524in}{0.696655in}}{\pgfqpoint{1.340366in}{0.698497in}}%
\pgfpathcurveto{\pgfqpoint{1.342208in}{0.700338in}}{\pgfqpoint{1.343242in}{0.702836in}}{\pgfqpoint{1.343242in}{0.705441in}}%
\pgfpathcurveto{\pgfqpoint{1.343242in}{0.708046in}}{\pgfqpoint{1.342208in}{0.710544in}}{\pgfqpoint{1.340366in}{0.712385in}}%
\pgfpathcurveto{\pgfqpoint{1.338524in}{0.714227in}}{\pgfqpoint{1.336026in}{0.715262in}}{\pgfqpoint{1.333422in}{0.715262in}}%
\pgfpathcurveto{\pgfqpoint{1.330817in}{0.715262in}}{\pgfqpoint{1.328319in}{0.714227in}}{\pgfqpoint{1.326477in}{0.712385in}}%
\pgfpathcurveto{\pgfqpoint{1.324635in}{0.710544in}}{\pgfqpoint{1.323601in}{0.708046in}}{\pgfqpoint{1.323601in}{0.705441in}}%
\pgfpathcurveto{\pgfqpoint{1.323601in}{0.702836in}}{\pgfqpoint{1.324635in}{0.700338in}}{\pgfqpoint{1.326477in}{0.698497in}}%
\pgfpathcurveto{\pgfqpoint{1.328319in}{0.696655in}}{\pgfqpoint{1.330817in}{0.695620in}}{\pgfqpoint{1.333422in}{0.695620in}}%
\pgfpathclose%
\pgfusepath{stroke,fill}%
\end{pgfscope}%
\begin{pgfscope}%
\pgfpathrectangle{\pgfqpoint{0.750000in}{0.500000in}}{\pgfqpoint{4.650000in}{3.020000in}}%
\pgfusepath{clip}%
\pgfsetbuttcap%
\pgfsetroundjoin%
\definecolor{currentfill}{rgb}{0.121569,0.466667,0.705882}%
\pgfsetfillcolor{currentfill}%
\pgfsetlinewidth{1.003750pt}%
\definecolor{currentstroke}{rgb}{0.121569,0.466667,0.705882}%
\pgfsetstrokecolor{currentstroke}%
\pgfsetdash{}{0pt}%
\pgfpathmoveto{\pgfqpoint{1.156803in}{0.657695in}}%
\pgfpathcurveto{\pgfqpoint{1.159407in}{0.657695in}}{\pgfqpoint{1.161905in}{0.658730in}}{\pgfqpoint{1.163747in}{0.660572in}}%
\pgfpathcurveto{\pgfqpoint{1.165589in}{0.662414in}}{\pgfqpoint{1.166624in}{0.664912in}}{\pgfqpoint{1.166624in}{0.667516in}}%
\pgfpathcurveto{\pgfqpoint{1.166624in}{0.670121in}}{\pgfqpoint{1.165589in}{0.672619in}}{\pgfqpoint{1.163747in}{0.674461in}}%
\pgfpathcurveto{\pgfqpoint{1.161905in}{0.676303in}}{\pgfqpoint{1.159407in}{0.677337in}}{\pgfqpoint{1.156803in}{0.677337in}}%
\pgfpathcurveto{\pgfqpoint{1.154198in}{0.677337in}}{\pgfqpoint{1.151700in}{0.676303in}}{\pgfqpoint{1.149858in}{0.674461in}}%
\pgfpathcurveto{\pgfqpoint{1.148016in}{0.672619in}}{\pgfqpoint{1.146982in}{0.670121in}}{\pgfqpoint{1.146982in}{0.667516in}}%
\pgfpathcurveto{\pgfqpoint{1.146982in}{0.664912in}}{\pgfqpoint{1.148016in}{0.662414in}}{\pgfqpoint{1.149858in}{0.660572in}}%
\pgfpathcurveto{\pgfqpoint{1.151700in}{0.658730in}}{\pgfqpoint{1.154198in}{0.657695in}}{\pgfqpoint{1.156803in}{0.657695in}}%
\pgfpathclose%
\pgfusepath{stroke,fill}%
\end{pgfscope}%
\begin{pgfscope}%
\pgfpathrectangle{\pgfqpoint{0.750000in}{0.500000in}}{\pgfqpoint{4.650000in}{3.020000in}}%
\pgfusepath{clip}%
\pgfsetbuttcap%
\pgfsetroundjoin%
\definecolor{currentfill}{rgb}{0.121569,0.466667,0.705882}%
\pgfsetfillcolor{currentfill}%
\pgfsetlinewidth{1.003750pt}%
\definecolor{currentstroke}{rgb}{0.121569,0.466667,0.705882}%
\pgfsetstrokecolor{currentstroke}%
\pgfsetdash{}{0pt}%
\pgfpathmoveto{\pgfqpoint{1.159698in}{0.647134in}}%
\pgfpathcurveto{\pgfqpoint{1.162303in}{0.647134in}}{\pgfqpoint{1.164801in}{0.648169in}}{\pgfqpoint{1.166642in}{0.650011in}}%
\pgfpathcurveto{\pgfqpoint{1.168484in}{0.651852in}}{\pgfqpoint{1.169519in}{0.654351in}}{\pgfqpoint{1.169519in}{0.656955in}}%
\pgfpathcurveto{\pgfqpoint{1.169519in}{0.659560in}}{\pgfqpoint{1.168484in}{0.662058in}}{\pgfqpoint{1.166642in}{0.663900in}}%
\pgfpathcurveto{\pgfqpoint{1.164801in}{0.665741in}}{\pgfqpoint{1.162303in}{0.666776in}}{\pgfqpoint{1.159698in}{0.666776in}}%
\pgfpathcurveto{\pgfqpoint{1.157093in}{0.666776in}}{\pgfqpoint{1.154595in}{0.665741in}}{\pgfqpoint{1.152754in}{0.663900in}}%
\pgfpathcurveto{\pgfqpoint{1.150912in}{0.662058in}}{\pgfqpoint{1.149877in}{0.659560in}}{\pgfqpoint{1.149877in}{0.656955in}}%
\pgfpathcurveto{\pgfqpoint{1.149877in}{0.654351in}}{\pgfqpoint{1.150912in}{0.651852in}}{\pgfqpoint{1.152754in}{0.650011in}}%
\pgfpathcurveto{\pgfqpoint{1.154595in}{0.648169in}}{\pgfqpoint{1.157093in}{0.647134in}}{\pgfqpoint{1.159698in}{0.647134in}}%
\pgfpathclose%
\pgfusepath{stroke,fill}%
\end{pgfscope}%
\begin{pgfscope}%
\pgfpathrectangle{\pgfqpoint{0.750000in}{0.500000in}}{\pgfqpoint{4.650000in}{3.020000in}}%
\pgfusepath{clip}%
\pgfsetbuttcap%
\pgfsetroundjoin%
\definecolor{currentfill}{rgb}{0.121569,0.466667,0.705882}%
\pgfsetfillcolor{currentfill}%
\pgfsetlinewidth{1.003750pt}%
\definecolor{currentstroke}{rgb}{0.121569,0.466667,0.705882}%
\pgfsetstrokecolor{currentstroke}%
\pgfsetdash{}{0pt}%
\pgfpathmoveto{\pgfqpoint{1.909605in}{0.812754in}}%
\pgfpathcurveto{\pgfqpoint{1.912209in}{0.812754in}}{\pgfqpoint{1.914707in}{0.813789in}}{\pgfqpoint{1.916549in}{0.815631in}}%
\pgfpathcurveto{\pgfqpoint{1.918391in}{0.817473in}}{\pgfqpoint{1.919426in}{0.819971in}}{\pgfqpoint{1.919426in}{0.822575in}}%
\pgfpathcurveto{\pgfqpoint{1.919426in}{0.825180in}}{\pgfqpoint{1.918391in}{0.827678in}}{\pgfqpoint{1.916549in}{0.829520in}}%
\pgfpathcurveto{\pgfqpoint{1.914707in}{0.831361in}}{\pgfqpoint{1.912209in}{0.832396in}}{\pgfqpoint{1.909605in}{0.832396in}}%
\pgfpathcurveto{\pgfqpoint{1.907000in}{0.832396in}}{\pgfqpoint{1.904502in}{0.831361in}}{\pgfqpoint{1.902660in}{0.829520in}}%
\pgfpathcurveto{\pgfqpoint{1.900818in}{0.827678in}}{\pgfqpoint{1.899784in}{0.825180in}}{\pgfqpoint{1.899784in}{0.822575in}}%
\pgfpathcurveto{\pgfqpoint{1.899784in}{0.819971in}}{\pgfqpoint{1.900818in}{0.817473in}}{\pgfqpoint{1.902660in}{0.815631in}}%
\pgfpathcurveto{\pgfqpoint{1.904502in}{0.813789in}}{\pgfqpoint{1.907000in}{0.812754in}}{\pgfqpoint{1.909605in}{0.812754in}}%
\pgfpathclose%
\pgfusepath{stroke,fill}%
\end{pgfscope}%
\begin{pgfscope}%
\pgfpathrectangle{\pgfqpoint{0.750000in}{0.500000in}}{\pgfqpoint{4.650000in}{3.020000in}}%
\pgfusepath{clip}%
\pgfsetbuttcap%
\pgfsetroundjoin%
\definecolor{currentfill}{rgb}{0.121569,0.466667,0.705882}%
\pgfsetfillcolor{currentfill}%
\pgfsetlinewidth{1.003750pt}%
\definecolor{currentstroke}{rgb}{0.121569,0.466667,0.705882}%
\pgfsetstrokecolor{currentstroke}%
\pgfsetdash{}{0pt}%
\pgfpathmoveto{\pgfqpoint{1.165489in}{0.990856in}}%
\pgfpathcurveto{\pgfqpoint{1.168093in}{0.990856in}}{\pgfqpoint{1.170592in}{0.991891in}}{\pgfqpoint{1.172433in}{0.993733in}}%
\pgfpathcurveto{\pgfqpoint{1.174275in}{0.995574in}}{\pgfqpoint{1.175310in}{0.998072in}}{\pgfqpoint{1.175310in}{1.000677in}}%
\pgfpathcurveto{\pgfqpoint{1.175310in}{1.003282in}}{\pgfqpoint{1.174275in}{1.005780in}}{\pgfqpoint{1.172433in}{1.007621in}}%
\pgfpathcurveto{\pgfqpoint{1.170592in}{1.009463in}}{\pgfqpoint{1.168093in}{1.010498in}}{\pgfqpoint{1.165489in}{1.010498in}}%
\pgfpathcurveto{\pgfqpoint{1.162884in}{1.010498in}}{\pgfqpoint{1.160386in}{1.009463in}}{\pgfqpoint{1.158544in}{1.007621in}}%
\pgfpathcurveto{\pgfqpoint{1.156703in}{1.005780in}}{\pgfqpoint{1.155668in}{1.003282in}}{\pgfqpoint{1.155668in}{1.000677in}}%
\pgfpathcurveto{\pgfqpoint{1.155668in}{0.998072in}}{\pgfqpoint{1.156703in}{0.995574in}}{\pgfqpoint{1.158544in}{0.993733in}}%
\pgfpathcurveto{\pgfqpoint{1.160386in}{0.991891in}}{\pgfqpoint{1.162884in}{0.990856in}}{\pgfqpoint{1.165489in}{0.990856in}}%
\pgfpathclose%
\pgfusepath{stroke,fill}%
\end{pgfscope}%
\begin{pgfscope}%
\pgfpathrectangle{\pgfqpoint{0.750000in}{0.500000in}}{\pgfqpoint{4.650000in}{3.020000in}}%
\pgfusepath{clip}%
\pgfsetbuttcap%
\pgfsetroundjoin%
\definecolor{currentfill}{rgb}{0.121569,0.466667,0.705882}%
\pgfsetfillcolor{currentfill}%
\pgfsetlinewidth{1.003750pt}%
\definecolor{currentstroke}{rgb}{0.121569,0.466667,0.705882}%
\pgfsetstrokecolor{currentstroke}%
\pgfsetdash{}{0pt}%
\pgfpathmoveto{\pgfqpoint{5.006227in}{2.164707in}}%
\pgfpathcurveto{\pgfqpoint{5.016314in}{2.164707in}}{\pgfqpoint{5.025990in}{2.168715in}}{\pgfqpoint{5.033122in}{2.175848in}}%
\pgfpathcurveto{\pgfqpoint{5.040255in}{2.182981in}}{\pgfqpoint{5.044263in}{2.192656in}}{\pgfqpoint{5.044263in}{2.202743in}}%
\pgfpathcurveto{\pgfqpoint{5.044263in}{2.212831in}}{\pgfqpoint{5.040255in}{2.222506in}}{\pgfqpoint{5.033122in}{2.229639in}}%
\pgfpathcurveto{\pgfqpoint{5.025990in}{2.236772in}}{\pgfqpoint{5.016314in}{2.240780in}}{\pgfqpoint{5.006227in}{2.240780in}}%
\pgfpathcurveto{\pgfqpoint{4.996139in}{2.240780in}}{\pgfqpoint{4.986464in}{2.236772in}}{\pgfqpoint{4.979331in}{2.229639in}}%
\pgfpathcurveto{\pgfqpoint{4.972198in}{2.222506in}}{\pgfqpoint{4.968190in}{2.212831in}}{\pgfqpoint{4.968190in}{2.202743in}}%
\pgfpathcurveto{\pgfqpoint{4.968190in}{2.192656in}}{\pgfqpoint{4.972198in}{2.182981in}}{\pgfqpoint{4.979331in}{2.175848in}}%
\pgfpathcurveto{\pgfqpoint{4.986464in}{2.168715in}}{\pgfqpoint{4.996139in}{2.164707in}}{\pgfqpoint{5.006227in}{2.164707in}}%
\pgfpathclose%
\pgfusepath{stroke,fill}%
\end{pgfscope}%
\begin{pgfscope}%
\pgfpathrectangle{\pgfqpoint{0.750000in}{0.500000in}}{\pgfqpoint{4.650000in}{3.020000in}}%
\pgfusepath{clip}%
\pgfsetbuttcap%
\pgfsetroundjoin%
\definecolor{currentfill}{rgb}{0.121569,0.466667,0.705882}%
\pgfsetfillcolor{currentfill}%
\pgfsetlinewidth{1.003750pt}%
\definecolor{currentstroke}{rgb}{0.121569,0.466667,0.705882}%
\pgfsetstrokecolor{currentstroke}%
\pgfsetdash{}{0pt}%
\pgfpathmoveto{\pgfqpoint{1.342108in}{0.674691in}}%
\pgfpathcurveto{\pgfqpoint{1.345298in}{0.674691in}}{\pgfqpoint{1.348357in}{0.675958in}}{\pgfqpoint{1.350613in}{0.678214in}}%
\pgfpathcurveto{\pgfqpoint{1.352868in}{0.680469in}}{\pgfqpoint{1.354136in}{0.683529in}}{\pgfqpoint{1.354136in}{0.686719in}}%
\pgfpathcurveto{\pgfqpoint{1.354136in}{0.689909in}}{\pgfqpoint{1.352868in}{0.692968in}}{\pgfqpoint{1.350613in}{0.695224in}}%
\pgfpathcurveto{\pgfqpoint{1.348357in}{0.697480in}}{\pgfqpoint{1.345298in}{0.698747in}}{\pgfqpoint{1.342108in}{0.698747in}}%
\pgfpathcurveto{\pgfqpoint{1.338918in}{0.698747in}}{\pgfqpoint{1.335858in}{0.697480in}}{\pgfqpoint{1.333603in}{0.695224in}}%
\pgfpathcurveto{\pgfqpoint{1.331347in}{0.692968in}}{\pgfqpoint{1.330080in}{0.689909in}}{\pgfqpoint{1.330080in}{0.686719in}}%
\pgfpathcurveto{\pgfqpoint{1.330080in}{0.683529in}}{\pgfqpoint{1.331347in}{0.680469in}}{\pgfqpoint{1.333603in}{0.678214in}}%
\pgfpathcurveto{\pgfqpoint{1.335858in}{0.675958in}}{\pgfqpoint{1.338918in}{0.674691in}}{\pgfqpoint{1.342108in}{0.674691in}}%
\pgfpathclose%
\pgfusepath{stroke,fill}%
\end{pgfscope}%
\begin{pgfscope}%
\pgfpathrectangle{\pgfqpoint{0.750000in}{0.500000in}}{\pgfqpoint{4.650000in}{3.020000in}}%
\pgfusepath{clip}%
\pgfsetbuttcap%
\pgfsetroundjoin%
\definecolor{currentfill}{rgb}{0.121569,0.466667,0.705882}%
\pgfsetfillcolor{currentfill}%
\pgfsetlinewidth{1.003750pt}%
\definecolor{currentstroke}{rgb}{0.121569,0.466667,0.705882}%
\pgfsetstrokecolor{currentstroke}%
\pgfsetdash{}{0pt}%
\pgfpathmoveto{\pgfqpoint{2.616080in}{0.838678in}}%
\pgfpathcurveto{\pgfqpoint{2.618685in}{0.838678in}}{\pgfqpoint{2.621183in}{0.839712in}}{\pgfqpoint{2.623025in}{0.841554in}}%
\pgfpathcurveto{\pgfqpoint{2.624866in}{0.843396in}}{\pgfqpoint{2.625901in}{0.845894in}}{\pgfqpoint{2.625901in}{0.848498in}}%
\pgfpathcurveto{\pgfqpoint{2.625901in}{0.851103in}}{\pgfqpoint{2.624866in}{0.853601in}}{\pgfqpoint{2.623025in}{0.855443in}}%
\pgfpathcurveto{\pgfqpoint{2.621183in}{0.857285in}}{\pgfqpoint{2.618685in}{0.858319in}}{\pgfqpoint{2.616080in}{0.858319in}}%
\pgfpathcurveto{\pgfqpoint{2.613476in}{0.858319in}}{\pgfqpoint{2.610978in}{0.857285in}}{\pgfqpoint{2.609136in}{0.855443in}}%
\pgfpathcurveto{\pgfqpoint{2.607294in}{0.853601in}}{\pgfqpoint{2.606259in}{0.851103in}}{\pgfqpoint{2.606259in}{0.848498in}}%
\pgfpathcurveto{\pgfqpoint{2.606259in}{0.845894in}}{\pgfqpoint{2.607294in}{0.843396in}}{\pgfqpoint{2.609136in}{0.841554in}}%
\pgfpathcurveto{\pgfqpoint{2.610978in}{0.839712in}}{\pgfqpoint{2.613476in}{0.838678in}}{\pgfqpoint{2.616080in}{0.838678in}}%
\pgfpathclose%
\pgfusepath{stroke,fill}%
\end{pgfscope}%
\begin{pgfscope}%
\pgfpathrectangle{\pgfqpoint{0.750000in}{0.500000in}}{\pgfqpoint{4.650000in}{3.020000in}}%
\pgfusepath{clip}%
\pgfsetbuttcap%
\pgfsetroundjoin%
\definecolor{currentfill}{rgb}{0.121569,0.466667,0.705882}%
\pgfsetfillcolor{currentfill}%
\pgfsetlinewidth{1.003750pt}%
\definecolor{currentstroke}{rgb}{0.121569,0.466667,0.705882}%
\pgfsetstrokecolor{currentstroke}%
\pgfsetdash{}{0pt}%
\pgfpathmoveto{\pgfqpoint{0.961364in}{0.631292in}}%
\pgfpathcurveto{\pgfqpoint{0.963968in}{0.631292in}}{\pgfqpoint{0.966466in}{0.632327in}}{\pgfqpoint{0.968308in}{0.634169in}}%
\pgfpathcurveto{\pgfqpoint{0.970150in}{0.636010in}}{\pgfqpoint{0.971185in}{0.638509in}}{\pgfqpoint{0.971185in}{0.641113in}}%
\pgfpathcurveto{\pgfqpoint{0.971185in}{0.643718in}}{\pgfqpoint{0.970150in}{0.646216in}}{\pgfqpoint{0.968308in}{0.648058in}}%
\pgfpathcurveto{\pgfqpoint{0.966466in}{0.649899in}}{\pgfqpoint{0.963968in}{0.650934in}}{\pgfqpoint{0.961364in}{0.650934in}}%
\pgfpathcurveto{\pgfqpoint{0.958759in}{0.650934in}}{\pgfqpoint{0.956261in}{0.649899in}}{\pgfqpoint{0.954419in}{0.648058in}}%
\pgfpathcurveto{\pgfqpoint{0.952578in}{0.646216in}}{\pgfqpoint{0.951543in}{0.643718in}}{\pgfqpoint{0.951543in}{0.641113in}}%
\pgfpathcurveto{\pgfqpoint{0.951543in}{0.638509in}}{\pgfqpoint{0.952578in}{0.636010in}}{\pgfqpoint{0.954419in}{0.634169in}}%
\pgfpathcurveto{\pgfqpoint{0.956261in}{0.632327in}}{\pgfqpoint{0.958759in}{0.631292in}}{\pgfqpoint{0.961364in}{0.631292in}}%
\pgfpathclose%
\pgfusepath{stroke,fill}%
\end{pgfscope}%
\begin{pgfscope}%
\pgfpathrectangle{\pgfqpoint{0.750000in}{0.500000in}}{\pgfqpoint{4.650000in}{3.020000in}}%
\pgfusepath{clip}%
\pgfsetbuttcap%
\pgfsetroundjoin%
\definecolor{currentfill}{rgb}{0.121569,0.466667,0.705882}%
\pgfsetfillcolor{currentfill}%
\pgfsetlinewidth{1.003750pt}%
\definecolor{currentstroke}{rgb}{0.121569,0.466667,0.705882}%
\pgfsetstrokecolor{currentstroke}%
\pgfsetdash{}{0pt}%
\pgfpathmoveto{\pgfqpoint{1.996466in}{1.403487in}}%
\pgfpathcurveto{\pgfqpoint{2.001675in}{1.403487in}}{\pgfqpoint{2.006672in}{1.405556in}}{\pgfqpoint{2.010355in}{1.409240in}}%
\pgfpathcurveto{\pgfqpoint{2.014039in}{1.412923in}}{\pgfqpoint{2.016108in}{1.417919in}}{\pgfqpoint{2.016108in}{1.423128in}}%
\pgfpathcurveto{\pgfqpoint{2.016108in}{1.428338in}}{\pgfqpoint{2.014039in}{1.433334in}}{\pgfqpoint{2.010355in}{1.437017in}}%
\pgfpathcurveto{\pgfqpoint{2.006672in}{1.440701in}}{\pgfqpoint{2.001675in}{1.442770in}}{\pgfqpoint{1.996466in}{1.442770in}}%
\pgfpathcurveto{\pgfqpoint{1.991257in}{1.442770in}}{\pgfqpoint{1.986261in}{1.440701in}}{\pgfqpoint{1.982577in}{1.437017in}}%
\pgfpathcurveto{\pgfqpoint{1.978894in}{1.433334in}}{\pgfqpoint{1.976825in}{1.428338in}}{\pgfqpoint{1.976825in}{1.423128in}}%
\pgfpathcurveto{\pgfqpoint{1.976825in}{1.417919in}}{\pgfqpoint{1.978894in}{1.412923in}}{\pgfqpoint{1.982577in}{1.409240in}}%
\pgfpathcurveto{\pgfqpoint{1.986261in}{1.405556in}}{\pgfqpoint{1.991257in}{1.403487in}}{\pgfqpoint{1.996466in}{1.403487in}}%
\pgfpathclose%
\pgfusepath{stroke,fill}%
\end{pgfscope}%
\begin{pgfscope}%
\pgfpathrectangle{\pgfqpoint{0.750000in}{0.500000in}}{\pgfqpoint{4.650000in}{3.020000in}}%
\pgfusepath{clip}%
\pgfsetbuttcap%
\pgfsetroundjoin%
\definecolor{currentfill}{rgb}{0.121569,0.466667,0.705882}%
\pgfsetfillcolor{currentfill}%
\pgfsetlinewidth{1.003750pt}%
\definecolor{currentstroke}{rgb}{0.121569,0.466667,0.705882}%
\pgfsetstrokecolor{currentstroke}%
\pgfsetdash{}{0pt}%
\pgfpathmoveto{\pgfqpoint{1.095999in}{0.852119in}}%
\pgfpathcurveto{\pgfqpoint{1.098604in}{0.852119in}}{\pgfqpoint{1.101102in}{0.853154in}}{\pgfqpoint{1.102944in}{0.854996in}}%
\pgfpathcurveto{\pgfqpoint{1.104786in}{0.856837in}}{\pgfqpoint{1.105820in}{0.859336in}}{\pgfqpoint{1.105820in}{0.861940in}}%
\pgfpathcurveto{\pgfqpoint{1.105820in}{0.864545in}}{\pgfqpoint{1.104786in}{0.867043in}}{\pgfqpoint{1.102944in}{0.868885in}}%
\pgfpathcurveto{\pgfqpoint{1.101102in}{0.870726in}}{\pgfqpoint{1.098604in}{0.871761in}}{\pgfqpoint{1.095999in}{0.871761in}}%
\pgfpathcurveto{\pgfqpoint{1.093395in}{0.871761in}}{\pgfqpoint{1.090897in}{0.870726in}}{\pgfqpoint{1.089055in}{0.868885in}}%
\pgfpathcurveto{\pgfqpoint{1.087213in}{0.867043in}}{\pgfqpoint{1.086178in}{0.864545in}}{\pgfqpoint{1.086178in}{0.861940in}}%
\pgfpathcurveto{\pgfqpoint{1.086178in}{0.859336in}}{\pgfqpoint{1.087213in}{0.856837in}}{\pgfqpoint{1.089055in}{0.854996in}}%
\pgfpathcurveto{\pgfqpoint{1.090897in}{0.853154in}}{\pgfqpoint{1.093395in}{0.852119in}}{\pgfqpoint{1.095999in}{0.852119in}}%
\pgfpathclose%
\pgfusepath{stroke,fill}%
\end{pgfscope}%
\begin{pgfscope}%
\pgfpathrectangle{\pgfqpoint{0.750000in}{0.500000in}}{\pgfqpoint{4.650000in}{3.020000in}}%
\pgfusepath{clip}%
\pgfsetbuttcap%
\pgfsetroundjoin%
\definecolor{currentfill}{rgb}{0.121569,0.466667,0.705882}%
\pgfsetfillcolor{currentfill}%
\pgfsetlinewidth{1.003750pt}%
\definecolor{currentstroke}{rgb}{0.121569,0.466667,0.705882}%
\pgfsetstrokecolor{currentstroke}%
\pgfsetdash{}{0pt}%
\pgfpathmoveto{\pgfqpoint{1.334869in}{0.651455in}}%
\pgfpathcurveto{\pgfqpoint{1.337474in}{0.651455in}}{\pgfqpoint{1.339972in}{0.652490in}}{\pgfqpoint{1.341814in}{0.654331in}}%
\pgfpathcurveto{\pgfqpoint{1.343655in}{0.656173in}}{\pgfqpoint{1.344690in}{0.658671in}}{\pgfqpoint{1.344690in}{0.661276in}}%
\pgfpathcurveto{\pgfqpoint{1.344690in}{0.663880in}}{\pgfqpoint{1.343655in}{0.666378in}}{\pgfqpoint{1.341814in}{0.668220in}}%
\pgfpathcurveto{\pgfqpoint{1.339972in}{0.670062in}}{\pgfqpoint{1.337474in}{0.671097in}}{\pgfqpoint{1.334869in}{0.671097in}}%
\pgfpathcurveto{\pgfqpoint{1.332265in}{0.671097in}}{\pgfqpoint{1.329766in}{0.670062in}}{\pgfqpoint{1.327925in}{0.668220in}}%
\pgfpathcurveto{\pgfqpoint{1.326083in}{0.666378in}}{\pgfqpoint{1.325048in}{0.663880in}}{\pgfqpoint{1.325048in}{0.661276in}}%
\pgfpathcurveto{\pgfqpoint{1.325048in}{0.658671in}}{\pgfqpoint{1.326083in}{0.656173in}}{\pgfqpoint{1.327925in}{0.654331in}}%
\pgfpathcurveto{\pgfqpoint{1.329766in}{0.652490in}}{\pgfqpoint{1.332265in}{0.651455in}}{\pgfqpoint{1.334869in}{0.651455in}}%
\pgfpathclose%
\pgfusepath{stroke,fill}%
\end{pgfscope}%
\begin{pgfscope}%
\pgfpathrectangle{\pgfqpoint{0.750000in}{0.500000in}}{\pgfqpoint{4.650000in}{3.020000in}}%
\pgfusepath{clip}%
\pgfsetbuttcap%
\pgfsetroundjoin%
\definecolor{currentfill}{rgb}{0.121569,0.466667,0.705882}%
\pgfsetfillcolor{currentfill}%
\pgfsetlinewidth{1.003750pt}%
\definecolor{currentstroke}{rgb}{0.121569,0.466667,0.705882}%
\pgfsetstrokecolor{currentstroke}%
\pgfsetdash{}{0pt}%
\pgfpathmoveto{\pgfqpoint{1.038092in}{0.646654in}}%
\pgfpathcurveto{\pgfqpoint{1.040696in}{0.646654in}}{\pgfqpoint{1.043194in}{0.647689in}}{\pgfqpoint{1.045036in}{0.649531in}}%
\pgfpathcurveto{\pgfqpoint{1.046878in}{0.651372in}}{\pgfqpoint{1.047912in}{0.653871in}}{\pgfqpoint{1.047912in}{0.656475in}}%
\pgfpathcurveto{\pgfqpoint{1.047912in}{0.659080in}}{\pgfqpoint{1.046878in}{0.661578in}}{\pgfqpoint{1.045036in}{0.663420in}}%
\pgfpathcurveto{\pgfqpoint{1.043194in}{0.665261in}}{\pgfqpoint{1.040696in}{0.666296in}}{\pgfqpoint{1.038092in}{0.666296in}}%
\pgfpathcurveto{\pgfqpoint{1.035487in}{0.666296in}}{\pgfqpoint{1.032989in}{0.665261in}}{\pgfqpoint{1.031147in}{0.663420in}}%
\pgfpathcurveto{\pgfqpoint{1.029305in}{0.661578in}}{\pgfqpoint{1.028271in}{0.659080in}}{\pgfqpoint{1.028271in}{0.656475in}}%
\pgfpathcurveto{\pgfqpoint{1.028271in}{0.653871in}}{\pgfqpoint{1.029305in}{0.651372in}}{\pgfqpoint{1.031147in}{0.649531in}}%
\pgfpathcurveto{\pgfqpoint{1.032989in}{0.647689in}}{\pgfqpoint{1.035487in}{0.646654in}}{\pgfqpoint{1.038092in}{0.646654in}}%
\pgfpathclose%
\pgfusepath{stroke,fill}%
\end{pgfscope}%
\begin{pgfscope}%
\pgfpathrectangle{\pgfqpoint{0.750000in}{0.500000in}}{\pgfqpoint{4.650000in}{3.020000in}}%
\pgfusepath{clip}%
\pgfsetbuttcap%
\pgfsetroundjoin%
\definecolor{currentfill}{rgb}{0.121569,0.466667,0.705882}%
\pgfsetfillcolor{currentfill}%
\pgfsetlinewidth{1.003750pt}%
\definecolor{currentstroke}{rgb}{0.121569,0.466667,0.705882}%
\pgfsetstrokecolor{currentstroke}%
\pgfsetdash{}{0pt}%
\pgfpathmoveto{\pgfqpoint{1.000451in}{0.832917in}}%
\pgfpathcurveto{\pgfqpoint{1.003056in}{0.832917in}}{\pgfqpoint{1.005554in}{0.833952in}}{\pgfqpoint{1.007396in}{0.835793in}}%
\pgfpathcurveto{\pgfqpoint{1.009238in}{0.837635in}}{\pgfqpoint{1.010272in}{0.840133in}}{\pgfqpoint{1.010272in}{0.842738in}}%
\pgfpathcurveto{\pgfqpoint{1.010272in}{0.845342in}}{\pgfqpoint{1.009238in}{0.847841in}}{\pgfqpoint{1.007396in}{0.849682in}}%
\pgfpathcurveto{\pgfqpoint{1.005554in}{0.851524in}}{\pgfqpoint{1.003056in}{0.852559in}}{\pgfqpoint{1.000451in}{0.852559in}}%
\pgfpathcurveto{\pgfqpoint{0.997847in}{0.852559in}}{\pgfqpoint{0.995349in}{0.851524in}}{\pgfqpoint{0.993507in}{0.849682in}}%
\pgfpathcurveto{\pgfqpoint{0.991665in}{0.847841in}}{\pgfqpoint{0.990631in}{0.845342in}}{\pgfqpoint{0.990631in}{0.842738in}}%
\pgfpathcurveto{\pgfqpoint{0.990631in}{0.840133in}}{\pgfqpoint{0.991665in}{0.837635in}}{\pgfqpoint{0.993507in}{0.835793in}}%
\pgfpathcurveto{\pgfqpoint{0.995349in}{0.833952in}}{\pgfqpoint{0.997847in}{0.832917in}}{\pgfqpoint{1.000451in}{0.832917in}}%
\pgfpathclose%
\pgfusepath{stroke,fill}%
\end{pgfscope}%
\begin{pgfscope}%
\pgfpathrectangle{\pgfqpoint{0.750000in}{0.500000in}}{\pgfqpoint{4.650000in}{3.020000in}}%
\pgfusepath{clip}%
\pgfsetbuttcap%
\pgfsetroundjoin%
\definecolor{currentfill}{rgb}{0.121569,0.466667,0.705882}%
\pgfsetfillcolor{currentfill}%
\pgfsetlinewidth{1.003750pt}%
\definecolor{currentstroke}{rgb}{0.121569,0.466667,0.705882}%
\pgfsetstrokecolor{currentstroke}%
\pgfsetdash{}{0pt}%
\pgfpathmoveto{\pgfqpoint{2.511846in}{1.752709in}}%
\pgfpathcurveto{\pgfqpoint{2.514451in}{1.752709in}}{\pgfqpoint{2.516949in}{1.753744in}}{\pgfqpoint{2.518791in}{1.755585in}}%
\pgfpathcurveto{\pgfqpoint{2.520632in}{1.757427in}}{\pgfqpoint{2.521667in}{1.759925in}}{\pgfqpoint{2.521667in}{1.762530in}}%
\pgfpathcurveto{\pgfqpoint{2.521667in}{1.765134in}}{\pgfqpoint{2.520632in}{1.767633in}}{\pgfqpoint{2.518791in}{1.769474in}}%
\pgfpathcurveto{\pgfqpoint{2.516949in}{1.771316in}}{\pgfqpoint{2.514451in}{1.772351in}}{\pgfqpoint{2.511846in}{1.772351in}}%
\pgfpathcurveto{\pgfqpoint{2.509242in}{1.772351in}}{\pgfqpoint{2.506743in}{1.771316in}}{\pgfqpoint{2.504902in}{1.769474in}}%
\pgfpathcurveto{\pgfqpoint{2.503060in}{1.767633in}}{\pgfqpoint{2.502025in}{1.765134in}}{\pgfqpoint{2.502025in}{1.762530in}}%
\pgfpathcurveto{\pgfqpoint{2.502025in}{1.759925in}}{\pgfqpoint{2.503060in}{1.757427in}}{\pgfqpoint{2.504902in}{1.755585in}}%
\pgfpathcurveto{\pgfqpoint{2.506743in}{1.753744in}}{\pgfqpoint{2.509242in}{1.752709in}}{\pgfqpoint{2.511846in}{1.752709in}}%
\pgfpathclose%
\pgfusepath{stroke,fill}%
\end{pgfscope}%
\begin{pgfscope}%
\pgfpathrectangle{\pgfqpoint{0.750000in}{0.500000in}}{\pgfqpoint{4.650000in}{3.020000in}}%
\pgfusepath{clip}%
\pgfsetbuttcap%
\pgfsetroundjoin%
\definecolor{currentfill}{rgb}{0.121569,0.466667,0.705882}%
\pgfsetfillcolor{currentfill}%
\pgfsetlinewidth{1.003750pt}%
\definecolor{currentstroke}{rgb}{0.121569,0.466667,0.705882}%
\pgfsetstrokecolor{currentstroke}%
\pgfsetdash{}{0pt}%
\pgfpathmoveto{\pgfqpoint{1.512936in}{0.665856in}}%
\pgfpathcurveto{\pgfqpoint{1.515540in}{0.665856in}}{\pgfqpoint{1.518039in}{0.666891in}}{\pgfqpoint{1.519880in}{0.668733in}}%
\pgfpathcurveto{\pgfqpoint{1.521722in}{0.670575in}}{\pgfqpoint{1.522757in}{0.673073in}}{\pgfqpoint{1.522757in}{0.675677in}}%
\pgfpathcurveto{\pgfqpoint{1.522757in}{0.678282in}}{\pgfqpoint{1.521722in}{0.680780in}}{\pgfqpoint{1.519880in}{0.682622in}}%
\pgfpathcurveto{\pgfqpoint{1.518039in}{0.684464in}}{\pgfqpoint{1.515540in}{0.685498in}}{\pgfqpoint{1.512936in}{0.685498in}}%
\pgfpathcurveto{\pgfqpoint{1.510331in}{0.685498in}}{\pgfqpoint{1.507833in}{0.684464in}}{\pgfqpoint{1.505991in}{0.682622in}}%
\pgfpathcurveto{\pgfqpoint{1.504150in}{0.680780in}}{\pgfqpoint{1.503115in}{0.678282in}}{\pgfqpoint{1.503115in}{0.675677in}}%
\pgfpathcurveto{\pgfqpoint{1.503115in}{0.673073in}}{\pgfqpoint{1.504150in}{0.670575in}}{\pgfqpoint{1.505991in}{0.668733in}}%
\pgfpathcurveto{\pgfqpoint{1.507833in}{0.666891in}}{\pgfqpoint{1.510331in}{0.665856in}}{\pgfqpoint{1.512936in}{0.665856in}}%
\pgfpathclose%
\pgfusepath{stroke,fill}%
\end{pgfscope}%
\begin{pgfscope}%
\pgfpathrectangle{\pgfqpoint{0.750000in}{0.500000in}}{\pgfqpoint{4.650000in}{3.020000in}}%
\pgfusepath{clip}%
\pgfsetbuttcap%
\pgfsetroundjoin%
\definecolor{currentfill}{rgb}{0.121569,0.466667,0.705882}%
\pgfsetfillcolor{currentfill}%
\pgfsetlinewidth{1.003750pt}%
\definecolor{currentstroke}{rgb}{0.121569,0.466667,0.705882}%
\pgfsetstrokecolor{currentstroke}%
\pgfsetdash{}{0pt}%
\pgfpathmoveto{\pgfqpoint{2.165847in}{1.044623in}}%
\pgfpathcurveto{\pgfqpoint{2.168451in}{1.044623in}}{\pgfqpoint{2.170950in}{1.045657in}}{\pgfqpoint{2.172791in}{1.047499in}}%
\pgfpathcurveto{\pgfqpoint{2.174633in}{1.049341in}}{\pgfqpoint{2.175668in}{1.051839in}}{\pgfqpoint{2.175668in}{1.054444in}}%
\pgfpathcurveto{\pgfqpoint{2.175668in}{1.057048in}}{\pgfqpoint{2.174633in}{1.059546in}}{\pgfqpoint{2.172791in}{1.061388in}}%
\pgfpathcurveto{\pgfqpoint{2.170950in}{1.063230in}}{\pgfqpoint{2.168451in}{1.064264in}}{\pgfqpoint{2.165847in}{1.064264in}}%
\pgfpathcurveto{\pgfqpoint{2.163242in}{1.064264in}}{\pgfqpoint{2.160744in}{1.063230in}}{\pgfqpoint{2.158902in}{1.061388in}}%
\pgfpathcurveto{\pgfqpoint{2.157061in}{1.059546in}}{\pgfqpoint{2.156026in}{1.057048in}}{\pgfqpoint{2.156026in}{1.054444in}}%
\pgfpathcurveto{\pgfqpoint{2.156026in}{1.051839in}}{\pgfqpoint{2.157061in}{1.049341in}}{\pgfqpoint{2.158902in}{1.047499in}}%
\pgfpathcurveto{\pgfqpoint{2.160744in}{1.045657in}}{\pgfqpoint{2.163242in}{1.044623in}}{\pgfqpoint{2.165847in}{1.044623in}}%
\pgfpathclose%
\pgfusepath{stroke,fill}%
\end{pgfscope}%
\begin{pgfscope}%
\pgfpathrectangle{\pgfqpoint{0.750000in}{0.500000in}}{\pgfqpoint{4.650000in}{3.020000in}}%
\pgfusepath{clip}%
\pgfsetbuttcap%
\pgfsetroundjoin%
\definecolor{currentfill}{rgb}{0.121569,0.466667,0.705882}%
\pgfsetfillcolor{currentfill}%
\pgfsetlinewidth{1.003750pt}%
\definecolor{currentstroke}{rgb}{0.121569,0.466667,0.705882}%
\pgfsetstrokecolor{currentstroke}%
\pgfsetdash{}{0pt}%
\pgfpathmoveto{\pgfqpoint{1.547681in}{0.926528in}}%
\pgfpathcurveto{\pgfqpoint{1.550285in}{0.926528in}}{\pgfqpoint{1.552783in}{0.927563in}}{\pgfqpoint{1.554625in}{0.929405in}}%
\pgfpathcurveto{\pgfqpoint{1.556467in}{0.931246in}}{\pgfqpoint{1.557502in}{0.933745in}}{\pgfqpoint{1.557502in}{0.936349in}}%
\pgfpathcurveto{\pgfqpoint{1.557502in}{0.938954in}}{\pgfqpoint{1.556467in}{0.941452in}}{\pgfqpoint{1.554625in}{0.943294in}}%
\pgfpathcurveto{\pgfqpoint{1.552783in}{0.945135in}}{\pgfqpoint{1.550285in}{0.946170in}}{\pgfqpoint{1.547681in}{0.946170in}}%
\pgfpathcurveto{\pgfqpoint{1.545076in}{0.946170in}}{\pgfqpoint{1.542578in}{0.945135in}}{\pgfqpoint{1.540736in}{0.943294in}}%
\pgfpathcurveto{\pgfqpoint{1.538894in}{0.941452in}}{\pgfqpoint{1.537860in}{0.938954in}}{\pgfqpoint{1.537860in}{0.936349in}}%
\pgfpathcurveto{\pgfqpoint{1.537860in}{0.933745in}}{\pgfqpoint{1.538894in}{0.931246in}}{\pgfqpoint{1.540736in}{0.929405in}}%
\pgfpathcurveto{\pgfqpoint{1.542578in}{0.927563in}}{\pgfqpoint{1.545076in}{0.926528in}}{\pgfqpoint{1.547681in}{0.926528in}}%
\pgfpathclose%
\pgfusepath{stroke,fill}%
\end{pgfscope}%
\begin{pgfscope}%
\pgfpathrectangle{\pgfqpoint{0.750000in}{0.500000in}}{\pgfqpoint{4.650000in}{3.020000in}}%
\pgfusepath{clip}%
\pgfsetbuttcap%
\pgfsetroundjoin%
\definecolor{currentfill}{rgb}{0.121569,0.466667,0.705882}%
\pgfsetfillcolor{currentfill}%
\pgfsetlinewidth{1.003750pt}%
\definecolor{currentstroke}{rgb}{0.121569,0.466667,0.705882}%
\pgfsetstrokecolor{currentstroke}%
\pgfsetdash{}{0pt}%
\pgfpathmoveto{\pgfqpoint{1.227740in}{0.643774in}}%
\pgfpathcurveto{\pgfqpoint{1.230344in}{0.643774in}}{\pgfqpoint{1.232842in}{0.644809in}}{\pgfqpoint{1.234684in}{0.646650in}}%
\pgfpathcurveto{\pgfqpoint{1.236526in}{0.648492in}}{\pgfqpoint{1.237561in}{0.650990in}}{\pgfqpoint{1.237561in}{0.653595in}}%
\pgfpathcurveto{\pgfqpoint{1.237561in}{0.656199in}}{\pgfqpoint{1.236526in}{0.658697in}}{\pgfqpoint{1.234684in}{0.660539in}}%
\pgfpathcurveto{\pgfqpoint{1.232842in}{0.662381in}}{\pgfqpoint{1.230344in}{0.663416in}}{\pgfqpoint{1.227740in}{0.663416in}}%
\pgfpathcurveto{\pgfqpoint{1.225135in}{0.663416in}}{\pgfqpoint{1.222637in}{0.662381in}}{\pgfqpoint{1.220795in}{0.660539in}}%
\pgfpathcurveto{\pgfqpoint{1.218954in}{0.658697in}}{\pgfqpoint{1.217919in}{0.656199in}}{\pgfqpoint{1.217919in}{0.653595in}}%
\pgfpathcurveto{\pgfqpoint{1.217919in}{0.650990in}}{\pgfqpoint{1.218954in}{0.648492in}}{\pgfqpoint{1.220795in}{0.646650in}}%
\pgfpathcurveto{\pgfqpoint{1.222637in}{0.644809in}}{\pgfqpoint{1.225135in}{0.643774in}}{\pgfqpoint{1.227740in}{0.643774in}}%
\pgfpathclose%
\pgfusepath{stroke,fill}%
\end{pgfscope}%
\begin{pgfscope}%
\pgfpathrectangle{\pgfqpoint{0.750000in}{0.500000in}}{\pgfqpoint{4.650000in}{3.020000in}}%
\pgfusepath{clip}%
\pgfsetbuttcap%
\pgfsetroundjoin%
\definecolor{currentfill}{rgb}{0.121569,0.466667,0.705882}%
\pgfsetfillcolor{currentfill}%
\pgfsetlinewidth{1.003750pt}%
\definecolor{currentstroke}{rgb}{0.121569,0.466667,0.705882}%
\pgfsetstrokecolor{currentstroke}%
\pgfsetdash{}{0pt}%
\pgfpathmoveto{\pgfqpoint{1.824191in}{0.704454in}}%
\pgfpathcurveto{\pgfqpoint{1.827380in}{0.704454in}}{\pgfqpoint{1.830440in}{0.705722in}}{\pgfqpoint{1.832696in}{0.707977in}}%
\pgfpathcurveto{\pgfqpoint{1.834951in}{0.710233in}}{\pgfqpoint{1.836219in}{0.713292in}}{\pgfqpoint{1.836219in}{0.716482in}}%
\pgfpathcurveto{\pgfqpoint{1.836219in}{0.719672in}}{\pgfqpoint{1.834951in}{0.722732in}}{\pgfqpoint{1.832696in}{0.724988in}}%
\pgfpathcurveto{\pgfqpoint{1.830440in}{0.727243in}}{\pgfqpoint{1.827380in}{0.728511in}}{\pgfqpoint{1.824191in}{0.728511in}}%
\pgfpathcurveto{\pgfqpoint{1.821001in}{0.728511in}}{\pgfqpoint{1.817941in}{0.727243in}}{\pgfqpoint{1.815685in}{0.724988in}}%
\pgfpathcurveto{\pgfqpoint{1.813430in}{0.722732in}}{\pgfqpoint{1.812162in}{0.719672in}}{\pgfqpoint{1.812162in}{0.716482in}}%
\pgfpathcurveto{\pgfqpoint{1.812162in}{0.713292in}}{\pgfqpoint{1.813430in}{0.710233in}}{\pgfqpoint{1.815685in}{0.707977in}}%
\pgfpathcurveto{\pgfqpoint{1.817941in}{0.705722in}}{\pgfqpoint{1.821001in}{0.704454in}}{\pgfqpoint{1.824191in}{0.704454in}}%
\pgfpathclose%
\pgfusepath{stroke,fill}%
\end{pgfscope}%
\begin{pgfscope}%
\pgfpathrectangle{\pgfqpoint{0.750000in}{0.500000in}}{\pgfqpoint{4.650000in}{3.020000in}}%
\pgfusepath{clip}%
\pgfsetbuttcap%
\pgfsetroundjoin%
\definecolor{currentfill}{rgb}{0.121569,0.466667,0.705882}%
\pgfsetfillcolor{currentfill}%
\pgfsetlinewidth{1.003750pt}%
\definecolor{currentstroke}{rgb}{0.121569,0.466667,0.705882}%
\pgfsetstrokecolor{currentstroke}%
\pgfsetdash{}{0pt}%
\pgfpathmoveto{\pgfqpoint{0.967154in}{0.631292in}}%
\pgfpathcurveto{\pgfqpoint{0.969759in}{0.631292in}}{\pgfqpoint{0.972257in}{0.632327in}}{\pgfqpoint{0.974099in}{0.634169in}}%
\pgfpathcurveto{\pgfqpoint{0.975941in}{0.636010in}}{\pgfqpoint{0.976975in}{0.638509in}}{\pgfqpoint{0.976975in}{0.641113in}}%
\pgfpathcurveto{\pgfqpoint{0.976975in}{0.643718in}}{\pgfqpoint{0.975941in}{0.646216in}}{\pgfqpoint{0.974099in}{0.648058in}}%
\pgfpathcurveto{\pgfqpoint{0.972257in}{0.649899in}}{\pgfqpoint{0.969759in}{0.650934in}}{\pgfqpoint{0.967154in}{0.650934in}}%
\pgfpathcurveto{\pgfqpoint{0.964550in}{0.650934in}}{\pgfqpoint{0.962052in}{0.649899in}}{\pgfqpoint{0.960210in}{0.648058in}}%
\pgfpathcurveto{\pgfqpoint{0.958368in}{0.646216in}}{\pgfqpoint{0.957333in}{0.643718in}}{\pgfqpoint{0.957333in}{0.641113in}}%
\pgfpathcurveto{\pgfqpoint{0.957333in}{0.638509in}}{\pgfqpoint{0.958368in}{0.636010in}}{\pgfqpoint{0.960210in}{0.634169in}}%
\pgfpathcurveto{\pgfqpoint{0.962052in}{0.632327in}}{\pgfqpoint{0.964550in}{0.631292in}}{\pgfqpoint{0.967154in}{0.631292in}}%
\pgfpathclose%
\pgfusepath{stroke,fill}%
\end{pgfscope}%
\begin{pgfscope}%
\pgfpathrectangle{\pgfqpoint{0.750000in}{0.500000in}}{\pgfqpoint{4.650000in}{3.020000in}}%
\pgfusepath{clip}%
\pgfsetbuttcap%
\pgfsetroundjoin%
\definecolor{currentfill}{rgb}{0.121569,0.466667,0.705882}%
\pgfsetfillcolor{currentfill}%
\pgfsetlinewidth{1.003750pt}%
\definecolor{currentstroke}{rgb}{0.121569,0.466667,0.705882}%
\pgfsetstrokecolor{currentstroke}%
\pgfsetdash{}{0pt}%
\pgfpathmoveto{\pgfqpoint{1.576634in}{0.670657in}}%
\pgfpathcurveto{\pgfqpoint{1.579239in}{0.670657in}}{\pgfqpoint{1.581737in}{0.671692in}}{\pgfqpoint{1.583579in}{0.673534in}}%
\pgfpathcurveto{\pgfqpoint{1.585421in}{0.675375in}}{\pgfqpoint{1.586455in}{0.677873in}}{\pgfqpoint{1.586455in}{0.680478in}}%
\pgfpathcurveto{\pgfqpoint{1.586455in}{0.683083in}}{\pgfqpoint{1.585421in}{0.685581in}}{\pgfqpoint{1.583579in}{0.687422in}}%
\pgfpathcurveto{\pgfqpoint{1.581737in}{0.689264in}}{\pgfqpoint{1.579239in}{0.690299in}}{\pgfqpoint{1.576634in}{0.690299in}}%
\pgfpathcurveto{\pgfqpoint{1.574030in}{0.690299in}}{\pgfqpoint{1.571532in}{0.689264in}}{\pgfqpoint{1.569690in}{0.687422in}}%
\pgfpathcurveto{\pgfqpoint{1.567848in}{0.685581in}}{\pgfqpoint{1.566814in}{0.683083in}}{\pgfqpoint{1.566814in}{0.680478in}}%
\pgfpathcurveto{\pgfqpoint{1.566814in}{0.677873in}}{\pgfqpoint{1.567848in}{0.675375in}}{\pgfqpoint{1.569690in}{0.673534in}}%
\pgfpathcurveto{\pgfqpoint{1.571532in}{0.671692in}}{\pgfqpoint{1.574030in}{0.670657in}}{\pgfqpoint{1.576634in}{0.670657in}}%
\pgfpathclose%
\pgfusepath{stroke,fill}%
\end{pgfscope}%
\begin{pgfscope}%
\pgfpathrectangle{\pgfqpoint{0.750000in}{0.500000in}}{\pgfqpoint{4.650000in}{3.020000in}}%
\pgfusepath{clip}%
\pgfsetbuttcap%
\pgfsetroundjoin%
\definecolor{currentfill}{rgb}{0.121569,0.466667,0.705882}%
\pgfsetfillcolor{currentfill}%
\pgfsetlinewidth{1.003750pt}%
\definecolor{currentstroke}{rgb}{0.121569,0.466667,0.705882}%
\pgfsetstrokecolor{currentstroke}%
\pgfsetdash{}{0pt}%
\pgfpathmoveto{\pgfqpoint{1.546233in}{0.700421in}}%
\pgfpathcurveto{\pgfqpoint{1.548837in}{0.700421in}}{\pgfqpoint{1.551336in}{0.701455in}}{\pgfqpoint{1.553177in}{0.703297in}}%
\pgfpathcurveto{\pgfqpoint{1.555019in}{0.705139in}}{\pgfqpoint{1.556054in}{0.707637in}}{\pgfqpoint{1.556054in}{0.710242in}}%
\pgfpathcurveto{\pgfqpoint{1.556054in}{0.712846in}}{\pgfqpoint{1.555019in}{0.715344in}}{\pgfqpoint{1.553177in}{0.717186in}}%
\pgfpathcurveto{\pgfqpoint{1.551336in}{0.719028in}}{\pgfqpoint{1.548837in}{0.720063in}}{\pgfqpoint{1.546233in}{0.720063in}}%
\pgfpathcurveto{\pgfqpoint{1.543628in}{0.720063in}}{\pgfqpoint{1.541130in}{0.719028in}}{\pgfqpoint{1.539288in}{0.717186in}}%
\pgfpathcurveto{\pgfqpoint{1.537447in}{0.715344in}}{\pgfqpoint{1.536412in}{0.712846in}}{\pgfqpoint{1.536412in}{0.710242in}}%
\pgfpathcurveto{\pgfqpoint{1.536412in}{0.707637in}}{\pgfqpoint{1.537447in}{0.705139in}}{\pgfqpoint{1.539288in}{0.703297in}}%
\pgfpathcurveto{\pgfqpoint{1.541130in}{0.701455in}}{\pgfqpoint{1.543628in}{0.700421in}}{\pgfqpoint{1.546233in}{0.700421in}}%
\pgfpathclose%
\pgfusepath{stroke,fill}%
\end{pgfscope}%
\begin{pgfscope}%
\pgfpathrectangle{\pgfqpoint{0.750000in}{0.500000in}}{\pgfqpoint{4.650000in}{3.020000in}}%
\pgfusepath{clip}%
\pgfsetbuttcap%
\pgfsetroundjoin%
\definecolor{currentfill}{rgb}{0.121569,0.466667,0.705882}%
\pgfsetfillcolor{currentfill}%
\pgfsetlinewidth{1.003750pt}%
\definecolor{currentstroke}{rgb}{0.121569,0.466667,0.705882}%
\pgfsetstrokecolor{currentstroke}%
\pgfsetdash{}{0pt}%
\pgfpathmoveto{\pgfqpoint{1.521622in}{0.737385in}}%
\pgfpathcurveto{\pgfqpoint{1.524227in}{0.737385in}}{\pgfqpoint{1.526725in}{0.738420in}}{\pgfqpoint{1.528566in}{0.740262in}}%
\pgfpathcurveto{\pgfqpoint{1.530408in}{0.742103in}}{\pgfqpoint{1.531443in}{0.744602in}}{\pgfqpoint{1.531443in}{0.747206in}}%
\pgfpathcurveto{\pgfqpoint{1.531443in}{0.749811in}}{\pgfqpoint{1.530408in}{0.752309in}}{\pgfqpoint{1.528566in}{0.754151in}}%
\pgfpathcurveto{\pgfqpoint{1.526725in}{0.755992in}}{\pgfqpoint{1.524227in}{0.757027in}}{\pgfqpoint{1.521622in}{0.757027in}}%
\pgfpathcurveto{\pgfqpoint{1.519018in}{0.757027in}}{\pgfqpoint{1.516519in}{0.755992in}}{\pgfqpoint{1.514678in}{0.754151in}}%
\pgfpathcurveto{\pgfqpoint{1.512836in}{0.752309in}}{\pgfqpoint{1.511801in}{0.749811in}}{\pgfqpoint{1.511801in}{0.747206in}}%
\pgfpathcurveto{\pgfqpoint{1.511801in}{0.744602in}}{\pgfqpoint{1.512836in}{0.742103in}}{\pgfqpoint{1.514678in}{0.740262in}}%
\pgfpathcurveto{\pgfqpoint{1.516519in}{0.738420in}}{\pgfqpoint{1.519018in}{0.737385in}}{\pgfqpoint{1.521622in}{0.737385in}}%
\pgfpathclose%
\pgfusepath{stroke,fill}%
\end{pgfscope}%
\begin{pgfscope}%
\pgfpathrectangle{\pgfqpoint{0.750000in}{0.500000in}}{\pgfqpoint{4.650000in}{3.020000in}}%
\pgfusepath{clip}%
\pgfsetbuttcap%
\pgfsetroundjoin%
\definecolor{currentfill}{rgb}{0.121569,0.466667,0.705882}%
\pgfsetfillcolor{currentfill}%
\pgfsetlinewidth{1.003750pt}%
\definecolor{currentstroke}{rgb}{0.121569,0.466667,0.705882}%
\pgfsetstrokecolor{currentstroke}%
\pgfsetdash{}{0pt}%
\pgfpathmoveto{\pgfqpoint{1.549128in}{0.780303in}}%
\pgfpathcurveto{\pgfqpoint{1.552318in}{0.780303in}}{\pgfqpoint{1.555378in}{0.781571in}}{\pgfqpoint{1.557633in}{0.783826in}}%
\pgfpathcurveto{\pgfqpoint{1.559889in}{0.786082in}}{\pgfqpoint{1.561156in}{0.789142in}}{\pgfqpoint{1.561156in}{0.792332in}}%
\pgfpathcurveto{\pgfqpoint{1.561156in}{0.795522in}}{\pgfqpoint{1.559889in}{0.798581in}}{\pgfqpoint{1.557633in}{0.800837in}}%
\pgfpathcurveto{\pgfqpoint{1.555378in}{0.803092in}}{\pgfqpoint{1.552318in}{0.804360in}}{\pgfqpoint{1.549128in}{0.804360in}}%
\pgfpathcurveto{\pgfqpoint{1.545938in}{0.804360in}}{\pgfqpoint{1.542879in}{0.803092in}}{\pgfqpoint{1.540623in}{0.800837in}}%
\pgfpathcurveto{\pgfqpoint{1.538367in}{0.798581in}}{\pgfqpoint{1.537100in}{0.795522in}}{\pgfqpoint{1.537100in}{0.792332in}}%
\pgfpathcurveto{\pgfqpoint{1.537100in}{0.789142in}}{\pgfqpoint{1.538367in}{0.786082in}}{\pgfqpoint{1.540623in}{0.783826in}}%
\pgfpathcurveto{\pgfqpoint{1.542879in}{0.781571in}}{\pgfqpoint{1.545938in}{0.780303in}}{\pgfqpoint{1.549128in}{0.780303in}}%
\pgfpathclose%
\pgfusepath{stroke,fill}%
\end{pgfscope}%
\begin{pgfscope}%
\pgfpathrectangle{\pgfqpoint{0.750000in}{0.500000in}}{\pgfqpoint{4.650000in}{3.020000in}}%
\pgfusepath{clip}%
\pgfsetbuttcap%
\pgfsetroundjoin%
\definecolor{currentfill}{rgb}{0.121569,0.466667,0.705882}%
\pgfsetfillcolor{currentfill}%
\pgfsetlinewidth{1.003750pt}%
\definecolor{currentstroke}{rgb}{0.121569,0.466667,0.705882}%
\pgfsetstrokecolor{currentstroke}%
\pgfsetdash{}{0pt}%
\pgfpathmoveto{\pgfqpoint{1.721404in}{0.721063in}}%
\pgfpathcurveto{\pgfqpoint{1.724009in}{0.721063in}}{\pgfqpoint{1.726507in}{0.722098in}}{\pgfqpoint{1.728349in}{0.723940in}}%
\pgfpathcurveto{\pgfqpoint{1.730190in}{0.725781in}}{\pgfqpoint{1.731225in}{0.728280in}}{\pgfqpoint{1.731225in}{0.730884in}}%
\pgfpathcurveto{\pgfqpoint{1.731225in}{0.733489in}}{\pgfqpoint{1.730190in}{0.735987in}}{\pgfqpoint{1.728349in}{0.737829in}}%
\pgfpathcurveto{\pgfqpoint{1.726507in}{0.739670in}}{\pgfqpoint{1.724009in}{0.740705in}}{\pgfqpoint{1.721404in}{0.740705in}}%
\pgfpathcurveto{\pgfqpoint{1.718800in}{0.740705in}}{\pgfqpoint{1.716301in}{0.739670in}}{\pgfqpoint{1.714460in}{0.737829in}}%
\pgfpathcurveto{\pgfqpoint{1.712618in}{0.735987in}}{\pgfqpoint{1.711583in}{0.733489in}}{\pgfqpoint{1.711583in}{0.730884in}}%
\pgfpathcurveto{\pgfqpoint{1.711583in}{0.728280in}}{\pgfqpoint{1.712618in}{0.725781in}}{\pgfqpoint{1.714460in}{0.723940in}}%
\pgfpathcurveto{\pgfqpoint{1.716301in}{0.722098in}}{\pgfqpoint{1.718800in}{0.721063in}}{\pgfqpoint{1.721404in}{0.721063in}}%
\pgfpathclose%
\pgfusepath{stroke,fill}%
\end{pgfscope}%
\begin{pgfscope}%
\pgfpathrectangle{\pgfqpoint{0.750000in}{0.500000in}}{\pgfqpoint{4.650000in}{3.020000in}}%
\pgfusepath{clip}%
\pgfsetbuttcap%
\pgfsetroundjoin%
\definecolor{currentfill}{rgb}{0.121569,0.466667,0.705882}%
\pgfsetfillcolor{currentfill}%
\pgfsetlinewidth{1.003750pt}%
\definecolor{currentstroke}{rgb}{0.121569,0.466667,0.705882}%
\pgfsetstrokecolor{currentstroke}%
\pgfsetdash{}{0pt}%
\pgfpathmoveto{\pgfqpoint{1.152460in}{0.656735in}}%
\pgfpathcurveto{\pgfqpoint{1.155064in}{0.656735in}}{\pgfqpoint{1.157562in}{0.657770in}}{\pgfqpoint{1.159404in}{0.659612in}}%
\pgfpathcurveto{\pgfqpoint{1.161246in}{0.661454in}}{\pgfqpoint{1.162280in}{0.663952in}}{\pgfqpoint{1.162280in}{0.666556in}}%
\pgfpathcurveto{\pgfqpoint{1.162280in}{0.669161in}}{\pgfqpoint{1.161246in}{0.671659in}}{\pgfqpoint{1.159404in}{0.673501in}}%
\pgfpathcurveto{\pgfqpoint{1.157562in}{0.675342in}}{\pgfqpoint{1.155064in}{0.676377in}}{\pgfqpoint{1.152460in}{0.676377in}}%
\pgfpathcurveto{\pgfqpoint{1.149855in}{0.676377in}}{\pgfqpoint{1.147357in}{0.675342in}}{\pgfqpoint{1.145515in}{0.673501in}}%
\pgfpathcurveto{\pgfqpoint{1.143673in}{0.671659in}}{\pgfqpoint{1.142639in}{0.669161in}}{\pgfqpoint{1.142639in}{0.666556in}}%
\pgfpathcurveto{\pgfqpoint{1.142639in}{0.663952in}}{\pgfqpoint{1.143673in}{0.661454in}}{\pgfqpoint{1.145515in}{0.659612in}}%
\pgfpathcurveto{\pgfqpoint{1.147357in}{0.657770in}}{\pgfqpoint{1.149855in}{0.656735in}}{\pgfqpoint{1.152460in}{0.656735in}}%
\pgfpathclose%
\pgfusepath{stroke,fill}%
\end{pgfscope}%
\begin{pgfscope}%
\pgfpathrectangle{\pgfqpoint{0.750000in}{0.500000in}}{\pgfqpoint{4.650000in}{3.020000in}}%
\pgfusepath{clip}%
\pgfsetbuttcap%
\pgfsetroundjoin%
\definecolor{currentfill}{rgb}{0.121569,0.466667,0.705882}%
\pgfsetfillcolor{currentfill}%
\pgfsetlinewidth{1.003750pt}%
\definecolor{currentstroke}{rgb}{0.121569,0.466667,0.705882}%
\pgfsetstrokecolor{currentstroke}%
\pgfsetdash{}{0pt}%
\pgfpathmoveto{\pgfqpoint{3.161862in}{2.359314in}}%
\pgfpathcurveto{\pgfqpoint{3.170500in}{2.359314in}}{\pgfqpoint{3.178786in}{2.362746in}}{\pgfqpoint{3.184894in}{2.368854in}}%
\pgfpathcurveto{\pgfqpoint{3.191002in}{2.374963in}}{\pgfqpoint{3.194434in}{2.383248in}}{\pgfqpoint{3.194434in}{2.391887in}}%
\pgfpathcurveto{\pgfqpoint{3.194434in}{2.400525in}}{\pgfqpoint{3.191002in}{2.408810in}}{\pgfqpoint{3.184894in}{2.414919in}}%
\pgfpathcurveto{\pgfqpoint{3.178786in}{2.421027in}}{\pgfqpoint{3.170500in}{2.424459in}}{\pgfqpoint{3.161862in}{2.424459in}}%
\pgfpathcurveto{\pgfqpoint{3.153223in}{2.424459in}}{\pgfqpoint{3.144938in}{2.421027in}}{\pgfqpoint{3.138830in}{2.414919in}}%
\pgfpathcurveto{\pgfqpoint{3.132721in}{2.408810in}}{\pgfqpoint{3.129289in}{2.400525in}}{\pgfqpoint{3.129289in}{2.391887in}}%
\pgfpathcurveto{\pgfqpoint{3.129289in}{2.383248in}}{\pgfqpoint{3.132721in}{2.374963in}}{\pgfqpoint{3.138830in}{2.368854in}}%
\pgfpathcurveto{\pgfqpoint{3.144938in}{2.362746in}}{\pgfqpoint{3.153223in}{2.359314in}}{\pgfqpoint{3.161862in}{2.359314in}}%
\pgfpathclose%
\pgfusepath{stroke,fill}%
\end{pgfscope}%
\begin{pgfscope}%
\pgfpathrectangle{\pgfqpoint{0.750000in}{0.500000in}}{\pgfqpoint{4.650000in}{3.020000in}}%
\pgfusepath{clip}%
\pgfsetbuttcap%
\pgfsetroundjoin%
\definecolor{currentfill}{rgb}{0.121569,0.466667,0.705882}%
\pgfsetfillcolor{currentfill}%
\pgfsetlinewidth{1.003750pt}%
\definecolor{currentstroke}{rgb}{0.121569,0.466667,0.705882}%
\pgfsetstrokecolor{currentstroke}%
\pgfsetdash{}{0pt}%
\pgfpathmoveto{\pgfqpoint{2.972214in}{2.564299in}}%
\pgfpathcurveto{\pgfqpoint{2.980852in}{2.564299in}}{\pgfqpoint{2.989138in}{2.567731in}}{\pgfqpoint{2.995246in}{2.573839in}}%
\pgfpathcurveto{\pgfqpoint{3.001354in}{2.579948in}}{\pgfqpoint{3.004786in}{2.588233in}}{\pgfqpoint{3.004786in}{2.596872in}}%
\pgfpathcurveto{\pgfqpoint{3.004786in}{2.605510in}}{\pgfqpoint{3.001354in}{2.613795in}}{\pgfqpoint{2.995246in}{2.619904in}}%
\pgfpathcurveto{\pgfqpoint{2.989138in}{2.626012in}}{\pgfqpoint{2.980852in}{2.629444in}}{\pgfqpoint{2.972214in}{2.629444in}}%
\pgfpathcurveto{\pgfqpoint{2.963575in}{2.629444in}}{\pgfqpoint{2.955290in}{2.626012in}}{\pgfqpoint{2.949181in}{2.619904in}}%
\pgfpathcurveto{\pgfqpoint{2.943073in}{2.613795in}}{\pgfqpoint{2.939641in}{2.605510in}}{\pgfqpoint{2.939641in}{2.596872in}}%
\pgfpathcurveto{\pgfqpoint{2.939641in}{2.588233in}}{\pgfqpoint{2.943073in}{2.579948in}}{\pgfqpoint{2.949181in}{2.573839in}}%
\pgfpathcurveto{\pgfqpoint{2.955290in}{2.567731in}}{\pgfqpoint{2.963575in}{2.564299in}}{\pgfqpoint{2.972214in}{2.564299in}}%
\pgfpathclose%
\pgfusepath{stroke,fill}%
\end{pgfscope}%
\begin{pgfscope}%
\pgfpathrectangle{\pgfqpoint{0.750000in}{0.500000in}}{\pgfqpoint{4.650000in}{3.020000in}}%
\pgfusepath{clip}%
\pgfsetbuttcap%
\pgfsetroundjoin%
\definecolor{currentfill}{rgb}{0.121569,0.466667,0.705882}%
\pgfsetfillcolor{currentfill}%
\pgfsetlinewidth{1.003750pt}%
\definecolor{currentstroke}{rgb}{0.121569,0.466667,0.705882}%
\pgfsetstrokecolor{currentstroke}%
\pgfsetdash{}{0pt}%
\pgfpathmoveto{\pgfqpoint{1.706927in}{0.761730in}}%
\pgfpathcurveto{\pgfqpoint{1.712751in}{0.761730in}}{\pgfqpoint{1.718337in}{0.764044in}}{\pgfqpoint{1.722455in}{0.768162in}}%
\pgfpathcurveto{\pgfqpoint{1.726574in}{0.772280in}}{\pgfqpoint{1.728887in}{0.777867in}}{\pgfqpoint{1.728887in}{0.783691in}}%
\pgfpathcurveto{\pgfqpoint{1.728887in}{0.789514in}}{\pgfqpoint{1.726574in}{0.795101in}}{\pgfqpoint{1.722455in}{0.799219in}}%
\pgfpathcurveto{\pgfqpoint{1.718337in}{0.803337in}}{\pgfqpoint{1.712751in}{0.805651in}}{\pgfqpoint{1.706927in}{0.805651in}}%
\pgfpathcurveto{\pgfqpoint{1.701103in}{0.805651in}}{\pgfqpoint{1.695517in}{0.803337in}}{\pgfqpoint{1.691399in}{0.799219in}}%
\pgfpathcurveto{\pgfqpoint{1.687281in}{0.795101in}}{\pgfqpoint{1.684967in}{0.789514in}}{\pgfqpoint{1.684967in}{0.783691in}}%
\pgfpathcurveto{\pgfqpoint{1.684967in}{0.777867in}}{\pgfqpoint{1.687281in}{0.772280in}}{\pgfqpoint{1.691399in}{0.768162in}}%
\pgfpathcurveto{\pgfqpoint{1.695517in}{0.764044in}}{\pgfqpoint{1.701103in}{0.761730in}}{\pgfqpoint{1.706927in}{0.761730in}}%
\pgfpathclose%
\pgfusepath{stroke,fill}%
\end{pgfscope}%
\begin{pgfscope}%
\pgfpathrectangle{\pgfqpoint{0.750000in}{0.500000in}}{\pgfqpoint{4.650000in}{3.020000in}}%
\pgfusepath{clip}%
\pgfsetbuttcap%
\pgfsetroundjoin%
\definecolor{currentfill}{rgb}{0.121569,0.466667,0.705882}%
\pgfsetfillcolor{currentfill}%
\pgfsetlinewidth{1.003750pt}%
\definecolor{currentstroke}{rgb}{0.121569,0.466667,0.705882}%
\pgfsetstrokecolor{currentstroke}%
\pgfsetdash{}{0pt}%
\pgfpathmoveto{\pgfqpoint{1.249455in}{0.778863in}}%
\pgfpathcurveto{\pgfqpoint{1.252645in}{0.778863in}}{\pgfqpoint{1.255705in}{0.780131in}}{\pgfqpoint{1.257960in}{0.782386in}}%
\pgfpathcurveto{\pgfqpoint{1.260216in}{0.784642in}}{\pgfqpoint{1.261483in}{0.787702in}}{\pgfqpoint{1.261483in}{0.790891in}}%
\pgfpathcurveto{\pgfqpoint{1.261483in}{0.794081in}}{\pgfqpoint{1.260216in}{0.797141in}}{\pgfqpoint{1.257960in}{0.799397in}}%
\pgfpathcurveto{\pgfqpoint{1.255705in}{0.801652in}}{\pgfqpoint{1.252645in}{0.802920in}}{\pgfqpoint{1.249455in}{0.802920in}}%
\pgfpathcurveto{\pgfqpoint{1.246265in}{0.802920in}}{\pgfqpoint{1.243206in}{0.801652in}}{\pgfqpoint{1.240950in}{0.799397in}}%
\pgfpathcurveto{\pgfqpoint{1.238694in}{0.797141in}}{\pgfqpoint{1.237427in}{0.794081in}}{\pgfqpoint{1.237427in}{0.790891in}}%
\pgfpathcurveto{\pgfqpoint{1.237427in}{0.787702in}}{\pgfqpoint{1.238694in}{0.784642in}}{\pgfqpoint{1.240950in}{0.782386in}}%
\pgfpathcurveto{\pgfqpoint{1.243206in}{0.780131in}}{\pgfqpoint{1.246265in}{0.778863in}}{\pgfqpoint{1.249455in}{0.778863in}}%
\pgfpathclose%
\pgfusepath{stroke,fill}%
\end{pgfscope}%
\begin{pgfscope}%
\pgfpathrectangle{\pgfqpoint{0.750000in}{0.500000in}}{\pgfqpoint{4.650000in}{3.020000in}}%
\pgfusepath{clip}%
\pgfsetbuttcap%
\pgfsetroundjoin%
\definecolor{currentfill}{rgb}{0.121569,0.466667,0.705882}%
\pgfsetfillcolor{currentfill}%
\pgfsetlinewidth{1.003750pt}%
\definecolor{currentstroke}{rgb}{0.121569,0.466667,0.705882}%
\pgfsetstrokecolor{currentstroke}%
\pgfsetdash{}{0pt}%
\pgfpathmoveto{\pgfqpoint{1.867621in}{0.858073in}}%
\pgfpathcurveto{\pgfqpoint{1.875649in}{0.858073in}}{\pgfqpoint{1.883349in}{0.861263in}}{\pgfqpoint{1.889026in}{0.866939in}}%
\pgfpathcurveto{\pgfqpoint{1.894702in}{0.872616in}}{\pgfqpoint{1.897892in}{0.880316in}}{\pgfqpoint{1.897892in}{0.888343in}}%
\pgfpathcurveto{\pgfqpoint{1.897892in}{0.896371in}}{\pgfqpoint{1.894702in}{0.904071in}}{\pgfqpoint{1.889026in}{0.909748in}}%
\pgfpathcurveto{\pgfqpoint{1.883349in}{0.915424in}}{\pgfqpoint{1.875649in}{0.918613in}}{\pgfqpoint{1.867621in}{0.918613in}}%
\pgfpathcurveto{\pgfqpoint{1.859594in}{0.918613in}}{\pgfqpoint{1.851894in}{0.915424in}}{\pgfqpoint{1.846217in}{0.909748in}}%
\pgfpathcurveto{\pgfqpoint{1.840541in}{0.904071in}}{\pgfqpoint{1.837351in}{0.896371in}}{\pgfqpoint{1.837351in}{0.888343in}}%
\pgfpathcurveto{\pgfqpoint{1.837351in}{0.880316in}}{\pgfqpoint{1.840541in}{0.872616in}}{\pgfqpoint{1.846217in}{0.866939in}}%
\pgfpathcurveto{\pgfqpoint{1.851894in}{0.861263in}}{\pgfqpoint{1.859594in}{0.858073in}}{\pgfqpoint{1.867621in}{0.858073in}}%
\pgfpathclose%
\pgfusepath{stroke,fill}%
\end{pgfscope}%
\begin{pgfscope}%
\pgfpathrectangle{\pgfqpoint{0.750000in}{0.500000in}}{\pgfqpoint{4.650000in}{3.020000in}}%
\pgfusepath{clip}%
\pgfsetbuttcap%
\pgfsetroundjoin%
\definecolor{currentfill}{rgb}{0.121569,0.466667,0.705882}%
\pgfsetfillcolor{currentfill}%
\pgfsetlinewidth{1.003750pt}%
\definecolor{currentstroke}{rgb}{0.121569,0.466667,0.705882}%
\pgfsetstrokecolor{currentstroke}%
\pgfsetdash{}{0pt}%
\pgfpathmoveto{\pgfqpoint{2.642139in}{0.812947in}}%
\pgfpathcurveto{\pgfqpoint{2.645329in}{0.812947in}}{\pgfqpoint{2.648388in}{0.814215in}}{\pgfqpoint{2.650644in}{0.816470in}}%
\pgfpathcurveto{\pgfqpoint{2.652900in}{0.818726in}}{\pgfqpoint{2.654167in}{0.821786in}}{\pgfqpoint{2.654167in}{0.824976in}}%
\pgfpathcurveto{\pgfqpoint{2.654167in}{0.828165in}}{\pgfqpoint{2.652900in}{0.831225in}}{\pgfqpoint{2.650644in}{0.833481in}}%
\pgfpathcurveto{\pgfqpoint{2.648388in}{0.835736in}}{\pgfqpoint{2.645329in}{0.837004in}}{\pgfqpoint{2.642139in}{0.837004in}}%
\pgfpathcurveto{\pgfqpoint{2.638949in}{0.837004in}}{\pgfqpoint{2.635889in}{0.835736in}}{\pgfqpoint{2.633634in}{0.833481in}}%
\pgfpathcurveto{\pgfqpoint{2.631378in}{0.831225in}}{\pgfqpoint{2.630111in}{0.828165in}}{\pgfqpoint{2.630111in}{0.824976in}}%
\pgfpathcurveto{\pgfqpoint{2.630111in}{0.821786in}}{\pgfqpoint{2.631378in}{0.818726in}}{\pgfqpoint{2.633634in}{0.816470in}}%
\pgfpathcurveto{\pgfqpoint{2.635889in}{0.814215in}}{\pgfqpoint{2.638949in}{0.812947in}}{\pgfqpoint{2.642139in}{0.812947in}}%
\pgfpathclose%
\pgfusepath{stroke,fill}%
\end{pgfscope}%
\begin{pgfscope}%
\pgfpathrectangle{\pgfqpoint{0.750000in}{0.500000in}}{\pgfqpoint{4.650000in}{3.020000in}}%
\pgfusepath{clip}%
\pgfsetbuttcap%
\pgfsetroundjoin%
\definecolor{currentfill}{rgb}{0.121569,0.466667,0.705882}%
\pgfsetfillcolor{currentfill}%
\pgfsetlinewidth{1.003750pt}%
\definecolor{currentstroke}{rgb}{0.121569,0.466667,0.705882}%
\pgfsetstrokecolor{currentstroke}%
\pgfsetdash{}{0pt}%
\pgfpathmoveto{\pgfqpoint{2.326541in}{1.101789in}}%
\pgfpathcurveto{\pgfqpoint{2.335374in}{1.101789in}}{\pgfqpoint{2.343845in}{1.105298in}}{\pgfqpoint{2.350091in}{1.111544in}}%
\pgfpathcurveto{\pgfqpoint{2.356336in}{1.117789in}}{\pgfqpoint{2.359845in}{1.126261in}}{\pgfqpoint{2.359845in}{1.135093in}}%
\pgfpathcurveto{\pgfqpoint{2.359845in}{1.143926in}}{\pgfqpoint{2.356336in}{1.152398in}}{\pgfqpoint{2.350091in}{1.158643in}}%
\pgfpathcurveto{\pgfqpoint{2.343845in}{1.164889in}}{\pgfqpoint{2.335374in}{1.168398in}}{\pgfqpoint{2.326541in}{1.168398in}}%
\pgfpathcurveto{\pgfqpoint{2.317709in}{1.168398in}}{\pgfqpoint{2.309237in}{1.164889in}}{\pgfqpoint{2.302991in}{1.158643in}}%
\pgfpathcurveto{\pgfqpoint{2.296746in}{1.152398in}}{\pgfqpoint{2.293237in}{1.143926in}}{\pgfqpoint{2.293237in}{1.135093in}}%
\pgfpathcurveto{\pgfqpoint{2.293237in}{1.126261in}}{\pgfqpoint{2.296746in}{1.117789in}}{\pgfqpoint{2.302991in}{1.111544in}}%
\pgfpathcurveto{\pgfqpoint{2.309237in}{1.105298in}}{\pgfqpoint{2.317709in}{1.101789in}}{\pgfqpoint{2.326541in}{1.101789in}}%
\pgfpathclose%
\pgfusepath{stroke,fill}%
\end{pgfscope}%
\begin{pgfscope}%
\pgfpathrectangle{\pgfqpoint{0.750000in}{0.500000in}}{\pgfqpoint{4.650000in}{3.020000in}}%
\pgfusepath{clip}%
\pgfsetbuttcap%
\pgfsetroundjoin%
\definecolor{currentfill}{rgb}{0.121569,0.466667,0.705882}%
\pgfsetfillcolor{currentfill}%
\pgfsetlinewidth{1.003750pt}%
\definecolor{currentstroke}{rgb}{0.121569,0.466667,0.705882}%
\pgfsetstrokecolor{currentstroke}%
\pgfsetdash{}{0pt}%
\pgfpathmoveto{\pgfqpoint{1.043882in}{0.642843in}}%
\pgfpathcurveto{\pgfqpoint{1.049407in}{0.642843in}}{\pgfqpoint{1.054707in}{0.645038in}}{\pgfqpoint{1.058614in}{0.648945in}}%
\pgfpathcurveto{\pgfqpoint{1.062521in}{0.652851in}}{\pgfqpoint{1.064716in}{0.658151in}}{\pgfqpoint{1.064716in}{0.663676in}}%
\pgfpathcurveto{\pgfqpoint{1.064716in}{0.669201in}}{\pgfqpoint{1.062521in}{0.674501in}}{\pgfqpoint{1.058614in}{0.678407in}}%
\pgfpathcurveto{\pgfqpoint{1.054707in}{0.682314in}}{\pgfqpoint{1.049407in}{0.684509in}}{\pgfqpoint{1.043882in}{0.684509in}}%
\pgfpathcurveto{\pgfqpoint{1.038357in}{0.684509in}}{\pgfqpoint{1.033058in}{0.682314in}}{\pgfqpoint{1.029151in}{0.678407in}}%
\pgfpathcurveto{\pgfqpoint{1.025244in}{0.674501in}}{\pgfqpoint{1.023049in}{0.669201in}}{\pgfqpoint{1.023049in}{0.663676in}}%
\pgfpathcurveto{\pgfqpoint{1.023049in}{0.658151in}}{\pgfqpoint{1.025244in}{0.652851in}}{\pgfqpoint{1.029151in}{0.648945in}}%
\pgfpathcurveto{\pgfqpoint{1.033058in}{0.645038in}}{\pgfqpoint{1.038357in}{0.642843in}}{\pgfqpoint{1.043882in}{0.642843in}}%
\pgfpathclose%
\pgfusepath{stroke,fill}%
\end{pgfscope}%
\begin{pgfscope}%
\pgfpathrectangle{\pgfqpoint{0.750000in}{0.500000in}}{\pgfqpoint{4.650000in}{3.020000in}}%
\pgfusepath{clip}%
\pgfsetbuttcap%
\pgfsetroundjoin%
\definecolor{currentfill}{rgb}{0.121569,0.466667,0.705882}%
\pgfsetfillcolor{currentfill}%
\pgfsetlinewidth{1.003750pt}%
\definecolor{currentstroke}{rgb}{0.121569,0.466667,0.705882}%
\pgfsetstrokecolor{currentstroke}%
\pgfsetdash{}{0pt}%
\pgfpathmoveto{\pgfqpoint{1.973303in}{0.810688in}}%
\pgfpathcurveto{\pgfqpoint{1.982694in}{0.810688in}}{\pgfqpoint{1.991701in}{0.814419in}}{\pgfqpoint{1.998342in}{0.821060in}}%
\pgfpathcurveto{\pgfqpoint{2.004982in}{0.827700in}}{\pgfqpoint{2.008713in}{0.836707in}}{\pgfqpoint{2.008713in}{0.846098in}}%
\pgfpathcurveto{\pgfqpoint{2.008713in}{0.855489in}}{\pgfqpoint{2.004982in}{0.864496in}}{\pgfqpoint{1.998342in}{0.871137in}}%
\pgfpathcurveto{\pgfqpoint{1.991701in}{0.877777in}}{\pgfqpoint{1.982694in}{0.881508in}}{\pgfqpoint{1.973303in}{0.881508in}}%
\pgfpathcurveto{\pgfqpoint{1.963912in}{0.881508in}}{\pgfqpoint{1.954905in}{0.877777in}}{\pgfqpoint{1.948265in}{0.871137in}}%
\pgfpathcurveto{\pgfqpoint{1.941624in}{0.864496in}}{\pgfqpoint{1.937893in}{0.855489in}}{\pgfqpoint{1.937893in}{0.846098in}}%
\pgfpathcurveto{\pgfqpoint{1.937893in}{0.836707in}}{\pgfqpoint{1.941624in}{0.827700in}}{\pgfqpoint{1.948265in}{0.821060in}}%
\pgfpathcurveto{\pgfqpoint{1.954905in}{0.814419in}}{\pgfqpoint{1.963912in}{0.810688in}}{\pgfqpoint{1.973303in}{0.810688in}}%
\pgfpathclose%
\pgfusepath{stroke,fill}%
\end{pgfscope}%
\begin{pgfscope}%
\pgfpathrectangle{\pgfqpoint{0.750000in}{0.500000in}}{\pgfqpoint{4.650000in}{3.020000in}}%
\pgfusepath{clip}%
\pgfsetbuttcap%
\pgfsetroundjoin%
\definecolor{currentfill}{rgb}{0.121569,0.466667,0.705882}%
\pgfsetfillcolor{currentfill}%
\pgfsetlinewidth{1.003750pt}%
\definecolor{currentstroke}{rgb}{0.121569,0.466667,0.705882}%
\pgfsetstrokecolor{currentstroke}%
\pgfsetdash{}{0pt}%
\pgfpathmoveto{\pgfqpoint{2.894038in}{1.025099in}}%
\pgfpathcurveto{\pgfqpoint{2.900929in}{1.025099in}}{\pgfqpoint{2.907539in}{1.027837in}}{\pgfqpoint{2.912411in}{1.032710in}}%
\pgfpathcurveto{\pgfqpoint{2.917284in}{1.037583in}}{\pgfqpoint{2.920022in}{1.044192in}}{\pgfqpoint{2.920022in}{1.051083in}}%
\pgfpathcurveto{\pgfqpoint{2.920022in}{1.057974in}}{\pgfqpoint{2.917284in}{1.064584in}}{\pgfqpoint{2.912411in}{1.069456in}}%
\pgfpathcurveto{\pgfqpoint{2.907539in}{1.074329in}}{\pgfqpoint{2.900929in}{1.077067in}}{\pgfqpoint{2.894038in}{1.077067in}}%
\pgfpathcurveto{\pgfqpoint{2.887147in}{1.077067in}}{\pgfqpoint{2.880537in}{1.074329in}}{\pgfqpoint{2.875665in}{1.069456in}}%
\pgfpathcurveto{\pgfqpoint{2.870792in}{1.064584in}}{\pgfqpoint{2.868054in}{1.057974in}}{\pgfqpoint{2.868054in}{1.051083in}}%
\pgfpathcurveto{\pgfqpoint{2.868054in}{1.044192in}}{\pgfqpoint{2.870792in}{1.037583in}}{\pgfqpoint{2.875665in}{1.032710in}}%
\pgfpathcurveto{\pgfqpoint{2.880537in}{1.027837in}}{\pgfqpoint{2.887147in}{1.025099in}}{\pgfqpoint{2.894038in}{1.025099in}}%
\pgfpathclose%
\pgfusepath{stroke,fill}%
\end{pgfscope}%
\begin{pgfscope}%
\pgfpathrectangle{\pgfqpoint{0.750000in}{0.500000in}}{\pgfqpoint{4.650000in}{3.020000in}}%
\pgfusepath{clip}%
\pgfsetbuttcap%
\pgfsetroundjoin%
\definecolor{currentfill}{rgb}{0.121569,0.466667,0.705882}%
\pgfsetfillcolor{currentfill}%
\pgfsetlinewidth{1.003750pt}%
\definecolor{currentstroke}{rgb}{0.121569,0.466667,0.705882}%
\pgfsetstrokecolor{currentstroke}%
\pgfsetdash{}{0pt}%
\pgfpathmoveto{\pgfqpoint{1.334869in}{0.653534in}}%
\pgfpathcurveto{\pgfqpoint{1.341760in}{0.653534in}}{\pgfqpoint{1.348370in}{0.656272in}}{\pgfqpoint{1.353243in}{0.661145in}}%
\pgfpathcurveto{\pgfqpoint{1.358115in}{0.666017in}}{\pgfqpoint{1.360853in}{0.672627in}}{\pgfqpoint{1.360853in}{0.679518in}}%
\pgfpathcurveto{\pgfqpoint{1.360853in}{0.686409in}}{\pgfqpoint{1.358115in}{0.693018in}}{\pgfqpoint{1.353243in}{0.697891in}}%
\pgfpathcurveto{\pgfqpoint{1.348370in}{0.702764in}}{\pgfqpoint{1.341760in}{0.705502in}}{\pgfqpoint{1.334869in}{0.705502in}}%
\pgfpathcurveto{\pgfqpoint{1.327978in}{0.705502in}}{\pgfqpoint{1.321369in}{0.702764in}}{\pgfqpoint{1.316496in}{0.697891in}}%
\pgfpathcurveto{\pgfqpoint{1.311623in}{0.693018in}}{\pgfqpoint{1.308886in}{0.686409in}}{\pgfqpoint{1.308886in}{0.679518in}}%
\pgfpathcurveto{\pgfqpoint{1.308886in}{0.672627in}}{\pgfqpoint{1.311623in}{0.666017in}}{\pgfqpoint{1.316496in}{0.661145in}}%
\pgfpathcurveto{\pgfqpoint{1.321369in}{0.656272in}}{\pgfqpoint{1.327978in}{0.653534in}}{\pgfqpoint{1.334869in}{0.653534in}}%
\pgfpathclose%
\pgfusepath{stroke,fill}%
\end{pgfscope}%
\begin{pgfscope}%
\pgfpathrectangle{\pgfqpoint{0.750000in}{0.500000in}}{\pgfqpoint{4.650000in}{3.020000in}}%
\pgfusepath{clip}%
\pgfsetbuttcap%
\pgfsetroundjoin%
\definecolor{currentfill}{rgb}{0.121569,0.466667,0.705882}%
\pgfsetfillcolor{currentfill}%
\pgfsetlinewidth{1.003750pt}%
\definecolor{currentstroke}{rgb}{0.121569,0.466667,0.705882}%
\pgfsetstrokecolor{currentstroke}%
\pgfsetdash{}{0pt}%
\pgfpathmoveto{\pgfqpoint{3.011301in}{0.998730in}}%
\pgfpathcurveto{\pgfqpoint{3.019329in}{0.998730in}}{\pgfqpoint{3.027029in}{1.001920in}}{\pgfqpoint{3.032706in}{1.007596in}}%
\pgfpathcurveto{\pgfqpoint{3.038382in}{1.013273in}}{\pgfqpoint{3.041572in}{1.020973in}}{\pgfqpoint{3.041572in}{1.029000in}}%
\pgfpathcurveto{\pgfqpoint{3.041572in}{1.037028in}}{\pgfqpoint{3.038382in}{1.044728in}}{\pgfqpoint{3.032706in}{1.050405in}}%
\pgfpathcurveto{\pgfqpoint{3.027029in}{1.056081in}}{\pgfqpoint{3.019329in}{1.059271in}}{\pgfqpoint{3.011301in}{1.059271in}}%
\pgfpathcurveto{\pgfqpoint{3.003274in}{1.059271in}}{\pgfqpoint{2.995574in}{1.056081in}}{\pgfqpoint{2.989897in}{1.050405in}}%
\pgfpathcurveto{\pgfqpoint{2.984221in}{1.044728in}}{\pgfqpoint{2.981031in}{1.037028in}}{\pgfqpoint{2.981031in}{1.029000in}}%
\pgfpathcurveto{\pgfqpoint{2.981031in}{1.020973in}}{\pgfqpoint{2.984221in}{1.013273in}}{\pgfqpoint{2.989897in}{1.007596in}}%
\pgfpathcurveto{\pgfqpoint{2.995574in}{1.001920in}}{\pgfqpoint{3.003274in}{0.998730in}}{\pgfqpoint{3.011301in}{0.998730in}}%
\pgfpathclose%
\pgfusepath{stroke,fill}%
\end{pgfscope}%
\begin{pgfscope}%
\pgfpathrectangle{\pgfqpoint{0.750000in}{0.500000in}}{\pgfqpoint{4.650000in}{3.020000in}}%
\pgfusepath{clip}%
\pgfsetbuttcap%
\pgfsetroundjoin%
\definecolor{currentfill}{rgb}{0.121569,0.466667,0.705882}%
\pgfsetfillcolor{currentfill}%
\pgfsetlinewidth{1.003750pt}%
\definecolor{currentstroke}{rgb}{0.121569,0.466667,0.705882}%
\pgfsetstrokecolor{currentstroke}%
\pgfsetdash{}{0pt}%
\pgfpathmoveto{\pgfqpoint{1.414493in}{0.852312in}}%
\pgfpathcurveto{\pgfqpoint{1.417682in}{0.852312in}}{\pgfqpoint{1.420742in}{0.853580in}}{\pgfqpoint{1.422998in}{0.855835in}}%
\pgfpathcurveto{\pgfqpoint{1.425253in}{0.858091in}}{\pgfqpoint{1.426521in}{0.861150in}}{\pgfqpoint{1.426521in}{0.864340in}}%
\pgfpathcurveto{\pgfqpoint{1.426521in}{0.867530in}}{\pgfqpoint{1.425253in}{0.870590in}}{\pgfqpoint{1.422998in}{0.872846in}}%
\pgfpathcurveto{\pgfqpoint{1.420742in}{0.875101in}}{\pgfqpoint{1.417682in}{0.876369in}}{\pgfqpoint{1.414493in}{0.876369in}}%
\pgfpathcurveto{\pgfqpoint{1.411303in}{0.876369in}}{\pgfqpoint{1.408243in}{0.875101in}}{\pgfqpoint{1.405987in}{0.872846in}}%
\pgfpathcurveto{\pgfqpoint{1.403732in}{0.870590in}}{\pgfqpoint{1.402464in}{0.867530in}}{\pgfqpoint{1.402464in}{0.864340in}}%
\pgfpathcurveto{\pgfqpoint{1.402464in}{0.861150in}}{\pgfqpoint{1.403732in}{0.858091in}}{\pgfqpoint{1.405987in}{0.855835in}}%
\pgfpathcurveto{\pgfqpoint{1.408243in}{0.853580in}}{\pgfqpoint{1.411303in}{0.852312in}}{\pgfqpoint{1.414493in}{0.852312in}}%
\pgfpathclose%
\pgfusepath{stroke,fill}%
\end{pgfscope}%
\begin{pgfscope}%
\pgfpathrectangle{\pgfqpoint{0.750000in}{0.500000in}}{\pgfqpoint{4.650000in}{3.020000in}}%
\pgfusepath{clip}%
\pgfsetbuttcap%
\pgfsetroundjoin%
\definecolor{currentfill}{rgb}{0.121569,0.466667,0.705882}%
\pgfsetfillcolor{currentfill}%
\pgfsetlinewidth{1.003750pt}%
\definecolor{currentstroke}{rgb}{0.121569,0.466667,0.705882}%
\pgfsetstrokecolor{currentstroke}%
\pgfsetdash{}{0pt}%
\pgfpathmoveto{\pgfqpoint{1.841563in}{1.537356in}}%
\pgfpathcurveto{\pgfqpoint{1.844753in}{1.537356in}}{\pgfqpoint{1.847812in}{1.538623in}}{\pgfqpoint{1.850068in}{1.540879in}}%
\pgfpathcurveto{\pgfqpoint{1.852324in}{1.543134in}}{\pgfqpoint{1.853591in}{1.546194in}}{\pgfqpoint{1.853591in}{1.549384in}}%
\pgfpathcurveto{\pgfqpoint{1.853591in}{1.552574in}}{\pgfqpoint{1.852324in}{1.555633in}}{\pgfqpoint{1.850068in}{1.557889in}}%
\pgfpathcurveto{\pgfqpoint{1.847812in}{1.560145in}}{\pgfqpoint{1.844753in}{1.561412in}}{\pgfqpoint{1.841563in}{1.561412in}}%
\pgfpathcurveto{\pgfqpoint{1.838373in}{1.561412in}}{\pgfqpoint{1.835313in}{1.560145in}}{\pgfqpoint{1.833058in}{1.557889in}}%
\pgfpathcurveto{\pgfqpoint{1.830802in}{1.555633in}}{\pgfqpoint{1.829535in}{1.552574in}}{\pgfqpoint{1.829535in}{1.549384in}}%
\pgfpathcurveto{\pgfqpoint{1.829535in}{1.546194in}}{\pgfqpoint{1.830802in}{1.543134in}}{\pgfqpoint{1.833058in}{1.540879in}}%
\pgfpathcurveto{\pgfqpoint{1.835313in}{1.538623in}}{\pgfqpoint{1.838373in}{1.537356in}}{\pgfqpoint{1.841563in}{1.537356in}}%
\pgfpathclose%
\pgfusepath{stroke,fill}%
\end{pgfscope}%
\begin{pgfscope}%
\pgfpathrectangle{\pgfqpoint{0.750000in}{0.500000in}}{\pgfqpoint{4.650000in}{3.020000in}}%
\pgfusepath{clip}%
\pgfsetbuttcap%
\pgfsetroundjoin%
\definecolor{currentfill}{rgb}{0.121569,0.466667,0.705882}%
\pgfsetfillcolor{currentfill}%
\pgfsetlinewidth{1.003750pt}%
\definecolor{currentstroke}{rgb}{0.121569,0.466667,0.705882}%
\pgfsetstrokecolor{currentstroke}%
\pgfsetdash{}{0pt}%
\pgfpathmoveto{\pgfqpoint{1.098895in}{1.157916in}}%
\pgfpathcurveto{\pgfqpoint{1.101499in}{1.157916in}}{\pgfqpoint{1.103998in}{1.158951in}}{\pgfqpoint{1.105839in}{1.160793in}}%
\pgfpathcurveto{\pgfqpoint{1.107681in}{1.162635in}}{\pgfqpoint{1.108716in}{1.165133in}}{\pgfqpoint{1.108716in}{1.167737in}}%
\pgfpathcurveto{\pgfqpoint{1.108716in}{1.170342in}}{\pgfqpoint{1.107681in}{1.172840in}}{\pgfqpoint{1.105839in}{1.174682in}}%
\pgfpathcurveto{\pgfqpoint{1.103998in}{1.176523in}}{\pgfqpoint{1.101499in}{1.177558in}}{\pgfqpoint{1.098895in}{1.177558in}}%
\pgfpathcurveto{\pgfqpoint{1.096290in}{1.177558in}}{\pgfqpoint{1.093792in}{1.176523in}}{\pgfqpoint{1.091950in}{1.174682in}}%
\pgfpathcurveto{\pgfqpoint{1.090109in}{1.172840in}}{\pgfqpoint{1.089074in}{1.170342in}}{\pgfqpoint{1.089074in}{1.167737in}}%
\pgfpathcurveto{\pgfqpoint{1.089074in}{1.165133in}}{\pgfqpoint{1.090109in}{1.162635in}}{\pgfqpoint{1.091950in}{1.160793in}}%
\pgfpathcurveto{\pgfqpoint{1.093792in}{1.158951in}}{\pgfqpoint{1.096290in}{1.157916in}}{\pgfqpoint{1.098895in}{1.157916in}}%
\pgfpathclose%
\pgfusepath{stroke,fill}%
\end{pgfscope}%
\begin{pgfscope}%
\pgfpathrectangle{\pgfqpoint{0.750000in}{0.500000in}}{\pgfqpoint{4.650000in}{3.020000in}}%
\pgfusepath{clip}%
\pgfsetbuttcap%
\pgfsetroundjoin%
\definecolor{currentfill}{rgb}{0.121569,0.466667,0.705882}%
\pgfsetfillcolor{currentfill}%
\pgfsetlinewidth{1.003750pt}%
\definecolor{currentstroke}{rgb}{0.121569,0.466667,0.705882}%
\pgfsetstrokecolor{currentstroke}%
\pgfsetdash{}{0pt}%
\pgfpathmoveto{\pgfqpoint{1.135087in}{0.678051in}}%
\pgfpathcurveto{\pgfqpoint{1.138277in}{0.678051in}}{\pgfqpoint{1.141337in}{0.679318in}}{\pgfqpoint{1.143592in}{0.681574in}}%
\pgfpathcurveto{\pgfqpoint{1.145848in}{0.683830in}}{\pgfqpoint{1.147115in}{0.686889in}}{\pgfqpoint{1.147115in}{0.690079in}}%
\pgfpathcurveto{\pgfqpoint{1.147115in}{0.693269in}}{\pgfqpoint{1.145848in}{0.696329in}}{\pgfqpoint{1.143592in}{0.698584in}}%
\pgfpathcurveto{\pgfqpoint{1.141337in}{0.700840in}}{\pgfqpoint{1.138277in}{0.702107in}}{\pgfqpoint{1.135087in}{0.702107in}}%
\pgfpathcurveto{\pgfqpoint{1.131897in}{0.702107in}}{\pgfqpoint{1.128838in}{0.700840in}}{\pgfqpoint{1.126582in}{0.698584in}}%
\pgfpathcurveto{\pgfqpoint{1.124326in}{0.696329in}}{\pgfqpoint{1.123059in}{0.693269in}}{\pgfqpoint{1.123059in}{0.690079in}}%
\pgfpathcurveto{\pgfqpoint{1.123059in}{0.686889in}}{\pgfqpoint{1.124326in}{0.683830in}}{\pgfqpoint{1.126582in}{0.681574in}}%
\pgfpathcurveto{\pgfqpoint{1.128838in}{0.679318in}}{\pgfqpoint{1.131897in}{0.678051in}}{\pgfqpoint{1.135087in}{0.678051in}}%
\pgfpathclose%
\pgfusepath{stroke,fill}%
\end{pgfscope}%
\begin{pgfscope}%
\pgfpathrectangle{\pgfqpoint{0.750000in}{0.500000in}}{\pgfqpoint{4.650000in}{3.020000in}}%
\pgfusepath{clip}%
\pgfsetbuttcap%
\pgfsetroundjoin%
\definecolor{currentfill}{rgb}{0.121569,0.466667,0.705882}%
\pgfsetfillcolor{currentfill}%
\pgfsetlinewidth{1.003750pt}%
\definecolor{currentstroke}{rgb}{0.121569,0.466667,0.705882}%
\pgfsetstrokecolor{currentstroke}%
\pgfsetdash{}{0pt}%
\pgfpathmoveto{\pgfqpoint{1.423179in}{1.577499in}}%
\pgfpathcurveto{\pgfqpoint{1.427690in}{1.577499in}}{\pgfqpoint{1.432017in}{1.579291in}}{\pgfqpoint{1.435207in}{1.582481in}}%
\pgfpathcurveto{\pgfqpoint{1.438397in}{1.585671in}}{\pgfqpoint{1.440189in}{1.589998in}}{\pgfqpoint{1.440189in}{1.594509in}}%
\pgfpathcurveto{\pgfqpoint{1.440189in}{1.599021in}}{\pgfqpoint{1.438397in}{1.603348in}}{\pgfqpoint{1.435207in}{1.606538in}}%
\pgfpathcurveto{\pgfqpoint{1.432017in}{1.609727in}}{\pgfqpoint{1.427690in}{1.611520in}}{\pgfqpoint{1.423179in}{1.611520in}}%
\pgfpathcurveto{\pgfqpoint{1.418668in}{1.611520in}}{\pgfqpoint{1.414340in}{1.609727in}}{\pgfqpoint{1.411151in}{1.606538in}}%
\pgfpathcurveto{\pgfqpoint{1.407961in}{1.603348in}}{\pgfqpoint{1.406168in}{1.599021in}}{\pgfqpoint{1.406168in}{1.594509in}}%
\pgfpathcurveto{\pgfqpoint{1.406168in}{1.589998in}}{\pgfqpoint{1.407961in}{1.585671in}}{\pgfqpoint{1.411151in}{1.582481in}}%
\pgfpathcurveto{\pgfqpoint{1.414340in}{1.579291in}}{\pgfqpoint{1.418668in}{1.577499in}}{\pgfqpoint{1.423179in}{1.577499in}}%
\pgfpathclose%
\pgfusepath{stroke,fill}%
\end{pgfscope}%
\begin{pgfscope}%
\pgfpathrectangle{\pgfqpoint{0.750000in}{0.500000in}}{\pgfqpoint{4.650000in}{3.020000in}}%
\pgfusepath{clip}%
\pgfsetbuttcap%
\pgfsetroundjoin%
\definecolor{currentfill}{rgb}{0.121569,0.466667,0.705882}%
\pgfsetfillcolor{currentfill}%
\pgfsetlinewidth{1.003750pt}%
\definecolor{currentstroke}{rgb}{0.121569,0.466667,0.705882}%
\pgfsetstrokecolor{currentstroke}%
\pgfsetdash{}{0pt}%
\pgfpathmoveto{\pgfqpoint{1.459371in}{0.874255in}}%
\pgfpathcurveto{\pgfqpoint{1.463489in}{0.874255in}}{\pgfqpoint{1.467439in}{0.875891in}}{\pgfqpoint{1.470351in}{0.878803in}}%
\pgfpathcurveto{\pgfqpoint{1.473263in}{0.881715in}}{\pgfqpoint{1.474899in}{0.885665in}}{\pgfqpoint{1.474899in}{0.889783in}}%
\pgfpathcurveto{\pgfqpoint{1.474899in}{0.893902in}}{\pgfqpoint{1.473263in}{0.897852in}}{\pgfqpoint{1.470351in}{0.900764in}}%
\pgfpathcurveto{\pgfqpoint{1.467439in}{0.903676in}}{\pgfqpoint{1.463489in}{0.905312in}}{\pgfqpoint{1.459371in}{0.905312in}}%
\pgfpathcurveto{\pgfqpoint{1.455253in}{0.905312in}}{\pgfqpoint{1.451303in}{0.903676in}}{\pgfqpoint{1.448391in}{0.900764in}}%
\pgfpathcurveto{\pgfqpoint{1.445479in}{0.897852in}}{\pgfqpoint{1.443843in}{0.893902in}}{\pgfqpoint{1.443843in}{0.889783in}}%
\pgfpathcurveto{\pgfqpoint{1.443843in}{0.885665in}}{\pgfqpoint{1.445479in}{0.881715in}}{\pgfqpoint{1.448391in}{0.878803in}}%
\pgfpathcurveto{\pgfqpoint{1.451303in}{0.875891in}}{\pgfqpoint{1.455253in}{0.874255in}}{\pgfqpoint{1.459371in}{0.874255in}}%
\pgfpathclose%
\pgfusepath{stroke,fill}%
\end{pgfscope}%
\begin{pgfscope}%
\pgfpathrectangle{\pgfqpoint{0.750000in}{0.500000in}}{\pgfqpoint{4.650000in}{3.020000in}}%
\pgfusepath{clip}%
\pgfsetbuttcap%
\pgfsetroundjoin%
\definecolor{currentfill}{rgb}{0.121569,0.466667,0.705882}%
\pgfsetfillcolor{currentfill}%
\pgfsetlinewidth{1.003750pt}%
\definecolor{currentstroke}{rgb}{0.121569,0.466667,0.705882}%
\pgfsetstrokecolor{currentstroke}%
\pgfsetdash{}{0pt}%
\pgfpathmoveto{\pgfqpoint{1.171280in}{0.668737in}}%
\pgfpathcurveto{\pgfqpoint{1.173884in}{0.668737in}}{\pgfqpoint{1.176382in}{0.669772in}}{\pgfqpoint{1.178224in}{0.671613in}}%
\pgfpathcurveto{\pgfqpoint{1.180066in}{0.673455in}}{\pgfqpoint{1.181101in}{0.675953in}}{\pgfqpoint{1.181101in}{0.678558in}}%
\pgfpathcurveto{\pgfqpoint{1.181101in}{0.681162in}}{\pgfqpoint{1.180066in}{0.683661in}}{\pgfqpoint{1.178224in}{0.685502in}}%
\pgfpathcurveto{\pgfqpoint{1.176382in}{0.687344in}}{\pgfqpoint{1.173884in}{0.688379in}}{\pgfqpoint{1.171280in}{0.688379in}}%
\pgfpathcurveto{\pgfqpoint{1.168675in}{0.688379in}}{\pgfqpoint{1.166177in}{0.687344in}}{\pgfqpoint{1.164335in}{0.685502in}}%
\pgfpathcurveto{\pgfqpoint{1.162493in}{0.683661in}}{\pgfqpoint{1.161459in}{0.681162in}}{\pgfqpoint{1.161459in}{0.678558in}}%
\pgfpathcurveto{\pgfqpoint{1.161459in}{0.675953in}}{\pgfqpoint{1.162493in}{0.673455in}}{\pgfqpoint{1.164335in}{0.671613in}}%
\pgfpathcurveto{\pgfqpoint{1.166177in}{0.669772in}}{\pgfqpoint{1.168675in}{0.668737in}}{\pgfqpoint{1.171280in}{0.668737in}}%
\pgfpathclose%
\pgfusepath{stroke,fill}%
\end{pgfscope}%
\begin{pgfscope}%
\pgfpathrectangle{\pgfqpoint{0.750000in}{0.500000in}}{\pgfqpoint{4.650000in}{3.020000in}}%
\pgfusepath{clip}%
\pgfsetbuttcap%
\pgfsetroundjoin%
\definecolor{currentfill}{rgb}{0.121569,0.466667,0.705882}%
\pgfsetfillcolor{currentfill}%
\pgfsetlinewidth{1.003750pt}%
\definecolor{currentstroke}{rgb}{0.121569,0.466667,0.705882}%
\pgfsetstrokecolor{currentstroke}%
\pgfsetdash{}{0pt}%
\pgfpathmoveto{\pgfqpoint{1.179966in}{0.673790in}}%
\pgfpathcurveto{\pgfqpoint{1.183649in}{0.673790in}}{\pgfqpoint{1.187182in}{0.675253in}}{\pgfqpoint{1.189787in}{0.677858in}}%
\pgfpathcurveto{\pgfqpoint{1.192391in}{0.680462in}}{\pgfqpoint{1.193855in}{0.683995in}}{\pgfqpoint{1.193855in}{0.687679in}}%
\pgfpathcurveto{\pgfqpoint{1.193855in}{0.691362in}}{\pgfqpoint{1.192391in}{0.694895in}}{\pgfqpoint{1.189787in}{0.697500in}}%
\pgfpathcurveto{\pgfqpoint{1.187182in}{0.700104in}}{\pgfqpoint{1.183649in}{0.701568in}}{\pgfqpoint{1.179966in}{0.701568in}}%
\pgfpathcurveto{\pgfqpoint{1.176282in}{0.701568in}}{\pgfqpoint{1.172749in}{0.700104in}}{\pgfqpoint{1.170145in}{0.697500in}}%
\pgfpathcurveto{\pgfqpoint{1.167540in}{0.694895in}}{\pgfqpoint{1.166077in}{0.691362in}}{\pgfqpoint{1.166077in}{0.687679in}}%
\pgfpathcurveto{\pgfqpoint{1.166077in}{0.683995in}}{\pgfqpoint{1.167540in}{0.680462in}}{\pgfqpoint{1.170145in}{0.677858in}}%
\pgfpathcurveto{\pgfqpoint{1.172749in}{0.675253in}}{\pgfqpoint{1.176282in}{0.673790in}}{\pgfqpoint{1.179966in}{0.673790in}}%
\pgfpathclose%
\pgfusepath{stroke,fill}%
\end{pgfscope}%
\begin{pgfscope}%
\pgfpathrectangle{\pgfqpoint{0.750000in}{0.500000in}}{\pgfqpoint{4.650000in}{3.020000in}}%
\pgfusepath{clip}%
\pgfsetbuttcap%
\pgfsetroundjoin%
\definecolor{currentfill}{rgb}{0.121569,0.466667,0.705882}%
\pgfsetfillcolor{currentfill}%
\pgfsetlinewidth{1.003750pt}%
\definecolor{currentstroke}{rgb}{0.121569,0.466667,0.705882}%
\pgfsetstrokecolor{currentstroke}%
\pgfsetdash{}{0pt}%
\pgfpathmoveto{\pgfqpoint{1.423179in}{1.045394in}}%
\pgfpathcurveto{\pgfqpoint{1.431817in}{1.045394in}}{\pgfqpoint{1.440103in}{1.048826in}}{\pgfqpoint{1.446211in}{1.054934in}}%
\pgfpathcurveto{\pgfqpoint{1.452319in}{1.061042in}}{\pgfqpoint{1.455751in}{1.069328in}}{\pgfqpoint{1.455751in}{1.077966in}}%
\pgfpathcurveto{\pgfqpoint{1.455751in}{1.086605in}}{\pgfqpoint{1.452319in}{1.094890in}}{\pgfqpoint{1.446211in}{1.100999in}}%
\pgfpathcurveto{\pgfqpoint{1.440103in}{1.107107in}}{\pgfqpoint{1.431817in}{1.110539in}}{\pgfqpoint{1.423179in}{1.110539in}}%
\pgfpathcurveto{\pgfqpoint{1.414540in}{1.110539in}}{\pgfqpoint{1.406255in}{1.107107in}}{\pgfqpoint{1.400147in}{1.100999in}}%
\pgfpathcurveto{\pgfqpoint{1.394038in}{1.094890in}}{\pgfqpoint{1.390606in}{1.086605in}}{\pgfqpoint{1.390606in}{1.077966in}}%
\pgfpathcurveto{\pgfqpoint{1.390606in}{1.069328in}}{\pgfqpoint{1.394038in}{1.061042in}}{\pgfqpoint{1.400147in}{1.054934in}}%
\pgfpathcurveto{\pgfqpoint{1.406255in}{1.048826in}}{\pgfqpoint{1.414540in}{1.045394in}}{\pgfqpoint{1.423179in}{1.045394in}}%
\pgfpathclose%
\pgfusepath{stroke,fill}%
\end{pgfscope}%
\begin{pgfscope}%
\pgfpathrectangle{\pgfqpoint{0.750000in}{0.500000in}}{\pgfqpoint{4.650000in}{3.020000in}}%
\pgfusepath{clip}%
\pgfsetbuttcap%
\pgfsetroundjoin%
\definecolor{currentfill}{rgb}{0.121569,0.466667,0.705882}%
\pgfsetfillcolor{currentfill}%
\pgfsetlinewidth{1.003750pt}%
\definecolor{currentstroke}{rgb}{0.121569,0.466667,0.705882}%
\pgfsetstrokecolor{currentstroke}%
\pgfsetdash{}{0pt}%
\pgfpathmoveto{\pgfqpoint{1.188652in}{0.666337in}}%
\pgfpathcurveto{\pgfqpoint{1.191256in}{0.666337in}}{\pgfqpoint{1.193755in}{0.667371in}}{\pgfqpoint{1.195596in}{0.669213in}}%
\pgfpathcurveto{\pgfqpoint{1.197438in}{0.671055in}}{\pgfqpoint{1.198473in}{0.673553in}}{\pgfqpoint{1.198473in}{0.676157in}}%
\pgfpathcurveto{\pgfqpoint{1.198473in}{0.678762in}}{\pgfqpoint{1.197438in}{0.681260in}}{\pgfqpoint{1.195596in}{0.683102in}}%
\pgfpathcurveto{\pgfqpoint{1.193755in}{0.684944in}}{\pgfqpoint{1.191256in}{0.685978in}}{\pgfqpoint{1.188652in}{0.685978in}}%
\pgfpathcurveto{\pgfqpoint{1.186047in}{0.685978in}}{\pgfqpoint{1.183549in}{0.684944in}}{\pgfqpoint{1.181707in}{0.683102in}}%
\pgfpathcurveto{\pgfqpoint{1.179866in}{0.681260in}}{\pgfqpoint{1.178831in}{0.678762in}}{\pgfqpoint{1.178831in}{0.676157in}}%
\pgfpathcurveto{\pgfqpoint{1.178831in}{0.673553in}}{\pgfqpoint{1.179866in}{0.671055in}}{\pgfqpoint{1.181707in}{0.669213in}}%
\pgfpathcurveto{\pgfqpoint{1.183549in}{0.667371in}}{\pgfqpoint{1.186047in}{0.666337in}}{\pgfqpoint{1.188652in}{0.666337in}}%
\pgfpathclose%
\pgfusepath{stroke,fill}%
\end{pgfscope}%
\begin{pgfscope}%
\pgfpathrectangle{\pgfqpoint{0.750000in}{0.500000in}}{\pgfqpoint{4.650000in}{3.020000in}}%
\pgfusepath{clip}%
\pgfsetbuttcap%
\pgfsetroundjoin%
\definecolor{currentfill}{rgb}{0.121569,0.466667,0.705882}%
\pgfsetfillcolor{currentfill}%
\pgfsetlinewidth{1.003750pt}%
\definecolor{currentstroke}{rgb}{0.121569,0.466667,0.705882}%
\pgfsetstrokecolor{currentstroke}%
\pgfsetdash{}{0pt}%
\pgfpathmoveto{\pgfqpoint{1.378300in}{0.688928in}}%
\pgfpathcurveto{\pgfqpoint{1.383825in}{0.688928in}}{\pgfqpoint{1.389125in}{0.691123in}}{\pgfqpoint{1.393032in}{0.695030in}}%
\pgfpathcurveto{\pgfqpoint{1.396938in}{0.698937in}}{\pgfqpoint{1.399133in}{0.704236in}}{\pgfqpoint{1.399133in}{0.709762in}}%
\pgfpathcurveto{\pgfqpoint{1.399133in}{0.715287in}}{\pgfqpoint{1.396938in}{0.720586in}}{\pgfqpoint{1.393032in}{0.724493in}}%
\pgfpathcurveto{\pgfqpoint{1.389125in}{0.728400in}}{\pgfqpoint{1.383825in}{0.730595in}}{\pgfqpoint{1.378300in}{0.730595in}}%
\pgfpathcurveto{\pgfqpoint{1.372775in}{0.730595in}}{\pgfqpoint{1.367476in}{0.728400in}}{\pgfqpoint{1.363569in}{0.724493in}}%
\pgfpathcurveto{\pgfqpoint{1.359662in}{0.720586in}}{\pgfqpoint{1.357467in}{0.715287in}}{\pgfqpoint{1.357467in}{0.709762in}}%
\pgfpathcurveto{\pgfqpoint{1.357467in}{0.704236in}}{\pgfqpoint{1.359662in}{0.698937in}}{\pgfqpoint{1.363569in}{0.695030in}}%
\pgfpathcurveto{\pgfqpoint{1.367476in}{0.691123in}}{\pgfqpoint{1.372775in}{0.688928in}}{\pgfqpoint{1.378300in}{0.688928in}}%
\pgfpathclose%
\pgfusepath{stroke,fill}%
\end{pgfscope}%
\begin{pgfscope}%
\pgfpathrectangle{\pgfqpoint{0.750000in}{0.500000in}}{\pgfqpoint{4.650000in}{3.020000in}}%
\pgfusepath{clip}%
\pgfsetbuttcap%
\pgfsetroundjoin%
\definecolor{currentfill}{rgb}{0.121569,0.466667,0.705882}%
\pgfsetfillcolor{currentfill}%
\pgfsetlinewidth{1.003750pt}%
\definecolor{currentstroke}{rgb}{0.121569,0.466667,0.705882}%
\pgfsetstrokecolor{currentstroke}%
\pgfsetdash{}{0pt}%
\pgfpathmoveto{\pgfqpoint{1.069941in}{0.641854in}}%
\pgfpathcurveto{\pgfqpoint{1.072545in}{0.641854in}}{\pgfqpoint{1.075044in}{0.642888in}}{\pgfqpoint{1.076885in}{0.644730in}}%
\pgfpathcurveto{\pgfqpoint{1.078727in}{0.646572in}}{\pgfqpoint{1.079762in}{0.649070in}}{\pgfqpoint{1.079762in}{0.651674in}}%
\pgfpathcurveto{\pgfqpoint{1.079762in}{0.654279in}}{\pgfqpoint{1.078727in}{0.656777in}}{\pgfqpoint{1.076885in}{0.658619in}}%
\pgfpathcurveto{\pgfqpoint{1.075044in}{0.660461in}}{\pgfqpoint{1.072545in}{0.661495in}}{\pgfqpoint{1.069941in}{0.661495in}}%
\pgfpathcurveto{\pgfqpoint{1.067336in}{0.661495in}}{\pgfqpoint{1.064838in}{0.660461in}}{\pgfqpoint{1.062996in}{0.658619in}}%
\pgfpathcurveto{\pgfqpoint{1.061155in}{0.656777in}}{\pgfqpoint{1.060120in}{0.654279in}}{\pgfqpoint{1.060120in}{0.651674in}}%
\pgfpathcurveto{\pgfqpoint{1.060120in}{0.649070in}}{\pgfqpoint{1.061155in}{0.646572in}}{\pgfqpoint{1.062996in}{0.644730in}}%
\pgfpathcurveto{\pgfqpoint{1.064838in}{0.642888in}}{\pgfqpoint{1.067336in}{0.641854in}}{\pgfqpoint{1.069941in}{0.641854in}}%
\pgfpathclose%
\pgfusepath{stroke,fill}%
\end{pgfscope}%
\begin{pgfscope}%
\pgfpathrectangle{\pgfqpoint{0.750000in}{0.500000in}}{\pgfqpoint{4.650000in}{3.020000in}}%
\pgfusepath{clip}%
\pgfsetbuttcap%
\pgfsetroundjoin%
\definecolor{currentfill}{rgb}{0.121569,0.466667,0.705882}%
\pgfsetfillcolor{currentfill}%
\pgfsetlinewidth{1.003750pt}%
\definecolor{currentstroke}{rgb}{0.121569,0.466667,0.705882}%
\pgfsetstrokecolor{currentstroke}%
\pgfsetdash{}{0pt}%
\pgfpathmoveto{\pgfqpoint{1.137983in}{0.652581in}}%
\pgfpathcurveto{\pgfqpoint{1.144362in}{0.652581in}}{\pgfqpoint{1.150482in}{0.655116in}}{\pgfqpoint{1.154993in}{0.659627in}}%
\pgfpathcurveto{\pgfqpoint{1.159504in}{0.664138in}}{\pgfqpoint{1.162039in}{0.670258in}}{\pgfqpoint{1.162039in}{0.676638in}}%
\pgfpathcurveto{\pgfqpoint{1.162039in}{0.683017in}}{\pgfqpoint{1.159504in}{0.689137in}}{\pgfqpoint{1.154993in}{0.693648in}}%
\pgfpathcurveto{\pgfqpoint{1.150482in}{0.698159in}}{\pgfqpoint{1.144362in}{0.700694in}}{\pgfqpoint{1.137983in}{0.700694in}}%
\pgfpathcurveto{\pgfqpoint{1.131603in}{0.700694in}}{\pgfqpoint{1.125483in}{0.698159in}}{\pgfqpoint{1.120972in}{0.693648in}}%
\pgfpathcurveto{\pgfqpoint{1.116461in}{0.689137in}}{\pgfqpoint{1.113926in}{0.683017in}}{\pgfqpoint{1.113926in}{0.676638in}}%
\pgfpathcurveto{\pgfqpoint{1.113926in}{0.670258in}}{\pgfqpoint{1.116461in}{0.664138in}}{\pgfqpoint{1.120972in}{0.659627in}}%
\pgfpathcurveto{\pgfqpoint{1.125483in}{0.655116in}}{\pgfqpoint{1.131603in}{0.652581in}}{\pgfqpoint{1.137983in}{0.652581in}}%
\pgfpathclose%
\pgfusepath{stroke,fill}%
\end{pgfscope}%
\begin{pgfscope}%
\pgfpathrectangle{\pgfqpoint{0.750000in}{0.500000in}}{\pgfqpoint{4.650000in}{3.020000in}}%
\pgfusepath{clip}%
\pgfsetbuttcap%
\pgfsetroundjoin%
\definecolor{currentfill}{rgb}{0.121569,0.466667,0.705882}%
\pgfsetfillcolor{currentfill}%
\pgfsetlinewidth{1.003750pt}%
\definecolor{currentstroke}{rgb}{0.121569,0.466667,0.705882}%
\pgfsetstrokecolor{currentstroke}%
\pgfsetdash{}{0pt}%
\pgfpathmoveto{\pgfqpoint{2.052927in}{1.009329in}}%
\pgfpathcurveto{\pgfqpoint{2.059035in}{1.009329in}}{\pgfqpoint{2.064894in}{1.011756in}}{\pgfqpoint{2.069213in}{1.016075in}}%
\pgfpathcurveto{\pgfqpoint{2.073532in}{1.020394in}}{\pgfqpoint{2.075959in}{1.026253in}}{\pgfqpoint{2.075959in}{1.032361in}}%
\pgfpathcurveto{\pgfqpoint{2.075959in}{1.038469in}}{\pgfqpoint{2.073532in}{1.044328in}}{\pgfqpoint{2.069213in}{1.048647in}}%
\pgfpathcurveto{\pgfqpoint{2.064894in}{1.052966in}}{\pgfqpoint{2.059035in}{1.055393in}}{\pgfqpoint{2.052927in}{1.055393in}}%
\pgfpathcurveto{\pgfqpoint{2.046818in}{1.055393in}}{\pgfqpoint{2.040960in}{1.052966in}}{\pgfqpoint{2.036640in}{1.048647in}}%
\pgfpathcurveto{\pgfqpoint{2.032321in}{1.044328in}}{\pgfqpoint{2.029894in}{1.038469in}}{\pgfqpoint{2.029894in}{1.032361in}}%
\pgfpathcurveto{\pgfqpoint{2.029894in}{1.026253in}}{\pgfqpoint{2.032321in}{1.020394in}}{\pgfqpoint{2.036640in}{1.016075in}}%
\pgfpathcurveto{\pgfqpoint{2.040960in}{1.011756in}}{\pgfqpoint{2.046818in}{1.009329in}}{\pgfqpoint{2.052927in}{1.009329in}}%
\pgfpathclose%
\pgfusepath{stroke,fill}%
\end{pgfscope}%
\begin{pgfscope}%
\pgfpathrectangle{\pgfqpoint{0.750000in}{0.500000in}}{\pgfqpoint{4.650000in}{3.020000in}}%
\pgfusepath{clip}%
\pgfsetbuttcap%
\pgfsetroundjoin%
\definecolor{currentfill}{rgb}{0.121569,0.466667,0.705882}%
\pgfsetfillcolor{currentfill}%
\pgfsetlinewidth{1.003750pt}%
\definecolor{currentstroke}{rgb}{0.121569,0.466667,0.705882}%
\pgfsetstrokecolor{currentstroke}%
\pgfsetdash{}{0pt}%
\pgfpathmoveto{\pgfqpoint{1.294334in}{1.317009in}}%
\pgfpathcurveto{\pgfqpoint{1.297524in}{1.317009in}}{\pgfqpoint{1.300583in}{1.318276in}}{\pgfqpoint{1.302839in}{1.320532in}}%
\pgfpathcurveto{\pgfqpoint{1.305095in}{1.322787in}}{\pgfqpoint{1.306362in}{1.325847in}}{\pgfqpoint{1.306362in}{1.329037in}}%
\pgfpathcurveto{\pgfqpoint{1.306362in}{1.332227in}}{\pgfqpoint{1.305095in}{1.335287in}}{\pgfqpoint{1.302839in}{1.337542in}}%
\pgfpathcurveto{\pgfqpoint{1.300583in}{1.339798in}}{\pgfqpoint{1.297524in}{1.341065in}}{\pgfqpoint{1.294334in}{1.341065in}}%
\pgfpathcurveto{\pgfqpoint{1.291144in}{1.341065in}}{\pgfqpoint{1.288084in}{1.339798in}}{\pgfqpoint{1.285829in}{1.337542in}}%
\pgfpathcurveto{\pgfqpoint{1.283573in}{1.335287in}}{\pgfqpoint{1.282306in}{1.332227in}}{\pgfqpoint{1.282306in}{1.329037in}}%
\pgfpathcurveto{\pgfqpoint{1.282306in}{1.325847in}}{\pgfqpoint{1.283573in}{1.322787in}}{\pgfqpoint{1.285829in}{1.320532in}}%
\pgfpathcurveto{\pgfqpoint{1.288084in}{1.318276in}}{\pgfqpoint{1.291144in}{1.317009in}}{\pgfqpoint{1.294334in}{1.317009in}}%
\pgfpathclose%
\pgfusepath{stroke,fill}%
\end{pgfscope}%
\begin{pgfscope}%
\pgfpathrectangle{\pgfqpoint{0.750000in}{0.500000in}}{\pgfqpoint{4.650000in}{3.020000in}}%
\pgfusepath{clip}%
\pgfsetbuttcap%
\pgfsetroundjoin%
\definecolor{currentfill}{rgb}{0.121569,0.466667,0.705882}%
\pgfsetfillcolor{currentfill}%
\pgfsetlinewidth{1.003750pt}%
\definecolor{currentstroke}{rgb}{0.121569,0.466667,0.705882}%
\pgfsetstrokecolor{currentstroke}%
\pgfsetdash{}{0pt}%
\pgfpathmoveto{\pgfqpoint{1.146669in}{0.661789in}}%
\pgfpathcurveto{\pgfqpoint{1.150352in}{0.661789in}}{\pgfqpoint{1.153885in}{0.663252in}}{\pgfqpoint{1.156490in}{0.665856in}}%
\pgfpathcurveto{\pgfqpoint{1.159094in}{0.668461in}}{\pgfqpoint{1.160558in}{0.671994in}}{\pgfqpoint{1.160558in}{0.675677in}}%
\pgfpathcurveto{\pgfqpoint{1.160558in}{0.679361in}}{\pgfqpoint{1.159094in}{0.682894in}}{\pgfqpoint{1.156490in}{0.685498in}}%
\pgfpathcurveto{\pgfqpoint{1.153885in}{0.688103in}}{\pgfqpoint{1.150352in}{0.689566in}}{\pgfqpoint{1.146669in}{0.689566in}}%
\pgfpathcurveto{\pgfqpoint{1.142985in}{0.689566in}}{\pgfqpoint{1.139452in}{0.688103in}}{\pgfqpoint{1.136848in}{0.685498in}}%
\pgfpathcurveto{\pgfqpoint{1.134243in}{0.682894in}}{\pgfqpoint{1.132780in}{0.679361in}}{\pgfqpoint{1.132780in}{0.675677in}}%
\pgfpathcurveto{\pgfqpoint{1.132780in}{0.671994in}}{\pgfqpoint{1.134243in}{0.668461in}}{\pgfqpoint{1.136848in}{0.665856in}}%
\pgfpathcurveto{\pgfqpoint{1.139452in}{0.663252in}}{\pgfqpoint{1.142985in}{0.661789in}}{\pgfqpoint{1.146669in}{0.661789in}}%
\pgfpathclose%
\pgfusepath{stroke,fill}%
\end{pgfscope}%
\begin{pgfscope}%
\pgfsetbuttcap%
\pgfsetroundjoin%
\definecolor{currentfill}{rgb}{0.000000,0.000000,0.000000}%
\pgfsetfillcolor{currentfill}%
\pgfsetlinewidth{0.803000pt}%
\definecolor{currentstroke}{rgb}{0.000000,0.000000,0.000000}%
\pgfsetstrokecolor{currentstroke}%
\pgfsetdash{}{0pt}%
\pgfsys@defobject{currentmarker}{\pgfqpoint{0.000000in}{-0.048611in}}{\pgfqpoint{0.000000in}{0.000000in}}{%
\pgfpathmoveto{\pgfqpoint{0.000000in}{0.000000in}}%
\pgfpathlineto{\pgfqpoint{0.000000in}{-0.048611in}}%
\pgfusepath{stroke,fill}%
}%
\begin{pgfscope}%
\pgfsys@transformshift{0.954125in}{0.500000in}%
\pgfsys@useobject{currentmarker}{}%
\end{pgfscope}%
\end{pgfscope}%
\begin{pgfscope}%
\definecolor{textcolor}{rgb}{0.000000,0.000000,0.000000}%
\pgfsetstrokecolor{textcolor}%
\pgfsetfillcolor{textcolor}%
\pgftext[x=0.954125in,y=0.402778in,,top]{\color{textcolor}\rmfamily\fontsize{10.000000}{12.000000}\selectfont \(\displaystyle {0}\)}%
\end{pgfscope}%
\begin{pgfscope}%
\pgfsetbuttcap%
\pgfsetroundjoin%
\definecolor{currentfill}{rgb}{0.000000,0.000000,0.000000}%
\pgfsetfillcolor{currentfill}%
\pgfsetlinewidth{0.803000pt}%
\definecolor{currentstroke}{rgb}{0.000000,0.000000,0.000000}%
\pgfsetstrokecolor{currentstroke}%
\pgfsetdash{}{0pt}%
\pgfsys@defobject{currentmarker}{\pgfqpoint{0.000000in}{-0.048611in}}{\pgfqpoint{0.000000in}{0.000000in}}{%
\pgfpathmoveto{\pgfqpoint{0.000000in}{0.000000in}}%
\pgfpathlineto{\pgfqpoint{0.000000in}{-0.048611in}}%
\pgfusepath{stroke,fill}%
}%
\begin{pgfscope}%
\pgfsys@transformshift{1.677973in}{0.500000in}%
\pgfsys@useobject{currentmarker}{}%
\end{pgfscope}%
\end{pgfscope}%
\begin{pgfscope}%
\definecolor{textcolor}{rgb}{0.000000,0.000000,0.000000}%
\pgfsetstrokecolor{textcolor}%
\pgfsetfillcolor{textcolor}%
\pgftext[x=1.677973in,y=0.402778in,,top]{\color{textcolor}\rmfamily\fontsize{10.000000}{12.000000}\selectfont \(\displaystyle {500}\)}%
\end{pgfscope}%
\begin{pgfscope}%
\pgfsetbuttcap%
\pgfsetroundjoin%
\definecolor{currentfill}{rgb}{0.000000,0.000000,0.000000}%
\pgfsetfillcolor{currentfill}%
\pgfsetlinewidth{0.803000pt}%
\definecolor{currentstroke}{rgb}{0.000000,0.000000,0.000000}%
\pgfsetstrokecolor{currentstroke}%
\pgfsetdash{}{0pt}%
\pgfsys@defobject{currentmarker}{\pgfqpoint{0.000000in}{-0.048611in}}{\pgfqpoint{0.000000in}{0.000000in}}{%
\pgfpathmoveto{\pgfqpoint{0.000000in}{0.000000in}}%
\pgfpathlineto{\pgfqpoint{0.000000in}{-0.048611in}}%
\pgfusepath{stroke,fill}%
}%
\begin{pgfscope}%
\pgfsys@transformshift{2.401821in}{0.500000in}%
\pgfsys@useobject{currentmarker}{}%
\end{pgfscope}%
\end{pgfscope}%
\begin{pgfscope}%
\definecolor{textcolor}{rgb}{0.000000,0.000000,0.000000}%
\pgfsetstrokecolor{textcolor}%
\pgfsetfillcolor{textcolor}%
\pgftext[x=2.401821in,y=0.402778in,,top]{\color{textcolor}\rmfamily\fontsize{10.000000}{12.000000}\selectfont \(\displaystyle {1000}\)}%
\end{pgfscope}%
\begin{pgfscope}%
\pgfsetbuttcap%
\pgfsetroundjoin%
\definecolor{currentfill}{rgb}{0.000000,0.000000,0.000000}%
\pgfsetfillcolor{currentfill}%
\pgfsetlinewidth{0.803000pt}%
\definecolor{currentstroke}{rgb}{0.000000,0.000000,0.000000}%
\pgfsetstrokecolor{currentstroke}%
\pgfsetdash{}{0pt}%
\pgfsys@defobject{currentmarker}{\pgfqpoint{0.000000in}{-0.048611in}}{\pgfqpoint{0.000000in}{0.000000in}}{%
\pgfpathmoveto{\pgfqpoint{0.000000in}{0.000000in}}%
\pgfpathlineto{\pgfqpoint{0.000000in}{-0.048611in}}%
\pgfusepath{stroke,fill}%
}%
\begin{pgfscope}%
\pgfsys@transformshift{3.125669in}{0.500000in}%
\pgfsys@useobject{currentmarker}{}%
\end{pgfscope}%
\end{pgfscope}%
\begin{pgfscope}%
\definecolor{textcolor}{rgb}{0.000000,0.000000,0.000000}%
\pgfsetstrokecolor{textcolor}%
\pgfsetfillcolor{textcolor}%
\pgftext[x=3.125669in,y=0.402778in,,top]{\color{textcolor}\rmfamily\fontsize{10.000000}{12.000000}\selectfont \(\displaystyle {1500}\)}%
\end{pgfscope}%
\begin{pgfscope}%
\pgfsetbuttcap%
\pgfsetroundjoin%
\definecolor{currentfill}{rgb}{0.000000,0.000000,0.000000}%
\pgfsetfillcolor{currentfill}%
\pgfsetlinewidth{0.803000pt}%
\definecolor{currentstroke}{rgb}{0.000000,0.000000,0.000000}%
\pgfsetstrokecolor{currentstroke}%
\pgfsetdash{}{0pt}%
\pgfsys@defobject{currentmarker}{\pgfqpoint{0.000000in}{-0.048611in}}{\pgfqpoint{0.000000in}{0.000000in}}{%
\pgfpathmoveto{\pgfqpoint{0.000000in}{0.000000in}}%
\pgfpathlineto{\pgfqpoint{0.000000in}{-0.048611in}}%
\pgfusepath{stroke,fill}%
}%
\begin{pgfscope}%
\pgfsys@transformshift{3.849517in}{0.500000in}%
\pgfsys@useobject{currentmarker}{}%
\end{pgfscope}%
\end{pgfscope}%
\begin{pgfscope}%
\definecolor{textcolor}{rgb}{0.000000,0.000000,0.000000}%
\pgfsetstrokecolor{textcolor}%
\pgfsetfillcolor{textcolor}%
\pgftext[x=3.849517in,y=0.402778in,,top]{\color{textcolor}\rmfamily\fontsize{10.000000}{12.000000}\selectfont \(\displaystyle {2000}\)}%
\end{pgfscope}%
\begin{pgfscope}%
\pgfsetbuttcap%
\pgfsetroundjoin%
\definecolor{currentfill}{rgb}{0.000000,0.000000,0.000000}%
\pgfsetfillcolor{currentfill}%
\pgfsetlinewidth{0.803000pt}%
\definecolor{currentstroke}{rgb}{0.000000,0.000000,0.000000}%
\pgfsetstrokecolor{currentstroke}%
\pgfsetdash{}{0pt}%
\pgfsys@defobject{currentmarker}{\pgfqpoint{0.000000in}{-0.048611in}}{\pgfqpoint{0.000000in}{0.000000in}}{%
\pgfpathmoveto{\pgfqpoint{0.000000in}{0.000000in}}%
\pgfpathlineto{\pgfqpoint{0.000000in}{-0.048611in}}%
\pgfusepath{stroke,fill}%
}%
\begin{pgfscope}%
\pgfsys@transformshift{4.573366in}{0.500000in}%
\pgfsys@useobject{currentmarker}{}%
\end{pgfscope}%
\end{pgfscope}%
\begin{pgfscope}%
\definecolor{textcolor}{rgb}{0.000000,0.000000,0.000000}%
\pgfsetstrokecolor{textcolor}%
\pgfsetfillcolor{textcolor}%
\pgftext[x=4.573366in,y=0.402778in,,top]{\color{textcolor}\rmfamily\fontsize{10.000000}{12.000000}\selectfont \(\displaystyle {2500}\)}%
\end{pgfscope}%
\begin{pgfscope}%
\pgfsetbuttcap%
\pgfsetroundjoin%
\definecolor{currentfill}{rgb}{0.000000,0.000000,0.000000}%
\pgfsetfillcolor{currentfill}%
\pgfsetlinewidth{0.803000pt}%
\definecolor{currentstroke}{rgb}{0.000000,0.000000,0.000000}%
\pgfsetstrokecolor{currentstroke}%
\pgfsetdash{}{0pt}%
\pgfsys@defobject{currentmarker}{\pgfqpoint{0.000000in}{-0.048611in}}{\pgfqpoint{0.000000in}{0.000000in}}{%
\pgfpathmoveto{\pgfqpoint{0.000000in}{0.000000in}}%
\pgfpathlineto{\pgfqpoint{0.000000in}{-0.048611in}}%
\pgfusepath{stroke,fill}%
}%
\begin{pgfscope}%
\pgfsys@transformshift{5.297214in}{0.500000in}%
\pgfsys@useobject{currentmarker}{}%
\end{pgfscope}%
\end{pgfscope}%
\begin{pgfscope}%
\definecolor{textcolor}{rgb}{0.000000,0.000000,0.000000}%
\pgfsetstrokecolor{textcolor}%
\pgfsetfillcolor{textcolor}%
\pgftext[x=5.297214in,y=0.402778in,,top]{\color{textcolor}\rmfamily\fontsize{10.000000}{12.000000}\selectfont \(\displaystyle {3000}\)}%
\end{pgfscope}%
\begin{pgfscope}%
\definecolor{textcolor}{rgb}{0.000000,0.000000,0.000000}%
\pgfsetstrokecolor{textcolor}%
\pgfsetfillcolor{textcolor}%
\pgftext[x=3.075000in,y=0.223766in,,top]{\color{textcolor}\rmfamily\fontsize{10.000000}{12.000000}\selectfont Number of lines}%
\end{pgfscope}%
\begin{pgfscope}%
\pgfsetbuttcap%
\pgfsetroundjoin%
\definecolor{currentfill}{rgb}{0.000000,0.000000,0.000000}%
\pgfsetfillcolor{currentfill}%
\pgfsetlinewidth{0.803000pt}%
\definecolor{currentstroke}{rgb}{0.000000,0.000000,0.000000}%
\pgfsetstrokecolor{currentstroke}%
\pgfsetdash{}{0pt}%
\pgfsys@defobject{currentmarker}{\pgfqpoint{-0.048611in}{0.000000in}}{\pgfqpoint{-0.000000in}{0.000000in}}{%
\pgfpathmoveto{\pgfqpoint{-0.000000in}{0.000000in}}%
\pgfpathlineto{\pgfqpoint{-0.048611in}{0.000000in}}%
\pgfusepath{stroke,fill}%
}%
\begin{pgfscope}%
\pgfsys@transformshift{0.750000in}{0.635833in}%
\pgfsys@useobject{currentmarker}{}%
\end{pgfscope}%
\end{pgfscope}%
\begin{pgfscope}%
\definecolor{textcolor}{rgb}{0.000000,0.000000,0.000000}%
\pgfsetstrokecolor{textcolor}%
\pgfsetfillcolor{textcolor}%
\pgftext[x=0.583333in, y=0.587607in, left, base]{\color{textcolor}\rmfamily\fontsize{10.000000}{12.000000}\selectfont \(\displaystyle {0}\)}%
\end{pgfscope}%
\begin{pgfscope}%
\pgfsetbuttcap%
\pgfsetroundjoin%
\definecolor{currentfill}{rgb}{0.000000,0.000000,0.000000}%
\pgfsetfillcolor{currentfill}%
\pgfsetlinewidth{0.803000pt}%
\definecolor{currentstroke}{rgb}{0.000000,0.000000,0.000000}%
\pgfsetstrokecolor{currentstroke}%
\pgfsetdash{}{0pt}%
\pgfsys@defobject{currentmarker}{\pgfqpoint{-0.048611in}{0.000000in}}{\pgfqpoint{-0.000000in}{0.000000in}}{%
\pgfpathmoveto{\pgfqpoint{-0.000000in}{0.000000in}}%
\pgfpathlineto{\pgfqpoint{-0.048611in}{0.000000in}}%
\pgfusepath{stroke,fill}%
}%
\begin{pgfscope}%
\pgfsys@transformshift{0.750000in}{1.115891in}%
\pgfsys@useobject{currentmarker}{}%
\end{pgfscope}%
\end{pgfscope}%
\begin{pgfscope}%
\definecolor{textcolor}{rgb}{0.000000,0.000000,0.000000}%
\pgfsetstrokecolor{textcolor}%
\pgfsetfillcolor{textcolor}%
\pgftext[x=0.444444in, y=1.067666in, left, base]{\color{textcolor}\rmfamily\fontsize{10.000000}{12.000000}\selectfont \(\displaystyle {100}\)}%
\end{pgfscope}%
\begin{pgfscope}%
\pgfsetbuttcap%
\pgfsetroundjoin%
\definecolor{currentfill}{rgb}{0.000000,0.000000,0.000000}%
\pgfsetfillcolor{currentfill}%
\pgfsetlinewidth{0.803000pt}%
\definecolor{currentstroke}{rgb}{0.000000,0.000000,0.000000}%
\pgfsetstrokecolor{currentstroke}%
\pgfsetdash{}{0pt}%
\pgfsys@defobject{currentmarker}{\pgfqpoint{-0.048611in}{0.000000in}}{\pgfqpoint{-0.000000in}{0.000000in}}{%
\pgfpathmoveto{\pgfqpoint{-0.000000in}{0.000000in}}%
\pgfpathlineto{\pgfqpoint{-0.048611in}{0.000000in}}%
\pgfusepath{stroke,fill}%
}%
\begin{pgfscope}%
\pgfsys@transformshift{0.750000in}{1.595950in}%
\pgfsys@useobject{currentmarker}{}%
\end{pgfscope}%
\end{pgfscope}%
\begin{pgfscope}%
\definecolor{textcolor}{rgb}{0.000000,0.000000,0.000000}%
\pgfsetstrokecolor{textcolor}%
\pgfsetfillcolor{textcolor}%
\pgftext[x=0.444444in, y=1.547724in, left, base]{\color{textcolor}\rmfamily\fontsize{10.000000}{12.000000}\selectfont \(\displaystyle {200}\)}%
\end{pgfscope}%
\begin{pgfscope}%
\pgfsetbuttcap%
\pgfsetroundjoin%
\definecolor{currentfill}{rgb}{0.000000,0.000000,0.000000}%
\pgfsetfillcolor{currentfill}%
\pgfsetlinewidth{0.803000pt}%
\definecolor{currentstroke}{rgb}{0.000000,0.000000,0.000000}%
\pgfsetstrokecolor{currentstroke}%
\pgfsetdash{}{0pt}%
\pgfsys@defobject{currentmarker}{\pgfqpoint{-0.048611in}{0.000000in}}{\pgfqpoint{-0.000000in}{0.000000in}}{%
\pgfpathmoveto{\pgfqpoint{-0.000000in}{0.000000in}}%
\pgfpathlineto{\pgfqpoint{-0.048611in}{0.000000in}}%
\pgfusepath{stroke,fill}%
}%
\begin{pgfscope}%
\pgfsys@transformshift{0.750000in}{2.076008in}%
\pgfsys@useobject{currentmarker}{}%
\end{pgfscope}%
\end{pgfscope}%
\begin{pgfscope}%
\definecolor{textcolor}{rgb}{0.000000,0.000000,0.000000}%
\pgfsetstrokecolor{textcolor}%
\pgfsetfillcolor{textcolor}%
\pgftext[x=0.444444in, y=2.027783in, left, base]{\color{textcolor}\rmfamily\fontsize{10.000000}{12.000000}\selectfont \(\displaystyle {300}\)}%
\end{pgfscope}%
\begin{pgfscope}%
\pgfsetbuttcap%
\pgfsetroundjoin%
\definecolor{currentfill}{rgb}{0.000000,0.000000,0.000000}%
\pgfsetfillcolor{currentfill}%
\pgfsetlinewidth{0.803000pt}%
\definecolor{currentstroke}{rgb}{0.000000,0.000000,0.000000}%
\pgfsetstrokecolor{currentstroke}%
\pgfsetdash{}{0pt}%
\pgfsys@defobject{currentmarker}{\pgfqpoint{-0.048611in}{0.000000in}}{\pgfqpoint{-0.000000in}{0.000000in}}{%
\pgfpathmoveto{\pgfqpoint{-0.000000in}{0.000000in}}%
\pgfpathlineto{\pgfqpoint{-0.048611in}{0.000000in}}%
\pgfusepath{stroke,fill}%
}%
\begin{pgfscope}%
\pgfsys@transformshift{0.750000in}{2.556067in}%
\pgfsys@useobject{currentmarker}{}%
\end{pgfscope}%
\end{pgfscope}%
\begin{pgfscope}%
\definecolor{textcolor}{rgb}{0.000000,0.000000,0.000000}%
\pgfsetstrokecolor{textcolor}%
\pgfsetfillcolor{textcolor}%
\pgftext[x=0.444444in, y=2.507841in, left, base]{\color{textcolor}\rmfamily\fontsize{10.000000}{12.000000}\selectfont \(\displaystyle {400}\)}%
\end{pgfscope}%
\begin{pgfscope}%
\pgfsetbuttcap%
\pgfsetroundjoin%
\definecolor{currentfill}{rgb}{0.000000,0.000000,0.000000}%
\pgfsetfillcolor{currentfill}%
\pgfsetlinewidth{0.803000pt}%
\definecolor{currentstroke}{rgb}{0.000000,0.000000,0.000000}%
\pgfsetstrokecolor{currentstroke}%
\pgfsetdash{}{0pt}%
\pgfsys@defobject{currentmarker}{\pgfqpoint{-0.048611in}{0.000000in}}{\pgfqpoint{-0.000000in}{0.000000in}}{%
\pgfpathmoveto{\pgfqpoint{-0.000000in}{0.000000in}}%
\pgfpathlineto{\pgfqpoint{-0.048611in}{0.000000in}}%
\pgfusepath{stroke,fill}%
}%
\begin{pgfscope}%
\pgfsys@transformshift{0.750000in}{3.036125in}%
\pgfsys@useobject{currentmarker}{}%
\end{pgfscope}%
\end{pgfscope}%
\begin{pgfscope}%
\definecolor{textcolor}{rgb}{0.000000,0.000000,0.000000}%
\pgfsetstrokecolor{textcolor}%
\pgfsetfillcolor{textcolor}%
\pgftext[x=0.444444in, y=2.987900in, left, base]{\color{textcolor}\rmfamily\fontsize{10.000000}{12.000000}\selectfont \(\displaystyle {500}\)}%
\end{pgfscope}%
\begin{pgfscope}%
\pgfsetbuttcap%
\pgfsetroundjoin%
\definecolor{currentfill}{rgb}{0.000000,0.000000,0.000000}%
\pgfsetfillcolor{currentfill}%
\pgfsetlinewidth{0.803000pt}%
\definecolor{currentstroke}{rgb}{0.000000,0.000000,0.000000}%
\pgfsetstrokecolor{currentstroke}%
\pgfsetdash{}{0pt}%
\pgfsys@defobject{currentmarker}{\pgfqpoint{-0.048611in}{0.000000in}}{\pgfqpoint{-0.000000in}{0.000000in}}{%
\pgfpathmoveto{\pgfqpoint{-0.000000in}{0.000000in}}%
\pgfpathlineto{\pgfqpoint{-0.048611in}{0.000000in}}%
\pgfusepath{stroke,fill}%
}%
\begin{pgfscope}%
\pgfsys@transformshift{0.750000in}{3.516184in}%
\pgfsys@useobject{currentmarker}{}%
\end{pgfscope}%
\end{pgfscope}%
\begin{pgfscope}%
\definecolor{textcolor}{rgb}{0.000000,0.000000,0.000000}%
\pgfsetstrokecolor{textcolor}%
\pgfsetfillcolor{textcolor}%
\pgftext[x=0.444444in, y=3.467958in, left, base]{\color{textcolor}\rmfamily\fontsize{10.000000}{12.000000}\selectfont \(\displaystyle {600}\)}%
\end{pgfscope}%
\begin{pgfscope}%
\definecolor{textcolor}{rgb}{0.000000,0.000000,0.000000}%
\pgfsetstrokecolor{textcolor}%
\pgfsetfillcolor{textcolor}%
\pgftext[x=0.388888in,y=2.010000in,,bottom,rotate=90.000000]{\color{textcolor}\rmfamily\fontsize{10.000000}{12.000000}\selectfont Time (ms)}%
\end{pgfscope}%
\begin{pgfscope}%
\pgfsetrectcap%
\pgfsetmiterjoin%
\pgfsetlinewidth{0.803000pt}%
\definecolor{currentstroke}{rgb}{0.000000,0.000000,0.000000}%
\pgfsetstrokecolor{currentstroke}%
\pgfsetdash{}{0pt}%
\pgfpathmoveto{\pgfqpoint{0.750000in}{0.500000in}}%
\pgfpathlineto{\pgfqpoint{0.750000in}{3.520000in}}%
\pgfusepath{stroke}%
\end{pgfscope}%
\begin{pgfscope}%
\pgfsetrectcap%
\pgfsetmiterjoin%
\pgfsetlinewidth{0.803000pt}%
\definecolor{currentstroke}{rgb}{0.000000,0.000000,0.000000}%
\pgfsetstrokecolor{currentstroke}%
\pgfsetdash{}{0pt}%
\pgfpathmoveto{\pgfqpoint{5.400000in}{0.500000in}}%
\pgfpathlineto{\pgfqpoint{5.400000in}{3.520000in}}%
\pgfusepath{stroke}%
\end{pgfscope}%
\begin{pgfscope}%
\pgfsetrectcap%
\pgfsetmiterjoin%
\pgfsetlinewidth{0.803000pt}%
\definecolor{currentstroke}{rgb}{0.000000,0.000000,0.000000}%
\pgfsetstrokecolor{currentstroke}%
\pgfsetdash{}{0pt}%
\pgfpathmoveto{\pgfqpoint{0.750000in}{0.500000in}}%
\pgfpathlineto{\pgfqpoint{5.400000in}{0.500000in}}%
\pgfusepath{stroke}%
\end{pgfscope}%
\begin{pgfscope}%
\pgfsetrectcap%
\pgfsetmiterjoin%
\pgfsetlinewidth{0.803000pt}%
\definecolor{currentstroke}{rgb}{0.000000,0.000000,0.000000}%
\pgfsetstrokecolor{currentstroke}%
\pgfsetdash{}{0pt}%
\pgfpathmoveto{\pgfqpoint{0.750000in}{3.520000in}}%
\pgfpathlineto{\pgfqpoint{5.400000in}{3.520000in}}%
\pgfusepath{stroke}%
\end{pgfscope}%
\begin{pgfscope}%
\pgfsetbuttcap%
\pgfsetmiterjoin%
\definecolor{currentfill}{rgb}{1.000000,1.000000,1.000000}%
\pgfsetfillcolor{currentfill}%
\pgfsetfillopacity{0.800000}%
\pgfsetlinewidth{1.003750pt}%
\definecolor{currentstroke}{rgb}{0.800000,0.800000,0.800000}%
\pgfsetstrokecolor{currentstroke}%
\pgfsetstrokeopacity{0.800000}%
\pgfsetdash{}{0pt}%
\pgfpathmoveto{\pgfqpoint{0.847222in}{2.634969in}}%
\pgfpathlineto{\pgfqpoint{1.430556in}{2.634969in}}%
\pgfpathquadraticcurveto{\pgfqpoint{1.458334in}{2.634969in}}{\pgfqpoint{1.458334in}{2.662747in}}%
\pgfpathlineto{\pgfqpoint{1.458334in}{3.422778in}}%
\pgfpathquadraticcurveto{\pgfqpoint{1.458334in}{3.450556in}}{\pgfqpoint{1.430556in}{3.450556in}}%
\pgfpathlineto{\pgfqpoint{0.847222in}{3.450556in}}%
\pgfpathquadraticcurveto{\pgfqpoint{0.819444in}{3.450556in}}{\pgfqpoint{0.819444in}{3.422778in}}%
\pgfpathlineto{\pgfqpoint{0.819444in}{2.662747in}}%
\pgfpathquadraticcurveto{\pgfqpoint{0.819444in}{2.634969in}}{\pgfqpoint{0.847222in}{2.634969in}}%
\pgfpathclose%
\pgfusepath{stroke,fill}%
\end{pgfscope}%
\begin{pgfscope}%
\definecolor{textcolor}{rgb}{0.000000,0.000000,0.000000}%
\pgfsetstrokecolor{textcolor}%
\pgfsetfillcolor{textcolor}%
\pgftext[x=0.904128in,y=3.298549in,left,base]{\color{textcolor}\rmfamily\fontsize{10.000000}{12.000000}\selectfont \# Lints}%
\end{pgfscope}%
\begin{pgfscope}%
\pgfsetbuttcap%
\pgfsetroundjoin%
\definecolor{currentfill}{rgb}{0.121569,0.466667,0.705882}%
\pgfsetfillcolor{currentfill}%
\pgfsetlinewidth{1.003750pt}%
\definecolor{currentstroke}{rgb}{0.121569,0.466667,0.705882}%
\pgfsetstrokecolor{currentstroke}%
\pgfsetdash{}{0pt}%
\pgfsys@defobject{currentmarker}{\pgfqpoint{-0.009821in}{-0.009821in}}{\pgfqpoint{0.009821in}{0.009821in}}{%
\pgfpathmoveto{\pgfqpoint{0.000000in}{-0.009821in}}%
\pgfpathcurveto{\pgfqpoint{0.002605in}{-0.009821in}}{\pgfqpoint{0.005103in}{-0.008786in}}{\pgfqpoint{0.006944in}{-0.006944in}}%
\pgfpathcurveto{\pgfqpoint{0.008786in}{-0.005103in}}{\pgfqpoint{0.009821in}{-0.002605in}}{\pgfqpoint{0.009821in}{0.000000in}}%
\pgfpathcurveto{\pgfqpoint{0.009821in}{0.002605in}}{\pgfqpoint{0.008786in}{0.005103in}}{\pgfqpoint{0.006944in}{0.006944in}}%
\pgfpathcurveto{\pgfqpoint{0.005103in}{0.008786in}}{\pgfqpoint{0.002605in}{0.009821in}}{\pgfqpoint{0.000000in}{0.009821in}}%
\pgfpathcurveto{\pgfqpoint{-0.002605in}{0.009821in}}{\pgfqpoint{-0.005103in}{0.008786in}}{\pgfqpoint{-0.006944in}{0.006944in}}%
\pgfpathcurveto{\pgfqpoint{-0.008786in}{0.005103in}}{\pgfqpoint{-0.009821in}{0.002605in}}{\pgfqpoint{-0.009821in}{0.000000in}}%
\pgfpathcurveto{\pgfqpoint{-0.009821in}{-0.002605in}}{\pgfqpoint{-0.008786in}{-0.005103in}}{\pgfqpoint{-0.006944in}{-0.006944in}}%
\pgfpathcurveto{\pgfqpoint{-0.005103in}{-0.008786in}}{\pgfqpoint{-0.002605in}{-0.009821in}}{\pgfqpoint{0.000000in}{-0.009821in}}%
\pgfpathclose%
\pgfusepath{stroke,fill}%
}%
\begin{pgfscope}%
\pgfsys@transformshift{1.013889in}{3.153488in}%
\pgfsys@useobject{currentmarker}{}%
\end{pgfscope}%
\end{pgfscope}%
\begin{pgfscope}%
\definecolor{textcolor}{rgb}{0.000000,0.000000,0.000000}%
\pgfsetstrokecolor{textcolor}%
\pgfsetfillcolor{textcolor}%
\pgftext[x=1.263889in,y=3.104877in,left,base]{\color{textcolor}\rmfamily\fontsize{10.000000}{12.000000}\selectfont \(\displaystyle {0}\)}%
\end{pgfscope}%
\begin{pgfscope}%
\pgfsetbuttcap%
\pgfsetroundjoin%
\definecolor{currentfill}{rgb}{0.121569,0.466667,0.705882}%
\pgfsetfillcolor{currentfill}%
\pgfsetlinewidth{1.003750pt}%
\definecolor{currentstroke}{rgb}{0.121569,0.466667,0.705882}%
\pgfsetstrokecolor{currentstroke}%
\pgfsetdash{}{0pt}%
\pgfsys@defobject{currentmarker}{\pgfqpoint{-0.032572in}{-0.032572in}}{\pgfqpoint{0.032572in}{0.032572in}}{%
\pgfpathmoveto{\pgfqpoint{0.000000in}{-0.032572in}}%
\pgfpathcurveto{\pgfqpoint{0.008638in}{-0.032572in}}{\pgfqpoint{0.016924in}{-0.029140in}}{\pgfqpoint{0.023032in}{-0.023032in}}%
\pgfpathcurveto{\pgfqpoint{0.029140in}{-0.016924in}}{\pgfqpoint{0.032572in}{-0.008638in}}{\pgfqpoint{0.032572in}{0.000000in}}%
\pgfpathcurveto{\pgfqpoint{0.032572in}{0.008638in}}{\pgfqpoint{0.029140in}{0.016924in}}{\pgfqpoint{0.023032in}{0.023032in}}%
\pgfpathcurveto{\pgfqpoint{0.016924in}{0.029140in}}{\pgfqpoint{0.008638in}{0.032572in}}{\pgfqpoint{0.000000in}{0.032572in}}%
\pgfpathcurveto{\pgfqpoint{-0.008638in}{0.032572in}}{\pgfqpoint{-0.016924in}{0.029140in}}{\pgfqpoint{-0.023032in}{0.023032in}}%
\pgfpathcurveto{\pgfqpoint{-0.029140in}{0.016924in}}{\pgfqpoint{-0.032572in}{0.008638in}}{\pgfqpoint{-0.032572in}{0.000000in}}%
\pgfpathcurveto{\pgfqpoint{-0.032572in}{-0.008638in}}{\pgfqpoint{-0.029140in}{-0.016924in}}{\pgfqpoint{-0.023032in}{-0.023032in}}%
\pgfpathcurveto{\pgfqpoint{-0.016924in}{-0.029140in}}{\pgfqpoint{-0.008638in}{-0.032572in}}{\pgfqpoint{0.000000in}{-0.032572in}}%
\pgfpathclose%
\pgfusepath{stroke,fill}%
}%
\begin{pgfscope}%
\pgfsys@transformshift{1.013889in}{2.959815in}%
\pgfsys@useobject{currentmarker}{}%
\end{pgfscope}%
\end{pgfscope}%
\begin{pgfscope}%
\definecolor{textcolor}{rgb}{0.000000,0.000000,0.000000}%
\pgfsetstrokecolor{textcolor}%
\pgfsetfillcolor{textcolor}%
\pgftext[x=1.263889in,y=2.911204in,left,base]{\color{textcolor}\rmfamily\fontsize{10.000000}{12.000000}\selectfont \(\displaystyle {20}\)}%
\end{pgfscope}%
\begin{pgfscope}%
\pgfsetbuttcap%
\pgfsetroundjoin%
\definecolor{currentfill}{rgb}{0.121569,0.466667,0.705882}%
\pgfsetfillcolor{currentfill}%
\pgfsetlinewidth{1.003750pt}%
\definecolor{currentstroke}{rgb}{0.121569,0.466667,0.705882}%
\pgfsetstrokecolor{currentstroke}%
\pgfsetdash{}{0pt}%
\pgfsys@defobject{currentmarker}{\pgfqpoint{-0.045005in}{-0.045005in}}{\pgfqpoint{0.045005in}{0.045005in}}{%
\pgfpathmoveto{\pgfqpoint{0.000000in}{-0.045005in}}%
\pgfpathcurveto{\pgfqpoint{0.011936in}{-0.045005in}}{\pgfqpoint{0.023384in}{-0.040263in}}{\pgfqpoint{0.031823in}{-0.031823in}}%
\pgfpathcurveto{\pgfqpoint{0.040263in}{-0.023384in}}{\pgfqpoint{0.045005in}{-0.011936in}}{\pgfqpoint{0.045005in}{0.000000in}}%
\pgfpathcurveto{\pgfqpoint{0.045005in}{0.011936in}}{\pgfqpoint{0.040263in}{0.023384in}}{\pgfqpoint{0.031823in}{0.031823in}}%
\pgfpathcurveto{\pgfqpoint{0.023384in}{0.040263in}}{\pgfqpoint{0.011936in}{0.045005in}}{\pgfqpoint{0.000000in}{0.045005in}}%
\pgfpathcurveto{\pgfqpoint{-0.011936in}{0.045005in}}{\pgfqpoint{-0.023384in}{0.040263in}}{\pgfqpoint{-0.031823in}{0.031823in}}%
\pgfpathcurveto{\pgfqpoint{-0.040263in}{0.023384in}}{\pgfqpoint{-0.045005in}{0.011936in}}{\pgfqpoint{-0.045005in}{0.000000in}}%
\pgfpathcurveto{\pgfqpoint{-0.045005in}{-0.011936in}}{\pgfqpoint{-0.040263in}{-0.023384in}}{\pgfqpoint{-0.031823in}{-0.031823in}}%
\pgfpathcurveto{\pgfqpoint{-0.023384in}{-0.040263in}}{\pgfqpoint{-0.011936in}{-0.045005in}}{\pgfqpoint{0.000000in}{-0.045005in}}%
\pgfpathclose%
\pgfusepath{stroke,fill}%
}%
\begin{pgfscope}%
\pgfsys@transformshift{1.013889in}{2.766142in}%
\pgfsys@useobject{currentmarker}{}%
\end{pgfscope}%
\end{pgfscope}%
\begin{pgfscope}%
\definecolor{textcolor}{rgb}{0.000000,0.000000,0.000000}%
\pgfsetstrokecolor{textcolor}%
\pgfsetfillcolor{textcolor}%
\pgftext[x=1.263889in,y=2.717531in,left,base]{\color{textcolor}\rmfamily\fontsize{10.000000}{12.000000}\selectfont \(\displaystyle {40}\)}%
\end{pgfscope}%
\end{pgfpicture}%
\makeatother%
\endgroup%

    \caption[Linting time (Theories smaller than 3000 lines)]{Time taken to lint a theory depending on its
    length and the number of lints (restricted to
    theories shorter than 3000 lines) } \label{fig:timing_small}
\end{figure}

To get more insight on where time is spent during linting, the BDD
session from the AFP is linted with the VisualVM\footnote{\url{https://visualvm.github.io/}} profiler attached. For reference, the 
session has a total of 11058 lines of theory from which the linter generated
72 suggestions. Data from the profiler suggests that \SI{31.67}{\percent} of the time 
is spent by the linter trying the different checks, and the rest
\SI{68.33}{\percent} is spent by the reporter converting the results to the required
format (JSON in this case). The reporter spent almost all the time 
converting text offsets to the "line, column" format. Text offsets represent
how many characters are there before a certain position,
which represents the way position information is communicated within Isabelle.
The more usual "line, column" format makes it easier for users to navigate
the sources and facilitates integration with other tools 
like Language Server Protocol clients.
Creating reporters that do not convert
text offsets to lines and columns could cause a significant speedup
in \textit{relative} terms. In absolute terms, for example, it took the 
linter
a total of 1039 milliseconds to process the BDD session, meaning
around 710 milliseconds were spent in the reporter. This is not
a noticeable difference, especially when considering the time taken to
also generate the snapshots: around 2 minutes and 31 seconds are spent
in total during the invocation of the tool. The time effectively needed
by the linter to process the theories is rather insignificant.
