\chapter{Conclusion}\label{chapter:conclusion}

The lint support in automatic proof assistants is underwhelming when compared with
programming languages. Although guides and standards exist, they could not offer
automated feedback to users: it is the responsibility of the proof writers and maintainers to
learn about these standards and try to abide by them.

In this thesis, we attempt to fill this gap. We explored and implemented a linter for 
Isabelle in Isabelle/Scala, that involves 20 checks. In addition, it provides
Isabelle/jEdit integration to assist with interactive proof development
and a command line tool available through \texttt{isabelle lint}.
\textit{Extensibility} and \textit{configurability} were two main goals of
the design. With the help of the linter, we generated 252 suggestions
for Isabelle/HOL and 575 for a select entries from the Archive of Formal
Proofs, that could potentially benefit maintainability and readability.

\section{Future Work}
Although the linter is usable in its current state, the goal was to ``get the ball
rolling'' and provide a proof of concept on how a linter for Isabelle might be
developed. This seems a successful attempt that provides a solid
ground for improvements and further development, such as:
\begin{itemize}
    \item Applying the suggestions generated by the linter
    \item Developing more lints
    \item Exploring more lint types, such as the ones for the inner syntax
    \item Exploring CI/CD integration
    \item Expanding the support to Isabelle/VSCode by integrating
    with the Isabelle/PIDE language server protocol
\end{itemize}
It might also be interesting to consider further expanding the Isabelle ecosystem, 
by:
\begin{itemize}
    \item Developing and using a component that parses the whole
    Isabelle syntax to an \textit{AST}
    \item Exploring a formatter or lints that are specialized in how theories 
    should be formatted
\end{itemize}
Lastly, it is absolutely crucial to interact with the Isabelle community to
further evaluate the usability of the linter, get feedback on the lints 
implemented, and explore what extra features are requested.