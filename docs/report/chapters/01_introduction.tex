\chapter{Introduction}\label{chapter:introduction}

Isabelle makes it possible to formalize and verify mathematical concepts in an
expressive and interactive way by constructing machine-verifiable proofs: these can
be, for example, executed to ensure that a change in an implementation does not
affect the correctness of the underlying algorithm.
However, not all proofs are created equal: some may be harder to read,
or be more prone to break because of new releases or minor changes
in their dependencies.
Consider this hypothetical example \cite{klein_2015}:
\begin{lstlisting}
apply clarsimp
apply (rule my_rule)
 apply (fastforce simp: foo)
proof safe
  fix new
  assume "something surpising"
  show "unforeseen"
\end{lstlisting}
The proof starts with iterative tactic applications (the \texttt{apply} commands) and then switches
to a structured Isar proof, where the goals are explicitly stated.
The structured proof is prone to break if the goals generated by the apply-script
slightly change, which may occur, for instance, due to improvements in the
simplifier. It is also hard to read and understand without running 
Isabelle. A better alternative might be to rewrite this into a fully structured 
proof.

Correctly identifying these problematic constructs -- especially if they are hidden
deep inside a lengthy theory with complicated formalizations -- requires time, 
effort, and an experienced eye that knows the standards to adhere. For these reasons, 
this manual process does not scale.

In this thesis, we aim to help make Isabelle proofs more future-proof by developing a 
linter. Linters are static analysis tools that help catch bugs and warn about bad
practices.
By filling this gap in the Isabelle ecosystem, it can help catch
problems that might hinder readability and maintainability as early as possible e.g.\ 
interactively while the proof is being written, or at a later point on finished theories.
The linter is implemented in \textit{Isabelle/Scala} and
can be used with \textit{Isabelle/jEdit} or through a dedicated command line 
tool,
\texttt{isabelle lint}. It includes 20 checks, mainly from 
\textit{Gerwin Klein's Style Guide for Isabelle/HOL} \cite{klein_2015,klein_2015_2}.

\subsubsection{Outline}
First, \Cref{chapter:related,chapter:preliminaries} examine the required background through 
covering linters in programming languages, illustrating how the quality of
formalizations is maintained in proof assistants, and introducing key concepts
of the Isabelle environment needed throughout the thesis.

After that, \Cref{chapter:implementation} details the implementation of the linter and outlines 
its architecture and the design choices that shaped its development.

Next, \Cref{chapter:evaluation} evaluates the performance of the linter and the quality of 
some selected theories (Isabelle/HOL and some entries from the
Archive of Formal Proofs (AFP)\footnote{\url{https://www.isa-afp.org/}} with the help of the developed lints.

Finally, the \hyperref[chapter:conclusion]{last Chapter} summarizes the work and gives an outlook to motivate
further developments of the linter.