\appendix
\chapter{Lints and Bundles}

Currently, the linter has 20 lints implemented, mainly from 
\textit{Gerwin's Style Guide for Isabelle/HOL} \cite{klein_2015,klein_2015_2}. These are
concise descriptions of each of the checks implemented:
\paragraph{Apply-Isar Switch}\label{lint:applyisarswitch}
Severity: Medium

Switching form an apply script to a structured Isar proof results in an overall 
proof that is hard to read without relying on Isabelle. The Isar proof is also
sensitive to the output of the apply script, and might therefore break easily.
This lints finds such instances.

\paragraph{Auto Structural Composition}
Severity: Low

The check suggests to replace \texttt{apply (auto; ...)}, which might be hard to
reason about: structural composition applies the method on the right-hand side only
to the new goals generated by \texttt{auto}.

\paragraph{Axiomatization with where}
Severity: High

The \texttt{axiomatization} command can potentially introduce inconsistencies into 
the logic when coupled with a where clause. It should be avoided, unless when used
to formalize a logic. This lint detects this problem.

\paragraph{Bad style command}
Severity: Medium

Detects \texttt{back} and \texttt{apply\_end} commands.
\paragraph{Complex Isar initial method}\label{lint:complexisar}
Severity: Medium

Warns about using complex methods in the \texttt{proof} command, like 
\texttt{proof auto}, which can make it difficult to read the proof and 
understanding its goals.

\paragraph{Complex method}\label{lint:complexmethod}
Severity: Medium

Finds complex methods. A complex method is defined as a method that:

\begin{itemize}
    \item has more than one modifier (\texttt{?}, \texttt{+} or \texttt{[]})
    \item or, has modifiers that are not in the outmost level (e.g. \texttt{auto[3] | blast})
    \item or, has three or more combinators (\texttt{|}, \texttt{;} or \texttt{,}) 
\end{itemize}


\paragraph{Counter-example finder}\label{lint:counterexample}
Severity: High

Detects commands that find counter-examples like \texttt{nitpick}.

\paragraph{Diagnostic command}
Severity: Low

Finds diagnostic commands like \texttt{ML\_val}, \texttt{welcome} or 
\texttt{find\_consts}.

\paragraph{Force failure}
Severity: Low

Some proof methods, like \texttt{simp}, might not solve its goal. However, when
this is expected, it might be a good idea to combine it with the
\texttt{fail} method (e.g. \texttt{apply (simp; fail)}) in order to force it to fail
if the goal is not solved. This facilitates finding where proofs breaks because of
changes in the simp set or in the \texttt{simp} tactic itself.
The lint allows for a way to find these \texttt{simp} invocations.


\paragraph{Global attribute changes}
Severity: Low

Declaring lemmas as \texttt{simp} just to be used locally in some some
proofs and then deleting
them from the simp set is to be avoided, since it affects the global simp set and
might make it hard to merge theories together. Instead, these changes should
be localized, for example through \texttt{context}. This lint detects such 
declarations.

\paragraph{Global attribute on unnamed lemma} \label{lint:globalattr}
Severity: High

Finds unnamed lemmas with the \texttt{simp} attribute. This anti-pattern makes
it hard to remove this lemma from the simp set when needed.

\paragraph{Implicit rule}\label{lint:implicitrule}
Severity: Medium

Finds usage of the \texttt{rule} method without explicitly stating which rule
to use. These invocations might fail when the rule discovery strategy changes, and
figuring out which rule worked initially might difficult, making the proof hard to
maintain.
\paragraph{Lemma-transforming attribute}\label{lint:lemmatrans}
Severity: Medium

Warns about using the attributes \texttt{rule\_tac} and \texttt{simplified} on 
lemmas.

\paragraph{Low-level apply chain}\label{lint:lowlevel}
Severity: Low

Detects long apply scripts that use low-level proof methods (like \texttt{rule} or
\texttt{clarsimp}) and are longer than 5 commands.

\paragraph{Proof-finder}
Severity: High

Finds proof-finder commands such as \texttt{try} and \texttt{sledgehammer}.

\paragraph{Short name}
Severity: Low

Checks for names used for \texttt{fun} or \texttt{definition} that are shorter than 
two characters. \textit{Note:} only detects short names that are declared at the
outer syntax level.

\paragraph{Unfinished proof}
Severity: High

Finds \texttt{sorry}, \texttt{oops} and \texttt{\textbackslash<proof>} commands.

\paragraph{Unrestricted auto}\label{lint:unrestrictedauto}
Severity: High

Using \texttt{auto} in the middle of a proof on all goals (i.e. unrestricted) might 
produce an
unpredictable proof state. It should rather be used as a \textit{terminal} proof
method, or be restricted to a set of goals that it fully solves. This lint finds such
unrestricted invocations.

\paragraph{Use by}\label{lint:useby}
Severity: Low

Finds potential \texttt{apply} scripts that can be replaced by the \texttt{by}
command.

\paragraph{Use Isar}
Severity: Low

Triggers whenever the \texttt{apply} command is used, and suggests to use a 
structured proof instead.

\paragraph{}
In addition to these checks, one more debugging lint is implemented:

\paragraph{Print AST}
Severity: Low

Prints the parsed AST.


\paragraph{}
The lints are grouped into \textit{bundles} to provide presets that can be used to
configure the linter. These are:

\paragraph{All}
Includes all lints.
\paragraph{Default}
A default set of lints.

Lints included: \textit{Apply-Isar switch}, \textit{Bad style command}, \textit{Complex Isar initial method}, \textit{Complex method}, \textit{Counter example finder}, \textit{Global attribute changes}, \textit{Global attribute on unnamed lemma}, \textit{Implicit rule}, \textit{Lemma transforming attribute}, \textit{Unrestricted auto}, \textit{Use by}
\paragraph{Foundational}
A set of lints that can be used while defining a new logic. Similar to the default bundle,
but without the \texttt{axiomatization\_with\_where} lint.

Lints included:
 \textit{Apply-Isar switch}, \textit{Bad style command}, \textit{Complex Isar initial method}, \textit{Complex method}, \textit{Global attribute changes}, \textit{Global attribute on unnamed lemma}, \textit{Implicit rule}, \textit{Lemma transforming attribute}, \textit{Unrestricted auto}, \textit{Use by}
 
\paragraph{Afp}
A set of lints that should be used when developing entries for the \textit{Archive of
formal proofs.}

Lints included: 
\textit{Apply-Isar switch}, \textit{Bad style command}, \textit{Complex Isar initial method}, \textit{compleX method}, \textit{Counter-example finder}, \textit{Global attribute changes}, \textit{Global attribute on unnamed lemma}, \textit{Implicit rule}, \textit{Lemma transforming attribute}, \textit{Unrestricted auto}, \textit{Use by}

\paragraph{Pedantic}
A set of lints that are too strict.

Lints included: \textit{Auto structural composition}, \textit{Force failure}, \textit{Low level apply chain}, \textit{Short name}, \textit{Use Isar}


\paragraph{Non interactive}

A set of lints to disable interactive commands.

Lints included: \textit{Counter example finder}, \textit{Diagnostic command}, \textit{Proof finder}, \textit{Unfinished proof}

\chapter{XML and JSON reporting example}\label{app:xml-json}
The following are prettified excerpts of the outputs of \texttt{isabelle lint} in
XML and JSON modes:
\begin{lstlisting}
$ isabelle lint -r json HOL
{
  "reports":[
    ...
    {
      "theory":"HOL.Inductive",
      "report":{
        "results":[
          {
            "name":"use_by",
            "stopPosition":"205:7",
            "stopOffset":6746,
            "edit":{
              "startOffset":6673,
              "stopOffset":6746,
              "replacement":"by (erule gfp_upperbound [THEN subsetD]) (
              erule imageI)",
              "msg":null
            },
            "startOffset":6673,
            "severity":"Low",
            "startPosition":"203:3",
            "commands":[
              -156471,
              -156473,
              -156475
            ]
          }
        ]
      },
      "timing":129
    },
    ...
  ]
}

\end{lstlisting}
\begin{lstlisting}
$ isabelle lint -r xml HOL
<reports>
  ...
  <report theory="HOL.Inductive" timing="129">
      <lint_result lint_name="use_by" 
      lint_message="Use &quot;by&quot; instead of a short apply-script." 
      lint_severity="Low" 
      lint_commands="-111928,-111930,-111932">At <lint_location 
           offset="6673" 
           end_offset="6746">203:3</lint_location>:
 Use &quot;by&quot; instead of a short apply-script.
    Consider: 
          <lint_edit 
           offset="6673" 
           end_offset="6746" 
           content="by (erule gfp_upperbound [THEN subsetD]) (erule imageI)">
            by (erule gfp_upperbound [THEN subsetD]) (erule imageI)
            </lint_edit>
    Name: use_by
    Severity: Low
      </lint_result>
  </report>
  ...
</reports>
\end{lstlisting}